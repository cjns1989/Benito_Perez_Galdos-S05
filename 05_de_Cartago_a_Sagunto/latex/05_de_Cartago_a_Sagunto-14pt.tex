\PassOptionsToPackage{unicode=true}{hyperref} % options for packages loaded elsewhere
\PassOptionsToPackage{hyphens}{url}
%
\documentclass[oneside,14pt,spanish,]{extbook} % cjns1989 - 27112019 - added the oneside option: so that the text jumps left & right when reading on a tablet/ereader
\usepackage{lmodern}
\usepackage{amssymb,amsmath}
\usepackage{ifxetex,ifluatex}
\usepackage{fixltx2e} % provides \textsubscript
\ifnum 0\ifxetex 1\fi\ifluatex 1\fi=0 % if pdftex
  \usepackage[T1]{fontenc}
  \usepackage[utf8]{inputenc}
  \usepackage{textcomp} % provides euro and other symbols
\else % if luatex or xelatex
  \usepackage{unicode-math}
  \defaultfontfeatures{Ligatures=TeX,Scale=MatchLowercase}
%   \setmainfont[]{EBGaramond-Regular}
    \setmainfont[Numbers={OldStyle,Proportional}]{EBGaramond-Regular}      % cjns1989 - 20191129 - old style numbers 
\fi
% use upquote if available, for straight quotes in verbatim environments
\IfFileExists{upquote.sty}{\usepackage{upquote}}{}
% use microtype if available
\IfFileExists{microtype.sty}{%
\usepackage[]{microtype}
\UseMicrotypeSet[protrusion]{basicmath} % disable protrusion for tt fonts
}{}
\usepackage{hyperref}
\hypersetup{
            pdftitle={DE CARTAGO A SAGUNTO},
            pdfauthor={Benito Pérez Galdós},
            pdfborder={0 0 0},
            breaklinks=true}
\urlstyle{same}  % don't use monospace font for urls
\usepackage[papersize={4.80 in, 6.40  in},left=.5 in,right=.5 in]{geometry}
\setlength{\emergencystretch}{3em}  % prevent overfull lines
\providecommand{\tightlist}{%
  \setlength{\itemsep}{0pt}\setlength{\parskip}{0pt}}
\setcounter{secnumdepth}{0}

% set default figure placement to htbp
\makeatletter
\def\fps@figure{htbp}
\makeatother

\usepackage{ragged2e}
\usepackage{epigraph}
\renewcommand{\textflush}{flushepinormal}

\usepackage{indentfirst}

\usepackage{fancyhdr}
\pagestyle{fancy}
\fancyhf{}
\fancyhead[R]{\thepage}
\renewcommand{\headrulewidth}{0pt}
\usepackage{quoting}
\usepackage{ragged2e}

\newlength\mylen
\settowidth\mylen{……………….}

\usepackage{stackengine}
\usepackage{graphicx}
\def\asterism{\par\vspace{1em}{\centering\scalebox{.9}{%
  \stackon[-0.6pt]{\bfseries*~*}{\bfseries*}}\par}\vspace{.8em}\par}

 \usepackage{titlesec}
 \titleformat{\chapter}[display]
  {\normalfont\bfseries\filcenter}{}{0pt}{\Large}
 \titleformat{\section}[display]
  {\normalfont\bfseries\filcenter}{}{0pt}{\Large}
 \titleformat{\subsection}[display]
  {\normalfont\bfseries\filcenter}{}{0pt}{\Large}

\setcounter{secnumdepth}{1}
\ifnum 0\ifxetex 1\fi\ifluatex 1\fi=0 % if pdftex
  \usepackage[shorthands=off,main=spanish]{babel}
\else
  % load polyglossia as late as possible as it *could* call bidi if RTL lang (e.g. Hebrew or Arabic)
%   \usepackage{polyglossia}
%   \setmainlanguage[]{spanish}
%   \usepackage[french]{babel} % cjns1989 - 1.43 version of polyglossia on this system does not allow disabling the autospacing feature
\fi

\title{DE CARTAGO A SAGUNTO}
\author{Benito Pérez Galdós}
\date{}

\begin{document}
\maketitle

\hypertarget{i}{%
\chapter{I}\label{i}}

Arriba otra vez, arriba, Tito pequeñín de cuerpo y de espíritu amplio y
comprensivo; sacude la pereza letal en que caíste después de los
acontecimientos ensoñados y maravillosos que te dieron la visión de un
espléndido porvenir; vuelve a tu normal conocimiento de los hechos
tangibles, que viste y apreciaste en la vida romántica del Cantón
cartaginés, y refiérelos conforme al criterio de honrada veracidad
desnuda que te ha marcado la excelsa maestra \emph{Doña Clío}. Abandona
los incidentes de escaso valor histórico que han ocurrido en los días de
tu descanso soñoliento, y acomete el relato de las altas contiendas
entre cantonales y centralistas, sin prodigar alabanzas dictadas por la
amistad o el amaneramiento retórico.

Obedezco al amigo que me despabila con sacudimiento de brazos y tirones
de orejas, cojo mi estilete y sigo trazando en caracteres duros la
historia de estos años borrascosos en que, por suerte o por desgracia,
me ha tocado vivir. Lo primero que sale a estas páginas llegó a mi
conocimiento por los ojos y por el tacto: fue la moneda que acuñaron los
cantonales para subvenir a las atenciones de la vida social. Consistió
la primera emisión en duros cuya ley superaba en una peseta a la ley de
los duros fabricados en la Casa de Moneda de Madrid. Las inscripciones
decían: por el anverso, \emph{Revolución Cantonal}.---\emph{Cinco
pesetas}; por el reverso, \emph{Cartagena sitiada por los
centralistas}.---\emph{Septiembre de 1873}.

Elogiando yo la perfección del cuño ante los amigos don Pedro Gutiérrez,
Fructuoso Manrique, el Brigadier Pernas y Manolo Cárceles, éste, con su
optimismo que a veces resultaba un tanto candoroso, me dijo: «Fíjese el
buen Tito en que ese trabajo lo han hecho los buenos chicos que en
nuestro presidio sufrían cadena por monederos falsos.» Puse yo un
comentario a esta declaración, diciendo que los tales artífices fueron
maestros antes de ser delincuentes, que en la prisión afinaron su
ingenio, y que la libertad les habilitó para servir a la República con
diligente honradez, cada cual según su oficio. «Así es---dijo
Cárceles,---y da gusto verles por ahí tan tranquilos, sin hacer daño a
nadie, procurando aparecer como los más fieles y útiles auxiliares del
naciente \emph{Anfictionado español}.» Antes de la emisión de la moneda
se pagaban los servicios con cachos de plata que luego se canjearon por
los flamantes y bien pronto acreditados duros de Cartagena.

En los mismos días me enteré por los amigos de la nueva organización que
se había dado a los altos Poderes Cantonalistas. Dimitió el Gobierno
Provisional, incorporándose a la Junta Soberana, que se fraccionó en las
siguientes Secciones: \emph{De Relaciones Cantonales}, Presidente Roque
Barcia, Secretario Andrés de Salas.---\emph{De Guerra}, Presidente
General Félix Ferrer, Secretario Antonio de la Calle.---\emph{De
Servicios Públicos}, Presidente Alberto Araus, Secretario Manuel F.
Herrero.---\emph{De Hacienda}, Presidente Alfredo Sauvalle, Secretario
Gonzalo Osorio.---\emph{De Justicia}, Presidente Eduardo Romero Germes,
Secretario Andrés Lafuente.---\emph{De Marina}, Presidente Brigadier
Bartolomé Pozas, Secretario Manuel Cárceles Sabater. Los cargos de
Presidente y Secretario de estas Secciones equivalían a los de Ministro
y Subsecretario de los diferentes ramos.

Sin puntualizar una por una las diversas expediciones marítimas que
efectuaron los barcos insurgentes a fines de Septiembre, procuro
corregir mi deficiente sentido cronológico y me apodero de algunas
fechas, claveteando en mi memoria la del 24 porque ella señala mi nada
lucida incorporación a la escuadra que fue al bombardeo de Alicante con
las miras que fácilmente supondrá el lector. Mi amigo Cárceles, que se
empeñaba en hacer de mí una figura heroica, me metió casi a empujones en
el \emph{Fernando el Católico}, vapor de madera, inválido y de perezosos
andares, el cual iba como transporte llevando gente de desembarco ganosa
de probar en una plaza rica la fortaleza de su brazo y el largor de sus
uñas. Al conducirme a bordo, Cárceles puso en mi compañía para mi guarda
y servicio a un presidiario joven, simpático y hablador, que desde el
primer momento me cautivó con su amena charla y la variedad de sus
disposiciones. Antes de bosquejar la figura picaresca de mi adlátere y
edecán, os diré que el Cantón creyó deber patriótico cambiar el nombre
del barco en que íbamos, pues aquello de \emph{Fernando}, con añadidura
de \emph{el Católico}, conservaba el sonsonete del destruido régimen
monárquico y religioso. Para remediar esto buscaron un nombre que
expresase las ideas de rebeldía triunfadora, y no encontraron mejor mote
que el estrambótico y ridículamente enigmático de \emph{Despertador del
Cantón}.

A la hora de navegar en el \emph{Despertador}, mi asistente o machacante
hizo cuanto pudo para mostrarse amigo, refiriéndome con donaire su corta
y patética historia. Resultó que hacía versos. En su infancia se reveló
sacando de su cabeza coplas de ciego; luego enjaretó madrigales,
letrillas y algunas composiciones de arte mayor que corrían manuscritas
entre el vecindario de su pueblo natal, la villa de Mula. Por algunos
trozos que me recitó comprendí que no le faltaban dotes literarias, pero
que las había cultivado sin escuela ni disciplina\ldots{} Casó muy joven
con moza bravía; surgieron disgustos, piques, celeras, choques
violentísimos con varias familias del pueblo. Cándido Palomo, que tal
era su nombre, alpargatero de oficio y en sus ocios poeta libre, llegó
una noche a su casa con el firme propósito de matar a su mujer; mas tuvo
la suerte de equivocarse de víctima y dio muerte a su suegra, que era la
efectiva causante de aquellos líos y el impulso inicial de la tragedia.
Cuando Palomo entró en presidio compuso un poema lacrimoso relatando su
crimen y proceso. Aunque plagado de imperfecciones, el poético engendro
me recordó el libro primero de \emph{Los Tristes} de Ovidio y aquel
verso que empieza \emph{Cum repeto noctem\ldots{}}

Con estas y otras divertidas confidencias de aquel ameno galopín, que
también repitió una letrilla y un romance burlesco que había dedicado a
cantar las malicias de su suegra días antes de despacharla para el otro
mundo, entretuvimos las horas lentas de la travesía, terminada a las
nueve y media de la noche frente a la ciudad del turrón, la dulce
Alicante. El primer cuidado del caudillo cantonal que nos mandaba (y
juro por la laguna Estigia que no sé quién era) fue notificar a los
cónsules que si la plaza no aprontaba buena porción de víveres y
pecunia, conforme al truculento \emph{ultimátum} formulado en viajes
anteriores, comenzaría el bombardeo al amanecer\ldots{} Llegado el
momento, colocadas en orden de batalla las naves guerreras con nuestro
\emph{Despertador} a retaguardia, intervino el Almirante de una escuadra
francesa surta en aquellas aguas, logrando con hábil gestión humanitaria
que se aplazase el bombardeo cuarenta y ocho horas. Pusiéronse a buen
recaudo los vecinos pacíficos de Alicante, y el Gobierno Central,
representado allí por mi amigo Maisonave y por un general cuyo nombre no
figura en mis anotaciones, se preparó para la defensa.

A las seis de la mañana del 27 rompieron el fuego las fragatas
\emph{Numancia}, \emph{Tetuán} y \emph{Méndez Núñez} con pólvora sola, y
como no izase Alicante bandera de parlamento se hicieron disparos con
bala contra el castillo y la ciudad. El castillo, visto desde la mar,
parecíame asentado en la cima de un alto monte de turrón, deleznable
conglomerado de avellanas y miel. A pesar de estas apariencias, nuestros
proyectiles no hicieron allí estrago visible. En la plaza advertimos
señales de gran sufrimiento, y las balas que de allá nos venían apenas
rasparon el blindaje de nuestra \emph{Numancia}. Como tampoco sufrieron
deterioro las inservibles carracas \emph{Tetuán} y \emph{Méndez Núñez},
envejecidas e inútiles en plena juventud, no pude ver en aquella militar
función más que un juego de chicos o un bosquejo parodial de página
histórica, para recreo de gente frívola que se entusiasma con vanos
ruidos y parambombas.

Cinco horas duró el simulacro, disparando nosotros ciento cincuenta
proyectiles que debieron de ser pelotas de mazapán. Total, que Alicante
no dio un cuarto y que nos marchamos con viento fresco, llevando a la
mar la jactanciosa hinchazón de nuestras fantasías. Mientras nosotros
navegábamos hacia Cartagena, ufanándonos de haber impuesto duro castigo
a la plaza centralista, las autoridades de ésta telegrafiaban a Madrid
extravagantes hipérboles del daño que nos habían causado: según ellas,
la obra muerta de nuestras naves estaba hecha pedazos y las cubiertas
sembradas de cadáveres; en tierra, don Eleuterio y el general, cuyo
nombre sigo ignorando, habían afrontado el bombardeo con espartano
heroísmo. Por una parte y otra era todo pueril vanidad y mentirosas
grandezas para engaño de los mismos que las propalaban.

En el viaje de regreso hice amistad con otro galeote, llamado de apodo
\emph{Pepe el Empalmao} por la desmedida talla de su cuerpo flaco y
anguloso. Aprovechando un rato en que mi machacante subió a cubierta,
dejándonos en el primer sollado, me dijo que la bravía mujer de Palomo,
guapa de suyo y mejorada en sus atractivos por los afeites y pulidas
ropas que a la sazón gastaba, hacía en Cartagena vida libre, requiriendo
el trato de señores ricos en casas discretas cuyas paredes eran
reservado encierro del escándalo. Añadió que si yo quería verla y juzgar
por mí mismo su buen apaño de rostro y hechuras, él no tendría
inconveniente en llevarme a donde pudiese encontrarla. El pobre Cándido
conocía el aprovechado mariposeo con que su mujer se ganaba la vida;
visitábala alguna vez; pero ella con buenas o malas razones, según el
viento o el humor reinantes, le apartaba de su lado, dándole algunos
dineros que eran el mejor específico para que el marido se curase del
molesto afán de sus visitas. Comprendí que \emph{Pepe el Empalmao} era
un sutil rufián y le prometí aceptar sus buenos servicios, tan
necesarios, como dice Cervantes, en toda república bien ordenada.

Retirado de mi presencia \emph{El Empalmao} por accidentes del servicio,
volvió junto a mí Cándido Palomo, al cual le faltó tiempo para brindarme
sabrosos apuntes históricos de su camarada. José Tercero, que tal era el
nombre del rufián, había ido a comer el bizcocho y el corbacho del
presidio por ejercer con demasiada sutileza las artes de corrupción,
asistido de una mala hembra, llamada por mal nombre \emph{Marigancho},
que purgaba sus delitos en la Galera de Alcalá de Henares\ldots{} Dejo a
un lado a éste y otros prójimos de interesante psicología, para seguir
desenredando la madeja histórica. La Junta Soberana resolvió canjear con
el comercio, por artículos de comer, beber y arder, gran copia de
materiales existentes en el Arsenal y fortificaciones: bronces, hierros,
maderas finísimas, y cuanto no tenía inmediata eficacia para la defensa
de la plaza. Acordó además la Junta reforzar la guardia de la fábrica de
desplatación y amenazar a varios industriales, entre ellos al marqués de
Figueroa, con el embargo de sus bienes si no pagaban a la Aduana, en el
término de cuatro días, los derechos de Arancel por la importación de
carbón y otros efectos.

Continuaron aquellos días las salidas por mar y tierra. Resistí a las
sugestiones de Gálvez para que le acompañase en una expedición que hizo
a Garrucha con el \emph{Despertador} y la fragata \emph{Tetuán}. Creí
más divertido para mí, y más eficaz para la misma Historia, salir por
las calles de la ciudad con mi amigo \emph{El Empalmao} a la fácil
conquista de Leonarda Bravo o \emph{Leona la Brava}, como vulgarmente
llamaban en Cartagena a la mujer de Palomo. Pronto la encontramos, que
para llegar a la gruta de tal Calipso no era menester larga exploración
por tierras desconocidas. En una casa recatada y silenciosa, medianera
con la vivienda y taller de las tres muchachas retozonas amigas de
Fructuoso, recibió mi visita. Era una mujer bonita y fresca, bien
aderezada para su oficio, cariñosa en el habla y modos, como a sus
livianos tratos correspondía.

Nada advertí en \emph{Leona} que justificara su fama de braveza. A mis
preguntas sobre esto me contestó que la ferocidad de su genio habíala
mostrado tan sólo en el tiempo que hizo vida conyugal con Palomo, por
ser éste un terrible celoso atormentador y un carácter capaz de apurar y
consumir a la misma paciencia. Pero que recobrada la libertad, y
respirando el libre ambiente del mundo para vivir del beneficio que su
propio mérito y gracias le granjeaban, se había trocado de leona
furibunda en oveja mansísima. Ni fue corta ni desabrida mi visita, sino,
antes bien, larga y placentera. . . . . . . . . . . . . .

No sé si en los medios o en el fin de nuestra accidental intimidad,
\emph{Leona} me dijo que no vivía donde estábamos sino en la parte alta
de Santa Lucía. Oyendo esto acordeme de la famosa fragua mitológica y de
la escuela de Floriana. A mis preguntas, sugeridas por el recuerdo de
aquellos lugares, contestó la moza que existía la fragua, que el
patinillo era secadero de una tintorería y la escuela depósito de cosas
de barco. Las maestras puercas y legañosas que allí daban lección a los
chicos harapientos del barrio, se habían largado a otra parte. Esto
avivó mi curiosidad y el deseo de reconocer aquellos lugares, y pidiendo
permiso a \emph{La Brava} para visitarla en su vivienda, nos despedimos
hasta una tarde próxima.

\hypertarget{ii}{%
\chapter{II}\label{ii}}

Que la Junta Suprema de Cartagena autorizase una función dramática en el
teatro Principal, representándose \emph{Juan de Lanuza} y destinando los
productos a los Hospitales, no merece largo espacio en estas crónicas.
Tampoco debo darlo a la expedición de Gálvez a Garrucha, extendiéndose a
Vera y Cuevas de Vera, donde tuvo lucido acogimiento y pudo afanar
dinero y provisiones de boca. La repetición de estas colectas a mano
armada las priva de interés en el ciclo cantonal.

Mejor alimento, lector voraz, siquiera sea de golosinas, te doy
contándote que guiado por mi embajador venustino José Tercero fui a
visitar a \emph{La Brava} en el altillo de Santa Lucía. Entramos a la
vivienda de la moza por la fragua de marras, en la que forjaban clavos
unos vulgarísimos y tiznados herreros, que ni la más remota semejanza
tenían con los gallardos alumnos de Vulcano, y menos con el Titán
hermosísimo en quien los ojos de mi fantasía vieron al creador de mil
hijas de recia voluntad.

Pasamos de allí al patinillo, donde unas mujeres con las manos
carminosas ponían al sol madejas de estambre recién teñido de colorado.
Entramos luego en lo que fue escuela, y vi el local repleto de barriles
de alquitrán, de viejas lonas y de montones de la filástica que se usa
para calafatear las embarcaciones. Ni rastro hallé de objetos escolares.
¡Y pensar que allí se me representó en carne viva la ideal Floriana,
educadora de pueblos, virgen y madre de las generaciones que han de
redimirnos! ¡Qué cosas vio mi espíritu en aquel mísero aposento, y qué
divinos embustes imaginó, pintándolos en la retina, el caldeado cerebro
de este antojadizo historiador!

Introdújome Tercero en un angosto pasillo, que era pórtico de humildes
viviendas numeradas. En la salita de una de éstas encontré a Leonarda
con el cabello suelto, en compañía de una mujer que no era peinadora
sino maestra, y que a mi amiga estaba dando lección de escritura.
\emph{La Brava}, con los dedos tiesos, llenos de tinta y torcida la
boca, hacía tembliqueantes palotes, poniendo en ello toda su alma. La
maestra, con dulce paciencia, guiando la mano de su discípula, la
corregía y amonestaba\ldots{} Pásmate, lector incrédulo, y abre tamaños
ojos al saber que en la profesora reconocí las facciones de \emph{Doña
Caligrafía}, ya envejecidas y deslustradas cual si hubiera pasado medio
siglo desde que la vi o creí verla en la compañía y séquito de la ideal
Floriana.

Deseosa de hablar conmigo, \emph{Leona} suspendió la lección,
despidiendo a la momificada pendolista y a Pepe \emph{el Empalmao}. Sin
más ropa que la camisa y una holgada bata de colorines; sin corsé, los
desnudos pies en chancletas, suelto el negro cabello abundante, Leonarda
ponía la menor veladura posible entre sus corporales hechizos y los ojos
del visitante. Afectuosa y comunicativa, me habló de esta manera:

«Veo que te asombras de que ande yo en estos jeribeques de la escritura.
Pues sabrás que no me contento con ser lo que soy al modo rústico y
ordinario. Me enloquece la ambición. Desde que me metí en este vivir
\emph{arrastrao}, la mirilla en que tengo puestos los ojos de mi alma es
Madrid\ldots{} Quiero \emph{dirme} a la Corte, donde podré ser mujer
alegre con más aquél que aquí, luciendo y aprovechando lo que Dios me ha
dado\ldots{} Comprenderás, querido Tito, que no puedo ir hecha una
burra, pues entonces no me saldría la cuenta, que aquél no es un público
de patanes sino de personas principales y de posibles. Yo sabía leer a
trompicones, y ahora esta pobre maestra que aquí has visto, vecina mía,
por dos reales que le doy un día sí y otro no me enseña la lectura de
corrido, y además me da lección de escritura, empezando por tirar de
palotes que es muy duro ejercicio\ldots{} Pienso yo que la ilustración
es necesaria aun para las que andamos en tratos\ldots{} ya me
entiendes\ldots{} En Madrid haré vida de libertad, pero mirando a lo
elegante y superfirolítico. Como en ello están todos mis pensamientos,
pongo gran atención en el habla de los señores con quienes una noche y
otra noche tengo algo que ver, y cuantas palabritas o frases les oigo,
que a mí me parecen finas, las atrapo y me las remacho en la memoria
para soltarlas cuando vengan a cuento. Ya sé decir: \emph{a tontas y
locas}, \emph{de lo lindo}, \emph{en igualdad de circunstancias},
\emph{partiendo del principio}, \emph{permítame usted que le diga},
\emph{mejorando lo presente}, \emph{tengo la evidencia}, \emph{seamos
imparciales}, \emph{bajo el prisma}, \emph{bajo la base\ldots»}

Discretísimo y práctico me pareció el anhelo de aquella pobre criatura,
que no sabiendo salir de su esfera mísera trataba de ennoblecerla y
darle asomos de dignidad. Felicité sinceramente a \emph{La Brava},
incitándola a que se esmerase en engalanar con flores, siquiera fuesen
de trapo, el camino vicioso que había de seguir, siempre que su destino
no le marcara otro mejor aunque menos bonito. Puso ella a sus
confidencias el remate de esta profecía: «Con lo poquito que ya sé, y lo
que he de aprender, no será difícil que en Madrid me salga un marqués
viejo, rico, baboso, a quien yo pueda manejar como un títere, que me
ponga casa elegante, con alfombras y cortinones de seda, y me vista con
toda la majeza del siglo. \emph{Pa} entonces tendré coche y me pasearé
muy repantigada por las alamedas que llaman el Retiro y la Fuente
Castellana\ldots» Después de esto vino la peinadora. Del tiempo
transcurrido desde la operación de aderezarse la hermosa cabellera hasta
que se puso a almorzar un excelente arroz con pescado, no debo decir
nada a mis lectores, pues la tela de la Historia tiene dobleces
impenetrables.

Vestida y calzada salió \emph{Leona} conmigo al patinillo, donde vimos
un sujeto en mangas de camisa, lavándose la cara en una pobre jofaina de
latón. Mi amiga le saludó risueña, como a vecino que en uno de los
cuartos de aquella humilde casa moraba. Apartándonos de él para dejarle
fregotearse a sus anchas las orejas y el pescuezo, \emph{La Brava} me
dijo: «Este tipo es otro presidiario suelto a quien sus compañeros de
\emph{gurapas} llamaban \emph{don Florestán de Calabria}, y por este
remoquete le conoce todo Cartagena. Es noble, según dice, y desciende de
príncipes napolitanos. Vino a cumplir condena de seis años por enmiendas
que hizo al testamento de una tía suya. Es hombre de historias, de
lenguas, y tan périto en la escritura que no hay letra ni rúbrica que no
imite.»

Al llegarnos otra vez a \emph{don Florestán}, ya estaba el hombre
frotándose las orejas con una toalla no muy limpia. Era un cincuentón de
mediana estatura, cabeza romántica del tipo usual allá por el 45,
ahuecada melena, bigote y perilla corta como los que usaron Espronceda y
los Madrazos. Presentado a él por \emph{Leona}, que le dio el nombre de
\emph{Florestán}, me dijo estrechándome la mano: «Ya le conocía a usted
de vista y por su fama de historiador, señor don Tito. Mucho gusto tengo
en ser su amigo; pero sepa ante todo que ese nombre que me ha dado doña
Leonarda es broma de compañeros maleantes. Yo me llamo Jenaro de
Bocángel, y mi linaje está entroncado con la nobleza española de Nápoles
y Sicilia. ¿Habrá usted oído hablar de los Duques de Amalfi? Pues de
ellos vengo yo por la rama paterna; con los ilustrísimos Marqueses de
Taormina, residentes en Palermo, estaba emparentada mi madre, doña
Celimena de Silva; y no falta en mi sangre algún glóbulo procedente de
la clarísima estirpe de los Escláfanis de Siracusa. Algo más de mi
persona y familia, así como de los vaivenes de mi existencia, he de
contarle a usted\ldots{} Antes le pido permiso para volver a mi aposento
y arreglarme un poco, pues no está bien que los caballeros se presenten
ante sus iguales con este desaliño de andar por casa. Hasta luego.»

Entró corriendo en su vivienda el tronado caballero. Mi amiga y yo nos
quedamos riendo de su estampa fachosa y de sus hinchazones nobiliarias.
Díjome \emph{La Brava} que \emph{don Florestán} era un infeliz de buena
pasta y corazón muy tierno, a pesar de haber cometido el desliz de
aquellas endiabladas escrituras que dieron con sus huesos en el
\emph{estaró}. Apenas transcurrido un cuarto de hora, que invertí dando
a \emph{La Brava} lecciones de lenguaje finústico, reapareció don Jenaro
de Bocángel abrochándose un levitín raído, con visos de ala de mosca. El
chaleco de colorines y el pantalón veraniego mostraban a la legua los
ultrajes del tiempo. Las botas eran de charol deslucido y cuarteado,
torcidos tacones y grietas que pronto serían ventanas; la camisa sin
almidón; la corbata de color de rosa, anudada con esmero y arte. En el
corto tiempo que consagró a su aliño, tuvo espacio Bocángel para peinar
y alisar su melena coquetona, para darse un poquito de negro humo en las
canas del bigote y un toque de rosicler barato en las mejillas.

Pegando la hebra cortésmente en nuestra charla, \emph{don Florestán} me
dijo: «Si como parece escribe usted los grandes anales de este Cantón
que tanto da que hablar al mundo, seguramente tendrá que ocuparse de mí.
Pues allá van datos de este aristócrata perseguido inicuamente por haber
tomado como buen caballero la defensa de la bondad y la rectitud. Me
soltaron de las prisiones no por la clemencia sino por la justicia, que
nunca debieron traerme a padecer entre ladrones y asesinos. No fui
criminal: fui amparador de los menesterosos, abogado de la verdad,
adalid del derecho. No me arrepiento de lo que hice, sino que de ello
estoy muy orgulloso, pues si mi tía doña Silvia Menéndez de Bocángel
procedió criminalmente privando del usufructo de sus riquezas a los
parientes más próximos, yo, Jenaro de Bocángel y de Silva, en
representación de toda la parentela pobre, salí a la palestra jurídica
inspirado por Dios y por todas las leyes divinas y humanas. No cerré
contra la injusticia armado de espada y lanzón. Mis armas fueron una
pluma bien cortada y el buril de la navajita con que grabé la figura y
lemas de varios sellos en la blandura de una patata. Resultó un codicilo
que tuvo en confusión al tribunal por largo tiempo\ldots{} Fui vencido;
la sociedad, que es muy perra y muy ladrona, me destrozó con las garras
de sus infames escribanos y leguleyos. Y no contenta con deshonrarme, me
encerró en presidio por seis años. Pero el varón justo no se acobarda
ante la adversidad, y aquí me tiene usted decidido a defender el derecho
de los humildes contra la soberbia y egoísmo de los poderosos
endiosados. Sostengo y sostendré que mi tía doña Silvia fue una solemne
bribona legando sus riquezas a una piara de frailes inmundos y de monjas
idiotas y puercas\ldots{} Conque\ldots{} aquí tiene usted, señor mío, un
tema tan admirable que si lo campanea en su Historia, como sabe hacerlo,
resonará en todas las naciones de Europa, Asia, África y América.»

Respondile socarronamente que trataría el asunto con entusiasmo,
poniendo en el mismo cuerno de la luna la abnegación y valentía del
caballero don Jenaro de Bocángel. Añadí que necesitando para llevarle a
mis historias un conocimiento fiel de la vida y costumbres del
personaje, de sus medios de existencia, de sus trabajos o quehaceres, le
pedía licencia para estar en su compañía algunos ratos. Él, con júbilo y
cortesanía, me respondió de esta manera: «No saldré en toda la tarde, ni
a prima noche. A su disposición me tiene para cuanto guste indagar
acerca de mí. No le ruego que me acompañe a la mesa porque ya sé que
almorzó con Leonardita; además mi comida es tan sobria que sería
penitencia demasiado dura para una persona como usted: un platito de
cocido, tres o cuatro ciruelas y un vaso de vino de Alicante. Vivo ¡ay!,
en estrechez indecorosa con dos pesetas diarias que me pasan unos
parientes de Madrid.»

Deseosa \emph{La Brava} de emprender su ronda vespertina por las calles
alegres de la metrópoli cantonal, se despidió de nosotros hasta la
noche, y yo me metí con don Jenaro en la mísera covacha donde escondía
su degenerada grandeza. Después que devoró con famélicas ansias el
comistraje que le sirvió una mujer desgreñada y andrajosa, mostrome el
caballero un montón de cartas recibidas de Madrid y las contestaciones
que él había ya medio escrito. Díjome que se consagraba exclusivamente
al magno asunto de humanidad y justicia por el cual había roto lanzas en
la ocasión que motivó la execrable sentencia. Hasta morir seguiría
luchando, y esperaba que un triunfo glorioso coronase al fin sus
trabajos y horrendo sacrificio. Entre varias cartas me leyó una que dijo
ser de una prima suya, señora linajuda que de su dorada opulencia había
descendido a la triste condición de patrona de huéspedes de a tres
pesetas.

De los trozos de cartas leídos, el más extraordinario, peregrino y
despampanante fue éste: «Ya puedo asegurar que antes de fin de año se
proclamará en Madrid el Cantón que llaman \emph{Carpetano}, centro y
cabeza, según me ha dicho mi sobrino Policarpo, de los demás Cantones de
la España. Entonces, Jenaro de mi vida, será la nuestra. Porque tú con
tus influencias y Policarpo con las suyas, que no son flojas, echaréis
por tierra esas leyes inhumanas que nos han despojado de lo nuestro para
dárselo a la mano muerta, como tú dices, o a la mano demasiado viva y
sucia, como digo yo\ldots{} Castelar está dado a los demonios. Ve venir
el Cantón y no le llega la camisa al cuerpo. Mi opinión es que si este
papagayo quiere hacerse cantonalista, para seguir en candelero, debéis
mandarle a escardar cebollinos.»

Después de celebrar con ditirambos de júbilo estas graves noticias, sin
poner en duda su certeza, agregó Bocángel que no era de su gusto el
nombre de \emph{Carpetano} con que los madrileños querían bautizar el
nuevo Cantón. Mejor sería llamarle \emph{Mantuano}, voz que se
acomodaría fácilmente al criterio del vulgo\ldots{} En el curso de
nuestra conversación me mostró luego el de \emph{Calabria} ejecutorias
de familia de los siglos {\textsc{xvii}} y {\textsc{xviii}}, escritas en
lengua italiana y fechadas en Palermo. A pesar de lo rancio del papel y
de lo arcaico de la escritura, no creo pecar de malicioso diciendo a mis
lectores que en los tales documentos había puesto su hábil mano el
propio \emph{don Florestán}, insuperable calígrafo según pude apreciar
por las diferentes obras de su pluma que pasaron ante mis ojos\ldots{}
Dejéle al fin en su febril tarea epistolar, doliéndome de la incurable
vesania de aquel pobre hombre, más digno de los cuidados de una casa de
orates que de los rigores del presidio.

Volvime al centro de la ciudad en busca de alguna noticia substanciosa o
siquiera chismes políticos dignos de ser contados. Cerca del Arsenal me
encontré a Fructuoso Manrique y al cartero Sáez, por los cuales supe que
los vigías del puerto señalaban hacia poniente tres barcos de gran porte
que, según creencia general, eran de la escuadra centralista mandada por
el contralmirante Lobo. Así en el Arsenal como en las calles de la
población advertí que pueblo y Milicias ardían en entusiasmo ante la
proximidad de una naval refriega con los buques del Gobierno, a los
cuales pensaban derrotar y destruir precipitando sus despojos en las
honduras del reino de Neptuno. Cené con Alberto Araus, Ministro de
\emph{Servicios Públicos} (léase Fomento), el cual participaba del
general furor y bélico optimismo, anhelando \emph{la más alta ocasión
que vieron los pasados siglos y esperan ver los venideros.} A este
propósito dijo: «En el nuevo Lepanto nosotros seremos la Cristiandad y
ellos la bárbara Turquía.»

Al retirarme a mi fonda encontré a \emph{La Brava} que iba de vuelta
para su casa. Acompañela hasta la plaza de la Merced, y sentados en un
banco charló conmigo de cosas diferentes, entreverando estas donosas
consultas: «Tú que eres tan sabio, don Tito, dime: ¿qué significa
\emph{inocular}?\ldots{} Explícame también qué quieren decir estas
palabritas: \emph{bajo el punto de vista económico\ldots»} Con toda la
claridad posible contesté a sus preguntas, y ella me dijo: «Yo me pensé
que \emph{económico} y \emph{economía} eran cosa de ahorrar; y eso bien
lo entiendo, que ahorrando estoy y todos los días meto en una media lo
que me sobra. Así voy \emph{ajuntando} para mi \emph{mantención} en
Madrid hasta que se me arregle el negocio. Por tu salud, Tito mío, no
digas nada a nadie, que si se entera ese granuja de Cándido será capaz
de ir tras de mí y darme la gran desazón\ldots{} Yo te aseguro que
\emph{Leona la Brava} dará que hablar en los Madriles. Y ahora te
suplico que mientras esté en Cartagena me des lección en todo lo tocante
a palabras finas, modos de saludar, de comer, de presentarse ante la
gente, con los toquecitos de gracia, chispa y salero que allí se estilan
entre personas que a un tiempo son alegres y de buena educación.
Enséñame todo esto, que ya te pagaré el favor algún día en \emph{parné}
del mejor cuño.»

Prometile ser su catedrático, siempre que ella se corrigiera de emplear
en la conversación dicharachos flamencos, y ella me dijo: «Por la gloria
de tu madre, Titín, pégame un cate siempre que me oigas decir alguna de
esas porquerías. Me propongo que no salgan de mi boca, y se me escapan
por la fuerza de la costumbre. ¡Estará bueno que en Madrid, cuando me
vea con personas bien habladas, suelte yo un \emph{diquelar}, un
\emph{mangue}, un \emph{cangrí\ldots!} Ten por seguro que la ambición de
esta borrica que quiere afinarse ha de ir muy lejos. Ya me estoy viendo
entre medio de \emph{tantismo} señorío. Me gustaría mucho trincar a uno
de esos marimandones que llaman hombres públicos, y embobarle de tal
modo que no se atreva a respirar sin mi licencia. Yo le daría la mar de
consejos, señalándole las teclas que habían de tañer para gobernar al
pueblo con decencia y justicia, con lo cual, figúrate, vendrían a
bailarme el agua todos los lambiones de la Política, saldría mi nombre
en los papeles y me daría más charol que un \emph{dichabaró}. ¡Ay, se me
ha escapado! Pégame, Tito. \emph{Dichabaró} quiere decir gobernador.»

No sigo relatando la evolución de esta lumia, que quería elevarse de un
salto en la escala social, porque otros hechos que parecen traer médula
histórica requieren mi atención. A las siete de la mañana del 11 de
Octubre salieron de Cartagena las fragatas \emph{Numancia}, \emph{Méndez
Núñez}, \emph{Tetuán} y el vapor \emph{Fernando el Católico}
\emph{(Despertador del Cantón)}, haciendo rumbo hacia cabo de Palos en
busca de la escuadra centralista, compuesta de las fragatas
\emph{Vitoria}, \emph{Almansa}, \emph{Navas de Tolosa}, \emph{Carmen},
las goletas \emph{Prosperidad} y \emph{Diana}, y los vapores
\emph{Cádiz} y \emph{Colón}, al mando del contralmirante don Miguel
Lobo.

\hypertarget{iii}{%
\chapter{III}\label{iii}}

Subime a Galeras para ver la función, que por las trazas había de ser
imponente, aunque ninguna de las dos escuadras era digna de tal nombre,
pues cada una contaba tan sólo con un barco de combate. En realidad, el
duelo se entablaba entre la \emph{Numancia} y la \emph{Vitoria}. Los
demás buques eran unas respetables \emph{potadas} que no servían más que
para hacer bulto. Ni con ayuda de los buenos catalejos del castillo pude
ver gran cosa; pero como el cartero Sáez y algunos de los Voluntarios y
soldados de la fortaleza tenían ojos de águila, con lo que ellos me
contaron y lo poco que yo pude distinguir aderezo mi relato en la
siguiente forma:

Eran las doce próximamente cuando la \emph{Numancia} se separó más de
una milla de sus inválidas compañeras, y a toda máquina se coló en medio
de los barcos centralistas. Luchó sola contra los buques de Lobo, que la
rodearon disparando sobre ella todos sus cañones. Mas era tal la pujanza
de la fragata, cuyo nombre se inmortalizó en la guerra del Pacífico, que
salió ilesa de aquella embestida temeraria. Hizo nutrido fuego con sus
baterías de babor y estribor, y rompiendo el cerco viró con rapidez, sin
cesar en sus disparos.

Llegaron después al combate las apreciables carracas \emph{Méndez Núñez}
y \emph{Tetuán}, y la \emph{Vitoria} dispuso sus garfios de abordaje
intentando hacerse con la más próxima, que era la segunda. Ésta disparó
sus andanadas con brío, causando algún estrago en la cubierta de la
\emph{Vitoria}, la cual, teniendo que acudir en auxilio de sus
compañeras centralistas a las que seguía cañoneando la \emph{Numancia},
no pudo realizar el abordaje ni hacer cosa de provecho. El vapor-goleta
\emph{Cádiz} izó bandera de parlamento cuando uno de sus tambores fue
destrozado por los disparos de la \emph{Numancia}. La \emph{Carmen} y la
\emph{Navas de Tolosa} sufrieron bastantes averías, y como por nuestra
parte la \emph{Tetuán} y la \emph{Méndez Núñez} habían agotado sus
escasas fuerzas, quedó concluso el combate poco después de las dos de la
tarde. Los barcos cantonales pusieron proa a \emph{Cartago Espartaria},
y Lobo se retiró mar afuera.

Se me olvidó decir, para terminar la descripción de aquel Lepanto en
zapatillas, que a bordo de la \emph{Numancia} iba el General Contreras,
y en las demás naves del Cantón varios individuos de la Junta Soberana.
Desde Galeras vi que al llegar al puerto los combatientes se les hacía
un recibimiento loco, con gran algazara de vítores, aplausos y otras
demostraciones, cual si volvieran de un Trafalgar al revés trayendo la
cabeza de Nelson. Estos ruidos de la pasión local y del entusiasmo
sectario son la música inevitable que ameniza nuestras civiles
contiendas por un sí o por un no\ldots{} Luego supe que los cantonales
traían cinco muertos, entre ellos don Miguel Moya, vocal de la Junta
Suprema o Soberana.

En el tiempo que estuve en el castillo de Galeras hice amistad con un
hombre muy avispado, cuyos ojos suplieron a los míos en la visión del
lejano combate. Su vista superaba a la de las gaviotas, y todo lo
refería como si los objetos se acercasen hasta ponerse a tiro de fusil.
El mismo me reveló con donosa franqueza, su condición de presidiario,
diciéndome que la condena había sido por diez años, y que sólo le
faltaban meses para cumplirla cuando el Cantón le puso en libertad. De
las causas que motivaron su encierro no me dijo nada ni osé yo
preguntarle. Era de buen talle y agradable presencia, uno de esos
hombres de naturaleza tan peregrina que a los sesenta años conservan una
dulce jovialidad y el contento de vivir. Sus canas se armonizaban con
sus ojos azules de expresión bondadosa, y su palabra era fácil, serena y
de perfecto casticismo en la dicción. David Montero, que así se
nombraba, había ejercido antes de su delito la profesión de mecánico,
dedicado casi exclusivamente a la compostura y arreglo de instrumentos
de náutica. Tal era en el Departamento la fama de su habilidad, que tuvo
siempre la tienda llena de sextantes, octantes, brújulas, barómetros
aneroides, y no faltaban cronómetros, pues era también consumado
relojero. Apurábanle sus clientes, y él, infatigable, a duras penas
cumplía aumentando las horas de trabajo.

Cuando bajábamos del castillo, David me contó que al entrar en
prisiones, otros mecánicos vinieron a suplirle, estableciéndose en
Cartagena. Él, en tanto, logró con su buena conducta que el jefe del
presidio le consintiera montar un reducido taller en las estancias altas
del penal, con lo que alivió la pesadumbre del ocio y la tristeza,
granjeándose algunos dineros para mejorar las condiciones materiales de
su vida.

Al despedirnos en la Plaza de las Monjas ofreciome su casa, situada en
lo más alto de la ciudad, no lejos de la vieja iglesia románica. Díjome
que gustaba de vivir lo más cerca del cielo, pues con la libertad le
habían entrado aficiones astronómicas. Prometí visitarle para conocer
sus nuevos estudios\ldots{} A poco de separarme de él para ir al
Ayuntamiento encontré a \emph{Pepe el Empalmao}, el cual me dijo que
David Montero fue condenado por dar alevosa muerte a su manceba y a una
\emph{guaja} con quien la sorprendió en malos pasos.

El entusiasmo de Cartagena por el primer choque naval continuó con
hervor creciente en los días sucesivos. El 14 de Octubre, la Junta
Soberana acordó un plan de combate: \emph{luchar hasta vencer o quedarse
sin un barco}, según la espartana frase de la \emph{Gaceta} del Cantón.
En la mañana del 15 salió la escuadra en busca de los barcos de Lobo,
que se hallaban a la vista. A retaguardia, en el famoso
\emph{Despertador}, iban el bíblico Roque Barcia y Manolo Cárceles, en
representación de la Junta Suprema, para hacer cumplir las disposiciones
estratégicas de ésta y resolver sobre cualquier incidencia que ocurriese
en el curso de la batalla. Navegaban los buques de combate en correcta
línea, y apenas divisaron los barcos centralistas éstos se pusieron en
orden conveniente para afrontar la lucha.

Cuando ya estaban los adversarios a tiro de cañón adelantose la
\emph{Tetuán} rompiendo el fuego contra \emph{la bárbara Turquía}, como
dijo Alberto Araus. Apenas recibieron los primeros balazos, las naves
centralistas viraron en redondo, poniendo rumbo al Sur en franca
retirada. Los cantonales las persiguieron cerca de cuarenta millas hasta
perderlas de vista, y regresaron a Cartagena, quedando roto el bloqueo
por mar. No hay que decir que cuando entraron en el puerto los que se
llamaban vencedores se repitieron las inevitables alharacas y la
greguería jubilosa.

Al consignar que a bordo de las naves cantonales iba lo más granado y
florido del personal revolucionario, debo decir y digo que el único
hombre de mar y de guerra marítima que a mi parecer merecía ser
recordado en la Historia era un tal Alberto Colau, contrabandista, hijo
de Alicante y tan familiarizado con las aguas mediterráneas y con los
peligros del navegar y del combatir, que entre toda la gente llegada de
diversas partes a la República Cartagenera no se pudiera encontrar quien
le igualase. Le conocí el mismo día 15, a poco de saltar en tierra, y
quedé maravillado de su espléndida y arrogante facha. No era menester
ciertamente el auxilio de la fantasía para ver en aquel hombre la
resurrección del tipo del corsario que en los tiempos de la piratería
heroica llenó los anales del mar Interno.

Descollaba Colau entre la muchedumbre por su robusta complexión y lucida
estatura, por su curtido rostro y el mirar flamígero de sus ojos negros.
Como el azabache eran también sus cabellos crespos, sus cejas pobladas y
el bigotazo que perpetuaba la tradición de la moda turquesca. Coronaba
su cráneo con el fez rojo, complemento, en cierto modo histórico, de la
figura de aquel Barbarroja redivivo. Andando los días se vio un gorro
colorado en el puente de la \emph{Numancia}, de donde vino el atribuir a
Contreras el uso de tal prenda. No; el fez no era de Contreras, sino de
Colau, y éste, a juicio de un historiador psicólogo, la figura más
saliente, pintoresca y castiza del Cantón Cartaginés.

La bravura pirática del arrogante aventurero se llama hoy contrabando,
que viene a ser lo mismo con diferencias de tiempo y lugares. En sus
faluchos de vela, Colau desafiaba las olas y la persecución de las
escampavías del Resguardo. Cuando la astucia no le bastaba y era preciso
emplear la violencia, no vacilaba en derramar sangre. Empezadas sus
correrías en Gibraltar, se trasladó luego a Orán, donde obtuvo provecho
mayor y campo de operaciones más extenso. De la costa argelina nos traía
tabaco, licores, telas, quincalla y otras mercancías vigiladas por
nuestros aduaneros. A los \emph{vistas} de acá, unas veces les cerraba
los ojos, y otras les rompía la cabeza. Con este ten con ten y un ardor
infatigable, hizo Colau en poco tiempo una fortunita y vivía en Orán
como un bajá, con su mujer y sus hijos, bien quisto de los franceses y
de la colonia española. De él se contaba que nunca se le acercó un
necesitado sin que al punto le socorriese, y en la misma Cartagena era
el amparador de todas las personas o familias que, perseguidas por el
Centralismo, se habían refugiado en la Plaza.

Con la fiereza del continente y rostro de Colau contrastaba la blandura
de su trato en la vida social. Era cariñosísimo y a veces hasta pueril.
Al estallar la revolución cartagenera se presentó en la Plaza ofreciendo
sus servicios a la Junta Revolucionaria, que los aceptó en el acto
dándole el mando de la fragata \emph{Tetuán}, la cual manejó y gobernó
desde el primer momento con la misma destreza que solía desplegar en el
gobierno y mando de sus faluchos\ldots{} Pasé una tarde con él y otros
amigos en el café de la Marina, charlando de aventuras guerreras en el
mar y en la costa. Colau nos refirió terribles episodios de su lucha
contra las olas embravecidas en los duros Levantes, que mil veces le
pusieron a dos dedos de caer en los profundos abismos. Nos contó también
alijos que por su descomunal audacia parecían fabulosos, y peripecias
trágicas de sus encontronazos con los aduaneros y demás patulea del
Fisco.

A la gentil cortesía de Cárceles debimos aquella tarde el obsequio de
jerez y pastelillos, y en la alegría del beber y del charlar suplicamos
al contrabandista nos dijese el porqué ostentaba en el ojal de su
chaqueta el botoncito rojo de la Legión de Honor. Con modestia ruda
evadió Colau la respuesta, queriendo llevar a otros asuntos el vago
coloquio. Pero Manolo Cárceles, tan indiscreto en aquel caso como amante
de la verdad, nos refirió el hecho heroico que había motivado aquella
distinción, empezando por decir que Francia no concede nunca tales
honores más que al mérito indudable.

Horroroso temporal de Levante descargó una tarde sobre Orán, con
furibundas rachas de viento y olas como montañas, que en pocos minutos
destrozaron la escollera del nuevo puerto en construcción. En lo más
duro de la borrasca presentose a la vista un trasatlántico francés, que
traía de Marsella pasajeros de diferentes clases sociales, y entre ellos
gran número de mujeres y niños\ldots{} Muy apurado venía el barco por
los accidentes de una tormentosa travesía, y al querer tomar puerto se
le vio a punto de zozobrar, estrellándose contra las peñas o los bloques
de la escollera destruida donde reventaban las olas. En el muelle estaba
casi todo el vecindario de Orán, con ansiedad y espanto, pues muchas
familias tenían seres queridos entre los pasajeros del vapor. Nadie
osaba intentar el salvamento, que era poco menos que imposible en
condiciones tan aterradoras.

De pronto apareció entre la multitud un hombre\ldots{} Este hombre era
Alberto Colau\ldots{} que con fuerte y altanera voz dijo así:
«¡Cobardes! Si no hay quien me siga yo iré solo a salvar los que pueda.
Si alguno me acompaña, mejor.» Cuatro o seis marineros se adelantaron,
dispuestos a secundar al español en su hazaña. Metiéronse todos en una
lancha grande, con vela y remos, y desafiaron impávidos el oleaje
furioso. Al cabo de algunos ratos de indecible angustia realizó Colau el
primer salvamento. En la segunda tentativa, que fue la más emocionante,
se veía desde el muelle la lancha de Colau, a veces balanceándose en la
cresta de una ola formidable, a veces precipitándose en la hondonada
líquida\ldots{} Por momentos desapareció\ldots{}

Creyeron los angustiados espectadores que no volvería; pero volvió,
¡hurra!, trayendo unas señoras lívidas y unos niños llorosos, mojados
todos hasta los huesos\ldots{} Los marineros bogaban con sereno coraje;
Colau, en pie, las melenas al aire, llevaba el timón, empuñando la caña
con tal fuerza que no le superara el propio Neptuno\ldots{} El tercer
viaje fue más benigno. Las mismas olas parecían inclinarse respetuosas
ante la intrepidez de aquellos hombres. Cuando terminó el salvamento y
pisaron tierra todos los náufragos del vapor, se produjo una
indescriptible escena sentimental: abrazos, besos, exclamaciones,
llantos de alegría. Alberto Colau, desentendiéndose de las
manifestaciones de cariño y gratitud, tomó con sereno continente el
camino de su casa.

«Ahí le tenéis---dijo Cárceles al poner término a su relato.---Ahí
tenéis al héroe, ostentando en su pecho la insignia de la Orden de
Caballería más acreditada que existe en la Edad Moderna, recompensa de
su esforzado ánimo y de su amor a la Humanidad.»

---Caballero fui siempre y caballero soy---dijo Colau, contraviniendo
discretamente su natural modestia.---\emph{La Orden del Contrabando}
pide arrojo temerario, paciencia en las adversidades, calma y tino
cuando sean menester, liberalidad, sangre fría, prendas que entiendo yo
son y han sido siempre la mejor gala y adorno del alma de los
caballeros.

\hypertarget{iv}{%
\chapter{IV}\label{iv}}

Fáltame decir, para redondear la personalidad de Colau, que en el trajín
del contrabando también comerciaba. En aquellos tiempos era muy estimado
en el Norte de África el aljófar, perlitas pequeñas y mal configuradas
con que las moras adornan y recaman sus chaquetillas, sus fajas y
babuchas. Como en España venía desmereciendo este artículo, multitud de
tratantes en pedrería iban de pueblo en pueblo comprándolo para llevarlo
a Marruecos y Argelia. A igual tráfico se dedicó Alberto Colau en
Cartagena, extendiéndose no más que a Lorca, Totana y Murcia. Redondeaba
su especulación trayendo de África zafiros y esmeraldas que en España
tenían cotización muy alta.

Dicho esto, añadiré que aquella misma noche cenábamos Fructuoso
Manrique, Cárceles, Alberto Colau y yo, en el propio café de la Marina,
cuando vimos entrar fachendosa y arrogante a \emph{La Brava}, que
agarrando con desgaire una silla se plantó en nuestro corro junto a
Colau, acometiéndole con esta viva requisitoria: «Eh, Alberto, cómprame
ahora mismo este aljófar que te traigo. Dispensen los señores y sigan
comiendo, que no vengo a cenar, sino a mi negocio.» Diciéndolo sacó un
envoltorio de papel de periódico en que guardaba un puñado de perlitas,
y así prosiguió: «Las he recogido entre mis amigas. A ver cuánto me vas
a dar, judío arrastrao. Yo quiero por ellas veinte \emph{chus}, o por lo
menos una \emph{jara}.»

Dejó Colau el tenedor, y risueño, sopesando la mercancía, dijo a la
moza: «Pero si esto no vale más de doscientos \emph{rumbeles} a todo
tirar. En fin, ya hablaremos. ¿Quieres cenar?» Rechazó \emph{La Brava}
con donosura el galante ofrecimiento, y todos reiteramos con alegre
algazara la invitación: «¿Quieres huevas de \emph{jumol}? ¿Una copa de
jerez? ¿Dátiles de mar? ¿Un pastelillo de estos de crema que están tan
ricos?»

---Bueno---exclamó \emph{Leona} arrimando su silla en el hueco que le
hicimos y cogiendo el primer plato vacío que encontró.---Venga alguna
cosita. Pero déjenme que siga con mi negocio. Yo todo lo miro ya
\emph{bajo el prisma} de mi \emph{economía}.

---Ya, ya sé por Dorita---dijo Fructuoso---que acumulas fondos para irte
a Madrid y hacerte un buen cartel en la \emph{cocotería} elegante.

---¡Calla, \emph{malange}, tú qué sabes de eso!---replicó ella,
atizándose una copa de Jerez.---Yo necesito cuartos porque me voy
volviendo muy regalona. Díganle a este perro de Colau que tenga
conciencia y me pague por el género lo que le pido.

---Yo te daría eso y más---repuso Alberto---si hicieras caso de mí. ¿Qué
demonio vas tú a pintar en los Madriles? Allí no hay más que pobretería
finchada y figurones políticos que no tienen ni un \emph{calé\ldots{}}
Repito lo que te he dicho mil veces. Cuando acabe este jollín del Cantón
en que estamos metidos, vente a Orán conmigo. Verás qué tierra, chica.
Allí encontrarás la mar de franceses tontos y ricos. ¡Qué fácilmente los
podías pescar, gitana, con el anzuelo de esa carita! Pues digo; si le
caes en gracia a uno de aquellos morazos podridos de dinero, que se
pirran por las españolas, ¡ay morena!, te cubres el riñón para toda la
vida.

---No me hables a mí de tierras extranjeras---contestó \emph{La
Brava}---Yo tiro siempre al españolismo\ldots{} \emph{La Madre Patria
necesita de todos sus hijos}, como dice don Roque\ldots{} y de todas sus
hijas, digo yo.

La respuesta de Alberto Colau a estas sesudas consideraciones fue coger
el papel donde estaba envuelto el aljófar, y sacar de su repleto bolso
varias monedas de oro y una de plata, que entregó a la mozanca,
añadiendo estas expresivas razones: «Pierdo dinero. Allá no pagan el
adarme de aljófar más que a seis pesetas. Pero en fin, para que no
chilles te doy la \emph{jara} y un \emph{chus} de propina.» Continuó la
conversación alegre. Mientras \emph{Leona} devoraba pastelillos, jamón
en dulce y otras frioleras, humedeciéndolas con Jerez, todos le
dirigíamos chicoleos, anunciándole los grandes éxitos que había de
obtener en Madrid. Ella nos atajó diciendo: «No hablen de eso, que el
diablo las carga. Estoy perdida si mi marido se entera. Cándido no me
deja vivir, me persigue, me acosa. Ese condenado \emph{parte del
principio} de que yo soy rica, y cuando me niego a darle dinero se pone
fosco\ldots{} Temo que el mejor día me mate como mató a mi madre\ldots{}
Si le da por seguirme a Madrid\ldots{} No quiero pensarlo\ldots{}
¡Sálveme la Virgen de la Caridad!»

Desde allí nos fuimos todos al teatro Principal, donde había función de
aficionados. Representaban un dramón, obra de dos autores indígenas,
titulado \emph{Glorias del Cantón y perfidias del Centralismo}. Camino
del teatro, \emph{La Brava}, cogiéndome del brazo y retrasándonos del
grupo, me dijo con misterio: «Explícame ahora mismo qué quiere decir
\emph{en tesis general}, porque anoche Juanito Pacheco, el hijo del
Marqués de Águilas, que es un chico que habla muy requintado y siempre
con mala idea, me dijo que yo y otras como yo éramos, \emph{en tesis
general}, lindas bestias sin alma. Lo de \emph{tesis} me ha escocido,
créelo. Dime si es alguna desvergüenza, porque yo no aguanto \emph{ancas
de nadie.»} Solté la risa y le contesté que no era fácil explicarle el
significado de la palabra \emph{tesis}, pues tendría yo que emplear en
mi lección otros vocablos incomprensibles para ella; que no hiciera
caso; que ya iría aprendiendo eso y mucho más en el trato con la gente
de Madrid.

Persistiendo Leonarda en sus anhelos instructivos, me dijo: «También
hablaron anoche de que a Pepito le da por \emph{la ironía}. Para mí que
\emph{la ironía} es como quien dice \emph{la viceversa de las cosas.»}

---Así es---repliqué yo.---Veo que tú sola vas aprendiendo con tu propia
inteligencia y criterio. ¡Adelante, mujer de los alegres destinos!

En esto llegamos al teatro. \emph{Leona} no quiso entrar. Su marido
hacía el papel de traidor centralista, y por bien que ella se escondiese
entre los espectadores no podría evitar que el indino saliera al público
para darle la matraca y corromperle las oraciones. La \emph{tesis
general} de Cándido Palomo era emborracharse todas las noches\ldots{}
Retirose mi amiga a su casa, muy satisfecha con la \emph{guita} que le
había sacado a Colau, y los demás entramos a ver la función. El frenesí
patriótico que en su drama pusieron los inocentes autores, no atenuaba
los disparates de fondo y forma. Sin pararnos en estos pelillos
aplaudimos hasta desollarnos las manos.

En los siguientes días supimos que el contralmirante Lobo dio cuenta de
su retirada al Ministro de Marina, en términos que ha conservado la
Historia para conocimiento de hombres y sucesos. Era Lobo un técnico
excelente, autor de obras muy estimables; mas en el mando naval no pudo
poner nunca su nombre a la altura de su suficiencia científica. He aquí
lo que telegrafió al señor Oreiro: «Hoy 15 de Octubre han salido otra
vez las fragatas insurrectas en orden de batalla. La \emph{Numancia} iba
un poco delante, pero sin romper la línea de los otros buques, y
formando con ellos un muro de hierro. Todos maniobraban muy bien y
parecían mandados por jefes expertos. En vista de lo cual, y teniendo
que reparar algunas averías y proveer de carbón, he ordenado partir con
rumbo a Gibraltar.»

Bañándose en agua de rosas quedaron los cantonales con la inexplicable
inhibición, por no darle otro nombre, del Contralmirante Lobo, y era
general creencia que ello se debió al respeto que le impuso el
acertadísimo plan y perfecta organización táctica de las naves de
Cartagena, obedientes a las órdenes del contrabandista. Los amigos y
admiradores de éste le dimos desde aquel día título y diploma de marino
de guerra, llamándole, entre veras y bromas, \emph{el Comodoro Colau}.
La mejor prueba de que Lobo no supo engallarse ante los barcos
cantonales en su segunda salida fue que le censuró duramente el General
Ceballos, sucesor de Martínez Campos en el mando de las tropas
sitiadoras de Cartagena. El Gobierno Central destituyó a Lobo en el
mando de la escuadra, nombrando para este puesto al Contralmirante
Chicarro. Fueron asimismo reemplazados el comandante de la \emph{Navas
de Tolosa} y el segundo de la \emph{Blanca}.

Fuera de la feliz aventura del \emph{Despertador del Cantón} que apresó
una goleta cargada de bacalao, lo que trajo gran alivio a la plaza mal
surtida de víveres, no hay sucesos dignos de mención hasta la salida de
la escuadra para Valencia con los mismos barcos y los propios jefes que
en las anteriores correrías llevara. Para el mejor desempeño de mis
deberes croniquiles embarqueme en el \emph{Católico Despertador},
desoyendo las amonestaciones de David Montero y de \emph{La Brava}, que
al despedirme en el Arsenal me vaticinaron una jugarreta del Destino.
\emph{Leona} había echado las cartas, y David consultado el inmenso
libro del firmamento. Ambos presagiaban que tendríamos unas miajas de
catástrofe. Pero yo, que nunca di crédito al lenguaje de las estrellas
ni al de los naipes, me agregué a la expedición tranquilo y confiado.
¡Ay, ay; cuán equivocado estaba yo y cuán en lo cierto aquellos buenos
amigos! Sabed, lectores compasivos, que cuando habíamos rebasado de
Alicante, montado ya el cabo Huertas\ldots{} Pero dejadme tomar aliento,
pues se trata de uno de los más apretados lances de mi vida.

El \emph{Despertador} iba de vanguardia, con mar llana y tiempo cerrado
de niebla. A la madrugada, cuando bajo cubierta dormían todos los
tripulantes, menos una veintena que huyendo de la pesada atmósfera de
cámaras y sollados subimos a pasar la noche con los que hacían servicio
a proa y en el puente, fuimos sorprendidos y aterrorizados por la visión
de un corpulento barco que se nos echaba encima. Era la \emph{Numancia}.
Nuestro timonel inició una virada rápida, mas con tan mala suerte que el
formidable espolón de la fragata embistió el costado de estribor de
nuestro barco, hizo añicos la rueda y abrió un inmenso boquete en el
departamento de calderas y máquinas. Aunque en la \emph{Numancia} dieron
contravapor apenas divisaron al \emph{Católico}, no se logró evitar el
desastre.

No podréis imaginar la confusión, el espanto de los que estábamos sobre
cubierta. El \emph{Despertador} se hundía rápidamente como un cesto
cargado de plomo. Empezó a salir gente por las escotillas. No hubo
tiempo de arriar nuestros botes, y si no es por los de la
\emph{Numancia}, que acudieron con presteza, todos habríamos perecido.
Ya tenía el \emph{Católico} la popa bajo el agua cuando yo salté, no sé
cómo ni por dónde, a un chinchorro que estuvo a punto de zozobrar por
los muchos hombres que en él se metieron. En tan horrible confusión caí
al agua y fui recogido por unos marineros que luego vi eran de la
\emph{Tetuán}, pues entre ellos estaba Alberto Colau. A éste debí mi
salvación, que todavía creo milagrosa. Mi primer pensamiento fue para
recordar las fatídicas predicciones de \emph{La Brava} y David Montero.

La escena era espantosa: vi a muchos infelices que nadaban
desesperadamente, tratando de agarrarse a los pocos salvavidas que
fueron arrojados desde el buque náufrago. Desgarradores gritos
aumentaban el horror de la catástrofe. Yo también grité llamando a mi
machacante\ldots{} ¡Cándidoo!\ldots{} ¡¡Palomo, Palomo!!\ldots{} Ni éste
me respondió ni le vi entre los que luchaban angustiosamente con las
negras aguas\ldots{} Cuando estábamos como a diez o doce brazas del
siniestro, noté que del \emph{Católico} sólo se veían ya los palos, la
chimenea y un poco del tambor de babor. Al reconocerme seguro en la
cubierta de la \emph{Tetuán}, tropecé con un contramaestre del
\emph{Despertador} y le pregunté por Palomo. «Dormido estaba como un
leño---me dijo.---Quise despertarle; le tiré de una pata; no rechistó.
\emph{Ajogado} estará.»

El primer cuidado de los supervivientes fue calcular el número de
víctimas. Unos decían que eran ciento y pico; otros que no pasaban de
treinta. Luego quedó fluctuando la cifra entre sesenta y setenta\ldots{}
Consagrado por todos un pensamiento de fúnebre despedida a los que
habían perecido y al pobre \emph{Despertador}, la escuadra cantonal
siguió su ruta. Llegamos al Grao de Valencia, donde estuvimos fondeados
tres días y medio. No pudiendo obtener de la plaza lo que pedíamos,
arramblamos con los barcos mercantes \emph{Darro}, \emph{Victoria},
\emph{Bilbao} y \emph{Extremadura}, cargados de víveres, tejidos y otros
artículos de comercio. Nuestro arribo a Cartagena fue el 22 de Octubre
si no me engaña mi flaco sentido en la cuestión de fechas. Salté en
tierra con botas prestadas y una gorra de marinero, pues perdí las
prendas mías equivalentes en las ansias del naufragio.

En la plaza de las Monjas encontré a \emph{La Brava}, que ya tenía
noticias del desastrado fin de su caro esposo. Inquieta y medrosica me
preguntó por él, y yo le dije sin preparación ni melindres que ya podía
tenerse por viuda. No se cuidó la buena moza de disimular su alegría, y
me consultó si estaba en el caso de vestirse de luto por el bien
parecer. Mi opinión fue que si tenía ropas negras debía ponérselas,
siquiera unos cuantos días, a lo que me respondió que algunos trapitos
conservaba del luto que llevó por su madre, añadiendo: «Con mi ropa
negra y la cara un poco afligida representaré muy a gusto lo que llama
Juanito Pacheco la \emph{comedia social}. \emph{En igualdad de
circunstancias}, igualdad de sentimientos y luto al canto. Ahora lo
llevaré como huérfana y como viuda, y tú podrás mirarme \emph{bajo el
prisma} que quieras.» Me acompañó hasta mi fonda en la calle del Cañón,
y por el camino me habló de este modo: «A pesar de lo que me has dicho,
no acabo de creer que ese posma de Cándido haya perecido. Tiene más
picardías que un gato soltero, y puede que se haya hecho el náufrago
para cuando una esté harta de llevar luto aparecerse en alguna isla
desierta de las que llaman Columbretes, o Filipinas de la mar Caribe.»

El 24 de Octubre apareció nuevamente en aguas de Cartagena la escuadra
centralista, al mando del Contralmirante Chicarro, reforzada con la
fragata \emph{Zaragoza}, que había venido de Cuba. Los barcos de
Chicarro cruzaban sin cesar frente a Escombreras; pero el bloqueo no era
de gran eficacia porque de noche, sin luces, entraban embarcaciones
menores que mantenían en regular abundancia el abasto de la ciudad.

En una de mis excursiones a Santa Lucía, visité al desdichado prócer
maniático \emph{don Florestán de Calabria}, a quien hallé muy abatido y
macilento por efecto del frío que vino con las primeras lluvias de
Noviembre. Envuelto en una manta vieja y rota continuaba arrimado a la
mesa en la fementida estancia que era su mísero albergue. Cubría sus
pies descalzos con una mugrienta toquilla de su casera, y no dejaba de
la mano la tarea de contestar con tembloroso pulso la copiosa
correspondencia de sus parientes de Madrid. Como en aquellos días de
recogimiento había dejado de pintarse la perilla y los pómulos,
tuviérasele por envejecido en dos o tres lustros.

Lástima grande me inspiró el caballero sin ventura, y atento a
remediarle volví aquel mismo día con la modesta ofrenda de unas babuchas
de orillo, un gorro de pellejo y un chaquetón, deslucido pero en buen
uso, que me compró para este fin Alonso Criado, el camarero de la fonda.
No necesito decir cuánto agradeció mi pobre amigo aquellas prendas,
demostrando su necesidad con las prisas que puso en estrenarlas.

Al estrecharme las manos con honda emoción se le saltaron las lágrimas,
y como advirtiese yo que al llanto siguieron desaforados bostezos,
comprendí que su mal no era sólo de frío sino de hambre. Saqué del
bolsillo algunas pesetas para ofrecérselas con efusión sincera; pero no
quiso tomarlas. Se puso de mil colores, rechazando el socorro. Su
delicadeza, su dignidad de hombre linajudo, le permitían quizás admitir
un obsequio de la amistad, siempre que éste fuera en especie; dinero
jamás admitiría. El oro y la plata ofrecidos a título de caridad
causábanle un horror invencible. Luego añadió: «Mi patrona o casera me
dará de comer mientras el bloqueo de la plaza impida la llegada del
correo que ha de traerme\ldots{} fondos.»

Pasado un rato me dijo: «Siéntese a mi lado un momento y le pondré al
tanto de las graves noticias que tengo de Madrid. Cierto es que Castelar
ha restablecido la disciplina, aplicando severos castigos; cierto es
también que ha reconstituido en su antiguo ser y estado el Cuerpo de
Artillería. Pero con todo esto sepa usted que el \emph{Cantón Mantuano}
será un hecho muy pronto. Nos lo traerá el mismo Castelar. Aquí tengo
textos fehacientes\ldots{} las cartas de mi sobrino Policarpo que está
muy bien enterado de todo y es el brazo derecho de don Emilio. ¿A que no
adivina usted quiénes ayudarán al Presidente a traernos el Cantón? Pues
los generales de más nota, y entre éstos el más decidido es\ldots{}
¿quién dirá usted?\ldots{} el General Pavía\ldots{} Don Manuel Pavía y
Alburquerque\ldots{} Eh, ¿qué tal?\ldots{} Aquí, aquí están los textos.
Véalos.»

\hypertarget{v}{%
\chapter{V}\label{v}}

Las visitas que en los siguientes días hice a \emph{don Florestán de
Calabria} me proporcionaron agradables ratos de parloteo con \emph{La
Brava} en su propia habitación. Mostraba \emph{Leona} bastante inquietud
ante el cerco que a la ciudad ponían las tropas centralistas mandadas
por Ceballos, activando cada día más los trabajos de fortificación y
atrincheramiento. \emph{«A mi juicio}---me dijo Leonarda torciendo la
boquita como hacía siempre que pronunciaba palabras escogidas---pronto
empezarán nuestros contrarios a zurrarnos \emph{de lo lindo}, y tanto
apretarán el sedio que no podrá entrar ni salir bicho viviente. Si
tuviera yo mi \emph{economía} en todo su \emph{pogeo}, quiero decir si
hubiera \emph{ajuntado} dinero bastante, mañana mismo saldría de
\emph{naja} para Madrid.» Respondile que tuviera sosiego porque el sitio
no había de ser muy duro. ¿Por qué no aplazar el viaje hasta fin de año?
En un momento de afectuosa intimidad me salió de la boca el chispazo de
estas palabritas: «No juraré yo, pecador de mí, que no te acompañe para
hacer tu presentación en el gran mundo, que solemos llamar
\emph{demi-monde.»}

Movido de no sé qué atracción inexplicable, visité también por aquellos
días a David Montero. Este hombre me interesaba enormemente por su
natural agudeza, por su vida laboriosa y trágica. Si eran dignos de
estima los pensamientos que en el curso de la conversación mostraba, no
lo eran menos los que a medias palabras y con velos de reserva dejaba
traslucir. Cuando le conocí se me mostró como habilísimo mecánico de
instrumentos menudos y sutiles. Después, en su casa, se me reveló como
astrónomo con puntas de nigromante. Últimamente advertí en su taller
apuntes, papeles llenos de guarismos y trazos lineales que indicaban
estudios de Aritmética y Geometría.

Una mañana, al traspasar los umbrales del hogar de Montero, situado como
he dicho en los altos de la vieja Catedral, tropecé de manos a boca con
una mujer que, si no era la propia \emph{Doña Aritmética} era el mismo
demonio, transfigurado para volverme tarumba. Trémulo y confuso le
pregunté: «¿Pero es usted \emph{Doña Aritmética?»} Y ella me contestó
entre asustada y burlona: «No señor; no me llamo Demetria, sino
Angustias para servir a Dios y a usted.» Repuesto de mi sorpresa pude
advertir que había semejanza de facciones entre la servidora de Floriana
y la criada de David, sólo que ésta era mucho más madura y peor
apañadita.

Poco después, cuando Montero me daba cuenta de la parte no reservada de
sus trabajos, entró a llevarle café otra anciana vestida de negro, en
quien de pronto vi pintiparada la imagen de \emph{Doña Geografía}.
También entonces expresé mi curiosidad, y ella repuso: «No me llamo
Sofía sino Consolación, y soy de Totana para lo que usted guste mandar.»

---Pues mire, don Tito---dijo a la sazón David, riendo.---En broma llamo
a esta buena mujer \emph{Doña Geografía}, porque sabe de memoria los
nombres de todos los pueblos del país murciano.

No era la primera vez que sufría yo tales equivocaciones. Algunos días
sentíame perseguido por fantasmas, reminiscencia de mi antigua
navegación por el inmenso piélago suprasensible.

Sin saber cómo, nuestra conversación recayó en el asunto del cerco de la
Plaza, mostrándose David algo pesimista sobre las consecuencias de esta
función militar, y no mal informado de los planes del Ejército sitiador.
Hizo breve semblanza del General Ceballos, del Brigadier Azcárraga y de
los Comandantes Generales de Artillería e Ingenieros Brigadier don
Joaquín Vivanco y Coronel don Juan Manuel Ibarreta, revelando
conocimiento directo de sus respectivos caracteres. Luego enumeró las
fuerzas Centralistas, según su parecer escasas pero bien disciplinadas.
Marcó después el contingente de las diversas Armas, con tal precisión y
seguridad en las cifras como si lo hubiera contado. Notando mi extrañeza
por la posesión que tenía de aquellos datos sin salir de la Plaza, me
dijo:

«Algunas mañanas me voy al castillo de Moros. En lo más alto de sus
muros he puesto un anteojo de mucho poder, con el cual veo los trabajos
que hacen los sitiadores. Ya sabe usted que la primera batería la tienen
emplazada en Las Guillerías. En ella hay cuatro piezas de a diez y seis.
El talud interior del espaldón está revestido de cestones, y las
cañoneras de sacos terreros. Han emplazado la segunda batería cerca de
las casas de don José Solano, artillándola con cinco obuses de a
veintiuno. El terraplén interior consta de tres planos diferentes.

»Más allá, junto a la ermita de San Ferreol, hay otra batería con seis
cañones de a diez y seis. Los revestimientos están hechos con cestones y
fajinas. La batería de la Piqueta, que está al lado de la finca de este
nombre, se halla provista de \emph{cubre-cabezas}, y tiene un través en
su centro que completa la protección del retorno de la derecha.»

---Ya veo, amigo David---le dije sin ocultar mi asombro,---que es usted
una enciclopedia. Yo le admiraba como mecánico y astrónomo, y ahora
resulta que es usted maestro también en el Arte de la Castrametación.

---La tristeza y el aislamiento---replicó él---nos lleva, señor don
Tito, a la variedad de los estudios. Hace unos días, hallándome hastiado
de trabajar sin fruto, sentí vivas ganas de tomar el tiento a las cosas
de Guerra\ldots{} Vea los libros que tengo aquí. Me los ha prestado el
Brigadier Pozas, que, según entiendo, no los ha leído ni por el
forro\ldots{} Si sigo en esta inacción que me entumece el cerebro, el
mejor día me encuentra usted entregado al Derecho canónico, o al
Ocultismo, que así llaman hoy a la Magia.

Con la idea de obtener de aquel hombre extraños hilos o hilachas para mi
tejido histórico, seguí visitando a Montero. Algunas mañanas no le
encontré en su casa. Esperábale, y al fin le veía llegar fatigado y
cubierto de polvo. Venía sin duda del campo reseco que a Cartagena
circunda. A las veces, no me hablaba de nada concerniente a las fuerzas
sitiadoras, sino de chismes y enredijos del interior de la ciudad; por
ejemplo: «Parece que hay sospechas de que Carreras, Pernas, Del Real y
otros militares, hociquean secretamente con el General Ceballos. Dicen
que corre el dinero\ldots{} Yo no lo creo. Tal infamia no es posible.»
Otros días se lanzaba desde luego, sin preámbulos, a departir sobre el
Arte de la Fortificación.

«Para proteger las baterías que acaban de emplazar---me dijo una
mañana,---y para oponerse a cualquier salida que intentemos los
cantonales, están los sitiadores haciendo espaldones sistema Pidoll,
modificado con pozos para los sirvientes de las piezas, que creo son de
las de a diez. Uno de los espaldones lo construyen entre el ferrocarril
y la finca de Bosch, otro en las inmediaciones de la casa de Calvet, y
otro junto a Roche Bajo. Parece ser que cuando terminen estas obras
empezará el bombardeo, y allá veremos quién puede más.»

\emph{Pepe el Empalmao}, a quien yo utilizaba mediante cortas dádivas
para recadillos y espionajes de diversa índole, aprovechó una tarde en
que nos encontramos enteramente solos para decirme con ronco sigilo
cavernoso: «Señor don Tito, ese David sale de madrugada, y escondiéndose
de la gente va al campo de los judíos Centralistas. Allí se pasa las
horas hablando con éste y con el otro, y mayormente con uno que llaman
\emph{el Azcárrago}. Esto se lo digo a usted sólo. Chitón y armas al
hombro.»

---Me parece, Peporro---contesté yo, para estimularle a mayores
confidencias,---me parece que no es David sólo. También tú y otros como
tú\ldots{} metéis la cuchara en la olla del enemigo.

---¡Señor!---exclamó furioso José, golpeándose el pecho con
rabia.---Llámeme lo que quiera menos traidor. Por la necesidad le presto
a usted y a otras personas servicios de tercería. Pero vender a mi
Cantón de mi alma\ldots{} ¡eso no lo hago por todo el oro del
\emph{Potosí sumarino}!

Buscando yo nutritivo condimento histórico, encontraba tan sólo
aguanosas y desabridas salsas. Por las tardes, en la redacción de
\emph{El Cantón Murciano}, Fructuoso Manrique y Manuel Cárceles me
referían los sucesos, abultándolos desaforadamente. Las cosas más
vulgares, en boca de aquellos patriotas ingenuos, eran trágicas, épicas
y de grandeza universal o cósmica. Un día de Noviembre, no importa la
fecha, leí en pruebas un artículo de Roque Barcia, que ofrezco a mis
lectores como muestra de la literatura política sentimental que hizo
estragos en aquellos tiempos. El insigne don Roque flaqueaba por la
entonación lacrimosa de sus escritos, inspirados en los trenos de
Isaías, o en los cánticos de David bailando delante del Arca Santa.

Decía Barcia en su artículo que pronto partiría de Cartagena, por la
necesidad de \emph{inflamar en todas partes el fuego sagrado del
Cantonalismo}. Al marchar a otras \emph{Regiones, donde estaba a punto
de sonar el grito}, rogaba a todos que se acordasen de él. Concluía así
la salmodia: «Cuando los niños de hoy pregunten a sus madres ¿dónde está
aquel hombre que nos dio tantos besos?, que les contesten: ¿vosotros no
sabéis la historia de aquel hombre?\ldots{} Pues era\ldots{} hijo, era
un pirata.»

El 26 de Noviembre (esta fecha es de las que no pueden escaparse de mi
memoria), a las siete de la mañana, rompieron el fuego contra la Plaza
las baterías Centralistas. Al bombardeo no precedió intimación ni aviso
alguno. El primer momento fue de estupor medroso en Cartagena. Pero el
vecindario y los defensores de la ciudad no tardaron en rehacerse:
hombres, mujeres, niños y ancianos corrían al Parque en busca de
proyectiles y sacos de pólvora, que llevaban a los baluartes de la
muralla. Yo fui también allá para enterarme de cuanto ocurría, y vi
actos hermosos que casi recordaban los de Zaragoza y Gerona.

Entre la muchedumbre encontré al veterano de Trafalgar, Juan Elcano, que
ansiaba reverdecer sus marchitos laureles. Gesticulando con sus manos
tembliconas me dijo que si le daban un puesto en la muralla cumpliría
como quien era. La persona del heroico viejo trajo a mi mente la imagen
de \emph{Mariclío}, con quien primera vez le vi comiendo
\emph{aladroque} en la puerta de un caserón de Santa Lucía. Al momento
le pregunté por la divina Madre, y afligido me contestó: «Ya no está la
Señora en Cartagena. Una noche, hallándonos todos sus amigos
\emph{acoderados} a ella, oyéndole contar cosas de los tiempos en que
era moza (y para mí que su mocedad la pasó en el Paraíso Terrenal), se
desapareció de nuestra vista y todos nos quedamos con la boca abierta,
mirando al cielo, porque nos pensemos que se había ido por los aires.
Una vieja sabidora que andaba siempre con \emph{Doña Mariana}, nos dijo:
`Bobalicones; aunque la Señora gusta de platicar con los humildes, no
creáis que es mujer; es Diosa'. Yo calculo, acá para entre mí, que
\emph{Doña Mariana} es el Verbo, o por mejor hablar, la Verba divina.»

Al atardecer de aquel mismo día supe que el veterano de Trafalgar,
consecuente con su destino heroico, había muerto en la muralla
defendiendo la idea cantonalista, última cristalización de su
patriotismo.

Continuó el bombardeo en lo restante de Noviembre, con mucha intensidad
durante el día, atenuándose algo por la noche. Los proyectiles de los
sitiadores producían más estragos en los edificios de la población que
en las fortalezas. La Junta Soberana recorría los castillos y baluartes
dando ánimos a los defensores de la Plaza. Ocasiones tuve yo de ver y
apreciar por mí mismo el tesón de los Cantonales ante los fuegos
Centralistas. Esta virtud les hacía merecedores de la independencia que
proclamaban. Había cesado el estruendo importuno de los vítores, arengas
y aplausos, y llegado el momento, la función guerrera desarrollábase
gravemente, con viril entereza que rayaba en heroísmo.

Accediendo a las súplicas de los Almirantes de las escuadras
extranjeras, el General Ceballos concedió armisticios de cuatro y seis
horas para que salieran de Cartagena los ancianos, niños y mujeres. Una
de éstas, la impaciente \emph{Leona}, se preparó para escabullirse
aprovechando alguna de aquellas claras. Pero yo la disuadí con la
promesa de acompañarla si hasta Navidad me esperaba.

A don Jenaro de Bocángel le vi en el baluarte de la Puerta de San José,
lacio, trémulo y despintado, no ciertamente con anhelos heroicos, sino
con la modesta pretensión de transportar agua, proyectiles y cuanto los
combatientes necesitasen. Llevaba las babuchas de orillo y el pardo
chaquetón que yo le regalé. En el corto diálogo que sostuvimos me dijo
que, según noticias transmitidas por la suegra de su sobrino, la
proclamación del \emph{Cantón Mantuano} dependía de que la indómita
\emph{Cartago} hiciese una defensa heroica, no dejando títere con cabeza
en el Ejército de Ceballos.

El 29 de Noviembre marchó la escuadra Centralista a repostarse de carbón
en Alicante. El 30 hicieron los Cantonales una salida desde el fuerte de
San Julián, causando 25 bajas a los batallones de Figueras y Galicia,
que mandó a su encuentro el General Ceballos. Como yo no cesaba en mis
investigaciones, allegando datos para los anales de \emph{Mariclío}, fui
a ver a David Montero, y éste me dijo que Ceballos, apretado por el
Gobierno para rendir la Plaza en pocos días y no teniendo bajo su mando
fuerzas suficientes para consumar empresa tan difícil, había presentado
la dimisión. No di crédito a esta noticia. Algunos días después volví a
visitar a Montero, encontrándole inquieto y caviloso. Díjome que en
sustitución de Ceballos vendría López Domínguez, General joven,
procedente del Cuerpo de Artillería, y sobrino de Serrano. No pude
arrancarle más confidencias, ni me dio el menor indicio de la fuente de
sus informes.

El 5 o el 6 de Diciembre, no acierto a puntualizar la fecha, subí de
nuevo a la guarida del mecánico, astrónomo y estratega. Al traspasar la
puerta saliéronme al encuentro, desoladas, las dos viejas a quienes mi
exaltada mente confundió con las vaporosas figuras de \emph{Doña
Aritmética} y \emph{Doña Geografía}, las cuales me manifestaron que
estaban solas pues don David, después de quemar todos sus papeles, se
había marchado una madrugada enviando luego el aviso verbal de que su
ausencia duraría largo tiempo. Aquellas pobres mujeres no sabían qué
hacer ni a qué santo encomendarse.

Del 12 al 13 llegó López Domínguez y tomó el mando de las fuerzas
sitiadoras. Ceballos había marchado ya, dejando interinamente al frente
del Ejército Centralista al General Pasarón. Con el nuevo caudillo
vinieron los Brigadieres López Pintos y Carmona en sustitución de
Azcárraga y Rodríguez de Rivera, que con Pasarón marcharon a Madrid. El
primer cuidado de López Domínguez fue recorrer la extensa línea de sitio
y revistar las tropas, a las que encontró animosas y disciplinadas.
Luego dio una proclama. Siguió después el bombardeo, notándose que la
Artillería Centralista hostigaba a la población sin hacer fuego contra
los castillos, lo que puso en cuidado a los jefes Cantonales por ver en
ello un indicio de secretas connivencias con las guarniciones de los
fuertes. Desde que comenzó el bombardeo de Cartagena en 26 de Noviembre
hasta que López Domínguez tomó el mando del Ejército Centralista, hizo
éste 9.297 disparos de cañón, y la Plaza, sus fortalezas y fragatas
10.159. ¡Una friolera!

En el curso de Diciembre, pude apreciar por observación directa ciertos
hechos que explican y corroboran la psicología de las guerras civiles en
España. Leed, amigos y parroquianos, lo que a continuación os refiere un
observador sincero de los hilos con que se atan y desatan las
revoluciones en los tiempos ardorosos y pasionales de nuestra Historia.
Cuando arreció el bombardeo pudo advertirse que los jefes de los
batallones de Iberia y Mendigorría, que como se recordará se habían
pronunciado en favor de los rebeldes de Cartagena, se mostraban
inclinados a una pronta capitulación. \emph{Tonete} Gálvez, que poseía
tanta bravura como agudeza y era el hombre de mando en la República
Cantonal, con dotes militares, con dotes de estadista y toda la malicia
y sagacidad que siempre han sido complemento de aquellas cualidades,
supo calar las intenciones de los individuos del Ejército que meses
antes, en los torbellinos de Julio y Agosto, se habían pasado al
Cantonalismo con armas y bagajes. Los vigilaba cauteloso y al fin
descubrió el enredo.

Desempeñando el Coronel Carreras las funciones de Sargento Mayor de la
Plaza, dispuso una noche, con el pretexto de defender a Santa Lucía, que
salieran el batallón de Mendigorría y Movilizados. Gálvez, noticioso de
que se dio a estas fuerzas el mismo santo y seña que tenían los
sitiadores para entrar en Cartagena, ordenó al instante la suspensión de
la salida, y puso presos al Sargento Mayor y a varios jefes y oficiales,
asegurándolos en el castillo de Galeras. Al enterarse el General
Contreras de lo que ocurría, subió presuroso al castillo para escuchar
las declaraciones de los detenidos. Encerrado Carreras en una estancia,
alguien observó que rompía papeles apresuradamente.

En esta operación fue sorprendido, y sus guardianes recogieron los
trozos de papel, entregándolos a Gálvez y Contreras, que tuvieron la
paciencia de unirlos para obtener el texto completo. Entonces se
comprobó que había sido vendida la Plaza: era aquel escrito una lista de
comprometidos a entregar Cartagena a los sitiadores, y consignaba las
recompensas de grados y el premio pecuniario que por su defección les
concedería el Gobierno Central. Ordenose en el acto la prisión de los
que aquel documento denunciaba, y dieron con sus huesos en Galeras
Pozas, Pernas, Perico del Real y otros muchos militares de diferente
rango y categoría.

Pocos días después de este grave suceso, supo Gálvez por un soplo que a
las doce de la noche tenían decidido embarcar y marcharse de Cartagena
algunos individuos de la Junta Soberana. Eran las ocho cuando, reunida
la Junta en el Ayuntamiento, se presentó \emph{Tonete} en el Salón de
sesiones, sin más escolta que su hijo Enrique, su sobrino Paco y el
capitán de Voluntarios Tomás Valderrábano. Llevaba Gálvez las manos en
los bolsillos del pantalón y en ellos dos pistolas amartilladas. Apenas
traspuso la puerta dijo a los reunidos: «No se mueva nadie. Al que
intente salir le levanto la tapa de los sesos, y si alguno se me escapa,
en la calle será recibido a tiros.»

---¿Puedo yo moverme?---preguntó el General Ferrer.

---Puede usted pasearse dentro de esta sala; pero nada más---contestó
Gálvez con sequedad y entereza, añadiendo sin más preámbulos.---Han sido
ustedes descubiertos, caballeros.

Quedaron corridos como monas los señores de la Junta que estaban en el
ajo. Estrechó \emph{Tonete} la mano a los que consideraba leales al
Cantón; a los demás dijo que quedaban en libertad, que podían ausentarse
de Cartagena previo aviso, y que sí alguno permanecía en la ciudad y
hacía traición a la Causa sería fusilado en el acto sin compasión.

\hypertarget{vi}{%
\chapter{VI}\label{vi}}

Ante sucesos de tal trascendencia no podía faltar la bíblica salmodia
del bueno de don Roque. Resonó en un escrito jeremíaco recomendando que
al imponer castigo a los desleales, se hiciera justicia magnánima,
generosa, clemente\ldots{} Decíase por aquellos días que López Domínguez
había pedido cuatro mil hombres de refuerzo al Gobierno Central, y que a
los apremios de éste para rendir la Plaza antes de 1.º de Enero, fecha
de la reunión de las Cortes, contestó que a tantos no se podía
comprometer. Con un mes largo por delante quizá podría rematar la
empresa.

Castelar ofreció mandar los refuerzos y seguía pidiendo \emph{rendición
a todo trance}, ya por la fuerza, ya por el soborno, o bien combinando
hábilmente ambos métodos de guerra\ldots{} A mediados del mes, los
sitiadores concentraron sus fuegos sobre los castillos de Atalaya, Moros
y Despeñaperros, y las puertas de San José y Madrid. La Plaza contestó
con brío, y los disparos de la escuadra Centralista contra San Julián
resultaron cortos y por tanto ineficaces.

Reunió a la sazón López Domínguez Consejo de Generales para determinar
el plan que habían de seguir, acordándose por el pronto la conveniencia
de un ataque vigoroso a San Julián, y conviniéndose en la urgencia suma
de reforzarla línea de bloqueo: ésta no era inferior a seis leguas, y si
no se neutralizaba la extensión con la intensidad, imposible alcanzar el
éxito con la rapidez que Castelar quería. Desplegaba López Domínguez
enorme actividad, supliendo con su cuidado y esfuerzo la escasez de los
medios de combate.

En Pormán celebró el General en Jefe una entrevista con el
Contralmirante Chicarro, el cual le dijo que le era dificilísimo el
bloqueo marítimo porque sus barcos andaban bastante menos que los barcos
rebeldes. Con tal Marina y un Ejército animoso, pero de contado
contingente, era obra de romanos rendir la más formidable plaza de
guerra que sin duda existe en el Mediterráneo. Si los Cantonales
hubieran tenido tanto seso como bravura en aquella última ocasión de su
loca rebeldía, no queda un centralista para contarlo.

Hasta el 28 de Diciembre transcurrieron los días sin ningún suceso
extraordinario. Continuaba incesante el fuego entre sitiadores y
sitiados. Éstos hicieron varias salidas y en una de ellas causaron diez
y ocho bajas a sus enemigos. Hacia el 22 recibieron los centralistas los
refuerzos que esperaban y con ellos veinticuatro piezas de Artillería de
diez y seis centímetros. El 24, un proyectil Armstrong disparado por la
fragata \emph{Tetuán}, que seguía mandada por el intrépido
contrabandista Colau, estalló en la batería número 3 del campo enemigo,
haciendo reventar cuatro granadas que dieron muerte a un oficial,
catorce artilleros e individuos de tropa, y tres paisanos. Y con esto,
amados lectores, llego al día 28, fecha culminante en mi memoria por ser
la fiesta de los Santos Inocentes, y porque en aquella madrugada, a
punto de salir el sol, nos escapamos de Cartagena \emph{Leona la Brava}
y yo, suceso a mi ver memorable que merece un rinconcito en estas
verídicas crónicas.

Mi escapatoria no fue secreta, pero tampoco me convino hacerla pública.
Sólo me despedí de Manolo Cárceles, a quien tantas atenciones debía. Al
abrazarnos, me dio con sus cariñosos adioses algunos recados verbales
para Estévanez, Castañé y Patricio Calleja. Prometile yo volver pronto,
pues me interesaba mucho el Cantón y quería presenciar hasta el fin su
arrogante defensa. En la respuesta de Cárceles creí advertir cierta
disminución del optimismo que había mostrado desde el comienzo de la
revolución cantonal: «Si nos vencen---me dijo,---y ello habrá de ser más
por la maña que por la fuerza, abandonaremos este volcán y nos iremos
tranquilamente al África en busca de mejor suelo para poder vivir. Si
vuelves, gran Tito, te vendrás con nosotros y nos haremos todos
africanos.»

Hasta la línea de bloqueo nos acompañó, al marcharnos \emph{La Brava} y
yo, mi leal mandadero \emph{Pepe el Empalmao}, a quien las fatigas del
sitio convirtieron de rufián en héroe. Su inveterada indolencia trocose
en actividad febril, su astucia de zorro en fiereza leonina. En los
baluartes de las puertas de San José o de Madrid afrontaba las balas
enemigas, con un desprecio de la vida que ya lo querrían para sí más de
cuatro figurones, de los que aspiran a merecer una línea en las altas
inscripciones de la Historia. Y no lo hacía por ambición ni propósito de
medro; no esperaba recompensa, ni galones, ni cintajos, ni cruces, ni
siquiera el aumento de un real en su miserable soldada. Hacíalo, sin
darse cuenta de ello, por la gloria, por un ideal que indeterminado y
confuso hervía dentro de aquel cerebro, que para muchos no era más que
una olla del más tosco barro. Como yo no quería partir sin saber algo
del pobre \emph{don Florestán de Calabria}, interrogué al
\emph{Empalmao}, que así me dijo:

«Ahora presta servicios de ranchero en las cocinas que ha mandado poner
la Junta Soberana en el sótano de la muralla de los Mártires. Allí le
tiene usted, con su mandil y su cucharón, revolviendo los peroles en que
nos hacen la bazofia con que matarnos el gusanillo. Don Jenaro, que no
sirve para militar sino para chupatintas, ha pedido a Contreras que le
nombre Memorialista Mayor de la República Cartagenera. Pero para mí que
se queda meneando el cazo toda su vida\ldots» Con esto nos despedimos
afectuosamente, y Leonarda y yo cogimos el tren de Madrid en la estación
de la Palma.

Ya estábamos instalados en un coche de segunda con la ilusión de ir
solitos todo el camino, y ya el tren se ponía en marcha, cuando vimos
que avanzaba presurosa y dando chillidos una pobre señora, cargada de
envoltorios, que intentó subir a nuestro departamento. Gracias al
auxilio que yo le presté pudo poner el pie en el estribo y posesionarse
de un asiento. Era una vieja de buenas carnes, vestida de negro. Al
fijarme en su rostro temblé de sorpresa y sobresalto: o yo estaba loco o
tenía frente a mí a la propia \emph{Doña Gramática}, si bien envejecida,
un poquito cargada de espaldas y tan descompuesta de facciones como de
vestimenta. Antes que yo pudiera decir palabra, soltó ella la suya
dejándome más absorto y alelado que antes, pues en cuanto abrió el pico
reconocí la tremebunda y retorcida sintaxis de la que en día no lejano
fue mi mayor suplicio. Volví a creer que me perseguían fantasmas al
escuchar de boca de la vetusta dama estas enfáticas razones:

«No agradeceré bastante al noble caballero la merced con que me ha
favorecido al prestarme ayuda para escalar, con la enfadosa carga de mis
achaques y de estos paquetes, el endiablado vehículo. No están ya mis
pobres huesos para tan vivos trotes\ldots{} Ello ha sido que, faltando
cortos minutos para la partida del tren, corrí a recoger estos livianos
bultos, que olvidados dejó mi señora en la covacha del jefe de la
estación, hombre descuidado al par que descortés, por quien a punto
estuve de perniquebrarme o de quedarme en tierra. Gracias a usted,
repito, y a esta hermosa dama cuyas manos diligentes me ayudaron a
subir, y Dios se lo pague, pude meterme en este coche zaguero, y salva
estoy aquí, aunque todavía no reparada del grave susto ¡ay de mí!, ni
del sofoco de estos cansados pulmones. ¡Ay, ay!\ldots»

Como he dicho, creí hallarme otra vez en pleno delirio y perseguido por
las visiones de antaño. Apenas recobré la palabra, que el azoramiento y
la confusión me habían quitado, dije a la para mí fantástica viajera:
«Señora; perdóneme si la interrogo con cierta indiscreción. ¿Es usted
\emph{Doña Gramática}, ilustre dama versada cual ninguna en los giros de
las sintaxis?»

---No me llamo \emph{Pragmática}---contestó ella con melindre---sino
Práxedes. No soy dama ilustre, aunque no hay bastardía en mi linaje, y
sólo acierta usted en que mi afición al estudio me ha enseñado a hablar
con discreta corrección y propiedad.

En tanto, \emph{Leona} no quitaba los ojos del rostro de la vieja, cuyo
hablar finísimo y entonado le colmó de asombro y embeleso. En el mirar
de mi amiga leía yo un afán ardiente de apropiarse los términos
exquisitos y la nobleza gramatical de nuestra compañera de coche.

«Cualesquiera que sean su nombre, estirpe y condición, señora mía---dije
yo a doña Práxedes,---nosotros estamos muy complacidos de haber trabado
conocimiento con usted. Juntos haremos este molesto viaje, honrándonos
mucho con su grata compañía.»

---¡Ay!, eso no podrá ser---replicó la enlutada dueña, arqueando las
cejas.---Y de veras lo siento, porque me hallo harto gustosa entre
personas tan hidalgas. En la primera parada que no sea corta tengo que
pasarme al coche donde va mi señora, la cual es de alcurnia tan alta que
no hay en la grandeza española quien pueda igualarse a ella. Va en el
departamento que lleva el rótulo \emph{Reservado de Señoras}. A su
servicio tiene damas y doncellas de singular hermosura.

Lo dicho por la vieja me adentró más en los delirios paganos. Pensé que
en el mismo tren iba \emph{Mariclío\ldots{}} quizás Floriana\ldots{}
¡Dios mío, qué horrible trastorno, mezcla de alegría y espanto! Si yo me
presentaba a la divina Madre y ésta me veía con \emph{La Brava}, sin
duda me reñiría duramente por mi liviandad\ldots{} Advertí que doña
Práxedes, risueña, no apartaba sus ojos inquisitivos del rostro de
\emph{Leona}. Sorprendida de su silencio pronunció estas palabras: «Y
esta joven tan hermosa y apuesta ¿no dice nada?» Mi compañera balbució
algunos monosílabos que no expresaron más que su timidez y el temor de
soltar algún disparate chulesco ante una tan refinada maestra de la
lengua castellana\ldots{} Intenté pedir a doña Práxedes más claras
referencias de aquella princesa de alto linaje que iba en el Reservado
de Señoras, con acompañamiento de bellas damas y lindísimas doncellas;
pero un escrúpulo invencible paralizó mi lengua, y seguí alelado y
taciturno.

Al fin, hostigada por la vieja redicha, pudo \emph{Leona} desatar el
nudo de su timidez, y pronunció algunas frases rebuscaditas para
demostrar que no era muda. «Nosotros vamos a Madrid---dijo haciendo con
sus rojos labios mohínes muy finústicos,---porque Cartagena es un
infierno en \emph{pequeña miniatura}. Allí la libertad es \emph{un
viceversa} del sosiego, o como quien dice, \emph{una ironía} que la
tiene a una siempre sobresaltada. En Madrid viviremos tranquilos porque
allí la libertad no hace daño a nadie. Además, como estamos bien
relacionados en la Corte, lo pasaremos al pelo.»

---Su esposo de usted tendrá, y esto lo colijo por su talante, porte y
lenguaje distinguido---dijo la vieja, clavando en mí sus miradas como
saetas,---tendrá de fijo, repito, una elevada posición.

---Regular---contestó \emph{Leona}, mordiendo su abanico para contener
la risa.---No diré que sea de las más ensalzadas, \emph{ni verbigracia}
cosa de poco más o menos. \emph{En el interín}, nos basta y nos sobra
para todas \emph{las circunstancias} de nuestra vida, y como no tenemos
sucesión, sucede que marchamos divinamente.

Aunque me mortificaba que \emph{Leona} me diputase por esposo permanente
y legítimo, no me pareció bien desmentir a mi amiga, y permanecí callado
largo rato, mientras ellas departían a su sabor. Leonarda, perdida
completamente la cortedad, hablaba a doña Práxedes de lo divertido que
era Madrid, donde había tanta aristocracia y tanta democracia. «Entre
otros mil atractivos---dijo,---Madrid tiene toros los lunes y domingos,
funciones en la mar de coliseos, misas de seis a doce en todas las
iglesias, y a cada dos por tres jaleo de revolución en las calles.»

Hasta la estación de Murcia, donde el tren paraba quince minutos, no se
atrevió doña Práxedes a bajar al andén para cambiar de coche. Despidiose
de nosotros con frase coruscante y ensortijada, deseándonos un viaje
dichoso y toda la ventura conyugal que por nuestra juventud y buenas
partes merecíamos. \emph{La Brava}, que en los últimos coloquios había
hecho muy buenas migas con aquella gramatical cotorra, tuvo gusto en
descender con ella y en llevarle \emph{los livianos bultos}, según la
clásica expresión de la matrona provecta. Era mi costilla \emph{per
accidens} vivaracha y curiosona, amiga de gulusmear y enterarse de todo.
Acompañó a la vieja hasta el Reservado de Señoras y, al abrirse la
portezuela para dar paso a doña Práxedes, exploró con rápida vista al
interior del departamento en que viajaban las misteriosas damas.

Pronto volvió a mi lado, contándome de este modo lo que había visto:
«Pues allí va una señorona con más años que Matusalén, alta y de buenas
hechuras. Su cara es blanca, con perfil de estatua: parece mismamente de
mármol. Viste de luto y tiene aire de reina que ha perdido el trono. En
el fondo del coche hay otras mujeres, y entre ellas una \emph{chavala}
guapísima\ldots{} como los propios ángeles. La \emph{gachí} parece una
diosa de las que he visto pintadas en un libro que tiene \emph{don
Florestán\ldots{}} No pude fijarme más porque ellas me miraban como
choteándose de mí. Me dio vergüenza y \emph{me retiré en buen orden a
mis posiciones}, como dice el ayudante de Contreras.»

Al partir el tren llenose nuestro coche de viajeros de Murcia, que
alborotaban hablando a gritos de las cosas del Cantón. Unos ponderaban a
Gálvez con extremadas hipérboles, asegurando que si le dejaran sería
pronto el dominador de toda España; otros, con desmayado pesimismo,
sostenían que el Cantón estaba perdido y que López Domínguez daría buena
cuenta de aquella gentecilla, entre Año Nuevo y Reyes. Yo me desentendí
de esta conversación, y reclinado en un ángulo del coche, mi mano en la
mano de Leonarda, permanecí largo rato soñoliento y meditabundo,
pensando en lo que mi amiga me contara de las damas que ocupaban el
Reservado de Señoras. ¿Iba \emph{Mariclío} en aquel departamento? ¿Era
Floriana la divina hermosura que \emph{Leona} comparó con las diosas?

En estas ideas y en dudas tan crueles fluctuaba mi espíritu, que ya se
asía fuertemente a la realidad rechazando toda relación con el mundo de
las quimeras, ya se lanzaba disparado a embelesarse con las hermosas
visiones Paganas y Mitológicas. Por momentos, el deseo y la curiosidad
me aguijoneaban para correr hacia el coche donde iban las misteriosas
viajeras; por momentos, el miedo a la desilusión y la idea de ser mal
recibido me retenían, sujetándome a la única diosa de que yo podía
disponer, \emph{Leona la Brava}, divinidad terrestre, pedestre y de
vuelo harto rastrero y prosaico.

En la estación de Hellín saqué un momento la cabeza por la ventanilla, y
vi pasar a un hombre de soberbia talla y formas escultóricas. ¿Era el
arrogante forjador de voluntades, padre presunto de las mil hijas de
Floriana que, después de echar toda el agua fría del mundo sobre mi
pasión por la Maestra educadora de pueblos, me arrojó desde lo alto de
un talud, cual si yo fuera un muñeco inservible o un despreciable
animalejo? Cuando advertí que el divino Titán, vestido con azul ropa de
maquinista, se acercaba al Reservado de Señoras y subiendo al estribo
departía con las incógnitas viajeras, llegué al colmo del espanto.
Tembloroso me arrebujé en la manta y cerré los ojos para reconcentrarme
de nuevo en mí mismo. «¿Qué te pasa?»---me preguntó \emph{Leona}. Y yo
respondí: «Me pasa\ldots{} me pasa que he visto cómo resucita el
Paganismo que creíamos muerto para siempre. Me pasa que he visto una
figura\ldots»

---¿Pero a quién, a quién has visto? ¿Quién ha resucitado?---exclamó
Leonarda con súbito terror, palideciendo.---¿Es mi marido que ha vuelto
ya de la isla desierta?

---No, hija, no. Tu marido\ldots{} se lo comieron los peces y lo han
digerido ya. La figura que he visto no es la de Cándido Palomo. Es la
del forjador atlético hijo de los Dioses, padre de las mil
maestras\ldots{} renovador del Paganismo\ldots{}

---¡Bah, bah!; ésas son coplas. ¿Ya estás otra vez con la tecla de los
\emph{paganistas}? Pues ya sabes que el mejor \emph{paganismo} es no
pagar a nadie y cobrar todo lo que se pueda.

\hypertarget{vii}{%
\chapter{VII}\label{vii}}

En Chinchilla, donde bajamos a confortar nuestros estómagos con el agua
de castañas almidonada que llaman café con leche los fondistas de las
estaciones, me puso la mano en el hombro un señor a quien al pronto no
conocí.

Era David Montero, totalmente transfigurado de ropa y rostro. Tenía la
facha de un clérigo vestido de seglar. Se había quitado barba y bigote,
y disimulaba con ligero tinte las canas de las sienes y de la nuca, bajo
un gorro de terciopelo negro como el que usan los párrocos de aldea.
«Hablemos quedito---me dijo sentándose junto a mí,---y no pronuncie
usted mi nombre. Ya ve que voy disfrazado. Me escapé hace días, y en
casa de un amigo de Balsicas me vestí de máscara para marcharme a
Madrid\ldots{} \emph{Leona} me mira sonriendo. Sin duda me ha conocido.
Adviértale que no venga ahora con aspavientos y que no me llame por mi
nombre\ldots{} Ya hablaremos, ya hablaremos. Dígame en qué departamento
van, y si es de segunda como el mío pasaré un rato con ustedes.»

Alegrándome mucho de ver a David, le indiqué que íbamos en el último
coche. Antes de partir el tren ya estábamos reunidos los tres y
entablábamos una grata conversación sin recelo de ser oídos, pues al
pasar de Chinchilla sólo quedaron en nuestro departamento dos viajeros,
que arrebujados en sus mantas dormían como lirones. «El Cantón está
perdido, señor don Tito---me dijo Montero con voz apagada.---Lo estuvo
desde 1.º de Diciembre. Ya sabrá usted la prisión de Carreras, Pozas y
demás individuos del Ejército.»

---Lo sé, lo sé---respondí.---Estoy bien enterado de todo. Desde que
López Domínguez tomó el mando de las fuerzas Centralistas, los militares
de la plaza se hacen cucamonas con los de fuera.

---¡A quién se lo cuenta usted!---repuso David.---Yo he tenido algún
trato con los Centralistas. Ello fue porque un primo mío, Carlos
Montero, está de mecánico en el Cuartel General, donde le estiman mucho
por los servicios que presta. He hablado con el Coronel Sánchez Molero,
que ayer me dijo: «La fiesta de Reyes la celebraremos dentro de la
Plaza.» He hablado también con López Domínguez, quien, generoso, y muy
satisfecho con las referencias que le dieron de mí, me aseguró que
pedirá mi indulto. Pero mientras esa gracia viene, yo me pongo en salvo,
amigo mío, que si se rinde Cartagena, lo primero que harán los
vencedores será meter en \emph{chirona} a toda la población penal. Y lo
que es a mí no me pescan.

---Muy bien, David---dije yo,---ha hecho usted muy bien: libertad y vida
nueva.

---Eso, eso---saltó \emph{La Brava} juguetona y alegre.---La idea de
pasar de un mundo a otro la tuvo antes que usted, amigo Montero, una
servidora. No más presidio: el mío era la pobreza, la vergüenza, el
andar siempre entre gente groserota y vil o entre señoritos babosos y
cargantes que todo lo ven \emph{bajo el prisma de la corcupicencia}.

No pudimos prolongar nuestro coloquio porque Montero se quedó en
Albacete, donde tenía un hermano. Allí descansaría breve tiempo,
trasladándose luego a Madrid sin abandonar las precauciones que
garantizaban su libertad. Díjome su nombre postizo, que era \emph{Simón
de la Roda}, añadiendo que se holgaría mucho de que nos viéramos en la
Villa y Corte. De su paradero darían razón en el taller de Calixto
Peñuela, un su amigo, famoso armero establecido en la calle de los
Reyes, número 15\ldots{} En Alcázar de San Juan, donde la parada fue muy
larga, no me fue posible reprimir mi curiosidad, y me lancé a una
indiscreta exploración del Reservado de Señoras, cuya portezuela estaba
abierta.

Con gran asombro vi que el coche se hallaba vacío. ¿Qué se hizo de las
misteriosas viajeras? ¿Se desvanecieron en los aires cual figuras que
tenían su domicilio en los espacios imaginarios, o eran seres de carne y
hueso que habían terminado su viaje? Busqué a las fantásticas damas a lo
largo del andén; luego en la Fonda, y no hallé rastro de las princesas o
señoras \emph{paganistas}, como decía \emph{La Brava}. Ésta, que era un
águila para las averiguaciones por su metimiento y natural comunicativo,
preguntó a un empleado del tren, el cual nos dijo secamente que el
Reservado de Señoras había venido vacío desde Cartagena. La mentira y la
verdad, enzarzadas y juguetonas, continuaban atormentando mi espíritu.

Nos hallábamos mi \emph{costilla falsa} y yo consumiendo sendos
chocolates con tortas de Alcázar, cuando se nos acercó un señor de más
que mediana edad, alto y de buen porte, suelto de ademanes y de lengua,
que saludó a \emph{Leona} con despejo y gracia, felicitándola por verla
camino de Madrid. Fue después al mostrador para pagar su gasto y el
nuestro, y yo pregunté a \emph{La Brava}: «¿Este caballero es Prefumo o
uno de los Paganes de Murcia?»

\emph{---Pagano} es y de los buenos---me contestó mi amiga
gozosa.---Pero no se llama Pagán. Y cuando el caballero volvía del
mostrador salió ella a su encuentro y hablaron un mediano rato lejos de
mí. Al meternos en nuestro coche para continuar el viaje, mi esposa
fortuita o accidental me dijo, con frase que por su extremada sinceridad
parecía candorosa, que el \emph{pagano} le había propuesto pasarse a su
departamento de primera y que él abonaría la diferencia del billete.

«¿Qué te parece, Tito?---agregó la moza con zalamería.---Sí tú lo
consientes, voy; si no, no. Te digo esto, Titín, porque el ir con ese
amigo me servirá para la introducción.»

---¿Qué quieres decir?

---Que para introducirme o como aquel que dice presentarse en la vida de
Madrid, ese caballero poderoso me hará un buen avío. Aconséjame si debo
ir o no. Aconséjame, hombre.

Con toda honradez y franqueza le contesté que siendo ella mujer libre y
árbitra de su destino, podía tomar la senda que más le conviniese para
el buen principio y orientación en la carrera que había emprendido. Mi
fácil consentimiento produjo en ella un ligero chispazo del amor propio
y fugaces monerías de coquetismo. Pero al fin quedó convencida, gracias
a la perfecta lucidez con que yo expresé la rectitud de mis intenciones.
Díjele que si en Madrid necesitaba de mí me encontraría en mi vivienda,
calle del Amor de Dios. Como \emph{La Brava} no dominaba el conocimiento
de los números, señalé la casa con la infalible indicación de que junto
a la puerta había una cacharrería y en ésta una tablilla anunciadora de
\emph{burras de leche\ldots{}} En Aranjuez se consumó nuestro divorcio.
No debo ocultar que si ella se fue un tanto pesarosa yo quedé
medianamente triste.

Llegué a Madrid solito y tan campante. Al tomar un coche de punto vi de
lejos a \emph{Leona la Brava} con el caballero \emph{pagano}, precedidos
de un mozo cargado de bultos, y disponiéndose a entrar en el ómnibus de
la Fonda Peninsular. En mi casa fui recibido con explosión de júbilo. A
Rosita encontré más espigada, a Nicanora más barriguda, y a Ido
transparente ya de puro espiritado. Una novedad de la vida hospederil me
contrarió mucho: la que yo llamaba mi habitación estaba ocupada por una
señora, a quien mis buenos patrones no podían echar para restituirme en
el usufructo de aquel cuarto. Era una dama recomendada por Delfina Gil,
la dulce beata traficante en ataúdes. ¿Era guapa aquella señora? Sí.
¿Joven? Regular, tal, cual\ldots{} En fin; ya la veríamos.

Ayudándome a quitarme la ropa de viaje, el seráfico Ido me dijo: «Ya
sabemos, señor don Tito, que los cabecillas cantonales le nombraron a
usted Embajador de Constantinopla, y que usted propuso al Gran Turco
pactar un \emph{Tratado de Alianza} con la República Cartagenera\ldots{}
No se ría, no venga negándolo; aquí todo se sabe\ldots{} Nos dijeron
también que estuvo en Roma tratando de conseguir del Papado que se
entendiera con Roque Barcia para establecer en Cartagena un catolicismo
suave y democrático. Ahora\ldots{} usted lo negará, porque diplomacia y
reserva son una misma cosa\ldots{} ahora, digo, viene usted a Madrid a
negociar con el Gobierno las paces con el Cantón en condiciones honrosas
para ambas partes\ldots{} No se haga de nuevas\ldots{} ¡Si aquí le están
esperando!\ldots{} Hace días estuvo en casa don Nicolás Estévanez a
preguntar cuándo volvía usted. Luego vino con la misma cantinela un
caballero que a mi parecer es el secretario del señor Maisonave,
Ministro de la Gobernación.»

---También vino---dijo Nicanora, que entraba con ropa limpia para
hacerme la cama---uno que debía de ser el propio Castelar\ldots{}

---Era él, era él---afirmó Ido dándose una palmada en la frente.---Era
don Emilio con barba postiza.

---No, José, no; estás trascordado---repuso Nicanora.---Aquel caballero
no traía barba\ldots{} Pero si no era don Emilio, era Carvajal
afeitadito\ldots{} También estuvieron aquí don Luis Blanc, don Serafín
de San José y un porción de santones, es a saber: el General Velarde,
Solís, Moreno Rodríguez, doña Candelarita la escritora, y un tal Robledo
Romero que me parece que es borbónico.

El mismo día de mi regreso al hogar patronil, hice conocimiento con la
señora que ocupaba mi habitación. Era una dama de agraciado rostro, de
estatura menos que mediana, edad incierta entre los treinta o treinta y
cinco, tipo de lugareña fina, modosa y bien criada, el habla dulce
aunque no exenta de viciosas concordancias, vestida con el hábito de los
Dolores, limpia, peinada con esmero y un poquito perfumada.

«No es la primera vez que veo a usted, señor Liviano---me dijo,
haciéndome sentar junto a ella en el sofá de los duros y punzantes
muelles .---Yo soy vizcaína, de un pueblo que llaman Elanchove, y en
Durango tuve el gusto de oír el discurso que usted nos echó sobre la
\emph{República Pontificia}, sermón bonita que si al pronto nos
entusiasmó, luego vimos que irreverente burla era\ldots{} Conozco a su
padre de usted que fuertecito todavía está, aunque resentido de sus
achaques. Trato mucho a su hermana Trigidia y a Ignacio Zubiri. Soy
amiga de Pepita Izco, y algo parienta del cura Choribiqueta. Me llamo
Silvestra Irigoyen, pero allá todos me conocen con el nombre familiar de
\emph{Chilivistra\ldots{}} Conque ya ve que nos conocemos\ldots{} Y
ahora sólo me falta decirle que esperaba su vuelta como agua de Mayo
para que me dé su auxilio \emph{poderosa} en la pretensión que traigo a
Madrid.»

Atento a la buena señora, y sintiéndome ya ¿por qué no decirlo?,
prendado de su modestia y dulzura melancólica, le dije que dispusiera de
mí a todo su talante y voluntad.

«Tanto Delfina como este señor Sagrario y doña Nicanora---prosiguió
\emph{Chilivistra}---me han dicho que a usted no le niega nada el
Gobierno. Cosa que pida es cosa lograda. Todos me aseguran que va usted
para Ministro, y que ha venido al arreglo de paces con el
\emph{Cantona.»}

Protestando con modestia de aquella supuesta privanza mía, le rogué que
me diera razón de su cuita o desventura, y ved aquí lo que me contestó,
echando por delante un gran suspiro: «Yo soy casada\ldots{} No podré
decir a usted si el casarme fue para mi felicidad o desdicha, pues de
todo hay. Mi marido es\ldots{} corazón de ángel y genio de todos los
demonios. Pruebas mil tengo de su cariño, y en mi cuerpo no faltan
señales de sus malos tratos. Se llama Gabino Zuricalday. En su familia
todos son carlistas netos\ldots{} Desde Febrero del año pasado mandaba
el 5.º Navarro. Cuentan que era una fiera en los combates\ldots{} Por
dejarse llevar de su arrojo le coparon con otros en un encuentro que
tuvieran con las avanzadas de Moriones cerca de Bacaicoa. Cuando le
llevaban preso a Pamplona quiso escaparse y\ldots{} ¡pim!, ¡pum!\ldots{}
sin lograr su objeto, Gabino mató a un guardia civil\ldots{} Milagro fue
que no le fusilaran. Hoy le tiene usted en la prisión militar de Logroño
esperando sentencia de un Consejo de guerra\ldots{} Más de un mes lleva
en este suplicio; pero ello va despacio. Militares hay del Ejército
\emph{liberala} que se interesan por él; mas no faltan otros que no
pararán hasta la vida quitarle\ldots{} Oído el parecer de mi familia, y
el consejo de mi confesor, vine a Madrid para poner cuanto esté de mi
parte en la santa obra de salvar a ese desgraciado.»

---Procede usted---le dije yo efusivamente apretándole las manos---como
esposa cristiana que olvida las ofensas y obra conforme a la divina ley
de amor. Porque si es verdad que su bello cuerpo conserva señales de
malos tratos\ldots{}

\emph{Chilivistra} me interrumpió diciendo con presteza: «Cardenales
fueron y \emph{tantas} que llevaba yo sobre mí todo el Sacro Colegio.
Más tiempo ha que no dolerme. Mi confesor, santo siervo de Dios y de don
Carlos, me ha dicho que perdone al marido mala que me ofendía\ldots{} y
ello no era más que cuando se arrebataba por la bebida o se
encalabrinaba porque le había soplado mal el naipe\ldots{} El Altísimo y
mi conciencia me gritan que emprenda la campaña de redención. Lo hago no
sólo por mí sino por el mi hijo\ldots{} Se me olvidó decirle que tenemos
un niño de siete años al cual he dejado en casa de los mis
padres\ldots{} ¡Ayúdeme usted, don Tito, en esta empresa cristiana, y si
en ella salimos \emph{triunfos} ganaremos el cielo!»

Lo que yo mayormente quería ganar era la ternura indecisa de sus ojos,
tras de los cuales entreveía los cielos infinitos del amor. «Señora
cristiana y dolorida---exclamé con arranque,---yo, como buen caballero,
me pongo al servicio de usted, y no tendré paz ni sosiego hasta que
rematemos el alto empeño de rescatar la vida de su esposo. Hoy mismo
veré a Sánchez Bregua, a Castelar. Mi grande amigo Emilio no me dará una
negativa\ldots»

\emph{Chilivistra} quedó muy complacida, y yo salí de su presencia
revolviendo en mi mente un plan de campaña que me pareció inspirado en
la lógica más pura. Con el súbito recuerdo de mis admirables éxitos, en
la primera mitad del año que expiraba, se renovó en mí la firme
convicción de que cuantas peticiones hiciese a los Ministros serían
inmediata y satisfactoriamente resueltas, por obra y gracia de mis
invisibles espíritus familiares. En aquel poder hermético confiaba yo
para conseguir la libertad del prisionero y hacerme dueño de su
interesante y acardenalada esposa.

Imaginando que me bastaría poner una expresiva carta a mi amigo
Eleuterio Maisonave para que el prodigio se realizase con la presteza
sobrenatural de marras, puse en ejecución mi pensamiento, y allá fue la
epístola que a mis queridos espíritus daba tarea en qué pasar el
rato\ldots{} Refrescado y vestido de limpio me eché a la calle en busca
de mis camaradas, y tuve la desgracia de no encontrar a ninguno.

Silvestra, sola o con Delfina, iba diariamente a misa, y las más de las
noches a los oficios que se celebraban en las iglesias próximas. Pero no
creáis, lectores píos, que era una de esas beatas apestosas y cargantes
que son verdadero antídoto contra el pecado. Largo espacio de la mañana
empleábalo en la limpieza y arreglo de su bella persona, y cuando salía
tan bien apañada y elegantita, daban ganas de ir en su seguimiento y
arrodillarse con ella ante los altares. El 1.º de Enero de 1874, se me
ocurrió salir en su acecho y la sorprendí hociqueando en la rejilla de
un confesonario. Mas no por esto se amenguaban su gracia y atractivos.
Algunas veces, después de dar un paseíto por el barrio, volvía trayendo
en su pañuelo naranjas o peladillas compradas en los puestos de Antón
Martín. Jamás conocí santurrona tan sugestiva y simpática.

Fiado en la intervención de mis amigos del otro mundo, daba yo a
\emph{Chilivistra} seguridades de un éxito feliz en nuestra empresa de
salvamento, y una tarde, acompañándola con su permiso a la iglesia de
Montserrat, donde había sermón y Manifiesto, pude advertir que cuando yo
le hablaba de la libertad de su marido no parecía tan contenta como era
de suponer. Llegué a formar la opinión de que los anhelos de la dama
dolorida y coquetona se satisfarían con obtener la vida de Zuricalday, y
conseguido esto\ldots{} que le mandaran lejos, lejos, a Filipinas por
ejemplo, poniendo así la mayor distancia posible entre el adorable
cuerpo de la señora y la mano impía del esposo.

No se me olvida la fecha de estas insignificantes ocurrencias y vanos
coloquios. Era el 2 de Enero. Deseoso de ponerme en contacto con mis
amigos me fui al Congreso, donde el invisible poder de \emph{Mariclío}
me llevó a presenciar los memorables acontecimientos de la noche del 2 y
madrugada del 3 de Enero de 1874\ldots{} ¡Dame tu aliento, sostén en mí
la acendrada devoción de la verdad, divina Madre y Maestra!

\hypertarget{viii}{%
\chapter{VIII}\label{viii}}

El primer amigo con quien tropecé en los pasillos fue Moreno Rodríguez,
a quien debí las referencias que me dieron un rumbo fijo en la corriente
histórica. Díjome que las mayores dificultades acumuladas sobre el
Gobierno Castelar provenían de la inquietud de los Intransigentes y de
la cuestión de los obispos. «Ya sabes---añadió---que sin aquiescencia de
Roma nombraron Arzobispo de Cuba al padre Llorente, íntimo de Martos, y
Obispo de Cebú al amigo Alcalá Zamora, demócrata de buena cepa, que
siendo diputado en las Constituyentes del 69 votó la Libertad de Cultos
vestido de clérigo. Sabes también que el Papa se negó a preconizar a
estos prelados, y que han pasado largos meses sin que el Gobierno
español y el Vaticano se entiendan.»

---Ya, ya lo sé---contesté yo.---Dicen que Pío IX está afligidísimo.

---Naturalmente---repuso mi amigo;---lo está siempre que no puede tener
a los países católicos bajo su sandalia. El nuestro se las mantiene
tiesas con Roma desde el 68, y por eso el Pontificado ha tenido que
cantar la palinodia, conviniendo un \emph{modus vivendi} con el Gobierno
Castelar para la provisión de las mitras vacantes, que son muchas. Los
jesuitas querían que el Papa nombrase los nuevos obispos arrebatando al
Gobierno el derecho de presentación, y hasta tenían preparada una
hornada de clérigos carcundas para encasquetarles la mitra. Pero Masttai
Ferretti vio que mermaban los chorros del dinero de San Pedro, y acabó
por entenderse bonitamente con la República española. Esto es un éxito
indudable del Gabinete Castelarino, ¿no te parece, querido Tito? Pues
verás qué amarguras y contratiempos le aguardan al bueno de don Emilio.
Salmerón está que echa bombas, y me parece que oigo ya los ruidos
lejanos de la tempestad que se acerca.

Poco después di de manos a boca con Pablito Nougués, que compartía con
Eugenio García Ruiz el fervor unitario. De lo que me contó el
inteligente y simpático periodista, redactor-jefe de \emph{El Pueblo},
deduje que la eterna discordia entre unitarios y federales era por
aquellos días violentísima. La más clara expresión del odio que unos a
otros se tenían es la frase pronunciada por un rabioso Intransigente:
«Entre una República que no sea Federal y la Monarquía, preferimos la
Monarquía.» Este relámpago no fue el último que me deslumbró aquella
tarde en la cálida atmósfera del Congreso.

En diferentes grupos, donde encontré amigos muy queridos, pude oír el
retumbar horrísono de la tempestad que se aproximaba. Salmerón, ya muy
esquinado con el Gobierno, estimando el \emph{Modus Vivendi} episcopal
supremo error y violación del credo republicano, escogió este tema para
cantarle a Castelar el \emph{De profundis} y dar con él en tierra.

Una Comisión de diputados se acercó a don Nicolás, rogándole que
depusiera su actitud contra el Gobierno. Mas no lograron rendir la
tenacidad del filósofo, que condensó su negativa en esta implacable
sentencia: \emph{Sálvense los principios y perezca la República}. Tal
fue el segundo relámpago deslumbrador que me anunciaba el rápido avance
de la tormenta. El espantable fallo del Presidente de las Cortes arrancó
lágrimas a los leales republicanos que más de una vez jugaron su vida en
las conspiraciones y en las barricadas.

No queriendo abandonar el Congreso entre la sesión de la tarde y la de
la noche tomé un piscolabis en la Cantina con Martínez Pacheco,
Castañeda, Olías, Morayta. Éste nos dijo que el voto de gracias al
Gobierno, que presentaron a primera hora de la tarde, se discutía
calurosamente. Castañeda refirió que estando aquella mañana en la casa
de Castelar, calle de Serrano, don Fernando Álvarez, pariente del gran
tribuno, y otros amigos allí presentes, aconsejaron al Presidente del
Poder Ejecutivo que se resolviera a dar el golpe de Estado. Don Emilio
contestó que su honor rechazaba no sólo la idea, sino hasta la frase
\emph{golpe de Estado}, y que a las Cortes iría sin vacilar, afrontando
todo lo que pudiera ocurrir.

Martínez Pacheco, uno de los políticos más ligados al jefe de la
Situación, nos contó sigilosamente que Castelar había conferenciado con
Pavía en el despacho de la Presidencia para informarle de los rumores
por todos oídos de que intentaban sublevarse contra las Cortes
Soberanas. El General lo negó en redondo. Don Emilio entonces le exigió
palabra de honor de que decía verdad. Pavía, dando su palabra, dijo
textualmente: «Jamás, jamás me sublevaré yo ejerciendo mando.» Oído esto
convinimos todos en que no había peligro por aquel lado. Don Manuel
Pavía y Alburquerque, ayudante de Prim, tuvo siempre estrechas
relaciones con los republicanos y era el General que más confianza podía
inspirar a todos.

En la sesión nocturna se fue avivando el debate, no sé si sobre la
proposición de Morayta y Olías o la indispensable de \emph{No ha lugar a
deliberar}. Subí a la tribuna de la Prensa y oí discursos de los
conservadores favorables al Gobierno. Romero Robledo dijo que habiendo
apoyado a Pi y Margall y a Salmerón cuando eran Poder, no podía negar su
voto al Gabinete Castelar. En el propio sentido habló don Agustín
Esteban Collantes, que sintetizó su pensamiento en esta frase feliz: «Si
un regimiento de Granaderos entrase por esas puertas y se hiciese dueño
del Poder, yo sería de los vencidos, ya triunfasen las turbas, ya los
Granaderos\ldots» Relámpago intenso que me hizo cerrar los ojos.

Defendió al Gobierno, entre otros, el eximio catedrático don Francisco
de Paula Canalejas, que fijó la cuestión política en estos precisos
términos: «Si el Ministerio debe caer, es preciso sepamos cuál es la
solución que ha de sustituirle.» Atacaron, sin acritud Benítez de Lugo,
y con sin igual dureza Corchado y Labra, quienes intentaron presentar a
Castelar como sospechoso a los republicanos. No pudiendo formar Gobierno
ningún hato suelto del rebaño parlamentario, se imponía un Gabinete
sintético o de conciliación; pero como era imposible armonizar la
Izquierda con el Centro, y la Derecha con los Intransigentes, resultaba
un embrollo de todos los diablos o un nudo que los dedos más hábiles no
podrían deshacer.

En esto sonó el primer trueno de la ya inminente tempestad. Salmerón,
que había dejado la silla presidencial, soltó en un escaño próximo al
reloj el raudal de su elocuencia altísona y majestuosa. Sus negros ojos
fulgurantes, su lucida estatura y la solemnidad de sus ademanes,
completaban el mágico efecto del orador sobre sus embelesados oyentes.
Mostrose ufano de haber contribuido a formar la Derecha, que definió de
este modo: «Partido eminentemente republicano, esencialmente democrático
en los principios, radical en las reformas, pero conservador en los
procedimientos; partido de paz, de orden, de imperio, de ley, de
autoridad.» A mi lado, los periodistas, comentando estas palabras,
dijeron que la Derecha no la había formado Salmerón con sus
vacilaciones, sino Castelar con su continua propaganda. Don Emilio era
el representante legítimo y autorizado de la Derecha.

Prosiguió el filósofo sosteniendo que Castelar había roto la órbita de
la política conservadora, y trató de probarlo exponiendo vagas
generalidades acerca del Ejército, del partido conservador monárquico,
de reformas administrativas y de economía de los gastos públicos, sin
aludir ni por asomo a la cuestión de los obispos, móvil, según creíamos,
de aquella gran borrasca. Se guardó muy bien de indicar cuáles eran las
economías y reformas administrativas que, según él, debió Castelar
implantar y no lo hizo. Tampoco dijo nada que permitiese apreciar la
diferencia entre las disposiciones referentes al Ejército dictadas por
don Emilio y las que él adoptó en el período de su mando.

Las únicas afirmaciones, por cierto nada tranquilizadoras, del orador
fueron éstas: «Soy inhábil, soy incapaz para el Gobierno mientras las
actuales condiciones no cambien: ni pretendo, ni demando, ni acepto el
Poder. Si no es posible salvar la situación presente dentro de la órbita
del Partido Republicano, antes que romperla nosotros con mano sacrílega,
digámoslo a la faz del país; declaremos que no es posible gobernar con
nuestros principios, con nuestros procedimientos: así quedará nuestra
conciencia tranquila de no haber profanado el Poder, de no haber hollado
nuestras sagradas convicciones.»

Aunque no sonaron fuertes aplausos, las señales de asentimiento que
advertimos en toda la Cámara, nos demostraron que había herido
gravemente al Gobierno el discurso del \emph{filósofo sin realidad},
según la sabida frase castelarina. Había llegado el momento supremo. El
Presidente del Poder Ejecutivo se levantó arrogante, ansioso de mostrar
en aquel torneo la pujanza de su nombre, de su elocuencia y de su honor,
como jefe de la democracia gubernamental.

Empezó su discurso el inmenso tribuno con estos ardientes apóstrofes:
«Soy sospechoso al Partido Republicano porque le digo que él solo no
puede salvar la República; porque le digo que está hondamente dividido y
perturbado; porque le digo la verdad, como se la dije a los Reyes, y
añado que no gobernará como no condene enérgicamente y para siempre a
esa demagogia.» \emph{(Señalando a la extrema izquierda.)}

Fijó luego su significación gubernamental, constante en su vida pública.
Sostuvo que nada hizo en el Gobierno que no hubiera defendido en la
oposición y expuesto en su programa al ser elevado al Poder. Notó los
servicios prestados por él a todos los Gobiernos de la República, de
quienes fue ministerial ardiente aun sin compartir sus opiniones, por no
mermarles autoridad. Luego prosiguió así: «Tenemos todo lo que hemos
predicado. Tenemos la Democracia, tenemos la Libertad, tenemos los
Derechos Individuales, tenemos la República. Dos reformas no más
necesitamos: la primera es la separación de la Iglesia del Estado; la
segunda es la abolición de la esclavitud en Cuba.»

El relampagueo y tronicio continuaban, con fulgores y sonidos más
próximos. Un diputado interrumpió: «¿Y la Federal?» Don Emilio repuso
con acento iracundo: «Eso\ldots{} eso es organización municipal y
provincial. Ya hablaremos más tarde; no merece la pena. ¡El más federal
tiene que aplazarla por diez años!» En los bancos de la Intransigencia
produjeron enorme tumulto las frases del tribuno. Una voz dijo: «¿Y el
proyecto de Constitución?» Castelar lanzó esta respuesta fulminante: «Le
enterrasteis en Cartagena.» \emph{(Sensación profunda en la Cámara y
contradictorias manifestaciones.)}

El Jefe del Gobierno puso término a su discurso con estas palabras: «El
Partido Republicano tiene que transformarse en dos grandes partidos: uno
de acción, progresivo, muy progresivo, a quien le parezcan estrechas y
mezquinas nuestras ideas; y otro pacífico, nada de dictatorial, nada de
autoritario, nada de arbitrario; legal, muy legal, demócrata, muy
demócrata, pero con grandes instintos de consolidación y
conservación\ldots{} Mi política es la natural y podréis maldecirla,
pero no sustituirla, porque ante la guerra no hay más política que la
guerra.»

Sin más dimes y diretes, porque Salmerón no rectificó y las Izquierdas
olfateando su triunfo no quisieron perder el tiempo, se dio por concluso
el debate. ¡A votar, a votar! Derrotado por 120 votos contra 100,
Castelar entregó a la Mesa la dimisión de todo el Gobierno\ldots{}
Aprobose la proposición de costumbre para elegir por papeletas firmadas
un nuevo Ministerio con las mismas facultades conferidas a los
anteriores, y se suspendió la sesión por más de dos horas para que los
diputados se pusieran de acuerdo\ldots{} Bajé de la Tribuna con mis
amigos periodistas, y en los pasillos y Salón de Conferencias oímos
ardorosos comentarios de la votación.

Alguien censuró con acritud a Figueras porque, si personalmente se
abstuvo, ordenó a sus parciales que votaran contra el Gobierno. También
votaron en contra Salmerón y sus adeptos, el Centro, la Izquierda y los
Intransigentes. Al lado de Castelar estuvieron, a más de sus amigos,
seis monárquicos y los Unitarios. Hallándome yo en medio de aquel
laberinto me encontré de improviso en los brazos de Estévanez. «Pero don
Nicolás---le dije,---¿qué es de su vida de usted? No le he visto en los
escaños.» Y él, con semblante triste y voz apagada, me contestó: «No he
venido más que a votar y me largo a escape. Mi suegra acaba de morir.
Adiós.»

Avanzaba la noche. Ya habían caído en las honduras del tiempo pasado las
horas del 2 de Enero de 1874 y entrábamos en la madrugada del 3. La
votación por papeletas se deslizaba lenta, triste, cadenciosa y
somnífera, reproduciendo en los espíritus la pesadez atmosférica de la
tempestad que sobre el Congreso se cernía. En los aires sobrevino el
silencio lúgubre que precede a los grandes estallidos de la
electricidad. No vean mis lectores en esto más que un fenómeno
subjetivo, producto de mi caldeada imaginación. La tempestad no estaba
en los aires sino en la Historia de España.

A una hora que debía de ser molesta para los trasnochadores más
empedernidos, las cinco o las seis de la madrugada, terminó la
parsimoniosa votación para elegir nuevo Gobierno, y se dio comienzo al
escrutinio con prolijos trámites a fin de garantir la más escrupulosa
exactitud. En esto estábamos cuando retumbó sobre nuestras cabezas un
trueno formidable. Retembló el edificio, se estremecieron todos los
corazones, vibraron todos los nervios\ldots{} Subió Salmerón a la
Presidencia y demudado, lívida la faz, centelleantes los ojos, dijo
solemnemente estas fatídicas palabras: «Señores diputados: hace pocos
momentos he recibido un recado u orden del Capitán General de
Madrid---creo que debe ser ex-Capitán General,---quien por medio de sus
ayudantes nos conmina para que desalojemos este local en un término
perentorio.»

\hypertarget{ix}{%
\chapter{IX}\label{ix}}

El rayo corrió por toda la Sala en menos de un segundo, levantando a
muchos de sus asientos, y oyéronse estas voces: «¡Nunca!, ¡nunca!»
Pareciome que en aquella fracción de segundo los pupitres, los divanes,
los candelabros, las luces de gas, las pinturas y adornos, los nombres
grabados en las lápidas conmemorativas y hasta los mudos maceros
gritaban también \emph{¡Nunca!}

Tratando de imponer silencio, Salmerón prosiguió así: «¡Orden, señores
diputados! La calma y la serenidad no deben apartarse de los ánimos
fuertes en circunstancias como ésta\ldots{} Me ha dicho el Capitán
General que si no se desaloja el Congreso en plazo perentorio lo ocupará
a viva fuerza\ldots{} Yo creo que es lo primero y lo que de todo punto
procede\ldots» Espantoso tumulto ahogó la voz del orador. Algunos
vociferaban: «¡Esto es una indignidad, una villanía! ¡Esto es una
traición infame!» El Presidente, en tanto, gritaba con voz estentórea:
«¡Orden, señores diputados, sírvanse oír la voz\ldots!» Continuó el
tumulto con creciente estruendo. Varios Intransigentes, en pie sobre sus
escaños, gesticulaban y decían: «Calma, señores, mucha calma.» Don
Eduardo Chao exclamó: «¡Esto es una cobardía miserable!» Y el filósofo
don Nicolás, reiterando sus exhortaciones, exclamaba a grito herido:
«¡Orden, orden, señores diputados! Vuelvo a recomendar la calma y la
serenidad. Sírvanse oír\ldots» Pero nadie le oyó.

Cuando por agotamiento físico se hizo un poco de silencio, prosiguió
Salmerón: «El Gobierno presidido por el ilustre patricio don Emilio
Castelar es todavía Gobierno y sus disposiciones habrá adoptado ya.
Entre tanto, yo creo que debemos seguir en sesión permanente, y seremos
fuertes para resistir hasta que nos desalojen por la violencia dando un
espectáculo que, aun cuando no sepan apreciarlo en lo que vale aquellos
que sólo pueden conseguir el triunfo por ciertos medios, las
generaciones futuras sabrán que los que éramos adversarios ahora hemos
estado unidos para defender la República.» Varios padres de la Patria
exclamaron: \emph{¡Todos!} \emph{¡Todos!} Y el Presidente contestó: «No
esperaba yo menos, señores diputados: ahora seremos todos unos.»

En los escaños retumbó el estruendoso clamor de \emph{¡Todos somos
unos!} \emph{¡Todos somos unos para defender la República!} Al oír esto
no pude contenerme. Se me encendió la sangre, y con toda la fuerza de
mis pulmones lancé al hemiciclo estas palabras: «¡A buenas horas mangas
verdes! Majaderos fuisteis; sed ahora ciudadanos y dejaos matar en
vuestros asientos.» En el espantoso vocerío perdiéronse mis apóstrofes.
Muchos diputados daban vivas a la Soberanía Nacional, a la Asamblea y a
la República. Salmerón echó el resto de su potente voz con estas frases
rotundas: «Se han borrado en este momento todas las diferencias que nos
separaban. Borradas estarán hasta tanto que no quede reintegrada esta
Cámara en la representación de la Soberanía Nacional\ldots» Otra vez,
sintiéndome coro, grité burlescamente: \emph{«¡Tarde piache!»} Mi
comentario familiar quedó ahogado en el estrépito de los aplausos que
corearon la vibrante protesta del gran metafísico.

Tocó la vez a Castelar, que dijo: «Yo creo que la sesión debe seguir
como si no sucediese nada fuera de esta Cámara. Puesto que aquí tenemos
libertad de acción, continuemos el escrutinio, sin que por eso el
Presidente del Poder Ejecutivo tenga que rehuir ninguna responsabilidad.
Yo he reorganizado el Ejército; pero lo he reorganizado no para volverse
contra la legalidad, sino para mantenerla.» Frenéticos aplausos
interrumpieron al colosal tribuno, que terminó de esta manera: «Ya,
señores diputados, no puedo hacer otra cosa que morir el primero con
vosotros.» \emph{(Inmensa emoción. Muchos se abalanzaron a abrazarle.)}

Don Eduardo Benot se puso en pie, y rojo de ira gritó: «¿Hay armas?
Vengan. ¡Nos defenderemos!»

\emph{Salmerón:} Sería inútil nuestra defensa y empeoraríamos nuestra
causa.

\emph{Una voz:} ¡Quia; ya no se puede empeorar!

\emph{Salmerón:} Digo que nosotros nos defenderemos con aquellas armas
que son las más poderosas en estos momentos: las de nuestro derecho, las
de nuestra dignidad, las de nuestra resignación para recibir semejantes
ultrajes.

\emph{Castelar:} Pero una cosa hay que hacer\ldots{}

\emph{Un diputado:} ¡Que se dé un Voto de confianza al Ministerio que ha
dimitido!

\emph{Castelar:} De ninguna manera; aunque la Cámara lo acordase, este
Gobierno no puede ser Gobierno, para que no se dijera nunca que había
sido impuesto por el temor de las armas a una Asamblea Soberana. Lo que
está pasando me inhabilita a mí perpetuamente para el Poder.

\emph{Varios diputados:} ¡No, no, que te creemos leal!

\emph{Castelar:} Así es, señores diputados, y a mí me toca demostrar que
yo no podía tener alguna parte en esto. Aquí, con vosotros los que
esperéis, moriré y moriremos todos.

\emph{Benot:} Morir, no: vencer.

\emph{Chao:} Ruego, señores diputados, que se expida un Decreto
declarando fuera de la ley al General Pavía, sujetándole a un Consejo de
guerra\ldots{} y si es necesario desligando a sus soldados del deber de
la obediencia.

\emph{Fernández Castañeda:} ¡Farsa! ¡Qué Decreto ni qué garambainas! Si
no disponemos ni de un cabo y cuatro soldados para que nos defiendan
¿cómo vamos a exonerar a nadie?

\emph{(Sánchez Bregua extiende y firma el Decreto. Varios diputados
solicitan ser ellos quienes lo entreguen a Pavía.)}

\emph{Calvo y Delgado:} \emph{(Despavorido. Penetrando en el Salón.)} La
Guardia Civil entra en el edificio, pregunta a los porteros la dirección
de esta Sala, y dice que se desaloje en el acto, de orden del Capitán
General.

\emph{Benítez de Lugo:} Que entre, y todo el mundo a sus asientos.

\emph{Salmerón:} Ruego que sólo esté en pie el señor diputado que se
halle en el uso de la palabra.

\emph{Benítez de Lugo:} Yo que en esta misma sesión he consumido un
turno contra la política del señor Castelar, pido que en este momento la
Cámara entera le dé un Voto de Confianza.

\emph{Castelar:} Ya no tendría fuerza y no me obedecerían.

\emph{Salmerón:} No tenemos más remedio que sucumbir ante la violencia,
pero ocupando cada cual su puesto. Vienen aquí y nos desalojan.
¿Acuerdan los señores diputados que debemos resistir? ¿Nos dejamos matar
en nuestros asientos?

\emph{Muchas voces:} ¡Sí, sí, todos!

\emph{(Algunos padres de la Patria desfilan silenciosos hacia las
puertas altas que dan al pasillo curvo.)}

\emph{Castelar:} Señor Presidente. Yo estoy en mi puesto y nadie me
arrancará de él. Yo declaro que aquí me quedo y que aquí moriré.

\emph{Un diputado:} ¡Ya entra la fuerza en el Salón!

\emph{Unos:} ¡Qué vergüenza!

\emph{Otros:} ¡Qué escándalo!

\emph{Varios:} Soldados: ¡Viva la República Federal! ¡Viva la Asamblea
Soberana!

Aparecieron por la puerta de la izquierda soldados con armas. Su aire
era tímido, receloso. En su actitud se conocía que traían orden de no
hacer daño. La grandeza del Salón, la muchedumbre de personas, las voces
airadas, les mantuvieron un instante en cierta perplejidad\ldots{}
¡Pobres hijos de España! ¡Y os sacaron de vuestros hogares para consumar
tal crimen!\ldots{} Algunos diputados se abalanzaron hacia la tropa,
agrediéndola con sus bastones y tratando de desarmarla. Entre aquel
torbellino se abrió paso el Coronel de la Guardia Civil, señor Iglesias,
alto, viejo, de blanco bigote y aire muy militar. Tricornio en mano
subió a la Presidencia y habló con Salmerón. Tanta gente se arremolinaba
en el alto estrado, que no pude distinguir la actitud de don Nicolás
ante el embajador de la fuerza bruta. Diputados, ujieres, taquígrafos,
se entremezclaban y corrían de un lado para otro en espantosa confusión.
Sólo permanecían en sus puestos, rígidos y mudos, los maceros, como esos
heraldos de piedra que decoran los regios sepulcros.

En esto sonó en los pasillos un tiro. Luego otro y otros\ldots{}
Terrible pánico. Por la puerta de la derecha salieron del Salón de
Sesiones muchos diputados: unos para evadirse lindamente; otros para ver
lo que ocurría entre la calle y el Salón de Sesiones. A escape bajé yo
de la Tribuna. En el pasillo de la Orden del Día vi que la tropa se
limitaba a indicar con la mano a los padres de la Patria la puerta de
salida. Algunos de los que habían jurado dejarse matar dentro del
Congreso antes de rendirse al imperio de la fuerza, recogieron sus
prendas de abrigo en el guardarropa y ganaron cabizbajos y silenciosos
la calle de Floridablanca. En cambio, los más exaltados trataban de
imponerse a los militares con razones iracundas y argumentos
contundentes.

Allí presencié una escena, que refiero para que se vea que la elevación
de sentimientos no dejó de manifestarse en los incidentes de aquella
memorable escena histórica. Emigdio Santamaría, hombre fornido, corto de
talla pero de fuerza hercúlea, arrebató su fusil a un sargento de
Infantería, en el pasillo circular. Consternado y casi lloroso quedó el
pobre sargento, considerándose sin honra por verse inerme e indefenso.
Como ya he dicho, tanto él como sus compañeros tenían orden de no
agredir a ningún diputado\ldots{} En esto intervino Antonio Fernández
Castañeda, representante de Santander en aquellas Cortes, el cual disipó
la ira acometedora de Santamaría con estos conceptos de Patria y
Humanidad que fielmente copio: «Amigo Emigdio, no tenemos medios hábiles
para sostener nuestro derecho. Tristísimo es decirlo, pero ya no hay
para nosotros más recurso que salir y callar, esperando el fallo de la
Historia. Lo que usted hace es una locura sin más consecuencia que
perjudicar a este pobre muchacho. ¡Devuélvale usted su fusil!» Emigdio
Santamaría, apagando los últimos resoplidos de su furia, entregó el arma
al sargento, que, con voz empañada por la emoción, dijo: «Gracias,
gracias, caballero.»

No era ésta la única prueba que de su comedimiento y claro juicio dieron
los buenos \emph{chicos del Ejército}. Obedecían a los autores de
aquella infamia sin desconocer que escarnecían a la Patria y pisoteaban
las Leyes.

Colándome en el Salón de Sesiones vi a don Nicolás ponerse el sombrero y
descender pausadamente de la Presidencia, seguido de los graves maceros.
En el banco azul, Castelar, con semblante dolorido y actitud de suprema
consternación, permanecía en su sitio como un estoico que apura el
cumplimiento del deber hasta el último instante. Rodeábanle sus amigos
más adictos y cariñosos. Dirigí una mirada al hemiciclo, y la soledad de
los escaños me dio la impresión del hielo de la muerte. Lucían los
mecheros de gas como funerarias antorchas\ldots{} Ya iban palideciendo
ante la claridad tenue del alba que por la claraboya cenital tímidamente
penetraba\ldots{}

Por fin, los fieles adeptos del gran tribuno consiguieron arrancarle de
su asiento, y sacarle de la \emph{Cámara ardiente} al pasillo. Abrieron
paso respetuosos los militares\ldots{} La que podríamos llamar procesión
de duelo se dirigió hacia la escalera y salida de la calle del Florín.
Seguí yo detrás, atraído por la solemnidad del suceso y por la figura de
\emph{Mariclío}, que creí distinguir junto a la persona triste y
agobiada del héroe vencido, Emilio Castelar.

En la calle, dudando yo si era real o imaginaria la presencia de la
excelsa Madre, acerqueme a ella. Iba vestida de negro, con la toca y
monjil que usaron las reinas viudas y las dueñas ricas, traje con que la
iconografía religiosa viste a Nuestra Señora de los Dolores. Suavemente
me dijo: «Vete a recorrer las calles que rodean a esta Casa profanada;
fíjate en las tropas que han acudido a consumar la fácil y criminal
hazaña. Repara bien dónde está \emph{el Pavía}, que verás a caballo,
rodeado de bayonetas y cañones, y de toda la máquina marcial hoy
dispuesta para matar mosquitos. Di a tus amigos los republicanos que
lloren sus yerros y procuren enmendarlos para cuando la rueda histórica
les traiga por segunda vez al punto de\ldots»

---Al punto de\ldots---repetí yo;---y al sonido de mi voz, como si ésta
fuera el canto del gallo que despide a las almas del otro mundo, la
Madre mil veces augusta desapareció de mi vista\ldots{} Corrí en
seguimiento de la comitiva de Castelar, y cuando ésta doblaba la esquina
de la calle del Sordo, una mano invisible me empujó hacia la plaza de
las Cortes.

La conciencia de mis deberes, como emborronador de páginas históricas,
me llevó a revistar las fuerzas apostadas a lo largo del palacio de
Medinaceli, calles de Floridablanca, Greda, Turco y Alcalá, hasta el
Ministerio de la Guerra. Allí, junto al jardín de Buenavista, vi a Pavía
y Alburquerque, rodeado de un Estado Mayor no menos nutrido y brillante
que el de Napoleón en la batalla de Austerlitz. Ya era día claro, aunque
nebuloso, tristísimo y glacial. Todo lo que pasó ante mis ojos, desde
los comienzos del escrutinio hasta mi salida del Congreso, se me
presentó con un carácter y matiz enteramente cómicos. Pensaba yo que en
las grandes crisis de las naciones, la tragedia debe ser tragedia, no
comedia desabrida y fácil en la que se sustituye la sangre con agua y
azucarillos. El grave mal de nuestra Patria es que aquí la paz y la
guerra son igualmente deslavazadas y sosainas. Nos peleamos por un
ideal, y vencedores y vencidos nos curamos las heridas del amor propio
con emplasto de arreglitos, y anodinas recetas para concertar nuevas
amistades y seguir viviendo en octaviana mansedumbre. En aquel día
tonto, el Parlamento y el pueblo fueron dos malos cómicos que no sabían
su papel, y el Ejército, suplantó, con sólo cuatro tiros al aire, la
voluntad de la Patria dormida.

Al volver hacia el Congreso decía yo para mi sayo, mirando al porvenir:
«Republicanos condenados hoy a larguísima noche: cuando veáis amanecer
vuestro día, sed astutos y trágicos.» En la calle del Turco me encontré
con Juanito Valero de Tornos, que siguió junto a mí, refiriéndome
detalles curiosos observados por él en las postrimerías del Parlamento
de la República. «Puedo asegurarte, querido Tito---me decía,---que el
truculento General Sánchez Bregua, en el azoramiento de su retirada
forzosa, se dejó olvidada la chistera en el Banco Azul. Yo no lo vi; me
lo contó Bernardo García, y lo tengo por exacto. De otro Ministro sé que
buscó refugio en las habitaciones altas, donde vive el Mayor, y allí
estuvo aguardando a que terminase la degollina\ldots{}

»Muchos diputados se agazaparon en las oficinas del \emph{Diario de las
Sesiones}, y por una ventana salieron a Floridablanca. Por la puerta que
da a la misma calle se escabulleron cantando bajito los que más habían
alborotado en los pasillos, queriendo desarmar a la tropa: eran Olías,
Casalduero, Díaz Quintero, el Marqués de la Florida y otros. Antonio
Orense dirigió algunas palabras enérgicas a los civiles que custodiaban
la puerta; pero éstos no le hicieron caso, y siguió su camino.

»Yo vi a don Nicolás Salmerón salir con el cuello del gabán levantado, y
tapándose la boca con un pañuelo. Le acompañaban Carratalá y Montero
Rodríguez, embozados en sus capas hasta los ojos\ldots{} Me consta
porque lo he visto, que León y Castillo, Antonio Matos, y Merelles, de
acuerdo con los conjurados, hacían frecuentes viajes del Congreso a
Buenavista para informar al General Pavía del momento preciso en que
debía dar el golpe. Ellos fueron los transmisores del estado agónico de
la pobre República. El Capitán General de Madrid no se puso en
movimiento hasta que supo que la enferma estaba dando las boqueadas.»

Anoto los informes de Juanito Valero, descontando de ellos el agridulce
que aquel ingenioso amigo ponía siempre en sus referencias políticas.
Como buen conservador y alfonsino, no perdía ripio para zaherir y
rebajar los caracteres de la gran familia republicano-democrática.

Cansado de correr en tonto por las calles, donde no veía más que tropas
fríamente alineadas e inactivas, sin ver asomar por ninguna parte la
cara iracunda del pueblo; asqueado del indigno suceso histórico que
llegó al brutal \emph{consummatum} sin dignidad por la parte ofendida ni
arrogancia por parte de los asesinos de la República, me fui a mi casa
con la esperanza de que un sueño profundo ahogara mi desaliento
tristísimo y dulcificase mi amargura. Pero mis nervios se opusieron
fieramente a que yo durmiera.

Hablé un rato con \emph{Chilivistra}, la cual, compuesta ya y vestida
con su hábito de los Dolores, me contó el sueño que había tenido aquella
madrugada. Soñó la pobre señora que don Carlos triunfante venía sobre
Madrid con poderosa hueste. Yo la tranquilicé diciéndole que la toma de
Madrid por el \emph{Niño Terso} no estaba tan próxima como ella había
visto en sueños.

Acompañé a mi dama hasta el oratorio del Olivar, y me fui a visitar a
Estévanez. En las calles no advertí el menor síntoma de inquietud ni
emoción por lo que había pasado en las Cortes. El vecindario se hallaba
tranquilo, las tiendas abiertas y todo el mundo en las ocupaciones
habituales de cada día. La casa de mi amigo don Nicolás estaba de duelo;
la madre política de cuerpo presente. No quise pasar, y aplacé mi visita
para el siguiente día\ldots{} Volví a divagar por la vía pública. En la
plaza del Ángel me encontré a Pepe Ferreras, con quien hablé de la
increíble tranquilidad que notaba en la población.

«Fíjese usted bien---me dijo el agudo periodista,---y notará más que
tranquilidad, alegría\ldots{} ¿Se asombra usted, querido Tito?\ldots{}
Aquí producen siempre regocijo los cambios de Gobierno, sobre todo
cuando son radicales y hay que mover todos los títeres. La mitad de las
personas que pasan a nuestro lado son cesantes que aguardan la formación
del nuevo Gobierno para pedir que los repongan. Esta situación hará un
desmoche tremendo\ldots{} Notará usted también que en las tiendas reina
cierto alborozo. Los tenderos salen a la puerta creyendo oír ya el voceo
de los extraordinarios de periódicos \emph{con el nuevo
Ministerio\ldots{}} Madrid se anima, el comercio se despereza, la
industria renace de sus cenizas como el Ave Fénix, los negocios se
desentumecen, y ya mañana las criadas irán a la compra con más dinero
del que suelen llevar a diario.»

Entramos en una sastrería, de cuyo dueño era Ferreras muy amigo. El
escuálido sastre, apenas le preguntamos su parecer sobre el cambio
político, nos dijo con semblante de júbilo: «Pues nada, señor don José y
la compañía, que estamos de enhorabuena; toda la calle lo está. El
cambio parece de esos que todo lo ponen al revés. Nos hallamos abocados
a una zafra que ha de ser magnífica y provechosa. Algo me ha de tocar a
mí de los encargos que han de caer sobre la sastrería de Madrid\ldots{}

»Antes de media semana habrá que tomar medidas para las 49 levitas de
los 49 gobernadores nuevos. De pantalones y chalecos negros, de ternos
de lanilla, tendremos tantísimos encargos que será fácil nos quedemos
sin género catalán, de ese que llamamos inglés. En el ramo de capas, que
es mi especialidad, espero que la cosecha será de las no vistas, pues el
invierno crudo y la crisis honda se han puesto de acuerdo para que la
gente tenga que abrigarse.

»Ya era tiempo, señor don José, pues en esta \emph{crujida} de la
República lo íbamos pasando muy mal. Los republicanos son muy buenos
chicos; pero con sus grescas escandalosas, su Pacto, sus Cantones, y la
maldita y arrastrada Igualdad, no traen más que hambre y mala ropa. Mis
compañeros y yo vivimos de vestir a los españoles. ¡Lucidos estaríamos
si nuestro negocio dependiera del lujo que gastan los
\emph{descamisados!»}

Nos despedimos del sastre. De madrugada había yo visto cómo se
empequeñecían las cosas grandes; acababa de ver cómo crecía y se
hinchaba lo infinitamente pequeño.

\hypertarget{x}{%
\chapter{X}\label{x}}

Después de enterarnos mi amigo Ferreras y yo del júbilo de los
sombrereros (que en tiempos de República el armatoste llamado
\emph{chistera} iba muy en desuso), entramos en el café de La Iberia,
donde tuvimos el feliz encuentro del bondadoso Llano y Persi, que nos
convidó a almorzar. Eran las doce. En el Congreso estaban reunidos el
Duque de la Torre, Cánovas, Sagasta, Martos, Becerra y algunos santones
más, civiles y militares, amasando el pastelón del nuevo Ministerio para
meterlo en el horno. Cánovas dijo que si no se proclamaba en el acto Rey
de España al Príncipe Alfonso, debía declararse por lo menos abolida y
conclusa la forma republicana. A esto no accedieron los altos
reposteros, y continuaron trabajando el hojaldre para darle una pronta
cochura y servirlo al país.

Ferreras, que era un águila para las indagaciones políticas, difirió por
un rato el almuerzo y se fue al profano Templo de las Leyes, de donde
volvió al cuarto de hora trayéndonos los nombres del nuevo Gabinete,
trazados por él con lápiz en un papelejo. Ante los amigos que formábamos
corrillo en dos mesas próximas leyó la esperada y emocionante lista, que
reproduzco para conocimiento de los papanatas del tiempo venidero:

Presidente del Poder Ejecutivo, General Serrano.---Presidente del
Gobierno y Ministro de la Guerra, General Zabala.---Estado,
Sagasta.---Marina, Topete. ---Hacienda, Echegaray.---Gobernación, García
Ruiz.---Gracia y Justicia, Martos. ---Fomento, Mosquera.---Ultramar,
Balaguer\ldots{} Almorzamos alegremente, y allí fue el acumular cálculos
sobre la vitalidad de la nueva Situación, sobre el atropellado asalto de
puestos oficiales y demás preparativos de la pública merienda
burocrática que se aproximaba. Llano y Persi nos contó que, cuando
Castelar iba del Congreso a su casa rodeado de amigos, a las siete y
media de la mañana, se le presentó un ayudante de Pavía, rogándole de
parte del General que continuase al frente del Gobierno. Don Emilio
contestó con frase desvergonzada, única respuesta que a tal ultraje
correspondía, y prosiguió inalterable y firme su retirada dolorosa.

Gratísima era la tertulia de La Iberia, donde se oían opiniones y
comentos dignos de ser grabados en los mármoles y bronces de nuestra
inmortal chismografía política. Pero yo, muerto de cansancio por no
haber pegado los ojos la noche anterior, me fui a mi casa, a punto que
atronaban las calles los voceadores de la \emph{Lista del nuevo
Ministerio\ldots{}} Tendido en mi cama y contagiado de la soñación de mi
vecina \emph{Chilivistra}, soñé que era yo sastre, y que estaba cortando
las 49 levitas para los 49 flamantes gobernadores de provincia. Luego
cambió el tema de mis cerebrales aberraciones, y soñé que la dolorida
dama se despojaba de su hábito negro para arrojarse en mis brazos
amantes. Por último, andando ya la noche, me atormentó la visión o
pesadilla del caso del \emph{Virginius}, que fue uno de los temas
tocados en la tertulia del café.

Dicha nave, arbolando bandera americana, fue apresada en aguas de
Jamaica por nuestra goleta \emph{Tornado}. Llevaba gran número de
filibusteros, norteamericanos, ingleses y españoles, dispuestos a
desembarcar en la Gran Antilla para favorecer la guerra contra España.
Conducidos a Santiago de Cuba los tripulantes y pasajeros del barco
insurgente, fueron fusilados la mayor parte de ellos, contraviniendo las
órdenes de Castelar al Capitán General Jovellar para que no se aplicara
la pena de muerte sin dar antes cuenta al Gobierno de Madrid. Ante la
horrenda tragedia de Santiago de Cuba, desperté en mi cama dando gritos
atroces: «¡Teneos, bárbaros! ¡No fusiléis!\ldots{} ¡A mí!\ldots{}
¡Socorro!\ldots{} ¡Clemencia!\ldots»

A mis voces acudió Ido del Sagrario en paños menores, alumbrado de un
candilejo, y me dijo: «¿Qué es eso, señor don Tito? ¿Qué le pasa?»

---Que están fusilando a los del \emph{Virginius}---repliqué yo
sentándome al borde del lecho.---Los tiros me han dejado sordo.

---¿Pero está usted en Babia?---murmuró mi patrón tembliqueando de
frío.---Lo del \emph{Virginius} está arreglado hace ya la mar de días,
según dijeron los papeles.

---No, no---exclamé yo lanzándome en pernetas a recorrer la estancia
.---En este cuarto estaban conferenciando ahora Castelar y míster
Sickles. Todavía estoy oyendo el traqueteo de la pata de palo que gasta
el Ministro de los Estados Unidos. De aquí pasó don Emilio al cuarto de
usted. Bien claro dijeron que es inevitable la guerra con la República
Norteamericana. ¡Jesús, qué calamidad! ¡Jesús, qué desastre! ¡Pobre
país, pobre España!

Con no poco esfuerzo me tranquilizó Ido, haciéndome volver a mi
camastro. La cuestión del \emph{Virginius} era ya cosa vieja. Castelar y
el cojo Sickles arregláronla con los bartolillos y bizcochos borrachos
que usa la diplomacia\ldots{}

El día siguiente, 4, lo pasé casi todo con Nicolás Estévanez. Embozados
en nuestras capitas nos fuimos a divagar por las calles, observando la
fisonomía y estado moral de esta compleja Villa. Hallábase el hombre en
un grado tal de desaliento y tristeza, que me fue imposible calmarle con
mis excitaciones a la paciencia filosófica. La inhibición del pueblo
ante el criminal golpe de Estado le ponía fuera de sí\ldots{} Más de una
vez le oí pronunciar estas frases que copio \emph{ad pedem literæ}:
\emph{Lo de ayer ha sido una increíble vergüenza\ldots{}} \emph{Todos
nos hemos portado como unos indecentes\ldots{}} Visitamos a no pocos
jefes y oficiales de la Milicia Nacional, para ver si los gorros
colorados se decidían a intentar un supremo esfuerzo. A todos les
encontramos indecisos y como atontados. Francisco Berenguer \emph{(el
Quito)} fue el único que, como siempre, se mostró resuelto a cualquier
barbaridad. Era popularísimo en la Latina y disponía de bastante gente.

Antes de tomar una resolución en asunto tan arriesgado, quiso Estévanez
ver a Salmerón, y allá nos fuimos. Dejele en la puerta de la casa y
quedé en esperarle en el café de Lepanto. A la media hora volvió el
infatigable republicano, diciéndome: «Farsa, farsa; no podemos hacer
nada. Salmerón ha recibido un mensaje de Moriones. El General en Jefe
del Ejército del Norte declara que no está dispuesto a reconocer el
Gobierno formado por Pavía. Pero encarga que no nos movamos para no
hacer fracasar sus intentos, y exige que se pongan de acuerdo los
desavenidos Salmerón, Pi, Figueras y Castelar\ldots{} Esto está perdido.
Cantemos a nuestra pobre República el debido responso.»

Pasados unos días me enteré de que las únicas poblaciones que
protestaron decorosamente contra el golpe de Estado fueron Valladolid,
Zaragoza y Barcelona. En la capital castellana pusieron sobre las armas
los Voluntarios de la República. El famoso General don Eulogio González
Iscar, familiarmente llamado \emph{Gonzalón} por su extremada
corpulencia, salió a calmar los ánimos. El gentío le acosó, rechazándole
con ultrajes; mas aunque amenazaba con fusilar a los revoltosos nada
hizo. El ruidoso motín, con sus incipientes barricadas, fue derivando
hacia la tibieza y por fin hacia la paz, convencidos los republicanos de
que la cosa no tenía remedio. En Zaragoza ocurrieron tentativas y
desmayos semejantes. En Barcelona los Batallones Catalanes que mandaba
el \emph{Xic de las Barraquetas}, armaron un cisco que dominó fácilmente
la tropa de la guarnición. El pueblo más deshonrado en aquellas vegadas
fue nuestro querido Madrid, dándonos el mal ejemplo de una resignación
musulmana. \emph{Estaba escrito} que las crisis políticas resolvían las
crisis del pequeño comercio y remediaban el hambre atrasada de sastres,
sombrereros, zapateros y patronas de huéspedes.

Una mañana llamó a la puerta de mi casa la \emph{Leona} cartagenera. No
tuve el gusto de recibirla porque el señor de Ido, oficioso y pudibundo,
conociendo por el trapío de la moza que ésta era de cuidado, le dijo que
yo estaba ausente y que hasta la noche no volvería. Pasado un cuarto de
hora salí a la calle y me la encontré en el portal: \emph{La Brava},
ducha ya en las mentiras cortesanas, había conocido el ardid de mi
filosófico patrón. Ella y yo nos alegramos de vernos, y apenas nos
saludamos hice propósito de acompañarla hasta su casa. Cuando pasábamos
juntitos a la acera de enfrente miré a mis balcones, y en uno de ellos
vi a \emph{Chilivistra} que nos \emph{guipaba} cautelosa y un tanto
ceñuda.

En el camino hacia la calle de la Victoria, donde Leonarda me dijo que
vivía, advertí que la mujer alegre no había perdido el tiempo en la obra
ciertamente admirable de su metamorfosis. En diez días de Madrid iba
vestida con traje flamante a la moda, y en lo referente a la adquisición
de palabras finas, sus progresos me colmaron de asombro. Ya sabía decir
\emph{hecatombe}, \emph{el punto de vista}, \emph{miel sobre hojuelas},
y otras majaderías usuales. Lo primero que me contó fue que el caballero
\emph{pagano} con quien llegó a Madrid le había servido de mucho para
orientarla en su nueva vida. Pero aquél tomó las de Villadiego, y ella
anduvo algunos días un poquito aperreada\ldots{} Después había tenido la
suerte de que le saliera un señorón muy bueno, que sólo con verla se
enamoró de ella como un colegial.

Parándose en medio de la calle, para hablarme con más reposo, \emph{La
Brava} continuó así su historia: «Mi señor es un personaje de la
Situación \emph{que acaba de salir ahora}, y está tan loco por mí que me
llama su tipo y otra cosa muy bonita\ldots{} a ver si me acuerdo\ldots{}
sí, eso es\ldots{} su ideal\ldots{} El nombre, Tito, no puedo decírtelo,
porque él es casado y\ldots{} debe una tener delicadeza y mirar \emph{el
punto de vista} de la familia y la sociedad\ldots{} Le han dado un
destino muy gordo\ldots{} Creo que cincuenta mil reales y manos
libres\ldots{} Ya le están haciendo un uniforme bordado y un sombrerote
con plumas, y todo esto, con el espadín y una banda amarilla, le sale
por más de diez mil reales. A mí me ha regalado este vestido. Ya
comprenderás que es rico\ldots{} rico por su mujer,
\emph{naturalmente.»}

Vivía \emph{Leona} en una casa equívoca. Al entrar con ella en su
habitación no vi más que a una mujer frescachona que me saludó con
amabilidad tan equívoca como la vivienda. Seguimos nuestra conversación
\emph{La Brava} y yo hablando de Cartagena y de las trifulcas que allí
dejamos. Mi amiga me dijo con viveza: «¿Pero no sabes?\ldots{} Si
tenemos aquí a la Ramira\ldots{} ¿No te acuerdas de la Ramira, una que
iba conmigo la noche que te acompañamos hasta la plaza de las
Monjas?\ldots{} Pues llegó ayer con un chico del ferrocarril\ldots{} En
casa está: voy a llamarla para que te cuente.» Salió un momento, y al
poco rato volvió acompañada de su amiga, que era menudita y graciosa.
«Siéntate aquí, Ramira---dijo \emph{Leona},---y cuéntale a don Tito el
incendio de la fragata. Verás, hijo, verás qué \emph{hecatombe.»}

«Pues señor---empezó diciendo la narradora;---la fragata \emph{Tetuán}
se ha quemado hace unos días. A las ocho de la noche comenzó el fuego, y
a la media hora las llamas llegaban al cielo. Era un espanto. Los que
estaban a bordo tuvieron que salvarse tirándose de cabeza a las lanchas.
Decían que si el incendio había sido por las estopas o por los
estopines. Los cañones se disparaban solos. La autoridad mandó que nadie
se acercase. La ciudad estaba aterrorizada. A media noche reventó la
santabárbara: la cubierta voló por los aires, hasta llegar a las
estrellas; se hicieron cisco los palos, el cordaje, cuanto a bordo
había, y el casco se fue a pique\ldots{} ¡Ay, Dios mío! ¡Los cristales
que se rompieron aquella noche cuando el retemblido!\ldots{} Puertas y
ventanas hubo que de la sacudida se arrancaron de por sí, saliéndose de
sus marcos.»

---Y fue milagro que no hubiera otras \emph{hecatombes}---añadió
Leonarda.---Según dice ésta, la \emph{Numancia}, que a la vera estaba de
la \emph{Tetuán}, tenía en las bodegas cuatro mil quintales de pólvora,
que hizo sacar del Parque tu amigo Cárceles porque contra el polvorín
tiraba siempre la tropa del Gobierno.

---Mientras duró el fuego de la \emph{Tetuán}---prosiguió
Ramira,---Cartagena estaba como en fiestas con luminarias. Toda la gente
se echó a la calle, y se veía lo mismito que en día claro. Los del
Gobierno no disparaban. Los de dentro hacían catálogos y calculorios
sobre el porqué del siniestro. Unos decían que el barco se quemó
\emph{de su motivo}; otros que había sido por mano de los que se fingen
amigos y son traidores. Lo cierto fue que cuando los fogoneros de la
\emph{Tetuán} vinieron a tierra los encerraron en el Presidio y se les
formó causa\ldots{} En cuantico que voló el barco y Cartagena se quedó a
obscuras, los de \emph{López Mínguez} arrearon de firme otra vez a
cañonazo limpio contra la pobre ciudad. Habíamos pasado de un infierno
con llamas a un infierno entre tinieblas.

Con esto puso fin a su relato la Ramira, porque ignoraba lo que después
de su salida del pueblo había pasado. Quiso \emph{Leona} invitarme a
almorzar, mas yo la convidé a ella, mandando traer dos cubiertos del
café del Pasaje. Informado por mi amiga de que su respetable adorador no
la visitaría en toda la tarde, permanecí junto a ella muy a gusto hasta
después de anochecido, admirando sus considerables adelantos en el arte
de hablar finamente y en otras preciosas y sutiles artes.

Cuando volví a mi casa, ¡ay de mí!, encontré a \emph{Chilivistra} con
unos morros de a cuarta que deslucían y afeaban su bello rostro. Mis
galanterías delicadas no lograron arrancar la máscara de su desapacible
seriedad. A fuerza de ruegos y arrumacos, pude oír de sus labios estas
amargas explicaciones: «Ya me he convencido, señor don Tito, de que no
debo confiar en el que se ofreció a prestarme auxilio con alma y vida en
mis tribulaciones. Permítame decirle que \emph{acción fea} es abandonar
a una dama en momentos de prueba, yéndose de paseo con una trotacalles
indecente.»

Iba yo a contestarle cuando me quitó la palabra de la boca para seguir
despotricando de esta manera: «¿A quién volverme ahora? ¿Con qué brazo
fuerte, con qué corazón generoso podré contar?»

---Con el mío, señora---exclamé, echando el resto de mis pelendengues
declamatorios y de mi hábil trasteo persuasivo. La domé, la convencí,
jurando y perjurando que la \emph{pelandusca} vino a pedirme socorro y
que sólo fui con ella hasta doblar la esquina de la calle de las
Huertas, desde donde marché al Ministerio de la Guerra. Con mohín
remilgado y pucheritos graciosos me contestó Silvestra lo que a la letra
copio: «¡Ay, Tito, Tito; no sabe usted cuán lacerado está hoy mi
corazón! Esta mañana, cuando volví del Oratorio, me dejó usted con la
palabra en la boca al intentar decirle\ldots»

---¿Qué, señora, qué?

---Allá voy. Tenga usted calma\ldots{} Pues mi confesor\ldots{} no, no,
me equivoco\ldots{} no fue mi confesor, fue el padre Carapucheta, Rector
del Oratorio, quien me aseguró que mi marido ha sido puesto en libertad
hace unos días\ldots{} Y usted que es el hombre del gran poder, usted
que todo lo arregla con una cartita ¿resulta que ahora no sabe una
palabra de esto?

---Perdone, señora. Se lo dice usted todo y no me deja meter
baza\ldots{} ¿Pues a qué fui yo hoy al Ministerio de la Guerra? ¿Qué me
dijo el Subsecretario?\ldots{} Me dijo, en nombre de mi amigo el General
don Juan de Zabala, que, atendida como siempre mi recomendación, había
sido indultado el capitán carlista Gabino Zuricalday. Eh\ldots{} ¿qué
tal?

---Está bien; pero aún no sabe usted lo mejor, quiero decir, lo peor. El
padre Carapucheta, que es hombre a quien no se le escapa nada de lo que
ocurre entre carlistas \emph{buenas} y \emph{malas} y tiene allá sin fin
de espías que le cuentan todo, me ha enterado de que Gabino, en cuanto
pescó \emph{la indulta}, se fue a mi pueblo, cogió al nuestro hijo y se
largó con él a la frontera de Francia, donde estará en espera de que don
Carlos le dé el mando de \emph{otro batallona}.

---Todo eso, Silvestra carísima---afirmé yo poniendo en mi rostro una
calma seráfica,---no es para que cojamos el cielo con las manos.
Serenidad, amiga mía. Lo primero es inquirir por ese clérigo Carapucheta
el lugar donde Zuricalday se encuentra, y seguirle los pasos hasta que
se agregue de nuevo al Ejército de don Carlos.

\emph{Chilivistra}, levantándose airosa y extendiendo hacia mí su brazo,
me dijo con rígida solemnidad: «¿Y podré yo contar, pobre mujer sola y
sin amparo, con un caballero hidalgo y valeroso que me asista en los
pasos arriesgados que son precisos para rescatar a mi hijito de las
manos de Gabino, \emph{forajida mala?»}

---Aun siendo preciso ir al mismo infierno, y pasar por entre todas las
catervas de diablos que andan sueltos por el mundo---exclamé yo, dándome
en el pecho un fuerte golpe,---aquí está el caballero, servidor y
esclavo de la dama dolorida.

---Mire lo que dice y a qué se compromete. Repetí yo, puesto en pie, con
hipérboles más deslumbradoras mi juramento, y en el calor de la
improvisación me lancé a darle un abrazo\ldots{} Del abrazo quise pasar
a darle un beso en la mejilla, pero ella desvió el rostro vivamente y me
quedé con las ganas\ldots{} Limitábame a besar ardorosamente sus lindas
manos, cuando me dijo con severa dulzura: «Admito muy agradecida su
oferta caballerosa, pero ello ha de ser sin el menor quebranto ni
perjuicio de mi honestidad\ldots{} La honestidad es lo primero\ldots{}
No habrá nada entre nosotros que no podamos decir a nuestros
confesores.»

Asentí, afirmé, corroboré con desaforados aspavientos.

\hypertarget{xi}{%
\chapter{XI}\label{xi}}

Mi primer cuidado en los días subsiguientes fue contener la impaciencia
de \emph{Chilivistra}, ganosa de lanzarse a románticas aventuras\ldots{}
Una noche, al salir del teatro del Príncipe, encontré a \emph{Leona} que
me soltó esta sorprendente noticia: «¿No sabes? Está aquí \emph{don
Florestán de Calabria}. Se ha escapado con un oficial de iberia, herido,
que viene a convalecer al lado de su familia. ¡Pobre don Jenaro! Ayer
tarde me tropecé con él en la calle. Al pronto no le conocí. Se ha
cortado las melenas, pero trae todavía la cara de hambre, los cachetes
dados de almazarrón y la perilla pintadita con el humo de la sartén. Me
dijo dónde vive, pero no \emph{me recuerdo\ldots{}} ¡Ay, ya doy con
ello!\ldots{} Vive con David Montero. Si tú sabes el domicilio de éste
podrás abocarte con el chiflado \emph{don Florestán\ldots{}} ¡Ah!,
también tienes aquí a Dorita, que rompió con Fructuoso por \emph{un
agravio contundente}, quiero decir \emph{bofetás\ldots{}} ¡Y qué cosas
cuentan de lo que en Cartagena ha pasado! Dice mi señor que aquello ha
sido el acabose de la \emph{apocalirsis.»}

Sin más averiguaciones me fui al día siguiente a la calle de los Reyes,
15, taller del armero Calixto Peñuela, famoso por su habilidad en la
compostura de escopetas de caza. Era éste un hombre de pocas palabras,
de corta estatura, calvo, afeitado. Entornaba los ojos para mirar por
ser corto de vista, y se cubría con un blusón o mandil azul hasta los
pies. En él vi el último representante vivo de aquellas ilustres
familias de armeros de Madrid, que tanta honra y prez dieron a su
industria en el siglo XVIII.

Su tienda era negra, desordenada, llena de piezas sueltas, de armas de
fuego en situación de reforma. Advertí que no tenía en el taller ninguna
silla, sin duda para que sus numerosos parroquianos no se sentaran a
darle conversación. Si el hombre era histórico, éralo también la casa,
que había pertenecido a don Francisco Goya.

Con el adusto artífice hablé lo preciso para formular mi pregunta, mas
sólo obtuve una respuesta rotundamente negativa: ignoraba quién era el
tal David Montero. Comprendiendo que quería guardar el incógnito a su
amigo, pronuncié el fingido nombre que el tal me confió en la estación
de Chinchilla: \emph{Simón de la Roda}. Al oírlo, Peñuela salió conmigo
a la puerta, y señalando calle abajo me dijo en forma seca y lacónica:
«En esta misma acera verá usted, tres casas más allá, una que no tiene
más que un piso alto, con un balcón y dos ventanuchos. En ese piso
hallará usted a \emph{Simón.»}

Al poco rato abrazaba yo a David, a quien encontré limando una pieza de
ajuste en un torno, junto a la ventana. No vestía ya de negro, y del
disfraz con que le vi en Chinchilla sólo conservaba el total rapado de
sus barbas. Apenas habíamos cambiado algunas impresiones sobre las cosas
de Cartagena, cuando vi entrar a \emph{don Florestán}, que venía de la
compra con su cesta al brazo. Al verme se deshizo en cumplimientos y
demostraciones de alegría, y habló de esta manera:

«Aún tengo tiempo de encender la lumbre\ldots{} Ya ve usted, señor don
Tito, en qué menesteres anda el pobre don Jenaro de Bocángel\ldots{} Esa
bigarda de Dorita, que pasa todas las noches corriendo las siete
partidas con bailarines, toreros y hombres de mal vivir, se acuesta
\emph{a la hora de las burras de leche}, y todavía la tiene usted
dormida como una marmota. Pero aquí está el hidalgo entre los hidalgos,
obligado a tirar de cacerola y soplillo, cosa tan contraria ¡oh Dios
mío!, a su abolengo y a su nombre\ldots{} Soportemos, aguantemos con
paciencia estas humillaciones, que pronto ha de llegar la buena\ldots{}
Habrá usted visto, señor historiador \emph{don Tito Livio}, que se
cumplieron mis predicciones: ya está establecido el \emph{Cantón
Mantuano}, aunque disimulado y so color de Centralismo para desorientar
a los \emph{alfonsainas.»}

---Sí, sí---dijo Montero, sarcástico;---¡bonito está el Cantón
Matritense, obra de Pavía, Serrano y García Ruiz!\ldots{} Coja usted la
cesta, \emph{don Florestán}, y váyase a la cocina, que yo cuidaré de
tirar de una pata a Dorita para que abra las pestañas, sacuda las
greñas, se ponga los huesos de punta y vaya a su obligación. ¡Hala
pronto, a la cocina, don Jenaro!

Rezongando se fue \emph{el de Calabria}, y David pasó a otro aposento.
Oí la voz descompuesta de Dorita maldiciendo a quien la despertaba.
Volvió Montero a mi lado\ldots{} Sentí el ruido que hacía la muchacha
lavoteándose la jeta y requiriendo su ropa y zapatillas. Pronto apareció
en la puerta alisándose las guedejas. «Este David tan súpito---exclamó
entre bostezos---no la deja a una vivir.» Luego advertí que metía sus
blanduras toráxicas dentro de un corsé muy deteriorado.

«Siéntese junto a mí, Tito---me dijo Montero.---Por esta gente y por
otros que han venido huyendo de la quema, sé lo que ha pasado en
Cartagena. En los primeros días de Enero arreció el fuego por una y otra
parte con intensidad aterradora\ldots{} Los cantonales izaron en todos
los fuertes bandera negra, y los Centralistas se apoderaron de la ermita
del monte Calvario, después de retirarse la poca fuerza que la
guarnecía. Me han dicho también que la \emph{Tetuán} no ardió por un
hecho casual. Cuentan que uno de los fogoneros de la fragata, encerrados
en el Presidio, fue malherido en el vientre por un casco de granada, y
que antes de morir confesó que había pegado fuego a las estopas de
limpiar las máquinas, después de rociarlas con petróleo, recibiendo por
este servicio treinta mil reales. Así me lo han referido; no respondo de
que ello sea cierto\ldots{}

»Por el teniente de Iberia que trajo a \emph{don Florestán}, he sabido
que López Domínguez recibió el día 3 un telegrama del General Pavía
dándole cuenta del golpe de Estado y diciéndole que tal acto fue tan
sólo una medida heroica para sacar a España del anarquismo y del caos.
Añadía el telegrama que acababa de formarse un Gobierno Nacional, y a
éste se adhirió aquel Ejército, sin más reserva que la del Coronel de
Ingenieros señor Ibarreta, el cual manifestó que su Cuerpo jamás se
había sublevado contra los Gobiernos constituidos.»

---Y en tanto---pregunté yo---¿siguieron bravamente unos y otros la
lucha emprendida?

---Sí---contestó David.---El día 4, los sitiadores rompieron un fuego
vivísimo contra el castillo de Galeras, y los sitiados reforzaron sus
medios de defensa montando un enorme cañón Barrios en el baluarte de la
puerta de Madrid. La jornada fue muy dura\ldots{} En ese día subió al
cielo de los inmortales el intrépido rufián don José Tercero \emph{El
Empalmao}.

---Lo que prueba, amigo mío---observé yo,---que toda una existencia de
acciones villanas puede ser redimida en una semana de sacrificios
heroicos.

---Así es---afirmó sentencioso David,---y no pocos ejemplos hay de ello
en la Historia.

---Tengo entendido que voló el Parque.

---Sí, el 6 al mediodía. El estruendo produjo efectos de terremoto.
Perecieron en el momento de la catástrofe más de cincuenta personas, y
otras tantas, espantosamente mutiladas, fueron extinguiéndose en los
días sucesivos. ¡Horrible, horrible!\ldots{} Lo más importante que vino
después fue que López Domínguez, apreciando los estragos que su
Artillería causó en los baluartes de Madrid y Muralla, amenazó con
emplazar cañones de gran calibre a setecientos metros de la Plaza, para
abrir brechas que facilitasen el asalto. Tales amenazas produjeron mayor
exaltación en las fuerzas Cantonales, y los presidiarios dijeron que
ellos serían los primeros en ocupar las brechas para recibir dignamente
a los sitiadores, \emph{sobre todo si venía delante la Guardia Civil.}

En esto llegó a nuestros oídos el rumorcillo de un altercado en lo
interior de la casa, y se nos presentó \emph{don Florestán}, compungido,
diciendo: «Señor Montero, señor don Tito: Dorita me ha pegado. Vean el
estropicio que me ha hecho en la frente con las tenazas. Y todo porque
quise arrimar a la lumbre el cazo en que hago mi café. Más que el golpe
he sentido que me haya llamado ladrón.»

Antes risueño que compadecido, Montero le incitó a llevar con paciencia
las genialidades de Dorita. Iguales exhortaciones le hice yo. Pero el
desdichado Bocángel, adoptando el tono patético y lacrimoso, se expresó
de esta manera: «¡No, señor Montero; las humillaciones que sufro aquí no
se compadecen con mi carácter altivo! El pan que como en su casa de
usted es demasiado amargo, y no pasa por mi gaznate sin producirme
bascas horribles. Ya sabe usted que mi prima, la dama ilustre que ha
venido a la triste condición de patrona de huéspedes, no quiere
admitirme en su casa si no le doy adelantadas las tres pesetas del
pupilaje. Pero hay Providencia, señor David, y un hombre como yo no
puede andar pidiendo limosna por las calles.»

---Eso no, eso no lo consentiremos---dije yo dando ánimos al infortunado
prócer.---¡Pues no faltaba más!

---Usted, señor don Tito, que sabe tanta Historia---prosiguió don Jenaro
,---no ignora que también tengo en mi abolengo ramificaciones con la
nobleza castellana. Por mi madre estoy emparentado con el famoso
personaje del siglo XVI Ruy Gómez de Silva, esposo de la Princesa de
Éboli, el cual Silva figura en la ópera que llaman \emph{Hernani}, donde
sale cantando por todo lo alto\ldots{} Pero dejo aparte estas grandezas
pasadas para repetir que hay Providencia. ¡Vaya si la hay! Sepan ustedes
que me ha salido una protectora sumamente caritativa, quien me ha
señalado un corto emolumento para vivir con el decoro que cumple a mi
linaje\ldots{} Y ahora, señor don David, agradeciéndole mucho su
hospitalidad, le pido licencia para recoger la balumba de mis papeles, y
me retiro de su casa.

Diole Montero el pasaporte con frases de afectuosa consideración, y don
Jenaro partió en seguimiento de su mejor acomodo\ldots{} Dos días me
bastaron para saber que la señora caritativa, ángel tutelar del \emph{de
Calabria}, era Leonarda Bravo, instalada ya en un pisito segundo de la
calle de Lope de Vega, frente a las Trinitarias. A visitarla fui una
tarde. La casa estaba bien arregladita de muebles, cortinas y alfombras,
y en ella campaba mi amiga como una reina que al trono de sus ilusiones
había subido dignamente. Ya conocía yo el buen corazón y natural
generoso de la hetaira lanzada con veloz carrera por el camino de la
ilustración. Lo primero que hizo al instalarse fue señalar a \emph{don
Florestán} dos pesetas diarias para que comiese en una taberna o figón;
luego le asignó una peseta más para que le diera lección de escritura,
dos horas al día, utilizando la consumada ciencia del eminente
calígrafo; y remató el favor concediéndole un cuarto interno de su casa
para que pasase las noches. Ahora dejo hablar a \emph{Leona}, que
completará estas interesantes noticias.

«No sólo me enseña la escritura---dijo ella sentándose en un blando
sillón---sino cosas tocantes a la poesía; porque has de saber, Tito de
mis pecados, que aquí trae mi señor las más de las noches a unos amigos,
que por las trazas deben ser gente de pluma, periodistas o autores de
comedias. Ello es que se ponen a decir versos, y a lo mejor salen
hablándome de estos o los otros poetas. Como yo estoy \emph{in albis} de
tal jerigonza me veo negra para poder contestarles. Pero ya verán qué
pronto me entero de todo eso y los dejo con la boca abierta\ldots{}
\emph{Don Florestán} me está enseñando nombres de poetas, y yo los
apunto para metérmelos en la memoria. Primero me ha enseñado los
españoles, y ahora está con los italianos que son los que mejor conoce,
cuatro no más según dice\ldots{} el Dante, el Ariosto, el Tasso, el
\emph{Petaca\ldots»}

---Petrarca, mujer, Petrarca---dije yo.---Ten cuidado, fijate bien.

---Ha sido una coladura---me contestó Leonarda.---Pero ya pongo en ello
mis cinco sentidos, y delante de gente no suelto uno de estos nombres
hasta que no estoy bien asegurada de las letras que tiene.

Felicité a mi amiga por el paso feliz que acababa de dar en su
regeneración mundana, y por sus adelantos en el arte de hablar bien, a
los que se unirían pronto algunos conocimientos literarios. En ella se
manifestaban, cada día más claramente, una inteligencia muy aguda y una
voluntad bien templada para la vida.

Ocasión es ésta de deciros algo del señor a cuya sombra realizaba
Leonarda sus planes educativos, y os daré clara razón de él, reservando
su nombre conforme a la delicada prescripción de su coima. Era el
empingorotado caballero un terrible burócrata, que siempre tenía puesto
en las situaciones liberales por su pericia en el mangoneo expedientil.
Conocíale yo de vista y no dejaba de admirar su corpulenta figura, su
pulida ropa, la mirada de protección y los andares majestuosos que
centuplicaban su indudable importancia. Bigote y perilla muy poblados y
teñidos de negro decoraban su rostro. En su pechera y en sus dedos
lucían brillantes espléndidos.

Pero lo más característico de tan imponente persona eran los sombreros
que usaba. La forma de tan descomunales chisteras estuvo muy en auge del
60 al 70: el primero que la llevó fue don José Salamanca. Adoptada
después por el \emph{Marqués del Bacalao}, Gándara, un conocido agente
de negocios y varios bolsistas y banqueros, siguió imperando en un corto
número de cabezas de notoria respetabilidad. Cuentan que fue Ministro un
sujeto por el solo mérito de usar aquella prenda, cuya especialidad
tenían los sombrereros Campo y Odone. Era un armatoste de alas anchas y
retorcidas por los lados, con alta copa cilíndrica semejante a la
chimenea de un vapor. El arrimo de \emph{La Brava} usó siempre la forma
más hiperbólica. Visto por detrás, el ajuste del sombrero en la cabeza
dejaba a la intemperie un segmento de la lustrosa calva del buen señor.
Completo en dos palabras el trazado de esta figura diciéndoos que era
uno de esos inconmensurables imbéciles que están siempre en candelero.

Visité yo algunas tardes a \emph{Leona}, hurtándole las vueltas al
caballero burócrata, para no tropezarme con él. Un día me recibió mi
amiga cuando terminaba su lección de escritura, y por cierto que
escribía ya gallardamente, con finos y elegantes trazos. ¡Vaya una
mujer! ¡Qué aplicación, qué tenacidad, qué inteligencia!\ldots{}

Viendo salir al pobrecillo \emph{don Florestán}, observamos que pisaba
con el contrafuerte. Movida a compasión, \emph{Leona} le llamó y le
dijo: \emph{«Florestancito}, no quiero verle más con esas botas;
tírelas, y aquí tiene tres duros para comprar unas nuevas.» Elogié yo su
caridad, presagiándole que por esta virtud, y por otras cosas que no son
virtud, llegaría seguramente a las mayores alturas de la esfera mundana.
Ella, riendo, me contestó: «Déjame a mí de alturas, Titillo, que yo,
\emph{con la modestia que me caracteriza}, andaré siempre a flor de
tierra.»

---No, \emph{Leona}---afirmé.---En ti se revela una cortesana de alto
vuelo, que será tal vez ornamento de la sociedad futura.

Disimulando con graciosos mohínes la hinchazón de su orgullo, me soltó
este verso, seguido de una fantástica cita literaria:
\emph{«\ldots Lástima grande---que no fuera verdad tanta
belleza\ldots{}} como dijo el Petrarca.»

Gozoso y echando facha con sus flamantes botas se me apareció una noche
\emph{don Florestán}, cerca de la casa en que moraba su protectora. Me
paró y entablamos el siguiente diálogo, que no carece de interés
histórico:

«Caballero don Tito, ¿va usted a casa de doña Leonarda?»

---No, hijo, que allí estará el señor del chisterómetro.

---En efecto, allí le tiene usted, acompañado de dos poetas tristes y
dos bolsistas alegres que hacen sus versos con números. Leonardita a
todos les oye y de todos aprende: ya sabe decir que el Interior está a
45,90, que los Bonos del Tesoro se cotizan a 33,12.

---Y a Montero ¿ha vuelto usted a verle?

---Sí señor, pero no en su casa. ¡Dios me libre de encontrarme con
Dorita, que es más mala que un dolor de muelas! He visto a don David en
un sotabanco de la calle de San Leonardo, donde mora una tal
\emph{Graziella}, italiana, que estaba en Cartagena y de allá vino
huyendo hace días.---¡Por Baco, por todos los númenes de Italia, qué
grata noticia me da usted! ¡Graziella en Madrid! Iré a verla
mañana\ldots{} ¿Habrá venido con el bestia de Perico?

---No señor. Ha venido con Fructuoso Manrique, ese caballerete semejante
a un palo del telégrafo que, según me dijo \emph{El Empalmao} (q. s. g.
h.), era novio de Dorita.

\emph{---Graziella} es mujer donosa y atractiva. Entiende de cábala y se
divierte con hechicerías que embelesan y cautivan el ánimo.

---¡A quién se lo cuenta usted!---exclamó \emph{don Florestán}.---En
Cartagena, mediante el estipendio de cinco duros, le hice yo una copia
del \emph{Manual Hebraico de Salomón Safetir,} donde están todos los
signos, trazos y garabatines que sirven para el barrunto y adivinación
de lo venidero, y para saber lo que está pasando a cien mil leguas de
distancia en la esfera terráquea\ldots{} Apenas llegó aquí, la
\emph{Graziella} puso taller y despacho de adivinanzas, con tan buena
mano que allí tiene un jubileo de mujeres del pueblo y de señoras de
alto copete, que van a que les eche las cartas para descubrir los
enredos de amantes o maridos.

---¿Estará haciendo su agosto?

---Ya lo creo. Cuando le pagan bien trae a capítulo a los animales del
Zodíaco, \emph{el Carnero}, \emph{el Toro}, \emph{el Escorpión},
\emph{el Macho Cabrío}, y a los que no son animales como \emph{Géminis}
y \emph{Libra} o \emph{la Balanza} que, entre paréntesis, es el signo
que presidió mi nacimiento, por lo cual estoy destinado a defender y
hacer triunfar la justicia. Mi misión es no tener descanso hasta
conseguir que la maldita mano muerta no se apodere por inicuos legados
de lo que no es suyo\ldots{} Cuando usted tenga un rato disponible le
daré a conocer las cartas que estoy escribiendo al General Pavía, al
General Serrano, al señor García Ruiz y al señor Martos, señalándoles el
camino que deben seguir para que las leyes tocantes a la herencia no
sean conforme al capricho de una vieja loca, sino ajustadas al fuero de
Naturaleza.

No se me cocía el pan hasta encararme con \emph{Graziella}, y allá me
fui a media mañana del día siguiente. El taller mágico de la italiana
diabólica radicaba en el piso más eminente de la casa en que vivió y
murió el buen don Hilario de la Peña. Cuando yo remontaba con dificultad
la escalera, mi audaz imaginación me hizo creer que ante mí corrían
negros y peludos diablillos\ldots{} En una estancia larga y de bajo
techo encontré a \emph{Graziella}, tan picaresca y sugestiva como
siempre, sentada a lo musulmán sobre un tapiz moruno. Vestía también al
uso marroquí, con chaquetilla roja recamada de aljófar, amplios calzones
y babuchas encarnadas. Entre sus piernas dormitaban dos gatos negros,
que a mi parecer, eran los mismos con quienes jugueteó el santo don
Hilario momentos antes de expirar. A un lado de la manga lucían dos
velas verdes. En el suelo vi un cuervo atado con delgada cadena, y un
búho que en platillo de barro comía su ración de carne cruda.

Al verme entrar, la diablesa soltó la risa y\ldots{}

\hypertarget{xii}{%
\chapter{XII}\label{xii}}

Yo también me reí viéndola con el atrezo y decorado de las hechiceras de
comedia de magia. «Esto, en verdad---me dijo,---no es para tomarlo a
guasa, porque gano el dinero a espuertas\ldots{} Ya puedes retirarte por
el foro: es la hora que he fijado para la entrada del público\ldots{} Mi
parroquia es la Humanidad que, como sabes, fue siempre tonta de remate.»
Respondile que haría mutis inmediatamente, pues mi visita no tenía más
objeto que ver a Fructuoso Manrique. ¿Estaba o no estaba en casa? Me
indicó \emph{Graziella} una puerta cercana, diciendo: «Por ahí pasas a
mi alcoba, y de ésta a otro aposento donde encontrarás a Manriquito
tumbado en un sofá de Vitoria. Ha pasado toda la noche fuera y está
rendido de cansancio. Él también desea mucho verte. Ya te dirá\ldots»

Momentos después había logrado despertar a Fructuoso, y platicábamos de
diversas cosas interesantes. Lo primero que me dijo fue que había pasado
la noche con Montero, en el domicilio de éste, y que ambos estaban
inquietos. Sentían cerca de sí el acecho policíaco como fugitivos del
Cantón. Se tranquilizó al saber mi amistad con un inspector de la
secreta, Serafín de San José, a quien yo había colocado tiempo atrás de
guardia de Orden Público. Aquella misma tarde procuraría verle, seguro
de tener a dicho individuo a nuestra completa devoción\ldots{} El
coloquio fue rodando por modo natural hacia los incidentes que
precedieron a la caída de Cartagena en poder de los Centralistas. A este
propósito, me refirió Manrique lo que a la letra copio:

«La defección del castillo de Atalaya, que está, como recordarás, en un
monte que domina el Arsenal, fue el principio del fin. Guarnecían
aquella posición fuerzas de Iberia y de Movilizados. A estos últimos los
mandaba un tal Joaquín Pagán, \emph{El Enlosador}, y a los primeros un
teniente llamado Ibarra. Según me dijo Cárceles, al Gobernador de la
fortaleza le ofrecieron los Centralistas diez mil duros. De esto no
puedo dar fe. Lo indubitable es que Ibarra y \emph{El Enlosador} estaban
en el ajo. Lo es también que un paisano, vecino de Quitapellejos, se
presentó en el Cuartel general de López Domínguez con el cuento de que
los de Atalaya se hallaban muertos de fatiga y de hambre, y que acaso se
rendirían si se les aseguraba que no serían fusilados. Contestó el
General en Jefe que concedería indulto a los paisanos, que a los
militares los pondría a disposición del Gobierno, y a los confinados los
encerraría de nuevo en el Presidio. Exceptuaba de la gracia de indulto a
todos los que pertenecieran o hubieran pertenecido a las llamadas Juntas
Supremas del Cantón.»

---Por algo que me dijo Montero, la rendición fue inmediata.

---No, no: espérate un poco. El 9 de Enero hubo un fuego vivísimo entre
los Centralistas y la Plaza. Sólo Atalaya permaneció inactivo y no fue
tampoco hostilizado\ldots{} El día 10, el Coronel Sánchez Mira y el
Brigadier Carmona celebraron una conferencia con los jefes del castillo
de Atalaya. A las ocho de la noche se reunían en una casa de campo
situada entre la fortaleza y las avanzadas del Ejército sitiador, y poco
después estaba concertada la entrega del castillo para las once y media
de aquella misma noche, no pidiendo los que se rendían más que el
indulto \emph{y algún socorro en metálico}.

Al llegar a este punto, oímos ruidillo de disputa en la puerta de la
casa. Creyendo escuchar una voz conocida corrí a satisfacer mi
curiosidad, y cuál no sería mi sorpresa al encararme con Celestina
Tirado que, actuando de portera en la consulta de quiromancia, trataba
de poner orden en el numeroso público, y alinearlo para formar cola. No
se hizo de nuevas al verme, y con su habitual socarronería me dijo: «Si
el caballero Tito viene también a que le adivinen, póngase en la
cola\ldots{} Hay señoras principales en la consulta.»

---No haré cola, señora doña Celestina---le dije muy quedamente,---si
usted me da razón de las damas ilustres que están dentro. Oigo aquí unas
vocecitas que\ldots{} o yo estoy loco o son de personas que conozco muy
bien.

Cautelosa y discreta me llevó la Tirado a las habitaciones interiores,
dejándome donde podía curiosear a mi sabor. Por una pequeña abertura de
la puerta del consultorio mágico vi a Delfina Gay y a
\emph{Chilivistra}, que aguardaban el oráculo del cuervo y el búho, y el
manejo de cartomancias que la pícara \emph{Graziella} se traía. Visto
esto, me volví de puntillas junto a Fructuoso, el cual prosiguió su
relato de esta manera:

«El castillo de Atalaya se rindió, y fue inútil la arriesgada tentativa
de Gálvez para recuperarlo. Como nota cómica de aquel indigno pasteleo
te contaré que el Gobernador de la fortaleza vendida a López Domínguez,
cuando le preguntó éste qué deseaba además del indulto y de los pocos
miles de reales con que había gratificado su infame traición, contestó
que deseaba le nombraran\ldots{} ¿qué dirás?\ldots{} ¡Administrador del
Matadero de Cartagena!

»Sigo mi cuento: al anochecer del 11 de Enero se presentó al General en
Jefe de los Centralistas una Comisión de la Cruz Roja, pidiéndole la
suspensión de hostilidades, y asegurándole que si era generoso con los
vencidos tal vez se conseguiría la capitulación de la Plaza. López
Domínguez contestó ofreciendo indulto para los que se rindieran. De esta
gracia quedaban exceptuados todos los individuos de la Junta Soberana,
sin perjuicio de recomendarlos a la benevolencia del Gobierno.

»Dio de plazo el General hasta las doce del siguiente día para la
entrega de Cartagena, ordenando a su Artillería suspender el fuego.
Luego se prorrogó el armisticio hasta las ocho de la mañana del 13.
Volvieron los de la Cruz Roja, con unos individuos que se atribuían la
representación del Ejército y de los Voluntarios Cantonales. Presentaron
a López Domínguez unas bases de Capitulación, que el General rechazó
indignado. Siguieron los tratos hasta primeras horas del día 13. López
Domínguez dijo que la Plaza tenía forzosamente que rendirse a
discreción, y que si se obstinaba en lo contrario la tomaría por asalto,
haciendo un duro escarmiento en los que intentasen una resistencia
inútil.

»La fiereza que en la mañana del 13 se manifestó en la Junta Soberana y
en todos los defensores de la idea cantonal, se fue trocando en
resignación estoica. Algunos querían rendirse, distinguiéndose en esta
actitud los militares; otros proponían furiosos seguir el ejemplo de
Numancia y Sagunto. Por sostener la no rendición hubo algún conato de
asesinar a Gálvez, y sus amigos tuvieron que llevarle casi a la fuerza a
bordo de la \emph{Numancia.»}

---No se puede negar---observé yo---que López Domínguez ha sabido
hacerse superior a la menguada fuerza de que disponía, y que sirvió
lealmente a la infantil, inestable República.

---Es verdad---afirmó Fructuoso.---Sigamos y acabemos. Llego al momento
más dramático y bello del Cantón Murciano, tan infantil e inestable como
la República Nacional de la que intentó desprenderse. La Junta Soberana
de Cartagena, los jefes de Voluntarios Cantonales y muchos de éstos,
además de los penados, no queriendo aceptar un perdón que jamás
solicitaran, resolvieron abandonar la Plaza con sus mujeres e hijos,
embarcándose en la \emph{Numancia}. Eran en total unos mil quinientos.
Confieso que no tuve valor para compartir la suerte de los que se
lanzaron con arrojo temerario al inmenso riesgo de la salida.

»Fuera esperaba la escuadra Centralista, compuesta de cinco fragatas,
entre ellas dos blindadas y otros barcos de guerra. Con los ojos llenos
de lágrimas me despedí de Manolo Cárceles, Gálvez, Contreras y demás
amigos, confundiendo en mis expresiones el sentimiento de mi cobardía y
el dolor de ver partir a tanta gente animosa que ponía la honra sobre la
vida y la expatriación sobre la libertad\ldots{} A las cuatro y media de
la tarde, mientras entraban en Cartagena parte de las tropas sitiadoras
y el General López Pintos se posesionaba del castillo de San Julián,
abandonado por su guarnición, levó anclas la nave intrépida que consignó
la última página del \emph{Cantón Cartaginés}. Desdicha fue para éste
que su postrer aliento sea el más interesante y hermoso en la Historia
de aquella turbulenta República.»

---Me han contado que en la boca del puerto embarrancó la fragata.

---Tocó ligeramente en el fondo con la proa; pero dio máquina atrás, y
con auxilio de un vapor se franqueó prontamente, saliendo mar afuera.
Desde el Empalmador Grande presencié la salida, imponente, grandiosa, en
medio de las aclamaciones de los que iban a bordo y del griterío de los
que quedábamos en tierra\ldots{} \emph{¡Viva el Cantón!} \emph{¡Viva
Cartagena!} \emph{¡Antes morir luchando que capitular!\ldots{}}
Claramente divisé el fez rojo del \emph{Comodoro Colau}, que sobre el
puente gobernaba el buque en la descomunal hazaña de la escapatoria.

»Al pasar de Escombreras, vieron los de la \emph{Numancia} la escuadra
Centralista formada en línea para cerrarle el paso. ¡Momento tan bello
que rayaba en lo sublime! Los barcos de Chicarro rompieron un fuego
horroroso contra la fugitiva\ldots{} Colau dio \emph{avante toda
máquina}, y viró rápidamente pasando como un rayo por entre la
\emph{Carmen} y la \emph{Zaragoza}, contra las cuales disparó sus dos
andanadas. Instantes después, la \emph{Numancia}, con veloz carrera,
apagadas las luces, se perdió en el horizonte\ldots{}

»Era la tarde fría, lluviosa y tristísima. El único consuelo de los que
permanecimos en tierra fue considerar los palmos de narices con que se
quedaron Chicarro y los suyos. Aún no habían vuelto de su asombro,
cuando la fragata que realizó el éxodo de los Cantonales al África
estaba ya en Orán.

»¡Adiós Cantón! ¡Adiós República ingenua y romántica, que a la Historia
diste más amenidad que altos y fecundos ejemplos! Tu existencia duró
seis meses y dos días\ldots»

Un rato se nos fue en inciertos cálculos sobre lo que hubiera podido
pasar en Orán a la llegada de la fragata. ¿Qué habría hecho el Gobierno
francés con los cabecillas, qué con los presidiarios?\ldots{} Divagando
estábamos cuando llegó David Montero, en quien advertimos mayor recelo
de los corchetes, que ya descaradamente le seguían los pasos. Para
sosegar a mis amigos salí a la busca de mi fiel esbirro Serafín de San
José, y no encontrándole en el Gobierno civil, me vi forzado a
personarme en la tienda de su esposa doña Cabeza (Concepción Jerónima).
Ya era yo sabedor de que se había restablecido felizmente la coyuntura
matrimonial.

Mi entrada en la tienda fue un éxito ruidoso, que casi trascendió a la
calle. Los dependientes me abrazaron, colmándome de felicitaciones, y al
punto bajó la rozagante doña Cabeza Ventosa de San José, quien, al
estrecharme ambas manos cariñosamente, se puso muy colorada de la
retozona emoción que al verme sentía. De boca de ella oí también
plácemes y albricias. Preguntando yo la razón de tales extremos, la
tendera me dijo: «Ya nos enteró don Francisco Bringas de que la
rendición de Cartagena no fue debida al cañoneo y artes guerreras de
López Domínguez, sino a la diplomacia de don Tito, que tiene en la
cabeza todo el talento de Dios.» El dependiente principal agregó con
petulancia: «Don Plácido Estupiñá supo de buena tinta, y así nos lo
comunicó, que el General Pavía quiso hacerle a usted Ministro, pero que
usted declinó esa honra \emph{con su habitual modestia}. Yo digo que
ello será en la primera crisis que \emph{haiga.»}

Como comprenderéis, lectores tan guasones como el que esto escribe, yo
dejé correr la bola, y afectando mucha prisa manifesté a la señora la
urgencia de hablar con su amante esposo. Por inmediatas referencias de
ella me enteré de que Serafín se había reformado; parecía otro hombre, y
al ascender a su actual posición su conducta y su porte eran de un
perfecto caballero. En tono reservado me dijo la que fue tiempo atrás
alivio de mis escaseces: «Como marido cumple, pero es tan \emph{Juan
Lanas} como siempre.»

En esto entró el ínclito San José; nos abrazamos, prodigándonos
recíprocas expresiones de cariño. Subimos al entresuelo, y reunidos los
tres, platicamos sobre el asunto que motivaba mi visita. Total, que
Serafín se prestó a ir conmigo a la calle de San Leonardo para devolver
la calma a mis amigos los emigrados de Cartagena.

«Ya sé---me dijo por el camino el complaciente policía,---ya sé que el
Gobierno le ha nombrado a usted Delegado Secreto con el fin de trabajar
la rendición de los carlistas, que nos están haciendo la santísima. Me
consta que el Zabala pone a disposición de usted trescientos mil duros
que ha de emplear \emph{paulativamente}, según se tercie, en el soborno
de los cabecillas que se quieran vender, y para mí que todos morderán el
queso. No hay hombre que pueda igualarse a usted en este fregado por su
talento macho, su agudeza y el meneo de los palillos en el juego de
convencer a la gente, por la buena cuando no por la mala. Como verá,
estoy bien enterado: seis millones de reales y manos libres para
contratar paces con los carlistas, como lo hizo tan limpiamente con los
Cantonales, mediante \emph{conquibus}. No ignoro tampoco que de aquí a
Julio tiene usted que dar por finiquita esta comisión. Seis meses y
cincuenta mil duros cada mes. ¿No es eso?»

A mi regocijada clientela no le ocultaré que también dejé correr esta
bola, a pesar de su descomunal magnitud. Cuando Serafín me propuso que
le llevara de auxiliar o secretario, le dije que ya pensaría en ello, y
tal y qué sé yo; pero que mayormente necesitaba un buen tesorero y
contador, muy experto en la Partida Doble. Pronto llegamos al eminente
piso de la calle de San Leonardo, y presentado Serafín a Fructuoso y a
Montero, quedamos acordes en la manera de asegurar a mis amigos su
omnímoda libertad en la Corte de las Españas. Retirose el bueno de San
José, diciéndome que estaba impaciente por tomar aquel mismo día una
provechosa lección de Partida Doble. David se fue a ver al armero
Calixto Peñuela para que le diese más trabajo, y Manrique salió en
requerimiento de sus antiguos camaradas, con idea de ser admitido en la
redacción de algún periódico mientras conseguía volver por los trámites
de costumbre al servicio de Telégrafos.

Quedeme solo con la hechicera y su ayudanta. Terminada la hora de
audiencia, presencié el recuento que hicieron de las ganancias de aquel
día. Luego las vi comer en el propio local donde tenían su consultorio
de adivinaciones. Apagaron las velas, sentáronse ambas a la turquesca,
el cuervo por un lado, el búho por otro, y con buen apetito aplicáronse
a devorar un oloroso guiso de carne y patatas y otros condumios que les
servía una criada algo gibosa, sin que faltaran las ricas uvas de cuelga
y el confortante Valdepeñas.

Celestina Tirado, que vestía falda y pañuelo al estilo gitano, me contó
que los dineros heredados del cura don Hilario se le habían ido entre
los dedos, porque se metió a fiadora y la desplumaron bonitamente,
dejándola por puertas. Desesperada y sin arrimo se acogió a la sabia
\emph{Graziella}, con quien se apañaba muy bien para hacer juntas el
oficio de brujas, granjería de mucho provecho en los reinos de España,
según ella había probado y visto por sus ojos más de una vez.

\emph{Graziella}, sin abandonar su traje moruno, se había recostado en
la alfombra después de la comida para fumar un cigarrillo, acariciando
el suave plumaje del búho, y en esta postura me dijo: «Más que de
Brujería debemos hablar de Ocultismo, que es ciencia flamante, muy
bonita, y yo sé de ella más que saben de Teología y Derecho Romano los
doctores de Salamanca. Por dominar esa ciencia heme dado buenos
atracones de lengua caldea, pues habéis de saber que de los caldeos y
egipcios ha venido esta divina monserga. Yo le digo a Celestina que no
necesitamos untarnos para salir por esos aires montadas en escobas y
llegarnos \emph{pian pianino} al cerro Zugarramurdi, donde nos espera el
\emph{Gran Cabrón} con toda su Corte de rabo y pezuña. Ésos son cuentos
viejos que ya están mandados recoger. Yo me voy de aquí a los antípodas,
o un poquito más allá si quiero, con sólo echar unas palabritas caldeas
sobre el humo de un braserillo en que pongo a quemar la muela del juicio
de un ahorcado que haya sido viudo tres veces y dos vértebras de una
urraca muerta en estado de virginidad. Yo me desentiendo del
\emph{Cabrío}, que ya está jubilado por viejo, y me pongo debajo del
patrocinio de Astarté, diosa de aquellos infiernos que sostienen buenas
relaciones con la Humanidad.»

---Pues aquí me tienes---dijo Celestina,---deseando meterme hasta las
cachas en la devoción de esa diosa \emph{Trastera}, y hoy empiezo a
rezarle padrenuestros y avemarías para que me tome en su gracia.

La profesora de Ocultismo me dio a renglón seguido prueba magnánima de
su confianza y del interés que se tomaba por mí. He aquí sus palabras:
«Hoy han estado en la consulta dos señoras amigas tuyas. La Delfina
quería cerciorarse de la fidelidad de un lindo coadjutor de San
Sebastián, con quien cambió promesas de cariño místico y rigurosamente
honesto. El dicho coadjutor se fue a Valladolid, donde al parecer se
halla en coqueteos igualmente místicos, puros y honestos, con otra dama
que allá tiene el negocio de ataúdes, según le han dicho a tu amiga en
un anónimo. La señora que por el habla me pareció vizcaína está
dislocada por ti, y anhela saber si puede contar con tu amor y tu
lealtad en un largo viaje que emprender quiere contigo. Yo les hice un
horóscopo con todas las de la ley, y ambas se fueron muy satisfechas. La
tuya llevó la seguridad de que estás enamoradísimo de ella y de que la
seguirás hasta el fin del mundo. La otra va dispuesta a cambiar de
coadjutor, pues en Madrid tiene donde escoger.» Último detalle de esta
referencia fue que la vizcaína le había pagado en plata y Delfina Gay en
calderilla.

Salí de aquella casa con mi espíritu en rotación vertiginosa. Bajando la
escalera creí que brincaban delante de mí negros animalejos con saltos
de batracio. Los peldaños vetustos de la casa de don Hilario gemían bajo
mis pies articulando frases que no entendí: sin duda me hablaban en
idioma caldeo. El fresco de la calle no despejó mi alocado
entendimiento. Éste se escapaba de la realidad, lanzándose con avidez
jubilosa a navegar por el insondable océano ultraterreno. Cerca ya de mi
casa, me parecían vanas y mentirosas las imágenes de los transeúntes que
mis ojos veían en derredor. Añadiré que aquel estado mental, sin duda de
carácter patológico, me transportaba suavemente a las penumbras de un
delicioso éxtasis. ¡Qué gusto mecerme en el vacío y subirme a las
estrellas, después de dar un puntapié al sólido asiento de la razón!

Lo primero que hice al entrar en la vivienda patronil fue interrogar
capciosamente a \emph{Chilivistra}, para cerciorarme de su visita al
sotabanco de las artes mágicas. ¡Grande sorpresa y mayor confusión mía!
O la vizcaína disimulaba con extrema sutileza, o la sesión de
Cartomancia y Brujería fue hechura quimérica de mis sentidos, sacados de
su orden natural por el influjo hermético de aquellas mujeres
diabólicas. Creció mi asombro cuando Silvestra me soltó estas
despampanantes revelaciones: «No por cábalas y sortilegios, que son
pecado mortal, sino por confidencias que acaba de hacer al señor Ido del
Sagrario un noble caballero de la Italia o de \emph{Palerma}, que se
llama, bien recuerdo el nombre, don Jenaro \emph{Bocadeángel}, sé que ha
tenido usted amores con una bestia hermosa, que ahora está estudiando
para señora fina y aristocrática. Daranle título de Duquesa de Mula.»

Rompió después \emph{Chilivistra} en un reír histérico. Yo me puse muy
serio ante aquel brusco retroceso a la realidad\ldots{} En el resto de
la tarde y a prima noche, logré con artificios de lenguaje, mezclando a
las patrañas la verdad, llevar el sosiego al ánimo de mi amiga. Sin
jactancia os aseguro que tuve un éxito de los más grandes de mi vida
enamoradiza y donjuanesca. La severidad de \emph{Chilivistra} se
descuajaba y desleía como un témpano de hielo rodeado de llamas\ldots{}
Sus resquemores contra \emph{Leona} quedaron reducidos a una infantil
celera por aventuras retrospectivas en que ninguna parte tuvo el corazón
de Proteo Liviano. Mi personalidad se creció a sus ojos, y echando el
resto de mi táctica seductora, la dejé totalmente sumisa, tierna y
acaramelada.

Aquella noche nos tuteamos por primera vez.

Y cuando nos entregábamos al descanso encadenó mi albedrío con un
emplazamiento perentorio: «¿Vendrás resueltamente conmigo en el viaje
que debo emprender para rescatar al hijo inocente del poder de un padre
loco?»

Mi contestación fue categórica y rotunda: «Al fin del mundo iré contigo.
No me arredran peligros ni distancias. Pasaremos si es preciso del mundo
real al mundo quimérico, que es la región de la verdad eterna.»

\hypertarget{xiii}{%
\chapter{XIII}\label{xiii}}

Casi automáticamente me llevaron mis pasos, no sé qué día, a la casa de
\emph{Leona}. El estado de constante alucinación, que balanceaba mi alma
en impresiones de susto y regocijo, sustraíame la noción del tiempo y me
daba sensaciones equivocadas de personas y lugares. La vivienda de
\emph{La Brava} se me antojó palacio suntuoso\ldots{} La señora no
estaba, según me dijo una linda criadita al abrirme la puerta. Pasé a la
sala y al punto se me apareció \emph{don Florestán}, en la misma facha y
pergenio con que le conocí en el patinillo de Santa Lucía. Las melenas
ahuecadas, según la moda del 40 al 50, ornaban otra vez su noble cabeza
siciliana. Había vuelto el rosicler a sus pómulos, y a su perilla el
negro humo de la sartén. Con voz opaca y un tanto medrosa me dijo:
«Estoy trazando un documento importantísimo, con escritura netamente
burocrática y todo primor de sellos y estampillas que han de darle la
debida eficacia como documento público\ldots{} Perdóneme que le deje un
momento, pues tengo que acabar mi trabajo ahora mismo. La señora no ha
de tardar; ha salido en coche.»

A punto que desaparecía de mi vista \emph{don Florestán}, se me presentó
Leonarda, en cuya persona vi la más exquisita elegancia y distinción.
¿Era ya Duquesa de Mula? Sentose a mi lado en un rico diván, y apenas me
habló de diferentes cosas, ora políticas, ora privadas, advertí la
discretísima forma y primor de su lenguaje. No usaba ya sin ton ni son
las palabras finas, sino que las seleccionaba, aplicándolas con arte a
la expresión de las ideas. Soñaba yo sin duda oyendo la dicción limpia
de \emph{Leona}, cual si pasara sobre ella toda la piedra pómez de la
Academia de la Lengua.

Díjome mi dulce amiga que no tardaría yo en llegar a \emph{la meta} de
mis ambiciones si seguía con paso firme la senda que un \emph{hado
propicio} me señalaba. Como yo me manifestase dispuesto a seguir todos
los caminos y veredas que los tales hados o hadas me señalaran, añadió
la ya retocada y pulida mujer: «Aunque no han de faltarte los medios
monetarios para \emph{dar cima} a empresa tan grande, padecerás un
ataque de \emph{inocencia paradisíaca} si crees que podrás salir de
Madrid sin numerario. Tú eres pobre, yo rica\ldots»

Diciendo esto sacó un portamonedas de malla de oro, y al ver yo que lo
abría para darme billetes y monedas, me levanté de súbito, protestando.
Mis primeras palabras, trémulas y confusas, fueron: «¿Eres tú, Leonarda,
la que a mi lado veo?\ldots{} ¿Cómo has subido tan pronto a la cima de
tus aspiraciones?\ldots{} ¿Andan también en esto los hados benignos y
las hadas traviesas?\ldots{} Si mis ojos no me engañan, esta vivienda
tuya es un lindo palacio\ldots{} Agradezco tu oferta. Pero no puedo, ni
debo, ni necesito aceptarla. Al mediar de todos los meses recojo yo en
la portería de la Academia de la Historia la cantidad que para mis
gastos asignada me tiene mi divina Madre\ldots{} ¿No la conoces?\ldots{}
Mi Madre vive lejos de aquí, y rara vez se deja ver en estos
barrios\ldots{} Pasa temporaditas en el Olimpo, con sus hermanas que,
naturalmente, son mis tías\ldots{} Algunas noches viene a esta casa mi
tía \emph{Doña Caliope} con los poetas que acá te trae de tertulia el
rimbombante señor de los desaforados sombreros\ldots{}

»Por descuido mío, por el desvanecimiento en que ahora está mi cabeza,
he dejado pasar cinco días sin recoger los dineros de la Mamá cien veces
augusta y soberana\ldots{} Allá me voy ahora mismo\ldots{} allá me
voy\ldots{} No me retengas; no dejes caer sobre mí el dulce peso de tu
cuerpo blando y amoroso\ldots{} No rodee mi cuello tu brazo\ldots{} no
me cautives\ldots{} Adiós, \emph{Leo\ldots»} Recuerdo haber oído la voz
tenue de Leonarda, diciéndome: «Adiós, Tito chiquitín y salado. Largo
tiempo estarás sin verme. Adiós.»

El encontronazo que di al entrar en la Academia de la Historia me
despertó. Había recorrido como máquina inconsciente un corto espacio de
las calles de Lope de Vega y el León. Una de las jambas graníticas que
forman la puerta de la antigua casa del \emph{Nuevo Rezado} me estropeó
el ala del sombrero, desollándome ligeramente una oreja\ldots{} Entré en
el portal de la Academia, y la portera, señora de mediano viso, afable y
un tanto redicha, me dio un paquetito rotulado a mi nombre con gallarda
escritura de Iturzaeta. Apresurábame a romper los sellos de lacre para
desentrañar lo que el paquete contenía, cuando la mano menudita de la
portera alargó hacia mí un pliego voluminoso que al punto reconocí como
de los llamados de oficio. En el sobre me daban tratamiento de
Ilustrísimo Señor, y vi un sello que decía: \emph{Presidencia del Poder
Ejecutivo}. «¿Qué será esto?»---me dije suspenso y turulato.

Como alma que lleva el diablo me eché a la calle, dándome un segundo
trastazo contra la jamba de berroqueña, y al doblar la esquina de la
calle de las Huertas metí el dedo en el sobre para rasgarlo y satisfacer
mi curiosidad. Hice propósito de irme a mi casa para examinar allí
detenidamente aquel embuchado misterioso; pero sumergido en la onda de
mi propio afán, seguí sin sentirlo por toda la calle de las Huertas
abajo. Lo primero que saqué del sobre fue un oficio, escrito en preciosa
letra de pendolista, con la mar de rasgueos y primores
caligráficos\ldots{} Al final me decían que me guardara Dios muchos
años, y que patatín y que patatán. Al principio leí que yo había sido
nombrado\ldots{} ¡Jesús, qué demonio será esto!\ldots{} Me dio en la
nariz olor de azufre, pez y otros ingredientes de la droguería infernal.

Con loca precipitación saqué del sobre otro papel. Era una carta firmada
por don Eugenio García Ruiz en la que éste me decía que el Consejo de
Ministros, después de la entrevista que yo celebré en la Presidencia con
los señores Serrano, Martos, Sagasta y el infrascrito, vistos mis
honrosos antecedentes, \emph{etcétera\ldots{}} examinadas mis altas
prendas de reserva y diplomacia, \emph{etcétera\ldots{}} acordado había
designarme como Delegado Secreto\ldots{}

Con mano convulsa, después de restregarme los ojos para convencerme de
que funcionaban en toda regla, saqué otro escrito del sobre y\ldots{}
¡Santa Bárbara!\ldots{} era un libramiento firmado por el Director del
Tesoro y el Ministro de Hacienda señor Echegaray\ldots{} ¡Ángeles
divinos, excelsa Madre: venid en mi socorro!\ldots{} Con sólo presentar
aquel documento en la Administración de la Hacienda Pública de Vitoria,
me serían entregados \emph{los primeros cincuenta mil duros, de los
trescientos mil} que yo debía emplear en la corruptela y soborno de
cabecillas carlistas\ldots{} Lo demás se me iría entregando en otras
Administraciones de Hacienda.

Poseído ya de una comezón epiléptica, metí todo en el sobre para leerlo
despacio en mi casa, y me encontré en el Prado, frente a la Platería de
Martínez. Me paré en firme, y un rato estuve haciendo cálculos
topográficos para ver qué camino había de tomar. Tras un largo discurrir
llegué a persuadirme de que por la calle de San Juan podía llegar
\emph{a la meta}, como decía mi amiga la Duquesa de Mula. Camino del
Amor de Dios, y pasando como un borracho de una acera a otra, tropecé
con varios transeúntes que me lanzaban hacia el arroyo.

Al cabo, encerrado en mi aposento patronil, traté de reconcentrar mi
pensamiento, apurando la lectura de los azufrados papelorios contenidos
en el sobre de oficio. Leí, releí: la duda y la certidumbre libraron
descomunal batalla en las sombrías regiones de mi espíritu. Lo que más
hondamente me alborotaba era el notición de mi conferencia con Serrano,
Sagasta, Martos y García Ruiz, en la Presidencia del Consejo, como
preliminar y fundamento del cargo de confianza con que el Gobierno me
favorecía. Para sacar de aquel abismo de confusiones la verdad que había
de tranquilizarme, me arrebujé en una manta, y hecho un ovillo me acosté
en mi lecho, amparándome de la obscuridad y un silencio absoluto con el
fin de que mi pensamiento trabajase a sus anchas\ldots{} Ahondando en el
problema llegué a creer que tal conferencia era verdad\ldots{} En esto,
entró en mi camarín Ido del Sagrario con la siguiente embajada, que
refiero sin dilación para solaz de mis regocijados lectores:

«¿Qué hay, carísimo don José?»---le dije fingiendo que despertaba.

---Ilustrísimo señor---me contestó,---ha estado aquí don Serafín de San
José. No le dejé pasar porque creí que Vuecencia no quería recibir a
nadie.

---A Serafín sí, sí---exclamé saltando de la cama.---¿Y no ha dicho si
está ya fuerte en la Partida Doble?

---Nada de eso me ha dicho, Ilustrísimo señor\ldots{} y no le apeo el
tratamiento aunque Vuecencia me lo mande\ldots{} El recado y comisión
que traía don Serafín era del tenor siguiente: Hallábase de guardia en
la Presidencia del Consejo el día en que Vuecencia celebró una larga
entrevista con el General Serrano y los Ministros de Gracia y Justicia,
Estado y Gobernación. Vio a Vuecencia entrar y salir. Uno de los
porteros de la Presidencia recogió un guante que a Su Ilustrísima se le
cayó al bajar la escalera. El susodicho guante pasó a las manos del
señor de San José para que se le entregase a Vuecencia\ldots{} y aquí lo
tenéis.

Mis asombrados ojos vieron el guante, pendiente de los trémulos dedos
del filósofo, y de ellos lo cogí, diciendo con toda la naturalidad que
afectar podía: «En efecto, lo eché de menos al volver a casa. Hágame el
favor, señor Sagrario, de buscar en el bolsillo de mi gabán el otro
guante, y cotéjelos a ver si\ldots»

---Aquí están los dos; son hermanos. El guante perdido y ahora
recuperado es el de la mano izquierda.

---Bien, bien. Que me pongan el almuerzo en seguida. Y ahora dígame otra
cosa: ¿está en casa doña Silvestra?

---No señor; hoy ha ido a confesar. Para mí que su conciencia está estos
días necesitada de un buen limpión\ldots{} Es un suponer: punto en
boca\ldots{} A Nicanora dijo esta mañana que quizás almorzaría con doña
Delfina. Si quiere usted verla váyase al almacén de féretros y allí le
darán razón.

Almorcé sin apetito, y por la tarde no vi mejor manera de pasar el rato
que lanzarme por calles y plazuelas, metiéndome más y más en la esfera
de la incongruencia que era en verdad un mundo delicioso, poblado de
indecibles encantos. A varios amigos encontré, y algunos de ellos me
felicitaron reservadamente\ldots{} «Ya sabemos que\ldots{} ¡Menuda
breva, amigo!\ldots» Al caer de la tarde, mis pasos automáticos me
llevaron a la calle de los Reyes. En la puerta de la armería de Calixto
Peñuela vi a \emph{Simón de la Roda} (Montero), que también me felicitó,
lamentándose de no poder acompañarme en mi diplomática expedición.

Seguí luego por la calle de San Bernardino. Al pasar por las Capuchinas
zumbaron en mis oídos voces, primero confusas, luego más claras, de mis
familiares espíritus, que alegremente me saludaban, celebrando con
blando gorjeo mi rápido avance en la esfera política y social. Aturdido
y como asustado de mí mismo me metí en un coche de los que en aquel
punto había y al cochero di las señas de mi casa, \emph{Amor de Dios,
12}. El vehículo corrió por las calles con un traqueteo espantoso que me
crispaba los nervios\ldots{} y no paró en la puerta de mi casa, sino en
Atocha, 3, tienda de ataúdes y coronas para muertos. Ya vi que los hados
me llevaban a donde querían. Entré, y a mi encuentro salió
\emph{Chilivistra}, que al verme se dispuso a volver conmigo a casa. Por
el camino, cogiéndome del brazo para que anduviera derecho, me dijo:

«Por mi parte ya tengo arregladas mis cosas. A ver si acabas tú de una
vez, para que partamos esta semana. Mañana no podemos irnos porque
quiero asistir a la novena de los Misterios Dolorosos de Nuestra Señora.
Pasado mañana tampoco, porque se celebra la fiesta de San Pedro Nolasco,
de quien era mi padre especial devoto, y pienso encargarle una misa que
oiremos los dos en la iglesia de las Trinitarias.»

Contestele yo que estaba en franquía para partir en globo, en
ferrocarril o a caballo, y correr con mi dama hasta el último rincón del
mundo. En casa ya, y sentaditos uno junto a otro en el sofá de los
muelles punzantes, me dijo \emph{Chilivistra}: «Aunque he confesado dos
veces, no creo tener mi conciencia enteramente limpia de pecado. Seamos
buenos, Tito, seamos juiciosos, y no nos lancemos a peligrosas aventuras
sin llevar nuestras almas bien confortadas en el santo temor de Dios.»
Asintiendo yo a cuanto me decía, todo mi afán era que diese la orden de
marcha la dulce, antojadiza y un tanto histérica señora de mis
atropellados pensamientos.

Un día entero me pasé en sueño profundo, durmiendo la mona que contraje
al sumergirme en las ondas en cierto modo alcohólicas del océano
suprasensible. El largo sueño agravó la intensa embriaguez de mi
espíritu, y por la noche, habiendo salido a que me diera el aire, me
creí convertido en pompa de jabón que flotaba sobre los transeúntes, al
ras de sus cabezas. Yo era una delgadísima esfera líquida, y temblando
me decía: «¡Ay, ay; si reviento al chocar con cualquiera de estas
cabezas, me deshago y no seré más que un salivazo mísero de agua
jabonosa!»

Por fin llegó el momento del anhelado éxodo. Precedidos de baúles y
maletas, salimos una tarde a punto de las siete para la estación de
Atocha, y nos empaquetamos en el correo de Aragón. Mi bendita compañera
se santiguó, una y otra vez, al ponerse el tren en marcha, y luego
siguió rezando hasta más allá de Alcalá de Henares.

Íbamos mi dama y yo solos en un departamento de primera. Observé que
Silvestra, al paso por algunas estaciones, consagraba devotas plegarias
entre dientes a los santos locales. En Sigüenza rezó a Santa Librada; en
Huerta a don Rodrigo Jiménez de Rada, creyéndole santo, y en Calatayud
dedicó extremados soliloquios y santiguaciones a los \emph{Divinos
Corporales}, confundiendo a Calatayud con Daroca. Así se lo dije,
añadiendo que el arzobispo de Toledo Jiménez de Rada no figuraba como
santo más que en el cielo de la Historia. En tanto, yo no perdía ripio
para proseguir las lecciones que le venía dando a fin de corregir sus
vicios de lenguaje, y debo hacer constar que ella demostró con su
aplicación el provecho que sacaba de tales enseñanzas.

Aunque salimos de Madrid con el propósito de hacer nuestro primer
descanso en Zaragoza, cambiamos de plan en Las Casetas, trasbordando al
tren de Castejón. Ya era día claro cuando corríamos por la ribera del
Ebro. Nuestro departamento iba mediado de viajeros, los cuales nos
informaron de que no se podía ir más allá de Tafalla por la línea de
Pamplona, y de que no había seguridad en la línea de Logroño y Miranda,
pues se decía que los carlistas de la Rioja Alavesa intentaban vadear el
río para ocupar a Cenicero. En vista de estas noticias y ansiando el
descanso, nos quedamos en Tudela, donde tranquilamente pasamos la noche.

En la intimidad, sintiéndome yo poseído, por no sé qué fenómeno
cerebral, de mi papel de Delegado Secreto, comuniqué a Silvestra todo el
intríngulis de mi Comisión diplomática para traer a la paz a los
cabecillas carlistas, mediante cebo contante y sonante. Más crédula que
yo mi antojadiza y nerviosa compañera, se apoderó gozosa de la noticia,
lanzándose a planear mi campaña, que fácilmente podía emparejarse con la
suya. «Creo yo---me dijo en tono de firmísima convicción---que ese
bandido de Cucala se venderá por veinte mil duros, o quizá por
menos\ldots{} ¿Está por aquí el Maestrazgo?»

---No, hija mía; el Maestrazgo lo hemos dejado a la espalda, al venir de
Las Casetas. Mi parecer es que el primer pez a quien hemos de echar el
anzuelo es el cura Santa Cruz, poniéndole una buena carnada de diez o
quince mil duros.

---Bastará con diez. Ya te diré yo cuál es el terreno en que opera ese
forajido, allá entre Tolosa, Betelu y la parte de Vera.

---Mi opinión\ldots{} ¿a ver qué te parece?\ldots{} es ofrecerle a Santa
Cruz los diez mil duros, dárselos, y en cuanto veamos que se los mete en
el bolsillo, cogerle, fusilarle, y en seguida quitarle el dinero, que
puede servirnos para otro.

---¡Muy bien, Tito: qué talento el tuyo!---exclamó \emph{Chilivistra}
navegando por el piélago inmenso del desatino.---Pero fíjate, debemos ir
primero contra los peces gordos. Si se consigue pescar a Dorregaray con
cuarenta mil duretes, a Cástor Andéchaga con veinticinco mil, y a otros
tales, habremos hecho más que cogiendo en la red a los bicharracos de
menor cuantía\ldots{} ¡Ah! Pero ahora caigo en que ante todo tenemos que
avistarnos con el Administrador de Rentas de Vitoria para que nos
entregue\ldots{}

---Ya, ya, el primer millón de reales---murmuré cayendo en honda
perplejidad. Y en mi mente se representó la imagen del Administrador de
Rentas como un ser escueto, peludo y rabilargo, que volvía del campo
solitario de Zugarramurdi.

\hypertarget{xiv}{%
\chapter{XIV}\label{xiv}}

Cediendo a los apremios de \emph{Chilivistra}, que mostraba impaciencia
febril, partimos en el primer tren del día siguiente hacia Logroño y
Miranda. Al pasar por Calahorra no olvidó Silvestra sus preces por los
santos patronos Emeterio y Celedonio, martirizados en aquella ciudad, y
cuyas cabezas fueron hasta Santander navegando por el Ebro, el
Mediterráneo y el Océano, en un barco de piedra. En Logroño, acordándose
mi amiga de la prisión de su marido, formuló mirando hacia el pueblo
este femenil apóstrofe: «¡Ah, pillastre! Más quiero verte vivo que
muerto; más atado que suelto por esos mundos, llevándote a mi pobre
hijo. Pero espérate un poco que ya te cogeremos, tunante\ldots{} Te
compraríamos por cinco mil duros si no supiéramos que habías de
jugártelos en seguida.»

Antes de llegar a la estación de Haro, tuvimos una detención de tres
horas largas en medio de la vía, sin que nadie supiera por qué. Los
viajeros, que entre unos y otros coches discurrían, hablaron de rotura
de máquina. Después se dijo que no llegaríamos a Miranda. Un señor que
entró en nuestro departamento porque en el suyo había demasiada gente,
nos contó que las tropas liberales habían desalojado de La Guardia a los
carlistas. Aquel buen señor, regordete, comunicativo y al parecer de
ideas avanzadas, dijo después: «Portugalete está en poder de los
carlistas. Ya se sabe que don Carlos ha repartido recompensas por ese
golpe de suerte: a Dorregaray le ha hecho Teniente General, y a Cástor
Andéchaga Mariscal de Campo. ¡Bonito se está poniendo esto! A Bilbao lo
tenemos cercado de carcundas. ¡Ay, mundo amargo, yo que tenía que ir
allá para mis negocios!\ldots{} ¿Van ustedes por casualidad a Vizcaya?»
Contestele que no por casualidad, sino por obligaciones ineludibles,
queríamos ir a Vitoria.

Nuestro desconocido acompañante, llevándose las manos a la cabeza,
aseguró que no podría ser sin llevar un salvoconducto del Estado Mayor
del maldito Treso, porque los carcas habían levantado la vía desde la
Puebla de Arganzón a Nanclares. Repuso a esto Silvestra que si no había
tren habría carros o borricos, y que de algún modo llegaríamos, pues nos
era indispensable abocarnos con el Administrador de Rentas de la
provincia de Álava\ldots{} Echado un remiendo provisional a la
locomotora, prosiguió el tren con marcha perezosa. Hacia las Conchas de
Haro se plantó de nuevo como un cojo dolorido de sus débiles piernas. La
segunda parada duró hasta el anochecer, y en ella tuvo tiempo el señor
regordete para darnos noticia descriptiva y topográfica de la cruel
guerra que asolaba el país.

No me detengo a referir los cuentos de aquel buen hombre porque me urge
deciros que llegamos a Miranda del Ebro entrada ya la noche, hartos del
tren y de su cojera insufrible. En la fonda de Guinea, donde nos
albergamos, diéronnos pormenores de la toma de La Guardia. Aunque
Moriones llevó consigo bastantes fuerzas para dominar la Rioja Alavesa,
aún quedaba en Miranda crecido número de tropas liberales.

A la mañana siguiente, dejando a \emph{Chilivistra} en el lecho con un
leve ataque de anginas, salí a recorrer el pueblo con idea de encontrar
entre la oficialidad de los Cuerpos allí estacionados algún amigo que me
orientase en la correría fantástica que había emprendido, acompañando a
una dolorida señora de buen palmito y un tantico alocada. Tan sólo
encontré a un Teniente de Puerto Rico llamado Palazuelos, a quien traté
mucho en Madrid, el cual me abrió ruta fácil hacia Vitoria con esta
indicación: «Proporciónese usted un carro, amigo mío, y agréguese mañana
a la impedimenta de mi batallón, que por orden de Moriones sale para la
capital de Álava.» Corrí a llevar esta feliz nueva a mi costilla
postiza, y me la encontré metida en fervorosos rezos a San Blas abogado
de los males de garganta (festividad del 3 de Febrero), con lo cual y
unas gargaritas de zumo de limón pensaba curarse totalmente de su
angina.

Por abreviar diré que San Blas y el zumo de limón triunfaron en la
garganta de \emph{Chilivistra}, y seguida al pie de la letra la
indicación del amigo Palazuelos, al anochecer del 4 nos aposentábamos en
la fonda de Quintanilla, en Vitoria\ldots{} Atormentado por la idea de
mi entrevista con el Administrador de Rentas, no pegué los ojos en toda
la noche. Silvestra durmió a pierna suelta\ldots{} En las primeras horas
de la mañana me incitó a levantarme con fuertes voces, diciéndome:
«Mientras yo me lavo y me arreglo, vete tú a presentar tu libramiento al
Administrador de Hacienda\ldots{} Despáchate, hombre, despáchate\ldots{}
Sacude la pereza. ¿Será preciso que te ayude a vestirte?\ldots{} Si
tuvieras mi genio ya estarías en la calle, atento a tu
obligación\ldots{} ¡Hala, hala, despabílate!\ldots{} ¡Ay, qué pelmazo,
Virgen Santa!\ldots{} Me desesperas\ldots»

Objeté yo que nada adelantaría con ir antes de las horas de oficina.
Pero ella, con ademán despótico y voces displicentes, me soltó esta
rociada: «Vete pronto, que algún tiempo has de necesitar para saber
dónde están esas oficinas. Coge tus papeles y no me vuelvas acá sin
traerte el millón de reales.»

No pasaré adelante sin daros detallada noticia del carácter complejo de
aquella mujer, estudiado por mí a medida que iba observando sus
diferentes facetas en el curso del trato íntimo. Era mimosa, blanda y
flexible, cuando en ella dominaba el instinto marital, o sea la
irresistible necesidad de aproximarse al hombre. Era ferozmente
autoritaria, tozuda y de palabra muy agria, cuando imperaba en ella la
soberbia. Su misticismo, o insana embriaguez de las devociones
supersticiosas, prevalecía tan pronto como se le apagaba el ardor de las
borracheras lúbricas.

En su conducta advertí una oscilación isócrona de péndulo: apenas se
levantaba un palmo del lodo en que arrastraba su liviandad, emprendía
rápido vuelo para subirse a una región de mentirosas estrellas, y de
allí caía otra vez al fango. Del mismo modo, los arrebatos de su
irritable amor propio alternaban en el curso diario de la vida con su
mórbida humildad de fémina caprichosa. Había yo notado que durante
semanas enteras comía vorazmente, sucediendo al buen apetito
abstinencias de anacoreta. La conocí tierna y amante; la padecí poseída
de celos absurdos y de locas envidias. En resumen; llegué a ver en ella
una especie de relicario diabólico en el que estaban contenidos los
siete pecados capitales.

Salí aquella mañana por las calles de Vitoria en estado de ánimo
semejante al de Sancho Panza cuando Don Quijote le envió al Toboso con
la carta para Dulcinea. Largo rato divagué movido por una extremada
confusión y perplejidad. ¿Presentaría mis documentos al Administrador de
Rentas? Sentado en un banco de la Plaza de la Constitución, por hacer
tiempo saqué mis papeles, y examinándolos una y otra vez, fijándome en
todos sus rasgos y primores de caligrafía, los diputé por buenos,
absolutamente fidedignos. Con esta idea me fui como una flecha hacia el
edificio donde me dijeron que radicaban el Gobierno civil y la
Administración de Hacienda. Pero al llegar a la puerta me sentí detenido
por una mano que llamaré invisible y misteriosa. Así son todas las manos
que en casos tales atajan a los personajes de novela, lanzados a veloz
carrera por un fuerte impulso del corazón. Supersticioso miedo invadió
mi alma. Oí la risilla de un diablo maleante y jovial, que a mi parecer
salió de las oficinas armado de látigo, más bien zorro para sacudir
muebles\ldots{}

Me retiré, invocando a \emph{Mariclío} para que de aquella horrible
turbación me sacase. Pero por más que la llamé con el pensamiento, y aun
con la voz, la Madre augusta no vino en mi auxilio. Decidí al cabo
volverme a la fonda, después de dar vueltas y más vueltas por las calles
circulares de la parte vieja de la ciudad, sin otro objeto que
justificar, con una prudente tardanza, el plan concebido para dar el
pego a \emph{Chilivistra\ldots{}} Encontré a ésta ya vestida con su
hábito negro de los Dolores, en el cual brillaba el emblema de plata: un
corazón atravesado por siete lindas espaditas. Advirtiendo en Silvestra
el temblor de labio, signo infalible del punto culminante de su
soberbia, me anticipé a su interrogación diciéndole con afectada
serenidad: «Pues verás, mujer, lo que me ha pasado.» Y ella, con seca
voz airada, balbució estas palabras: «Acaba pronto, majadero\ldots{}
¿Traes el millón?»

Me senté risueño, simulando cansancio para desarrollar mi plan
dialéctico, que fui exponiendo poco a poco en esta forma: «Espérate un
poco\ldots{} Verás\ldots{} Déjame tomar aliento\ldots{} El señor
Administrador es un caballero amabilísimo, pero\ldots» Interrumpiome
Silvestra con estas frases cortadas, que tartajosas salían de sus
labios: «Amabilísimo, sí\ldots{} Será un maula\ldots{} como tú\ldots{}
un perezoso\ldots{} Te habrá mandado que vuelvas\ldots{} Esa gentuza de
oficina siempre tiene en la boca el \emph{vuelva usted\ldots{}} ¿Y
cuándo?\ldots{} ¿Esta tarde?»

---Esta tarde no\ldots{} Pero no te sofoques, no te precipites. Siéntate
y hablaremos ---dije yo, viéndola correr y dar vueltas como una pantera
enjaulada.---Estas cosas no pueden resolverse de momento. Hay casos
excepcionales. Verás. El señor Administrador, que, lo repito, es hombre
muy fino, me ha mandado volver dentro de unos días\ldots{} Ten
calma\ldots{} Sin precisar cuántos días\ldots{} Es que ha tenido que dar
a las tropas de Moriones la paga de Noviembre y parte de la de
Diciembre. Ponte en su caso, mujer. Ayer hizo el arqueo, y sólo tiene en
Caja diez mil duros.

---¿Y por qué no te los ha dado ese bergante?

---Eres una pólvora. Espérate. Los diez mil duros están en calderilla.
¿Cómo quieres que\ldots?

Largo tiempo invertí en desfogar el encendido temperamento de aquella
hembra, que se ponía insufrible cuando le soplaba el viento de la
soberbia. Dos medios había para domarla: o apurar mis facultades
\emph{parlamentarias}, con refuerzo de halagos y carantoñas, o coger una
estaca y convencerla con razones contundentes. Este sistema radical no
lo había empleado nunca. Preferí en aquella ocasión el método de la
verbosidad dulzona, y a la media hora de aplicarlo ya estaba la señora
como un guante. Díjome que después de almorzar haría sus visitas a las
familias de Vitoria con quienes tenía conocimiento y amistad. Los
Baraonas eran los primeros a quienes pensaba visitar, porque con ellos
uníanla estrechos lazos de parentesco. Después se vería con los
Trapinedos, Prestameros y Romarates. De todas estas familias, que eran
fieles fanáticas del \emph{Dios, Patria y Rey}, esperaba obtener
salvoconductos para penetrar sin riesgo en el campo carlista. Cuando
comíamos me dijo que, por decoro y honestidad, no era prudente que yo
figurase como su acompañante. Pareciome muy sensata esta precaución y le
manifesté que si sus amistades y parentela le pagaban la visita, yo me
ocultaría discretamente.

Al disponer por la noche nuestra partida en dirección a Durango,
itinerario marcado por la terca vizcaína, ésta se rebelaba contra la
idea de dejar en Vitoria los diez mil duros, y en su desvarío llegó a
proponerme que cargáramos con la calderilla, aunque para ello tuviéramos
que alquilar cuantos carros fueran menester. Con nuevo gasto de saliva
la disuadí de aquel disparate, asegurándole que con mis libramientos en
regla bastaba para reducir a los cabecillas más inaccesibles al soborno.

En un mal carricoche, que alquilamos pagándolo muy bien, partimos de
madrugada por el camino real de Peña de Amboto y Ochandiano. Invertimos
casi todo el día en llegar a este último pueblo por entorpecimientos de
la carretera y por los sobresaltos que nos causaron algunas partidas
volantes, de las que logramos zafarnos gracias a los salvoconductos de
que se pertrechó en Vitoria la tozuda señora que me llevaba de rodrigón
o escudero.

En las agrias cuestas de la divisoria tuvimos que aplicar a nuestro
desvencijado carruaje la tracción de una pareja de bueyes. En otras
partes del camino, los deterioros causados por el temporal de lluvias
nos obligaron a recorrer a pie largos trayectos. Estos desavíos, y el
hambre que nos extenuaba por habérsenos olvidado la canasta de
provisiones, moviéronnos a guarecernos en la posada de Ochandiano para
comer tranquilamente y pasar la noche. Gozosos entramos a disfrutar del
abrigo de aquella casa, donde además de comodidades tuvimos agasajo y
cariño. La patrona, que era una mujer fresca, guapa y de gigantescas
hechuras, nos trató desde el primer momento con afabilidad campechana.
Apenas cruzados los primeros saludos entre la dueña del parador y
\emph{Chilivistra}, lanzáronse ambas a parlotear alegremente en lengua
vasca, dejándome casi a obscuras de cuanto decían.

La cena fue sabrosa, animada y familiar, sentándonos juntos en la misma
mesa la patrona con dos hijos suyos de corta edad, Silvestra, dos
hombrachos de boina blanca con insignias, de Teniente el uno de Capitán
el otro, y un servidor de ustedes. La posadera, cuyo asiento estaba
frontero al mío, blasonaba de persona cortés, dirigiéndome frases en
castellano macarrónico para indemnizarme del tedio que me producía el
asistir en silencio a una conversación en vascuence. «Esta señora---me
dijo mi dama---se llama Polonia Zuazu y es sobrina carnal de nuestro
amigo el cura Choribiqueta. Según ella, estás aburrido porque hablamos
una lengua que no entiendes, y yo le digo que no debemos hablar
castellano para que te acostumbres al son del habla nuestra y vayas
aprendiéndola.»

No refiero pormenores de aquella cena ni del franco regocijo que en ella
reinó, porque anhelo pasar rápidamente a otro pasaje más interesante.
Encendida la vela hospederil en candelero de cobre, Polonia nos guió a
la habitación que nos destinaba. Apenas encerrados en ella, vi que mi
compañera frente a mí se engallaba con ojos fulgurantes, y el temblor de
labio inseparable de sus accesos de ira. Absorto quedé al oír este
absurdo despropósito:

«Ya he sentido\ldots{} bien segura estoy\ldots{} que por debajo de la
mesa\ldots{} le pisabas el pie a Polonia\ldots{} No lo niegues: tengo yo
mucho pesquis para estas cosas\ldots{} Y ella, la muy puerca, se dejaba
caer pisándote a ti\ldots{} Es claro como el agua\ldots{} No se me han
escapado tampoco las miraditas que cruzabais ella y tú.»

Grave y firme rechacé la indigna suposición de Silvestra. Pero ella, más
enfoguetada en su imaginaria celera, prosiguió de este modo, agriando la
voz y sacudiendo mi brazo:

«La gran bribona me dijo que eres muy guapo\ldots{} Creerás tú que yo no
entiendo de estas cosas\ldots{} Claro: como soy santita no sé nada del
mundo\ldots{} Te equivocas, sinvergüenza\ldots{} Yo sé muy bien que las
gigantonas gustan de los enanitos\ldots{} y los chiquitines de las
marimachos\ldots{} Puedes irte con ella\ldots{} No temas nada\ldots{} El
marido está lejos: sirve como tambor mayor en el 6.º de Navarra.»

De toda mi serenidad y paciencia tuve que valerme para refrenar la
cólera. Cuantos argumentos me sugería la razón no bastaban para
desvanecer el ridículo supuesto de aquella hembra desconcertada. Llegué
a pensar que todo era invención caprichosa, histérica, para
mortificarme. Por fin, con rotunda frase corté la disputa. Ordené a
Silvestra que se acostara, y le dije que yo haría lo mismo, aplazando la
cuestión para el día siguiente. Por fortuna teníamos camas separadas.
\emph{Chilivistra} se desnudó aprisa, esparciendo su ropa por el cuarto,
y se metió en el lecho. Yo también me acosté.

Pero no pude disfrutar ni un momento de calma porque la furiosa mujer me
atormentó con fingidos lloriqueos, y con estos lastimeros reproches:
«Podías hacerte cargo, hombre desvanecido y sin seso, de que por culpa
tuya estoy yo en pecado mortal. Esto es tan verdad como Dios es mi
padre. Yo vivía en santa ignorancia de ciertos desvaríos, y tú has
venido con seducciones infernales a manchar mi conciencia. ¡Ay Virgen
mía! ¿Quién me había de decir que yo pasaría del estado angélico al
estado de condenación por las artes de este pillete vicioso, sin ley ni
Dios?»

Callado escuchaba yo tales desatinos, y mordiendo la sábana para no
disparatarme en denuestos contra Silvestra, me decía: «A esta loquinaria
le rompo yo un hueso antes que amanezca, y si logro contenerme, mañana
la dejo plantada, aquí o donde me parezca mejor.» Furiosa
\emph{Chilivistra} porque yo no quería contestar a sus invectivas, me
tiró una bota que vino a dar en mi frente. Más benigno que ella,
contesté a su disparo tirándole una almohada. No acabó aquí el
bombardeo. Viendo caer sobre mí la otra bota de ella, le arrojé yo las
dos mías, a lo que contestó \emph{la plaza enemiga} lanzándome un vaso
de agua que tenía en la mesa de noche.

Ya no pude aguantar más. Me levanté. Vistiéndome con calma vi que
Silvestra se volvía de cara a la pared y se arrebujaba en las sábanas,
como para prevenirse contra el vapuleo que merecía.

\hypertarget{xv}{%
\chapter{XV}\label{xv}}

Defendiéndome del frío con mi gabán y la manta de viaje me tendí en un
sofá de Vitoria, no sin requerir mi cachava, cuyo auxilio me pareció
necesario en expectación de lo que ocurrir pudiera. Contra lo que
esperaba, mi basilisco permaneció silencioso entre las sábanas, y a la
media hora el rumor de su respiración me advirtió que se había dormido.
Yo también descabecé algunos sueñecillos sobre el duro sofá.

Apenas entraron por las rendijas del balcón las primeras claridades del
alba, me sorprendió la voz de \emph{Chilivistra} en los tonos más dulces
que usar solía cuando su magín recobraba el normal equilibrio: «¡Ay,
Tito, ven! Hazme el favor. He despertado con terribles dolores en la
paletilla derecha. ¡Ay, ay! Ya se me corren por la espalda hacia el
costado. Acércate, dame unas friegas como tú sabes hacerlo, por toda
esta parte. Anda pronto, que no puedo respirar.»

Acudí a ella, y sin hablar palabra le di los deseados refregones,
recordando que había estado en un tris el dárselos de acebuche. «¡Ay,
Tito---me dijo plañidera,---qué arisco estás! Ni siquiera me preguntas
cómo he pasado la noche. Yo he dormido algo, ¿y tú?\ldots{} ¿Pero qué
haces, tonto? ¿Te vuelves al sofá sin decirme nada? Llégate otra vez
aquí y friégame más fuerte, que aún no se me ha quitado el dolor.»

Mientras yo le raspaba la piel con verdadero ahínco, la fierecilla me
habló de esta manera: «Ya recuerdo. Estás enojado por lo que pasó al
acostarnos. Tú eres un gran pillo, y yo me disloco cuando me figuro que
no me quieren\ldots{} En mi cama tengo una de tus botas y en la tuya
deben estar las dos mías. Vaya, no se hable más de eso, y veamos en todo
ello la fuerza del querer. Se me metió en la cabeza que le pisabas el
pie a Polonia; esta idea, y el decirme ella que eres muy guapín me
sacaron de quicio.»

Había pasado el arrechucho. La gata nerviosa pedía reconciliación con
suaves mayidos. Como siempre prefiero la situación de paz a la de
guerra, accedí a las paces para evitar mayores disgustos. Junto a ella
dormí largo rato, y ya serían las nueve cuando me despertó con fuertes
empujones, diciéndome: «¿No oyes tocar a misa? Levantémonos, vistámonos
a escape. Hoy no me quedo sin misa, y tú irás conmigo, que buena falta
nos hace a los dos.»

Al volver de la iglesia, la simpática Polonia nos dio el desayuno en la
planta baja de la casa, donde tenía taberna y estanco. Junto a nosotros
tomaba la mañana el fornido carlistón en quien vi la noche antes las
insignias de Teniente, el cual nos dijo que si a Durango íbamos él nos
llevaría gustoso. De diez a once saldría en aquella dirección
conduciendo un convoy de víveres. Aceptó Silvestra el galante
ofrecimiento, y poco después emprendimos nuestra marcha en un carro de
la impedimenta carlista. Nada de particular nos ocurrió en el camino. A
la caída de la tarde, cuando ya nos aproximábamos al fin de nuestro
viaje, paró el convoy junto a un robledal espeso. El Teniente, que iba a
caballo, se acercó a nuestro carro y nos dijo:

«Antes de seguir adelante, quiero decir a ustedes que yo me quedaré a
cenar esta noche en una casa de campo que encontraremos cerca de San
Pedro de Tavira. Es la quinta de Aizpurúa, hoy propiedad de mi prima
Pepita Izco. Sabiendo que son ustedes amigos de Pepita, les invito a que
pasen allí la noche. Estoy bien seguro de que en ello tendrá mucho gusto
mi parienta.»

Al oír mi dama el nombre de Pepita Izco palideció, y su labio temblicón
indicó la inminencia de otro estallido de celos. De un brinco descendió
del carro; yo hice lo mismo, tratando de contener los bufidos de su
enojo ante los soldados que ya se arremolinaban en torno nuestro. Sin
cuidarse del público que en derredor teníamos, el basilisco agarrome las
solapas del gabán y me increpó en esta forma desatinada y virulenta:
«¡Malvado!, anoche, mientras yo dormía, concertaste con este
Teniente\ldots{} ya lo veo, ya\ldots{} que te trajese a la casa de tu
antiguo amor, Pepita Izco\ldots{} ¡Bien, muy bien!\ldots{} ¿Es ello
propio de un caballero?»

Al decir esto me estrujaba, y llenando de arañazos mi rostro, me
desanudaba la corbata. Yo no hice más que rechazarla con alguna
violencia. El Teniente acudió a contenerla. Sofocado y casi sin aliento,
apenas pude formular algunas palabras en mi defensa. «Esta señora está
loca---afirmé.---Llévenla donde quieran. Yo me vuelvo a Ochandiano.» Y
dejando a Silvestra rodeada de los del convoy, fui a sacar del carro mi
maleta, para poner en ejecución inmediatamente lo que había dicho. En
esto, sentimos por el robledal toques de corneta y ruido de tropas. Era
un destacamento de la división de Lizárraga, que según después supe iba
a Portugalete.

Pronto se vio aquel trozo de la carretera lleno de soldados. El Capitán
que mandaba a los de Lizárraga reconoció al instante a la fierecilla, y
se fue hacia ella gozoso, saludándola con estas voces: «¡Oh,
\emph{Chilivistra}! ¿Tú aquí, mujer? ¿Qué te pasa, qué es esto?» Ella,
lívida, las manos en alto, la boca espumante, vociferaba contra mí con
los dicterios más atroces: infame, traicionero, burlador de mujeres
honradas, enviado de Satanás\ldots{}

En tanto, los del convoy me apartaban hacia otro lado, y por sus miradas
y actitudes comprendí que todos se ponían de parte de la señora.
Prodújose una confusión tan grande que no pude darme cuenta de lo que
pasaba. Luego vi que el convoy se ponía en marcha, llevándose al
basilisco en el mismo carro que hasta allí nos condujo. En pie seguía
dando gritos, entre los cuales percibí estos acentos trágicos:
«¡Matarle, fusilarle!»

El Capitán de la columna se llegó a mí, diciendo risueño y zumbón:
«Hola, Tito, gran Tito, ¿viene usted a proclamar la \emph{República
Pontificia}?» Fijándome en él caí en la cuenta de que era un muchacho
durangués, muy simpático por cierto, llamado Mendía y vecino de mi
hermana Trigidia. Al reconocerle abrí mis brazos con efusión,
diciéndole: «Amigo, deme usted un abrazo. ¡Qué alegría tan grande!»

---¿Alegría dice?---exclamó el Capitán.---¿Y quiere abrazarme? ¡Pero si
debe usted renegar de mí! Le tengo a usted por hombre sospechoso.
Conozco bien sus ideas, y seguramente no viene usted aquí a cosa buena.
Me veo, pues, precisado a detenerle. Venga usted conmigo.

---Deténgame y lléveme a donde quiera. Es usted mi salvador.

---¡Su salvador!\ldots{} ¿Por qué?

---Porque al librarme de esa tarasca me ha sacado de la más horrenda
esclavitud. Dice usted que me lleva preso, y yo digo que esa prisión
equivale a mi libertad.

El Capitán ordenó a un soldado que llevase mi maleta, indicándome que a
su lado marchara. Obedecí, y platicamos tranquilamente, andando por
senderos para mí desconocidos. Cerrada la noche, entramos por ásperas
cañadas entre matorrales espesos.

«Debe usted agradecerme, señor Tito---me dijo el Capitán,---que no le
haya dejado ir a Durango, donde tiene usted no pocos enemigos; hay allí
personas que desean cobrarle el bromazo que nos dio con aquella pamplina
del \emph{Imperio Hispano Pontificio}. Se ha librado usted de que le
contesten al discurso con una tanda de cardenales\ldots{} Además, le
diré por si lo ignora, que su padre don Matías Liviano no está ya en
Durango: hace un mes se fue con su hija Trigidia y sus nietos a Motrico,
buscando mayor sosiego. Ignacio Zubiri está en el Cuartel Real de don
Carlos.»

La noticia de la ausencia de mi padre y hermana turbó un poco mi
espíritu. Pero estas desazones, así como la idea de mi cautiverio, eran
compensadas por la felicidad de haber sacudido el insufrible yugo de
\emph{Chilivistra}. A las dos horas de camino por terreno quebrado,
vadeando arroyos y franqueando divisorias, empecé a sentir cansancio y
desaliento, dándome cuenta de la gravedad de mi situación\ldots{} ¿A
dónde me llevaban? ¿Qué sería de mí entre aquellos hombres fanáticos,
que subordinaban toda ley de humanidad a las absurdas pretensiones de un
Rey de fantasía?\ldots{} No estaba yo acostumbrado a las marchas
militares sin descanso ni respiro. Aquellos sectarios de inflamado
corazón y temple duro tenían piernas de acero. Para engañar el tiempo y
la fatiga amenizaban la constante andadura con alegres cantorrios.

El Capitán callaba, y de rato en rato, con frase breve, hacía por
estimularme a que pusiera mi paso perezoso al aire y compás de la
columna incansable. Ladridos de perros venían a nosotros de una parte y
otra, añadiendo las notas campesinas al tumulto de nuestras pisadas.
Avanzaba la noche, fría y obscura, sin que el formidable aliento de los
recios campeones, ávidos de tragarse las leguas y de medir con sus pies
el terreno sin fin, diera señales de amenguarse. A la madrugada, ya era
yo como un muerto que se movía por máquina\ldots{} Al clarear el alba
distinguí casas; vi algunos paisanos que salían a nuestro encuentro; oí
terminachos y salutaciones en vascuence. Entrábamos en un pueblo. Mis
pobres huesos dieron gracias a Dios.

«¿Descansaremos en este lugar?»---pregunté a Mendía. Y éste secamente me
respondió: «Nosotros no descansamos; hemos de seguir a marcha forzada
algunas horas más. Usted se queda aquí a disposición del Comandante de
la Fortaleza. Se registrará su maleta y su ropa a fin de saber qué
mensajes o encomiendas trae. Deseo que no resulte nada contra usted.
Adiós, amigo.»

En esto llegamos a una plazoleta empedrada y llena de baches. Vi
acercarse a unos hombres de boina, embozados en sus capotes. Uno de
ellos traía un farol que tristemente pestañeaba en la obscuridad, pues
la aurora, mensajera del rubicundo Febo, apenas hendía los horizontes
con sus dedos de rosa\ldots{}

Metiéronme por angosta puerta en una tenebrosa estancia, y a la luz del
farol macilento me tomaron el nombre, edad, profesión, etc. Mis
respuestas se ajustaron completamente a la verdad. Luego hicieron
registro escrupuloso en toda mi ropa, tentándola por una y otra parte,
por si entre los forros sonaba ruido de papeles. Los que yo llevaba en
el bolsillo, entre ellos mi credencial de Delegado Secreto y algunos
apuntes, los entregué antes que me los pidieran. Después me quitaron las
botas, sospechando que en ellas escondía algún parte o reservada
confidencia. Iguales pesquisas hicieron en el sombrero.

Cuando el registro hubo terminado, el que parecía jefe de los tres que
conmigo estaban, me dijo en mal castellano: «Aquí quedarte a las
resultas de lo que contenga el contenido de estas papelorias.» Sin más
razones, reintegrado en el uso de mis botas, gabán y sombrero,
lleváronme por un pasillo de dos ángulos y me metieron en un aposento
cuadrilongo, donde vi, a la luz del consabido farol, por un lado un mal
avío de estera, jergón y manta, y al otro una silla. En tan regio
alojamiento me dejaron, recomendándome la paciencia con frases medio
vascas, medio castellanas, y salieron cerrando la puerta con dos vueltas
de llave y corriendo un cerrojo, que rechinó como risotada del Infierno.

Reconociendo aquel antro con fugaz mirada, pude apreciar en uno de sus
muros una reja que daba al campo. El techo era de bóveda, las paredes
renegridas, el suelo mitad de ladrillos, mitad de tierra. Mis pobres
huesos me pedían el descanso, y yo lo pedí para ellos y para mi cerebro
al hinchado jergón, que por ser de hoja de maíz tocó diferentes piezas
de música cuando en él me acosté\ldots{} Creo que de un tirón dormí todo
el día y la noche siguiente. Anidaban en mi cárcel el tedio, la tristeza
y la desesperación. Pero yo saqué del fondo de mi alma el caudal
recóndito de mi estoicismo para defenderme de las ideas negras.

Corrían los días, sustrayéndome con su lentitud somnífera la noción
exacta de su valor cronométrico. El único ser humano que me visitaba era
una diligente abuelita, que me traía mi alimento por mañana y tarde:
medio pan y una ración de rancho, no mal guisado, ni tampoco escaso. Mi
carcelera, que no carecía de espíritu de caridad, solía dolerse de mí
con palabras dulces y consoladoras dichas en una mixtura de vascuence y
castellano que me hacía mucha gracia. Un día, no sé si al tercero de mi
prisión, o al octavo o al quinto, me obsequió con estas frases que
traducidas copio: «Mire, señor; le voy a traer, si usted quiere, a un
curita del pueblo para que le vaya preparando.»

---¿Preparándome?\ldots{} ¿Para qué?

---No se asuste, señor. Nuestra fe nos manda que tengamos la conciencia
siempre muy limpia de alas para poder volar hacia Dios cuando éste lo
disponga. Nadie se ve libre de un torozón o de un súpito a la cabeza.
Por eso le digo: ¿qué pierde con estar preparadito?

Llamaban a mi guardiana \emph{Maribatista}, y era tan buena que de su
cuenta me llevaba bizcochos, higos pasados, o alguna otra golosina para
mi regalo.

La primera visita que me hizo el jefe de la Fortaleza no fue anterior al
décimo día de mi cautiverio, según mis imperfectos cálculos del curso
del tiempo. Entró en mi calabozo una mañana, regañando con áspero acento
a dos tagarotes que le acompañaron hasta la puerta: «¡Pero qué brutos
\emph{seis}!---gritaba.---¿No \emph{vus} dije que metierais aquí un
\emph{talburete}? ¿Queréis que el preso y yo hablemos \emph{asentados}
en una sola silla?» Pronto trajeron una banqueta, y al punto quedé solo
con el terrible fantasmón que en aquel instante disponía de mi suerte.
Era un viejecillo seco, de alta estatura, de manos sarmentosas. Si por
su habla y acento se me reveló como hijo de Castilla, por su edad
entendí que era un veterano de la primera guerra, reducido en la segunda
a ejercer funciones sedentarias.

Con rudezas de forma, tras de las cuales traslucí un fondo de humanidad
y cortesía, me dijo el viejo carlistón que mis papeles entrañaban prueba
plena de intentos alevosos contra la causa del Rey, intentos que sin
duda venían de muy alto, por lo cual, él y sus compañeros habían
decidido remitir todo el papelorio al General en Jefe, a fin de que éste
resolviera lo \emph{procedente} en caso tan grave. Añadió que aún estaba
yo vivo \emph{motivado a que} él no quería cerrar mi boca antes que
Lizárraga, Elío o Dorregaray metieran sus dedos en ella, para saber de
dónde venía aquella infamia de querer comprar a los jefes carlistas con
el judío dinero liberal.

«Pues lléveme usted---dije yo con viveza,---lléveme pronto a presencia
de uno de esos Generales, ante quien declararé, como ante usted declaro,
que soy inocente y pruebas tengo de ello.» La respuesta de mi cancerbero
fue indecisa, con un dejo de sorna castellana: el General era quien
había de decidir si se dignaba escucharme o si por primera providencia
debía yo ser pasado por las armas\ldots{} Ya me lo dirían \emph{para mi
conocimiento y efectos consiguientes}.

\hypertarget{xvi}{%
\chapter{XVI}\label{xvi}}

No me afligieron más de la cuenta estos siniestros augurios. Envuelto en
la toga de mi resignación, esperaba sereno las derivaciones probables de
mi cautiverio. Además confiaba en el auxilio de mi divina Madre, que
seguramente no me dejaría perecer a manos de aquellos bárbaros. Una
noche desperté arrebatado de súbito alborozo y salté del jergón creyendo
ver, viendo mejor dicho, el rostro inefable de \emph{Mariclío} asomado
entre los barrotes de mi reja carcelaria. Palabras fervorosas se
escaparon de mis labios, y oí claramente esta contestación de la excelsa
Señora, mil veces augusta:

«Nada temas, hijo: yo estoy al cuidado de ti. Imita mi paciencia, imita
mi serenidad ante estas guerras tan inverosímiles ¡ay!, como verdaderas.
Estamos dentro de un absurdo vestido de realidad, Carnaval sangriento.
Escribiremos una Historia que no será creída por los venideros, y al
leerla, si es que la leen, pensarán que hemos escrito cuentos
disparatados para educar a los niños en la barbarie y en la
imbecilidad.»

Al recostarme de nuevo en mi jergón, dilucidaba yo con vagas
cavilaciones si lo que había oído me lo dijo la Madre o me lo cantaron
las armónicas hojas de maíz, gimiendo bajo mi cuerpo\ldots{} Rodaron
días sin otra visita que la de la señora \emph{Maribatista}, amén de las
que me hacían de noche alimañas audaces, ávidas de aprovechar los restos
de mi pitanza. La viejecilla continuaba dadivosa y afable, y me
entretenía con amena charla mientras trajinaba en mi calabozo haciendo
una limpieza elemental. Rara vez al traerme la comida dejaba de añadir
alguna fineza, y una tarde me obsequió muy gozosa con un pedazo de
mazapán y un Niño Jesús de alfeñique, obra de las monjas vecinas.

Hecho a la soledad y a la meditación pasaba yo mis horas revolviendo el
copioso archivo de mi vida pasada, rememorando mis adversidades y
bienandanzas, trazando síntesis históricas para un libro que seguramente
no escribiría nunca, y comunicándome por la fuerza expansiva de mi
espíritu con seres que me habían divertido sin hacerme ningún daño:
\emph{Leona la Brava}, \emph{don Florestán}, \emph{Graziella}, José Ido,
sin olvidar las pedantescas figuras simbólicas de Doña Gramática y sus
vetustas compañeras.

Una noche, después de beberme una botella de vino blanco que a
hurtadillas me llevó \emph{Maribatista}, mi encendido cerebro me trajo
la visita de seres, que si eran vivos fuera de allí, no eran dentro de
mi calabozo más que simples fenómenos espectrales. El primero que entró
fue Serafín de San José, el cual, fieramente, tirándome de los pies como
para despertarme, me decía: «Si me hubieras traído contigo como Contador
y maestro de Partida Doble, no te verías como te ves. Con la mitad del
dinero que te dio el Gobierno para la compra de cabecillas, habríamos
dado la paz a España\ldots{} y con la otra mitad nos hubiéramos
divertido tú y yo lindamente\ldots{} Contando con este negocio ofrecí yo
a Cabeza un aderezo de brillantes\ldots{} Y ahora ¿qué aderezo le daré,
como no sea una ristra de ajos?\ldots{} ¡Ja, ja!»

Se me apareció luego \emph{Graziella}, dando el brazo a un bulto negro
en quien vi un esbozo de la figura de don Hilario. La diablesa, con
mirada burlona, se sentó junto a mí, produciendo en la paja del jergón
un ligero estallido de risa. «Para que salgas de estos trances, Tito
salado---me dijo,---voy a ponerte en el dedo del corazón el anillo de
Astaroth, hijo de Astarté, la infernal divinidad que yo reverencio.»
Sentí en efecto el roce del anillo al entrar en mi dedo. El informe
bulto negro tiró del brazo de \emph{Graziella}, y ambos salieron dejando
tras de sí los ecos o salpicaduras de una cháchara zumbona.

No fue aquella noche sino otra, cuando la ingestión de medio azumbre de
chacolí, obsequio de \emph{Maribatista}, me produjo la visión de un
espantable murciélago que se coló por la reja, y después de chillar
revoloteando junto al techo, se posó cerca de mí, deslumbrándome con sus
ojos de fuego. Era el propio \emph{don Florestán}, con su melena,
perilla y pómulos pintados. De su hocico ratonil escuché estas grotescas
manifestaciones: «Acabo de escribir al Séptimo Carlos una carta de su
abuelo don Carlos María Isidro, en la que le dice que afane para sí todo
el dinero que traes y te ponga en libertad, dejándose de más guerras y
nombrándote su Chambelán Honorario.»

En una de las siestas que yo comúnmente dormía, me fueron a ver
\emph{Leona} y \emph{Doña Gramática}. Díjome la primera que ya era
Duquesa de Mula, y que para evitar la fealdad de esta palabra, la
concesión del título decía: \emph{Duquesa de la Mula del Nacimiento}.
Había tomado a \emph{Doña Gramática} como aya o maestra del buen decir
para no hacer mal papel entre la grandeza\ldots{}

Segunda y tercera visita recibí del áspero Comandante castellano, y en
ambas no hizo más que repetir o parafrasear lo que me había dicho en la
primera. Una mañana fui sorprendido por bullicio de multitudes,
congregadas en el campo que rodeaba mi cárcel. Más tarde, oí pasos y
voces de tropas en acción. Sonaron tiros lejanos, algún tiro próximo, y
a esto siguieron chillidos de mujeres no lejos de la reja de mi
calabozo\ldots{} Pensé que de aquella batahola podría resultar mi
liberación; pero no fue así.

Al anochecer entró en mi celda el Comandante, seguido de tres
descomunales guerrilleros, notificándome que el General de la División
reclamaba mi persona, para someterme a un interrogatorio \emph{conforme
había lugar en justicia}.

«¿Puedo saber a dónde voy?»---le pregunté. Y él, rígido y seco, me
contestó repitiendo el cuento del loro: \emph{«Usted, seor Tito, irá
aonde ó leven.»}

Laconismo tan áspero me enfadó; pero el estoicismo selló mis labios.
Sacáronme al pasillo y del pasillo a la calle, donde vi grupos de
soldados que se iban a poner en marcha. Despidiome el Comandante con una
mirada lastimera y un saludillo militar. En cambio, los adioses de
\emph{Maribatista} fueron de ternura casi materna, con el aditamento de
unas lonchas de jamón y unos bollitos, que me dio envueltos en un número
de \emph{El Cuartel Real}. Ya que la pobre mujer no pudo darme noticia
del lugar a donde me llevaban, por ella tuve conocimiento del tiempo que
había durado mi prisión. Cincuenta y dos días estuve recluido en aquel
antro que, visto por fuera, se me representó cual un resto vetusto de
construcción feudal. Como apenas podía yo tenerme a causa de mi dilatada
inmovilidad, me metieron en un carro de víveres, atándome los pies para
que no me fugara.

Y aquí me tenéis otra vez, llevado por valles y montes hacia lugares
desconocidos, donde se decidiría la solución adversa o favorable que mi
Destino me deparase. La noche era fría y clara, con hermosa luna
creciente, cielo limpio, atmósfera de hielo. Un individuo de los que
custodiaban el carro tuvo lástima de mí y me cubrió con una manta de
munición. Al abrigo de ésta traté de adormecerme. Tocándome las manos y
las sienes aprecié en mí un estado febril, y ello fue causa de que la
pesada modorra me trajera visiones fraguadas en mi propia caldera
cerebral, imágenes absurdas que al desvanecerse no dejaron rastro en mi
memoria.

No sé decir a mis compasivos lectores en qué día y hora terminó el
suplicio de mi segunda caminata, conducido por amenos valles y verdes
montes en un convoy carlista. Sólo apunto que el sol alumbraba en el
cenit cuando paró la caravana. ¿A qué lugar de Vasconia me habían
llevado? No lo sabía. También ignoraba si el General que reclamara mi
presencia era Lizárraga, Mendiri, Dorregaray o Cástor Andéchaga, pues
estos cuatro nombres sonaron en mis oídos durante la penosa marcha.

Desatados mis pies, dos mozarrones me llevaron en vilo a un aposento
bajo, espacioso y mal oliente. Yo no podía moverme, debilitado por la
inanición y abrasado por la fiebre intensísima. En mi horrible turbación
pude hacerme cargo de que me hallaba en un improvisado Hospital de
Sangre. Así me lo revelaron gemidos, ayes dolorosos que a mi lado
sonaban\ldots{} Un hombre, que por las trazas era médico, se acercó a
mí, y después de reconocerme minucioso, ordenó que me arropasen con
mantas o capotes, prescribiendo brebajes de quinina y alimentación muy
moderada.

Desde la visita del \emph{físico} ceso en las referencias directas de mi
persona porque estuve privado de conocimiento en largos días,
conservando sólo un brumoso recuerdo de la horrenda sed, del amargor de
la quina, y del repugnante gusto de los caldos que me daban.

Cuando mis sentidos empezaron a recobrarse, pude advertir que muchos de
mis compañeros de Hospital se morían lindamente, y oí los azadonazos de
los que a la parte de afuera les cavaban la sepultura. Otros,
destrozados por las balas, venían a sustituir a los fenecidos\ldots{}
Mujeres, que parecían monjas por su parda vestimenta y luengos rosarios,
andaban entre nosotros con blando pisar de alpargatas. Eran enfermeras
bondadosas, calladas y solícitas.

Mi renacer a la vida fue un vertiginoso cavilar sobre la impía guerra
civil, monstruo nefando que sólo me mostraba sus extremidades dolorosas.
Dos Ejércitos, dos familias militares, ambas enardecidas y heroicas, se
destrozaban fieramente por un \emph{quítame allá ese trono y un dame acá
ese altar}. No era fácil decir cuál de estos dos viejos muebles quedaba
más desvencijado y maltrecho en la lucha. En sin fin de páginas de la
Historia del mundo se ven hermosas querellas y tenacidades de una raza
por este o el otro ideal. Contiendas tan vanas y estúpidas como las que
vio y aguantó España en el siglo XIX, por ilusorios derechos de familia
y por unas briznas de Constitución, debieran figurar únicamente en la
historia de las riñas de gallos. Así lo pensaba yo en aquellas horas
siniestras de mi vida, y así lo pienso todavía.

Ahora voy a dar a mis joviales lectores un plato de gusto, contándoles
que una mañana fui conducido por las blandas mujerucas y algunos
militares de indecisa graduación a una estancia del piso alto, ancha y
luminosa, donde me dieron alimento escogido para fortalecerme en mí
convalecencia. Diéronme también una cama bien mullida, y en derredor mío
vi un mediano ajuar de cómodos mueblecitos. Encontrábame allí como el
pez en el agua y mi sorpresa fue tan grande como mi alegría cuando un
vejete modoso y limpio, de porte un tanto sacristanesco, y una monja
gordita, risueña y algo cojitranca, me dijeron que ya no corría peligro
de ser fusilado. Por mi vida se interesaban personajes altísimos, y aun
damas y princesas. No necesito decir cuánto me holgué de aquel feliz
cambiazo en mi destino\ldots{} No riáis, parroquianos maleantes que
entretenéis vuestra ociosidad con estas lecturas, no riáis y esperad lo
que resta de mi cuento.

Mis nuevos guardianes no sabían qué hacer para facilitar de un modo
grato mi reparación orgánica. Menudeaban las comiditas sabrosas,
alternadas con tragos de confortantes licores. De añadidura, me asearon
y compusieron, poniéndome muy elegantito. Por efecto de aquel dulce
trato y de las cosas estupendas que pasaron ante mi vista, hube de
reconocer en mí el trastorno más delicioso y la ensoñación más bella que
yo había gozado en mi existencia de historiador y de poeta. A la hora de
comer presentáronme cierto día una linda mesa pulquérrima con todo el
aderezo de vajilla y cristalería que pide un yantar lujoso. Mandáronme
sentar en el único sillón colocado a la cabecera de la mesa. Frente a
mí, a bastante distancia, había un gran ventanal, y junto a él extensa
hilera de figuras femeninas cuyos rostros no podía distinguir por estar
ellas de espaldas a la vivísima luz del sol. La figura del centro, si no
era \emph{Mariclío}, se le parecía mucho.

Dada la señal de empezar la comida por mis guardianes, que permanecían
en pie detrás de mí, avanzaron hacia la mesa dos señoras de las que
formaban fila junto al ventanal. La una era la titulada reina doña
Margarita de Parma, esposa de Carlos VII; la otra doña Isabel II, que
aunque destronada conservaba su rango mayestático. Ambas señoras
recibían de manos del maestresala y de la monja los platos exquisitos y
me los servían con soberana gentileza\ldots{} Yo no sabía qué decir ante
tan inauditos honores, y por no estar callado repetí con turbada voz los
famosos versos: \emph{Nunca fuera caballero---de damas tan bien
servido\ldots{}}

Del grupo de las señoras, destacáronse otras para compartir con las
Reinas el honor de servirme: eran la Infanta doña María Isabel
Francisca, viuda de Girgenti, y doña Blanca, esposa de don Alfonso de
Borbón y Este\ldots{} Las Reinas y Princesas, así como las otras damas
que ponían ante mí los ricos manjares, retirando después los platos ya
vacíos, me agraciaban con sonrisas y donosos mohínes sin pronunciar
palabra.

Inmóvil en su puesto ante el ventanón permanecía la Madre \emph{Clío},
como presidiendo la escena de cuento infantil en que yo era estupefacto
protagonista. No pude contener mis ganas de conversación, y desde mi
sitial dirigí a la Madre estas regocijadas expresiones: «Te veo, Señora,
sin distinguir claramente tu semblante augusto. Pero aunque no te viera
te reconocería por el bromazo que me das, ordenando que me sirvan de
comer testas más o menos coronadas y altísimas Princesas de sangre real.
Ello es el signo fantástico de la soberana protección que concedes a tu
siervo humilde, indigno amanuense de tus sacros Anales\ldots»

¡Jesús, qué delirio! Por Júpiter y don Pedro Calderón, ¿soñar es
vivir?\ldots{} Dormí hondamente la mona, empalmando la tarde con la
noche, y a la siguiente mañana, apenas me vestí y acicalé, llegose a mí
con su blando andar de alpargatas mi monjita guardiana, y así me dijo:
«Un ayudante del Teniente General don Antonio Dorregaray ha venido con
el recado de que éste le espera a usted para conferenciar.»

---¡Ah, no sabía\ldots!---exclamé requiriendo mi gabán y sombrero.

---¿Pero no sabe que llegó anoche el General? ¡Pues poco ruido que
hicieron las tropas al distribuirse en sus alojamientos! ¿Nada oyó
usted? Claro\ldots{} ha dormido entre tarde y noche diez y ocho horas
seguidas\ldots{}

Las últimas palabras de la buena señora fueron para decirme que
estábamos en el valle de Luyando, y que corría la segunda quincena de
Abril. Inmediatamente salí con el ayudante, que me llevó por la
carretera, sorteando baches y montones de grava. A un lado y otro vi
soldados que ocupaban caseríos y tiendas de campaña. En corto tiempo
llegamos a un grupo de casas, entre las cuales se destacaba una con gran
portalada señorial guarnecida de escudos. La muchedumbre de oficiales
que vi al entrar, me indicó que aquél era el alojamiento del Teniente
General Dorregaray. Subimos al primer piso, y el ayudante me metió en
una estancia que parecía biblioteca, con alta estantería de nogal
bruñido por el tiempo.

Retirose el ayudante, después de decirme que esperase un momento, y a
los diez minutos de estar allí vi aparecer al caudillo carlista don
Antonio Dorregaray, cuyo semblante conocía yo por los retratos que en
aquella época prodigaban los periódicos ilustrados. Era un hombre
fornido, membrudo, de negra y espesa barba partida, despejada frente y
expresivos ojos. Desde el primer momento advertí en él cierta
benevolencia mezclada de curiosidad. Hízome sentar frente a sí, junto a
una mesa donde vi números de \emph{El Cuartel Real}, una escribanía de
cobre con plumas de ave mojadas de tinta, y algunos pliegos sueltos a
medio escribir. Presidía la estancia un retrato litográfico de Carlos
VII, montado en brioso corcel de flotantes crines, que lanzaba por
narices y boca los vahos espumosos de su fogosidad.

\hypertarget{xvii}{%
\chapter{XVII}\label{xvii}}

Inició el General nuestro coloquio con estas palabras corteses: «Días ha
que deseaba yo hablar un rato con usted. Antes de tratar de los papeles
que se le recogieron al ser detenido, debo decirle que han llegado a mí
referencias de su persona. Por Carlos Calderón, a quien usted conoce, sé
que es usted historiador de nota.»

---De afición no más, mi General---respondí con modestia.---Mientras
llega el caso de examinar los hechos históricos, me dedico a estudiar
los caracteres que los producen. Al venir aquí me traje el bosquejo de
la figura militar de V. E., y si quiere le daré una muestra de la
escrupulosa fidelidad con que hago mis investigaciones.

---Suprima tratamientos y siga.

---Nació usted en Ceuta en 1823, y a los doce años ingresó como Cadete
de Infantería en el Ejército de don Carlos María Isidro. Tenía usted el
empleo de Subteniente cuando se acogió al Convenio de Vergara, pasando a
prestar servicio activo en el Ejército Nacional. Con el mismo grado de
Alférez guerreó usted a las órdenes de Oraa y Espartero para someter a
los carlistas que aún asolaban el Maestrazgo. Se batió usted en el sitio
de Castellote y en la toma de Morella\ldots{} El 48 y 49, siendo ya
Teniente, operó usted contra la facción Montemolinista que se organizó
en las Provincias Vascongadas, y por sus méritos le hicieron Capitán. En
Julio del 54 se adhirió usted al movimiento de Vicálvaro, a las órdenes
del General don Leopoldo O'Donnell, y ascendió a Comandante. Dos años
después se le condecoró con la Cruz de San Fernando de primera clase por
su animosa conducta en las turbulencias que ocurrieron en Madrid. El 59
fue usted a la guerra de África en el Batallón de Alcántara, uno de los
que componían la brigada de vanguardia del Primer Cuerpo, mandado por el
General Echagüe. Tomó usted parte en las más lucidas acciones de aquella
guerra, y el 9 de Enero del 60 se le dio, a petición suya, el mando de
las fuerzas de presidiarios armados. En 4 de Mayo se le nombró ayudante
de campo del General de la división en la que servía, y en este puesto
logró usted el grado de Teniente Coronel.

---¡Oh, qué hermosa guerra!---exclamó Dorregaray, dilatando su espíritu
en remembranzas placenteras.---¡El Serrallo, Castillejos, Montenegrón,
Tetuán!\ldots{} Siga, siga.

---Después de la guerra de África hizo usted servicio de guarnición en
diferentes poblaciones, demostrando siempre sus grandes conocimientos en
Táctica, Ordenanza y Ciencia militar. Poseía usted, además de la Cruz de
San Fernando concedida en 1856, la de San Hermenegildo, que le fue
otorgada el 58, y otra de San Fernando de primera clase, que se le dio
por su bravo comportamiento en la batalla de Wad-Ras. El 62 se le impuso
el hábito de la militar orden de Santiago\ldots{} Vea usted, mi General,
qué bien enterado estoy de los méritos y servicios del Teniente Coronel
don Antonio Dorregaray hasta que, en los últimos meses del 68, sus ideas
le llevaron a ingresar de nuevo en el Ejército absolutista.

---Está muy bien, señor mío---dijo Dorregaray, reforzando los conceptos
con expresivas cabezadas.---Si completa usted el estudio de las personas
con el examen imparcial de los hechos, será usted un historiador digno
de tal nombre.

---Me falta decir que conozco y trato a muchos distinguidísimos
militares que fueron y son amigos de usted: los hermanos Pieltain, Primo
de Rivera (Rafael y Fernando), Martínez Campos, Pavía y Alburquerque,
Nandín y Moya, ayudantes de Prim, Echagüe, Zabala, y algunos paisanos
ilustres como el Marqués de Beramendi, el Barón de Benifayó\ldots{}

---Bien, basta ya---dijo el caudillo realista cual sin quisiera apartar
de sus ojos una nube de tristeza.---Tengo mis afectos repartidos en uno
y otro campo\ldots{} Pero dejemos esto, y vamos al asunto que motiva
nuestra conferencia. Los papeles de usted\ldots{} ese extraño
nombramiento de Delegado Secreto para someter por el soborno a los jefes
carlistas, paréceme monstruosamente falso por la enormidad del intento,
y verosímil por la perfección de la escritura. Conozco muy bien la firma
de García Ruiz, que conmigo se ha carteado más de una vez; las firmas de
Echegaray y del Director del Tesoro también me son conocidas, y por
tanto\ldots{}

Hube de interrumpir al caudillo, anticipándole mi sincera y leal
explicación de aquellos farandulescos papeluchos. Eran un bromazo que me
dio al salir de Madrid el más sutil calígrafo que existe en estos
reinos. A la objeción lógica que vi apuntar en los labios de mi sagaz
interlocutor, me adelanté diciéndole: «Naturalmente, se asombra usted de
que yo, conociendo la falsedad de estos papeles, los haya traído conmigo
al pasar del campo liberal al campo absolutista. Comprenderá usted mi
torpeza cuando se entere de que padezco desvaríos mentales, que alteran
temporalmente mi fiel apreciación de las cosas, y cuando de añadidura
sepa que salí de Madrid bajo la sugestión insana de una mujer histérica,
antojadiza y atrabiliaria, que me hacía ver lo blanco negro\ldots»

---Ya, ya. ¿Hembra tenemos? ¡Malo, malo!---exclamó don Antonio,
conteniendo la risa y sacando del bolsillo del pecho los documentos de
autos.---Entre los papeles del señor don Proteo Liviano hay un
plieguecillo, escrito con lápiz en letra de mujer bastante garabatosa,
que dice así: \emph{Pesquemos primero a los pájaros gordos}. \emph{A
Dorregaray 50.000 duros\ldots{}} \emph{A Cástor Andéchaga 25.000\ldots{}
etc}.

---Me parece que con ese ridículo apunte de la dama estrambótica que me
acompañaba queda bien clara mi inocencia, y donde digo inocencia ponga
usted tontería o flaqueza mental.

Antes de que me lo preguntase le di cuenta de mis amores con
\emph{Chilivistra}, del endiablado carácter de ésta, no bien conocido
hasta que juntos emprendimos el viaje, de las querellas y ruidosas
trifulcas que nos separaron, largándose ella con mil demonios a no sé
dónde y cayendo yo en horrible cautiverio por más de dos meses, del cual
me sacó la magnanimidad del hombre generoso en cuya presencia estaba.

«Por lo que aquí hemos hablado---dijo Dorregaray,---y por los nuevos
informes que de usted me dio esta madrugada Carlos Calderón al partir
para Miravalles, queda usted indultado, señor don Proteo.»

En este punto se levantó, y rompiendo en cuatro pedazos los mágicos
documentos que me acreditaban como corruptor de caudillos facciosos en
el campo inmenso de la fantasía, los arrojó en el suelo con ademán
desabrido\ldots{} Creyéndome libre le pedí licencia para retirarme. Pero
él, deteniéndome con un gesto, me indicó que aún faltaba algún rabito
por desollar hasta poner término a nuestra entrevista.

«Ya sabe usted---me dijo---que hemos puesto sitio a Bilbao. Esta plaza
tan importante no tardará en ser nuestra. Ahora no se nos escapa como
se---¡Oh, qué hermosa guerra!---exclamó Dorregaray, dilatando su
espíritu en remembranzas placenteras.---¡El Serrallo, Castillejos,
Montenegrón, Tetuán!\ldots{} Siga, siga. nos escapó en los famosos días
de Luchana\ldots{} Sabrá usted también que Serrano y Concha embarcaron
en Santander para Castro Urdiales, y piensan atacarnos por las líneas de
Somorrostro.»

---Es la primera noticia que tengo de eso, mi General. Soy un pequeño
historiador que ignora la Historia viva que le rodea.

---¿Y tampoco sabe usted que con Serrano y Concha vienen Primo de
Rivera, Martínez Campos, Tassara, Echagüe y otros amigos míos\ldots?

---¡Qué he de saber, pobre de mí, si me han tenido ustedes más de dos
meses encerrado en Yurre y en Luyano!

---Pues si está usted a obscuras de todo lo que pueda interesarme---dijo
Dorregaray un tanto malhumorado,---quédese en libertad y tome la
dirección que más le convenga.

---Considere, mi General, que adonde quiera que vaya tendré que pasar
por entre tropas carlistas, y si éstas han de volver a encarcelarme
prefiero que sea usted el que disponga de mi suerte, llevándome consigo.

---Me refiero yo, señor Liviano---indicó don Antonio con un dejo de
socarronería,---que usted, hombre un tanto alocado y de imaginación que
tira siempre a los desvaríos, querrá irse con los suyos, que a estas
horas andarán por los vericuetos de Somorrostro. Yo le daré un
caballejo, unas alforjas con víveres y salvoconductos para que vaya
usted hasta Valmaseda, franqueándose de las tropas de Cástor Andéchaga o
Lizárraga, únicas que puede usted encontrar en ese camino. Desde
Valmaseda póngase usted en manos de la Providencia y de sus santos
tutelares para llegar a donde estén los suyos, a quienes tengo por tan
alocados y fantasiosos como usted. Dios se la depare buena\ldots{} Otra
cosa: si se tropieza usted con Arsenio Martínez Campos dígale que le
espero\ldots{} donde él verá. Adiós, amigo.

Con todo rendimiento me despedí de aquel hombre que tan gallarda y
generosamente se había portado conmigo. Para colmo de bondad cumplió al
instante su oferta, proporcionándome un caballo con alforjas a la grupa.
En ellas, junto con los víveres, acomodé mi ropa, desembarazándome del
estorbo de la maleta. El mismo ayudante que me llevó a presencia del
General, me entregó dos salvoconductos, en cuyas márgenes había trazado
Dorregaray expresivas líneas recomendándome a Lizárraga y Andéchaga.
Ganoso de aprovechar el tiempo despedime de mis buenos guardianes, y
entre alborozado y medroso partí hacia el valle de Llanteno, dirección
que me indicaron como la más fácil para llegar a Valmaseda.

No quiero entreteneros con pormenores de mi caminata, en la cual nada de
particular me ocurrió. Al otro día, cerca de Santa Coloma, encontré
tropas de Andéchaga. Hablé con el veterano cabecilla, que me acogió
hidalgamente, invitándome a seguir en su compañía. Así lo hice, y en el
lugar de Antuñano, el guerrillero me indicó la ruta más breve para
llegar a Valmaseda, donde quizás encontraría tropas e Lizárraga. Mi
jaco, que era una buena pieza, me llevó en algunas horas a la capital de
las Encartaciones, donde tuve la suerte de no topar con la facción de
Lizárraga y sí con un buen almuerzo caliente que me restauró de cuerpo y
espíritu. Eran las diez de la mañana de un día de Abril, cuyo número
estaría seguramente en los almanaques, pero no en mi flaca memoria.

Después de dar a mi valiente rocín el descanso y pienso que se le
debían, me lancé a la ventura por un camino que a mi parecer al
encuentro de Serrano y Concha me llevaba. La Providencia iba conmigo.
¿Iría también invisible mi excelsa Madre? Dígolo porque unos aldeanos, a
quienes pregunté si me había equivocado en el camino de Múzquiz, me
respondieron: «Va bien, señor; tuerza por la carretera que encontrará
pronto a mano derecha, y todo seguido llegará, si le dejan \emph{los
cristinos} que andan por esos montes.»

Seguí la indicada ruta, y al meterme en las encañadas de una sierra (que
según después supe se llama de Saldoja) me vi sorprendido por una
turbamulta de soldados carlistas a pie y a caballo, que en veloz
retirada venían hacia Valmaseda. Eran sin duda los vencidos en un
reciente combate. Sus caras atristadas, su andar presuroso, las
inflexiones de su lenguaje vasco, que unidas al ademán resultaban
inteligibles, me revelaron que iban en humillante fuga. Algunos me
injuriaron, en otros advertí una hostilidad nada tranquilizadora. Tuve
miedo de que, por lo menos, me quitaran el jaco, ya que no descargasen
en mi propia persona la rabia de su vencimiento\ldots{}

Cuando pasaban los últimos de la dispersa manada, mi buena suerte me
deparó a la derecha del camino una venta o parador. Picando espuelas a
ella me fui, con ánimo de guarecerme por si venía nuevo tropel de
guerreros desmandados. En la venta sólo había dos mujeres, las cuales, a
mis primeras palabras en demanda de hospitalidad me contestaron en
purísimo castellano y con acento muy cortés. Eran de Castro Urdiales,
hija y madre, y estaban solas porque los dos hombres de casa habían
tenido que ir con sus carros, llevados a la fuerza, a portear víveres y
municiones en un convoy de Mendiri.

«¡Ay, señor!---me dijo la más joven.---Desde ayer, por todo el terreno
de aquí a Somorrostro, en los altos de Las Muñecas y en la parte de
Montellano, no han cesado los tiros de fusil y los zambombazos de la
Artillería. Todavía hay para rato y no se sabe quién lleva las de
perder. Ha venido de Madrid, según dicen, un General que llaman el de la
Concha con otros tales. El Serrano parece que ahora va por delante.
¡Menudas trapatiestas vamos a tener, señor!»

La vieja, que con mirada de águila exploraba las lejanías, saltó
diciendo: «Me paiz que al carlista le zurran la badana. Hacia aquí
vienen algunos más huyendo de la quema. Por la encañada de allá abajo
veo un montón de ellos, \emph{espavoridos}, que corren buscando la
vuelta de Güeñes. Señor; si es usted moro de paz, puede guarecerse en el
pajar hasta que pase esta tremolina. Comida no tenemos, como no sea un
poco de cecina que asaremos en las brasas. Vino sí lo hay, y no faltan
cerezas en aguardiente.»

Cuando esto decía la buena mujer, arreció de un modo espantable el
tiroteo, y distinguíamos el humazo de los disparos como blancos vellones
que surgían incesantemente en los términos remotos. Quedeme en relativa
tranquilidad al abrigo del ventorro, y al amparo de aquellas tal vez
encantadas princesas, que así curaban de \emph{mi rocino} como de mi
humilde persona. Todo aquel día duró el estampido de las lejanas
batallas. La ventera más joven me señaló diferentes puntos de donde
venía ruido de volcanes en erupción, entre ellos unos picachos que a
mano derecha y a larga distancia se parecían, donde el humo de la
pólvora formaba espesa nube.

Relacionando días después aquella visión con lo que en el campo liberal
me contaron, vine a comprender que mi ventera me había señalado, sin
saberlo, el formidable paso del General Concha por los desfiladeros de
Las Muñecas. Como he dicho, todo el día siguió el tremendo chocar de
ambos ejércitos, y durante la noche, agazapado en el pajar, oí
distintamente el zumbido aterrador de los carlistas en retirada por los
caminos y veredas colindantes.

El día siguiente amaneció cerrado de nieblas. Desde muy temprano empezó
el fuego de fusilería y cañón. Salí de mi escondite, advirtiendo que el
ruido bélico se extendía marcadamente hacia mi derecha. Nada se veía.
Pedí a la mesonera anciana noticia de los lugares que la niebla
blanquecina en aquella dirección ocultaba, y me dijo: «Lo más cerca por
ese lado es Avellaneda; luego sigue Galdames de Suso y de Yuso; después
Abanto, y al cabo Portugalete.»

Arreció el rumor de batalla conforme avanzaba el día. Por la tarde
llegaron al parador dos viejos, con la noticia de que los carlistas
habían sido destrozados y de que el Ejército \emph{Cristino} también
tenía muchas bajas\ldots{} Horas después vimos que por una loma distante
pasaban de izquierda a derecha tropas que parecían liberales. No
pudiendo contener mi curiosidad impaciente enjaecé mi caballo, y
despidiéndome de las bondadosas mujeres, me lancé a buen trote en la
ruta que me pareció conducente al lugar de Avellaneda\ldots{} Antes de
anochecer me encontré cerca de los míos. Alegría retozona inundó mi
alma. Metiéndome entre ellos reconocí el Regimiento número 38,
\emph{León}.

\hypertarget{xviii}{%
\chapter{XVIII}\label{xviii}}

Al instante me puse al habla con los soldados que consideraba como mi
familia política y militar. Entre los oficiales reconocí a un joven
Teniente, sobrino de don Romualdo Palacios, el cual me dijo que las
divisiones de Letona y Martínez Campos estaban ya cerca de Portugalete,
pues las líneas carlistas habían sido forzadas y el enemigo, poniendo
pies en polvorosa, dejaba libre el camino de Bilbao. Descansamos algunas
horas en Avellaneda, y al salir de madrugada con el mismo Regimiento, vi
el suelo, a un lado y otro del camino, sembrado de cadáveres. A las
cuatro horas de marcha oí de nuevo fuego lejano. Dijéronme que hacia
Galdames de Suso se estaban batiendo todavía. Encontramos tropas que
creo eran de la retaguardia de Martínez Campos. Muchos hombres se
ocupaban en enterrar muertos. Era un espanto, un horror. ¿Y esto para
qué? ¿Qué finalidad tenían aquellos cruentos combates, con sacrificio de
tantas vidas generosas? Luego os diré, lectores de mi alma, las ideas
que empezaron a bullir en mi mente al presenciar la pavorosa escena.

Entre los oficiales que dirigían los enterramientos encontré a
Palazuelos, aquel Teniente que en Miranda facilitó mi viaje a Vitoria
con la enfadosa \emph{Chilivistra}. Abrazándome me dijo: «De
\emph{Puerto Rico} he pasado a \emph{Saboya} número 6, y aquí me tiene
usted, en la División de Martínez Campos.» Aquella misma tarde, pasado
Abanto, Palazuelos y dos oficiales más, despachando juntos y aprisa un
ligero tente-en-pie, me hicieron una descripción sintética de las bravas
acciones que franquearon el paso hacia la ría de Bilbao. Contáronme la
muerte de Andéchaga y el audaz movimiento del Marqués del Duero por la
cumbre de Las Muñecas, que envolvió al enemigo atacándole de flanco
hasta ponerle en dispersión presurosa.

Según el relato de aquellos amigos, las pérdidas nuestras habían sido
dolorosas. Mucho más lo fueron las de los carlistas. Los cadáveres eran
como jalones que marcaban el paso de la Historia en aquellos trágicos
días\ldots{} Amaneció el 1.º de Mayo, día feliz en concepto de los
liberales. Colocado yo en un altozano próximo al lugar de Cabreces,
viendo a nuestro Ejército en el término de aquella jornada truculenta,
lancé al aire vago y a los vapores de la tierra ensangrentada
pensamientos que si entonces tenían algo de profético, luego se
resolvieron en una apreciación clara y justa de la hispana vida. Sin
duda me inspiraba la Madre, cuyo aliento fecundo penetró en mi cerebro;
sin duda la Madre augusta me sugirió después el criterio clarísimo con
que, andando el tiempo, he podido juzgar los sucesos que entonces
vi\ldots{} Escribo estas líneas cuando el paso de los años y de
provechosas experiencias me ha dado toda la claridad necesaria para
iluminar el 2 de Mayo de 1874.

Ved aquí lo que pensaba y pienso: liberales y carlistas se desgarraron
cruel y despiadadamente por dos ideales que luego han venido a ser uno
solo. ¿Cabe mayor imbecilidad de una parte y otra? Los liberales
derramaban a torrentes su sangre y la sangre enemiga sin sospechar que
entronizaban lo mismo que querían combatir. Los carlistas se dejaban
matar estoicamente, ignorando que sus ideas, derrotadas en aquella
memorable fecha, reverdecerían luego con más fuerza de la que ellos, aun
victoriosos, les hubieran dado.

El 2 de Mayo, la suerte me deparó el honor de acompañar al General
Concha cuando en un vaporcito entró por la ría de Bilbao hasta llegar al
casco de la ciudad, recién liberada de un sofocante asedio. No puedo
describir el júbilo del vecindario. Era una locura, un delirio. Las
aclamaciones abrasaban el aire, infundiendo en las almas el fuego de una
nueva vida. Bilbao creía que inauguraba una era de grandeza nacional, de
cultura, de emancipación del pensamiento, de todo cuanto podían dar de
sí la pujanza mental y la nativa riqueza de aquel pueblo. Al recordar
hoy los sublimes momentos de aquel día, ayes de gozo, alaridos de
esperanza, me parece que oigo burlona carcajada del Destino. Sí, sí;
porque la Restauración primero, la Regencia después, se dieron prisa a
importar el jesuitismo y a fomentarlo hasta que se hiciera dueño de la
heroica Villa. Con él vino la irrupción frailuna y monjil, gobernó el
Papa, y las leyes teñidas de barniz democrático fueron y son una farsa
irrisoria.

Los desdichados carlistas, que entonces lloraron su retirada, vinieron
luego a instalarse sin rebozo en la ciudad opulenta, y a dar en ella
carta de naturaleza a las ideas sombrías que no pudieron imponer con las
armas. Pero si el hierro vizcaíno ha servido para forjar las cadenas que
cercan la vida de un pueblo llamado a influir derechamente en la
reconstrucción de España, también las almas oprimidas recibieron del
acero la dureza y temple con que han de romper algún día el asedio moral
que les ha puesto la barbarie\ldots{} Hablando de esto no hace mucho, la
excelsa Madre me dijo: «Tito del alma: aquellas peleas que viste el 74
fueron juego y travesuras de chicos mal criados.»

Pasados los ruidosos alegrones del 2 y 3 de Mayo en la invicta Villa, me
instalé en Portugalete, acomodándome en la propia casa donde se alojaban
el Teniente Palazuelos y otros amigos de \emph{Saboya} y \emph{Ciudad
Rodrigo}. En aquel período de descanso menudearon las comilonas en
diferentes sitios próximos a la ría, pues ya se sabe que donde hay
bilbaínos no pueden faltar las alegres cuchipandas campestres. En una de
éstas me contaron (no respondo de la veracidad) que los Generales
afectos a la dominación borbónica propusieron a Concha la proclamación
del Príncipe Alfonso, como el mejor entretenimiento para pasar el rato.
Mala cara puso el General en Jefe al oír tal despropósito, y aun se dijo
que reprendió ásperamente a los que con tanta prisa querían atropellar
los acontecimientos\ldots{}

El 13 de Mayo, bien presente tengo la fecha, emprendimos la
marcha\ldots{} El General Concha, con noble ardimiento, quería llevar la
guerra a lo que él llamaba \emph{el corazón del carlismo},
Navarra\ldots{} Acompañando a los de \emph{Saboya} me puse en camino,
montado en el trotón que me dio Dorregaray. Mi cabalgadura, con el largo
descanso y los buenos piensos, iba tomando trazas de corcel brioso y era
la envidia de mis amigos. Éstos, con graciosa burla, le pusieron el
nombre de \emph{Babieca}. Por la misma ruta que yo había traído fuimos
con otros muchos Regimientos y Batallones hasta Valmaseda, donde
pernoctamos. Al día siguiente recorríamos el Valle de Mena hasta Bercedo
y Medina de Pomar.

No describiré los movimientos de la numerosa hueste que llevaba consigo
don Manuel de la Concha\ldots{} Sólo diré que de Medina de Pomar
marchamos a Villasante y desde allí seguimos por el Valle de Tobalina,
orilla izquierda del Ebro, en dirección de Sobrón. Interpretando mal el
pensamiento de nuestro General pensé que nos llevaba a Miranda. Pero no
fue así. Desde Puentelarrá fuimos a Salinas de Añana; allí supe que
Concha, al frente de una división, había entrado en Orduña, donde impuso
un fuerte tributo, volando después la fábrica de pólvora. El 18 de Mayo,
se reunieron en Nanclares las diferentes fuerzas de aquel Ejército. El
19 estábamos todos en la capital de Álava.

En los cinco o seis días que pasé en Vitoria ocurrieron acontecimientos
históricos de extraordinaria importancia, y me apresuro a referir el que
estimo de mayor interés: mi repentino encuentro con la destornillada
mujer a quien los Anales de \emph{Clío} dieron el claro nombre de
\emph{Chilivistra}. Iba yo por la calle de la Zapatería, abstraído en
vagos pensamientos, cuando un siseo que no podía confundir con ninguna
otra expresión humana me obligó a detenerme. Era ella, ¡Dios!\ldots{}
Hacia mí vino presurosa, alargando los brazos como para estrecharme en
ellos. ¿Qué había de hacer yo? Dejarme abrazar, dejarme besuquear,
recibiendo en el rostro su saliva y sus lágrimas, y oír estas lastimeras
voces entremezcladas de amargor y dulzura:

«¡Ay, Tito de mi vida, lo que habrás sufrido!\ldots{} Cuéntame\ldots{}
¿Has estado preso en el campo carlista?\ldots{} Culpa mía fue tu
desgracia\ldots{} ¡Perdóname!\ldots{} Muy mal me porté contigo, lo
reconozco\ldots{} ¡Ay; cuando te cuente yo mis infortunios verás a qué
pruebas tan duras me ha sometido el Señor!\ldots{} ¡Oh, qué dicha
tenerte a mi lado!\ldots{} Hace días que no ceso de pedir a la Virgen
Santísima me conceda el favor inefable de recobrarte\ldots{} La Virgen
me ha oído\ldots{} y aquí estás\ldots{} aquí te tengo\ldots{} Dime tú
ahora: ¿has venido con ese Concha?\ldots»

Los atropellados conceptos de Silvestra no tuvieron fin hasta que accedí
a llevarla conmigo, colgada de mi brazo, por las calles curvas de la
ciudad vieja. Observé en \emph{Chilivistra} una desdichada
transformación de la persona en lo tocante a la vestimenta y aliño del
rostro. Venía mal trajeada, el cabello en desorden, ojerosa, revelando
el descuido de las artes de tocador con que acicalar y componer solía su
faz bella. Lo primero que me dijo al sosegar su ánimo fue que acababa de
salir del convento de las Brígidas, donde había permanecido tres semanas
en durísimos ejercicios espirituales, con toda la severidad de ayunos y
mortificaciones y el sin fin de rezos que le fueron impuestos por su
confesor. La causa de estos rigores me refirió en seguida con la
tranquilidad propia de un alma cristiana. Había sufrido tan áspera
penitencia para limpiar su alma de los pecados más graves a que nos
induce la humana flaqueza.

«¡Ay, Tito adorado!---prosiguió parándose frente a los pórticos de la
Colegiata de Santa María.---Entremos en la casa del Altísimo y en ella
te contaré\ldots{} Quiero que seas mi segundo confesor\ldots»

En la cavidad obscura del templo, Silvestra me guiaba como lazarillo,
pues mis ojos deslumbrados por la luz solar nada veían. Ella, como rata
de iglesia, iba fácilmente de una parte a otra en el recinto tenebroso.
Nos sentamos en un lustroso banco bajo el coro. En el fondo de la nave y
en alguna capilla distinguí macilentas luces, que con el tintineo de
campanillas me indicaron que había misa en algunos altares. Como
\emph{Chilivistra} había oído ya tres, puso más atención en mi persona
que en el Santo Sacrificio.

«Te contaré mis ansias---me dijo con susurro,---sin ocultarte los
horrendos pecados que me han traído a esta tribulación. Todo lo sabrás.
No quiero tener secretos para mi Tito, que es bueno, indulgente, y sabe
perdonar\ldots{} Pues verás: estuve unos días en Durango, otros en
Elanchove, donde me ocurrieron cosas que hoy tengo por secundarias y te
las contaré después. Vamos a lo principal, vamos a lo gordo. De mi
tierra me vine aquí, atraída por la amistad de mis parientes los
Baraonas, y al mes de estar en Vitoria haciendo vida de recogimiento y
devoción, conocí a un sujeto que dio en acosarme y perseguirme con
requerimientos amorosos. En todas las casas conocidas, así los Romarates
como los Trapinedos y los Prestameros, me lo encontraba. Es un hombre
que ya pasó de la juventud y aún no está en la madurez de la vida, muy
pulcro y atildado, de trato finísimo y palabra dulce y sonora, como
nacido en el riñón de Castilla, Ávila, patria de Santa Teresa de Jesús.»

---Y ese señor tan finústico---dije yo, poco interesado en aquella
historia,---¿será también místico y extático como su paisana?

---No te diré que sea místico---prosiguió \emph{Chilivistra},---pero de
palabritas devotas y de lindas frases tocantes a la Santa Religión, y
aun a la misma Teología, se valió el muy tuno para cortejarme\ldots{} No
te rías\ldots{} El buen señor estaba desatinado por mi frialdad y
resistencia. Me esperaba en la calle, y andando junto a mí, en voz baja
me decía cosas\ldots{} ¡Ay, Tito, qué cosas!\ldots{} La verdad\ldots{}
tiene el hombre una imaginación, una labia, un modo de expresarse
que\ldots{} vamos\ldots{} Yo, muerta de vergüenza, callaba y me ponía
muy colorada\ldots{} Una tarde me llevó a la Florida y nos internamos en
los paseos más reservados.

---Vaya, mujer, acaba pronto. ¡Tantos rodeos para venir a parar
en\ldots!

---Si el hombre se hubiera mantenido en el terreno del amor puro, o como
quien dice platónico, menos mal. Pero buscaba el melindre, quería
llevarme a la deshonestidad, al desenfreno, a la impureza\ldots{} Una
noche, paseándonos por la Plaza, sentía yo mucha sed porque había comido
bacalao asado\ldots{} Llevome a una Cervecería para que
refrescáramos\ldots{} ¡Ay, perdóneme Dios el mal pensamiento!\ldots{} Yo
creo que aquel hombre me echó en la copa de cerveza una droga
endiablada, incitativa y \emph{calórica}, que me trastornó por completo.

---En fin, que\ldots{}

---Sí, hijo, sí\ldots{} ¡Qué desgracia, ay!\ldots{} Como él es viudo y
vive solo, iba yo a su casa\ldots{} De este desvarío, que fue sin duda
obra del Enemigo Malo, resultó para mí el bochorno que puedes
imaginarte\ldots{} Todo el pueblo se enteró. Los Baraorias, los
Trapinedos, los Prestameros, los Romarates\ldots{} ¡ay!\ldots{} me
dieron de lado\ldots{} Ahora que conoces mi mal, Tito mío, te diré lo
que ha de causarte admiración y espanto. Aquel hombre que me arrastró al
pecado con maleficio y artes corruptoras es\ldots{} ¡asómbrate,
Tito!\ldots{} es el Administrador de Rentas de Vitoria.

Antes que compadecer a \emph{Chilivistra} sentime inclinado a reírme de
su simplicidad. Mi estupor subió de punto cuando me dijo, cambiando el
tono patético por el que familiarmente usamos en los negocios:
«Comprenderás que con Eulogio Mentirola, que así se llama el asaltador
de mi virtud, hablé de tu Delegación Secreta, y más de una vez me dijo
que tiene orden de pagar los libramientos y espera que tú vayas a
cobrarlos. Ya lo sabes. Si quieres, yo te llevaré a su casa o a su
oficina, identificaré tu persona y\ldots»

Para mi sayo me dije: «Esta mujer está loca rematada y lo mejor que
puedes hacer, Tito, es poner tierra por medio.» Y en alta voz proseguí:
«Pero tú, después que el confesor te sacó de ese oprobio y con la
penitencia y los ejercicios espirituales en las Brígidas has restaurado
tu pureza, ¿vuelves a caer en las garras del espíritu maligno?»

---¡Ay, hijo\ldots{} si supieras! Él me persigue, me acosa, no me deja
vivir\ldots{} Anhelo ser buena y no puedo\ldots{} Pero esto acabará, si
tú quieres, Titín. Decídete: te presentas a Mentirola, cobras el primer
libramiento y yo, aquí donde me ves, estoy dispuesta a ir contigo para
tender el anzuelo a Dorregaray\ldots{} Ya te dije que ése es el primero
a quien debes enganchar\ldots{} En Oñate le tienes: me consta.

Comprendiendo ya que la enajenación mental de la pobre Silvestra no
tenía remedio, la compadecí de veras. Díjome que vivía con la familia
del Capellán de las Brígidas y que a la mañana siguiente me visitaría en
mi hospedaje, fonda de Pallares. Dicho y hecho: estaba yo vistiéndome
cuando se metió en mi cuarto, y con lenguaje atropellado y febril, viva
expresión de su demencia, repitió la enmarañada historia: el
Administrador\ldots{} el libramiento\ldots{} los cincuenta mil
duros\ldots{} Oñate\ldots{} Dorregaray\ldots{}

Fingiendo pesadumbre le dije: «Hoy no puede ser. Dejémoslo para dentro
de unos días. ¿No sabes lo que pasa? Tenemos interceptado el camino de
Aránzazu y Oñate. Dorregaray, que ha sustituido a Elío en el mando en
jefe del Ejército carlista, ocupa los altos de Arlabán. Hoy saldrán de
aquí fuerzas considerables que manda Concha para batir a don Antonio si
se atreve a bajar al llano.» A esto añadí el socorrido embuste de que
tenía que unirme inmediatamente al Cuartel General de Concha: Don Manuel
me había llamado con urgencia, y tal y qué sé yo. De esta suerte logré
despachar a la pobre mujer, cuyo desconcierto cerebral influía, sin
darme cuenta de ello, en mi nada segura imaginación.

Oprimiendo los lomos de mi \emph{Babieca}, salí con la columna del
General Martínez Campos, una de las tres que mandó Concha al
reconocimiento de Arlabán. Fuimos hacia Arriaga y Urrúnaga, que los
carlistas abandonaron tras un ligero tiroteo. Echagüe se llegó por la
izquierda hasta Ulibarri Gamboa. Por el centro, otra columna avanzó
hasta Villarreal, al mando de no sé quién. Se vio claramente que
Dorregaray no aceptaba la batalla, permaneciendo en las alturas con sus
doce batallones. Al día siguiente, cuando regresábamos a Vitoria,
hervían en mi pensamiento las consideraciones escépticas que desde la
liberación de Bilbao formaban mi criterio sobre aquellas vesánicas
campañas.

En las alturas de Arlabán teníamos a Dorregaray, que empezó su carrera
en el absolutismo, y después de servir con gloria y provecho en el
Ejército liberal, volvió a la liza bajo las banderas de don Carlos. En
el llano de Álava, se agolpaban armados hasta los dientes los que
compartieron con don Antonio las fatigas de la guerra de África y de las
contiendas familiares del liberalismo. Habían sido amigos: lo serían
siempre\ldots{}

Con sutileza de imaginación introducíame yo en el cerebro del de arriba
y de los de abajo, y encontraba la percepción de un solo ideal. ¿Qué
querían, por qué peleaban? Debajo del emblema de la soberanía nacional
en los unos y del absolutismo en el otro, latía sin duda este común
pensamiento: establecer aquí un despotismo hipócrita y mansurrón que
sometiera la familia hispana al gobierno del patriciado absorbente y
caciquil. En esto habían de venir a parar las mareantes idas y venidas
de los Ejércitos, que unas veces peleaban con saña y otras se detenían,
como esquivando el venir a las manos.

Discurría yo, metido en las entendederas de aquellos hombres, que si por
el momento no era lógico el acuerdo entre ellos, no tardaría el tiempo
en dar realidad a mis maliciosas conjeturas. Concluirían por hacer
paces, reconociéndose grados y honores como en los días de Vergara, y la
pobre y asendereada España continuaría su desabrida Historia dedicándose
a cambiar de pescuezo a pescuezo, en los diferentes perros, los mismos
dorados collares.

\hypertarget{xix}{%
\chapter{XIX}\label{xix}}

Mayor interés que los toques proféticos que acabo de colocar a mis
lectores tiene en la Historia la noticia siguiente: cuando a partir
hacia Logroño me disponía, con el grueso del Ejército de Concha, volvió
a presentárseme \emph{Chilivistra}, ya restituida felizmente a su
prístino estado de compostura y arreglo personal. No era ya la figura
luctuosa, mísera y lastimera de los días anteriores. En su rostro
advertí los discretos afeites que comúnmente usaba. Venía risueña,
aliviada o quizás totalmente restablecida del dolor en que la
sumergieron sus deslices escandalosos con el Administrador de Rentas.
¿Fue todo ello una farsa, un caso más de las aberraciones histéricas?
Las personas atacadas de este mal inventan historias lúgubres,
aflictivas, y acaban por creérselas.

El lenguaje y actitud de la que fue mi costilla falsa eran de una
perfecta tranquilidad de espíritu, con ráfagas de alegría. Habíase
colocado de nuevo en el terreno de sus primitivos afanes, y ansiaba
continuar conmigo la odisea romántica en busca del errante marido y de
la inocente criatura. No quise contrariarla por temor a que saltase de
la mansedumbre a la cólera, mostrando una vez más el labio temblicón que
tanto miedo me inspiraba. Con buenas palabras la entretuve, y
acompañándola hasta su casa, allí la dejé asegurando que volvería por
ella. Mi vuelta fue la del humo\ldots{} Apresuré mi partida para
librarme de aquella desdichada cuyos desvaríos morbosos no podía yo
remediar, y me agregué a las primeras fuerzas que salieron en dirección
a la Rioja. Iba con el temor de que Silvestra se lanzara en mi
seguimiento, y adelanteme todo lo posible fiado en que, confundido entre
las tropas, no podría fácilmente encontrarme la que había venido a ser
enemiga de mi tranquilidad.

En Logroño supimos que los carlistas, rehaciéndose con tenaz esfuerzo
del descalabro de Bilbao, reorganizaban y fortalecían sus huestes para
salir al encuentro de Concha, en Navarra. Faltos de recursos, apelaban a
la munificencia de las Diputaciones Forales y al patriotismo de los
realistas pudientes; esquilmaban a los pueblos, y decididos a no
perdonar medio alguno para adquirir dinero, llegaron al extremo
increíble de afanar los fondos de la Santa Cruzada. Sin hacer caso del
Obispo, que puso el grito en el cielo al tener noticia de la exacción
sacrílega, conminaron a todos los párrocos a que aflojaran sin demora
\emph{los parneses} de la Bula, alegando que se trataba de defender la
Religión y que ya ajustarían ellos sus cuentas con el Papa.

En tanto, a espaldas de Concha surgían diferentes cabecillas aguerridos
y ligeros de pies, que asolaban las tierras de Burgos, Palencia y
Santander, mientras otros se corrían hacia el Alto Aragón.
Tranquilamente organizaba nuestro General en Jefe un poderoso Ejército,
con innúmeros batallones, muchas piezas de artillería Plasencia y Krupp,
y formidable contingente de caballería. Después de varias marchas y
contramarchas, que el mareo de mi cabeza no me permite referir, me
encontraba yo en el lugar de Allo hacia el 20 de junio. Me alojé con mis
amigos de \emph{Saboya} y \emph{Ciudad Rodrigo} en el mesón de \emph{La
Jarra}, plaza del Ayuntamiento. Nunca vi una casa más divertida, por el
sinnúmero de viajeros que salían y entraban durante el día y la noche.
La guerra aumentó la caterva de huéspedes: tan pronto invadían la posada
los oficiales carcas como los \emph{guiris}, que con tal nombre eran
conocidos en Navarra los liberales.

En el poco tiempo que allí estuve me sentí contento de la vida, gozando
de mi libertad sin ningún enojo, rodeado de muchachos simpáticos y
valientes a quienes miraba como a hermanos. Bestial apetito se despertó
en mí, y en todo el día no cesaba de meter algo en el estómago. Muy
tempranito me servían el desayuno: sopas de sartén con torreznos. A las
diez me regalaban con media \emph{pinta} de vino y una escudilla de
aceitunas. Al filo de las doce ya estaba en la mesa la sacramental sopa
de ajo; después el riquísimo \emph{Chilindrón}, un guiso de cordero con
\emph{pementonicos de cuerno de cabra}; luego las magras con tomate, y
de postre los blandos roscos y el mostillo dulzón.

Por la tarde me iba con los oficiales \emph{guiris} al casino de la
placeta, conocido por \emph{el de la Mormoña}. En él tomábamos café,
coñac y algún piscolabis, para conservar las fuerzas hasta la hora de la
cena. Ésta empezaba con la ensalada al uso navarro; seguía el abadejo en
ajo arriero, y el lomo con \emph{pementones} picantes. Y vengan
\emph{pintas} y más \emph{pintas} para remojar y reblandecer el
suculento comistraje, que terminaba con gran acopio de frutas secas y
del tiempo.

Conociendo mi carácter comprenderá el lector que una de mis primeras
ocupaciones en el simpático pueblo de Allo fue echarme una novia: tocole
la vez a una linda muchacha, llamada Ruperta, hija del Nuncio, nombre
con que es allí conocido el pregonero, que anda de calle en calle
anunciando al redoble de un tambor de llegada y venta de pescado fresco,
y dando publicidad a los edictos de la Alcaldía. Mostrábase la moza
blanda y accesible, y tales ventajas brindó el amor mío a su loca
imaginación que desdeñó los obsequios y la palabra de casamiento que le
había dado el \emph{Ministro}, remoquete con que designan en aquellas
tierras al alguacil de Ayuntamiento.

En fin, señores míos; las delicias de Allo, no menos gratas aunque sí
más breves que \emph{las delicias de Capua}, terminaron bruscamente con
el son guerrero de cajas y clarines en la madrugada del día del San
Juan, cuando aún ardía en la plaza del pueblo la enorme hoguera donde
hacen chocolate las mujeres, a las doce de aquella noche, para celebrar
la tradicional festividad.

La columna, división o lo que fuera se puso en marcha, y no me
preguntéis el derrotero que yo seguí caracoleando en mi \emph{Babieca}
porque la mente del buen Tito no dominaba todavía la fácil comprensión
de los movimientos militares\ldots{} Sólo supe de cierto que el General
Concha emprendió la marcha después de organizar en Tafalla una numerosa
hueste con la mar de batallones, que según después supe ascendían a
cuarenta y ocho con los que le mandaron de Bilbao, de Medina de Pomar y
de ambas Riojas. Las piezas de Artillería con que contaba eran, según
oí, veinte Plasencias y treinta y tantos Krupp. Del número de caballos
se hacían cálculos que me parecieron hiperbólicos.

El temporal de lluvias nos entorpeció algo el camino, y el 25 estábamos,
según creo, en las estribaciones del monte Esquinza. En mis cortos
alcances comprendí que se trataba de ocupar las entradas de Estella,
donde estaba Dorregaray con veintiocho batallones. Unidos al grueso de
la división de Martínez Campos escalamos sin dificultad las alturas del
monte, que tenían los carlistas abandonado. Seguimos nuestros
movimientos, y tras penosa marcha pernoctamos en Alloz. Otras fuerzas de
nuestra división quedáronse en Lácar. Según oí, las tropas de Echagüe
ocuparon a Murillo, y las de Rosell a Villatuerta y Arandigoya, después
de desalojar de allí a los carlistas. El General en Jefe no debía estar
lejos.

En una parada que hicimos entre Allo y el monte Esquinza, tomé a mi
servicio a un viejo muy despabilado, ágil, parlero y de carácter jovial,
ajustándole por \emph{ocho sueldos} diarios (léase reales) como
asistente o espolique. Llamábase de nombre Fermín y de apodo \emph{El
Sargentico}. Pronto eché de ver sus buenas cualidades: era un andarín
fabuloso, conocía palmo a palmo el suelo navarro, y daba razón de todos
los habitantes de los pueblos que recorríamos. Para que me fuera más
simpático figuraba entre los pocos \emph{guiris} que en tal terruño
existían. En los descansos cuidaba al \emph{Babieca} como si fuera hijo
suyo; en las lentas marchas me daba conversación, cautivándome con su
charla donosa; indicábame los nombres de los montes, pueblos y ríos que
encontrábamos al paso.

En Alloz, divagando por las calles, me dio cuenta minuciosa de todas las
chicas bonitas del pueblo, sus familias y viviendas. Ya me había
descubierto el flaco, y queriendo halagarme me ilustraba en todo lo
referente al bello sexo. Seco y avellanado, insensible al cansancio, así
como al frío y al calor, no llevaba más equipo que la camisa de lienzo,
el chaleco de pana, faja, calzón, peales, y en la cabeza \emph{el
zorongo}, que es un pañuelo de colores ceñido a estilo aragonés. Cuando
se le apagaba el cigarrillo a medio fumar se lo ponía detrás de la
oreja.

Salimos de Alloz y marchamos por terreno quebrado horas y horas, entre
pueblos cuyos nombres me iba diciendo mi espolique con la puntualidad de
un experto geógrafo. No me pidáis, lectores míos, que os dé cabal
noticia de los complicados movimientos tácticos de aquel nutrido
Ejército en extensión tan considerable. Estas complejas acciones de
guerra las describen los historiadores después que han sucedido,
valiéndose de planos y documentos guardados en los archivos del Estado
Mayor Central. \emph{A priori} y en el curso de los sucesos no hay quien
puntualice los varios accidentes marciales.

En la mañana del 26 me encontré, sin saber cómo ni por qué, en el
Cuartel General de don Manuel de la Concha. Éste tenía todo dispuesto
para dar la batalla; pero hubo de retrasarla por la tardanza de un
convoy que le era indispensable para racionar y municionar debidamente a
las tropas. La impaciencia y malhumor del General en Jefe se comunicaron
a cuantos estaban cerca de él. Por fin, a las tres de la tarde, en vista
de que el convoy no llegaba, ordenó atacar al enemigo. Yo me retiré a
retaguardia porque no había ido a la campaña con miras heroicas.
\emph{El Sargentico}, que todo lo sabía o lo adivinaba, me dijo que la
línea carlista se extendía desde Dicastillo hasta el puerto de Eraul, y
que el pueblo que atacaban los nuestros era Abárzuza. Hubo un momento en
que estuve muy cerca del General Concha; le vi a caballo, revestido de
su impermeable, echando los anteojos al lugar del combate.

No bien empezaron a disparar los cañones, estalló en los aires una
horrísona tempestad de truenos, rayos, centellas y demonios coronados.
El espectáculo que daban juntamente el cielo y la tierra, confundiendo
su furor y estruendo, pertenecía ¡vive Dios!, al orden de las cosas más
sublimes que pueden verse en la vida. No sabré yo deciros que mis ojos
percibieron los pormenores de la lucha, ni tampoco preciso el tiempo que
duró. Sólo sé que después de abrasar con incesante fuego a los pueblos
enemigos, lanzáronse contra ellos en frenética legión las tropas de los
Generales Echagüe y Martínez Campos. Al anochecer eran nuestros los
lugares de Abárzuza, Zurucuáin y Montalbán.

Llegada la hora del reposo, que tan bien habían ganado los esforzados
combatientes, consulté yo con mi espolique a dónde iríamos a repararnos
del cansancio, del hambre y la mojadura, y el buen Ferminico me dijo
guiñando el ojo: «Señor; vámonos a Zurucuáin, donde tenemos la posada de
mi primo Matías, que nos dará un trato superior. Además, para que usted
se alegre un poco, le diré que en ese pueblo hay chicas \emph{mucho
guapas}.»

Ni sosiego ni comodidad tuve en la posada de Zurucuáin por causa del
gran gentío que la invadió aquella noche, y en cuanto a las lindas mozas
de que me habló \emph{El Sargentico} declaro a fe de buen galanteador
que no las vi por ninguna parte. De madrugada supimos que el convoy que
esperaba el General Concha había llegado a Murillo, y que se habían
circulado órdenes a todo el Ejército para el combate del siguiente día.

En la mañana del 27, las tropas de Martínez Campos rompieron el fuego
amenazando con coronar la sierra de Estella, que domina el pueblo de
Zurucuáin. Mi amigo Palazuelos me dijo que el General en Jefe había dado
orden de no consumar la operación hasta que la columna que estaba en
Abárzuza tomase Murugarren y el caserío de Muru. La misma orden se dio a
los que atacaban al pueblo de Grocín. Martínez Campos repartió entre su
gente las primeras raciones del convoy, y los que operaban en Abárzuza
no pudieron ser racionados a tiempo. Por esta contrariedad, se pasó la
mayor parte del día sin hace otra cosa que entretener en fuego a los
carlistas mientras hacía sus preparativos el grueso del Ejército
liberal.

Por fin, a las cuatro de la tarde, comenzó el ataque. Don Manuel de la
Concha (y esto lo aseguro como historiador \emph{de visu}, pues no
estaba yo lejos de él) se situó con dos batallones y los Regimientos de
Caballería \emph{Numancia}, \emph{Pavía} y \emph{Talavera}, en una
excelente posición alta, donde se habían emplazado treinta cañones Krupp
para batir los atrincheramientos de Muru y Murugarren. Se rompió el
fuego y la artillería, corregida el alza, causó enormes estragos en las
trincheras carlistas. A galope tendido corrían los oficiales de Estado
Mayor con órdenes a las columnas que luchaban en Abárzuza, Villatuerta y
Zurucuáin, previniéndoles que sostuvieran el fuego sin tirarse a fondo
sobre el enemigo. Los carlistas tuvieron que abandonar sus trincheras
varias veces por el horrendo destrozo que en ellos hacían nuestras
granadas. Espantosa confusión se produjo en el campo enemigo. La
terrorífica escena ponía los pelos de punta.

El General Concha dio a sus edecanes breves y fulminantes órdenes. Éstos
las transmitieron con la velocidad del rayo al Brigadier Blanco y al
General Reyes. Momentos después, las masas de Infantería se lanzaban
como avalancha impetuosa en dos columnas, la una contra Murugarren, la
otra contra el caserío de Muru. Eran doce los batallones que avanzaban,
seis en cada columna. Los carlistas, sólo en Murugarren, tenían catorce
batallones.

En lo más recio del combate llegó un aviso del Brigadier Beaumont
comunicando que las fuerzas de su mando eran furiosamente atacadas por
los facciosos, los cuales habían abandonado sus trincheras para caer
contra Abárzuza. Con ayuda de un mal catalejo y por las explicaciones de
mi espolique, yo me daba cuenta de estas terribles peripecias. Los doce
batallones que avanzaban contra Murugarren y Muru fueron embestidos del
mismo modo que la columna Beaumont. El choque fue tremendo, como una
pelea de gigantes furiosos. Al cabo, los nuestros retrocedieron,
acuchillados a la bayoneta.

Los treinta cañones empleados en la altura escupían a torrentes la
mortífera metralla. Concha, con gesto de rabia y ronco acento imperioso,
daba órdenes y más órdenes. La formidable Artillería logró al fin
contener el ímpetu de los valientes realistas, obligándolos a buscar el
refugio de sus trincheras. Por segunda vez treparon nuestros soldados
con increíble arrojo por las fragosidades de Murugarren y Muru, y de
nuevo fueron atajados en su avance. Descompuestos retrocedieron hasta la
carretera. Pero los cañones, vomitando fuego, pusieron nuevamente a raya
a los bravos batallones de don Carlos. En tanto, hacia Zurucuáin y por
las líneas Villatuerta-Arandigoyen y Murillo-Grocín, oíamos fuerte
tiroteo. Eran las columnas allí destacadas, que entretenían a una parte
de la legión absolutista hasta que se les ordenase realizar acción más
decisiva.

Atento a los incidentes de la lucha, el General en Jefe ordenó que las
columnas de Reyes, Blanco y Beaumont se concentraran en una sola. La
concentración tardó en efectuarse por estar harto diseminadas estas
fuerzas. Pasaba el tiempo, caía la tarde, la artillería empezaba a
sentir escasez de municiones, apuntaban en nuestro Ejército síntomas de
desaliento, y el combate seguía sin resultado práctico.

Cansado de esperar a los batallones del General Reyes, se decidió Concha
a intentar el esfuerzo supremo. Dejó los tres Regimientos de Caballería
en la altura donde estaban emplazados los cañones, para que protegiesen
esta posición y aseguraran el flanco derecho. Llevose consigo los dos
batallones de Infantería y con ellos se unió a los diez y ocho que
acababan de reconcentrarse. Al frente de estas fuerzas se lanzó al
asalto, cuando ya el sol, enrojeciendo las nubes de Occidente, se hundía
en el horizonte. Arreció el combate con creciente furia. Las tropas de
Reyes no llegaban. Concha enviábale de continuo órdenes apremiantes para
que acudiera pronto en apoyo de sus movimientos. Y decidido a jugar el
todo por el todo, ascendió al frente de sus tropas hacia las trincheras
carlistas.

Ante el soberano arrojo del caudillo enardeciéronse los soldados, y
seguían a su General como si no hubieran sido arrollados momentos antes.
Yo, moviéndome a impulsos de una fuerza magnética, fui detrás de los
combatientes. Concha trepaba impertérrito, unas veces a pie y otras a
caballo, según los accidentes del terreno. Al llegar a cierta altura, el
General y los demás Jefes tuvieron que dejar los caballos al cuidado de
los ordenanzas. Con éstos quedé yo, teniendo de la brida a mi
\emph{Babieca}. Me uní a Ricardo Tordesillas, asistente de don Manuel de
la Concha, y ambos nos pusimos al amparo de unos árboles donde creíamos
librarnos de las balas enemigas.

La artillería continuaba teniendo a raya a los carlistas, que ya no se
atrevían a salir de sus trincheras. El avance de Concha fue tan rápido
que llegó a cincuenta metros del enemigo cuando aún no se le habían
incorporado los batallones del General Reyes. Por falta de este apoyo no
se pudo dar fin y remate al supremo esfuerzo. A las siete y media de la
tarde, Concha no tuvo más remedio que aplazar el ataque definitivo,
dando por frustrada en aquel día la operación. Empezó a descender,
dirigiéndose con los demás Jefes a donde aguardaban los caballos.

Llegó el General donde estábamos Tordesillas y yo, ocultos a la vista de
los demás asistentes por un matorral espeso. Con voz displicente dijo a
su ordenanza: «Ricardo, el caballo.» Éstas fueron las últimas palabras
que pronunció en el mundo de los vivos\ldots{} En el momento de cruzar
la pierna derecha por la grupa del caballo, una bala, que lo mismo pudo
venir del cielo que del mismo infierno, le atravesó el corazón. Con
débil gemido expiró el primer soldado español de aquellos maldecidos
tiempos.

\hypertarget{xx}{%
\chapter{XX}\label{xx}}

A las voces de Tordesillas acudieron los que estaban más próximos. El
cuerpo del General en Jefe cayó en tierra. Tal fue la consternación y el
espanto de los primeros espectadores de la terrible escena, que todos
quedaron un momento mudos. Los ayudantes de Concha, creyendo que aún
vivía el caudillo, le desabrocharon el impermeable y levita, haciendo
saltar botones y rasgando ojales. Nada vieron que no indicase la
seguridad de una muerte instantánea. Pronto se formó un grupo espeso en
el cual nadie osaba determinar cosa alguna. ¿Qué pensar, qué decir, qué
hacer\ldots?

Por fin, entre los ayudantes y Tordesillas discurrieron lo único
práctico en trance tan fatídico. Ante todo urgía apartar de allí el
cadáver. Con gran trabajo, por la pesadumbre del recio cuerpo exánime,
colocaron éste sobre un caballo y sigilosamente fue conducido al pueblo
de Abárzuza, evitando que las tropas pudieran darse cuenta de la
catástrofe. La triste caravana, fatal término y desenlace de un acto
militar que debió ser glorioso, deslizábase furtiva por los campos como
una decepción horrenda, o una burla del Destino que quiere sustraerse a
la mirada humana, y aun a los ojos de la Historia. La media luz
crepuscular, alumbrando este paso solemne y medroso, daba a la escena la
intensa melancolía de las grandezas caídas súbitamente en los abismos de
la nada.

El primer Jefe que se presentó en Abárzuza fue el General Echagüe, que
enterado del desastre tomó el mando del Ejército a pesar de hallarse muy
enfermo. No olvidaré nunca la cara del Conde del Serrallo cuando vio el
cadáver de su amigo y maestro. El dolor concentrado y mudo no tuvo jamás
expresión más fiel que la que le dieron aquellas facciones duras,
angulosas, de soldado curtido en cien combates. La primera determinación
de Echagüe fue convocar Consejo de Generales y Brigadieres. Se reunieron
sin demora los que estaban más cerca de Abárzuza: Beaumont, Burriel,
Reyes, Blanco, Bargés y el Coronel de Artillería señor Echaluce. Por
unanimidad acordose la retirada del Ejército a Tafalla para el amanecer
del siguiente día. Y al cabo se circularon órdenes a fin de que el
movimiento se realizase aquella misma noche.

Las tropas se pusieron en marcha. El desfile de las de la derecha fue
protegido por las del centro. Las de la izquierda mantuviéronse en sus
posiciones hasta que desfilaron todas las demás. El cadáver del Marqués
del Duero fue colocado con misterio sigiloso en un furgón de Artillería,
y los heridos quedaron en Abárzuza confiados a la humanidad del enemigo.
Como el éxito de la operación dependía del tiempo que se ganase y de que
los carlistas no advirtieran la retirada, se apresuró ésta todo lo
posible y se tomaron minuciosas precauciones. Determinose prohibir a los
vecinos de los pueblos por donde había de pasar la tropa el encender luz
ni fuego en las casas; se advirtió a todo el Ejército que nadie podía
fumar, del General en Jefe para abajo; se conminó con penas severísimas
al que imprudentemente produjera el menor ruido. De este modo, bajo la
protección del silencio y de las sombras, realizose el prodigio de que
antes de amanecer hubiera desfilado ya la muchedumbre armada, incluso la
Artillería y los convoyes, por delante de las posiciones de Villatuerta,
sin que los realistas sospechasen siquiera lo que ocurría en el campo
liberal.

Ya era día claro y nos aproximábamos a Oteiza cuando los carlistas se
dieron cuenta del fúnebre desfile. Tarde conoció el enemigo su engaño, y
fue inútil cuanto intentó para molestar a nuestras tropas. Las columnas
delanteras donde iba el furgón mortuorio avivaron el paso. Las de
retaguardia, combinadas con las fuerzas de Rosell y de Reyes, tomaron
posiciones y contuvieron el tardío movimiento de los soldados de
Dorregaray, retirándose después por escalones con el orden más perfecto.
No se perdió ni un hombre, ni un fusil, ni un cañón, ni una acémila, ni
un carro del convoy: la retirada dispuesta por Echagüe en Abárzuza fue
una brillante aunque triste página militar. En las encarnizadas acciones
del día 27, las bajas del Ejército de Concha habían sido: 121 oficiales
y 1.300 individuos de tropa fuera de combate, más 268 extraviados y
prisioneros.

Seguimos a buen andar, bordeando los montes de Baigorri; hicimos una
corta parada en Larraga para tomar alimento; y dejando a la derecha los
altos de Val de Ferrer, a media tarde llegamos a Tafalla, donde tuve el
descanso que mis asendereados huesos imperiosamente reclamaban. Mi
oficioso espolique me buscó cerca de la plaza un alojamiento muy
aceptable. Allí platiqué con mis amigos, comentando cada cual según su
entender las bravas refriegas y el inmenso desastre que mató en flor las
hermosas esperanzas del Ejército liberal. Enaltecieron todos el saber
estratégico, la genial maestría y la bravura del héroe muerto que
trajimos en mísero furgón, ocultándolo como si fuera un robo que se
había hecho a la Fatalidad.

Entre los oficiales que conmigo formaban corro alrededor de una mesa,
bebiendo y fumando, había un Teniente de Infantería muy desahogado,
sobrino según creo de una persona de alta significación en la política,
el cual, colmando de alabanzas la figura militar del Marqués del Duero,
aseguró (sabiéndolo de buena tinta) que el primer acto de éste al entrar
en Estella, si a entrar llegara, hubiera sido proclamar Rey de España al
Príncipe Alfonso. La irrespetuosa manifestación de aquel jovenzuelo
llevó nuestro coloquio al vértigo de las disputas políticas, y se oyeron
las opiniones más peregrinas, diferentes en estilo y criterio,
flemáticas unas, ardientes las otras. Queriendo yo poner término a la
controversia dije estas palabras: «Caballeros; no pierdan el tiempo
discutiendo lo que pudo pasar y no ha pasado\ldots{} Descuiden que todo
se andará. Lo que hizo Concha lo harán otros, y estas peleas horribles
acabarán poniéndose todos de acuerdo para llegar a un feliz arreglito,
cuya finalidad será que nos gobierne el Nuncio.»

Antes de entregarme al descanso fui al Ayuntamiento, a punto de las
diez, deseoso de presenciar las primeras honras que se tributaron al
grande hombre muerto, reuniendo en un solo acto el esplendor militar y
la escasa pompa religiosa que en aquel pueblo pudo ostentarse. Arreglado
y compuesto el cadáver, sin que desaparecieran las huellas de una muerte
gloriosa en el campo de batalla, le colocaron en un ataúd decoroso.
Paños negros y blandones encendidos completaban el triste cuadro. Las
facciones del héroe apenas habían sufrido alteración. Ignoro si hubo o
no embalsamamiento. Permanecía tal como le vi en el instante de caer del
caballo: el ceño fruncido, apretados los labios cual si aún durase el
dolor de la herida que le mató, el corto bigote rígido, la frente
surcada de arrugas. Por un momento creí yo adivinar dentro de aquel
cráneo la visión de su postrer arranque frustrado, y el agotamiento de
su voluntad al expirar el día.

Bien dijo el que dijo que tras de las pisadas duras de la tragedia suele
ir el blando paso de la comedia. Así lo quiere la complejidad tumultuosa
de nuestra vida, y yo lo confirmé aquella noche con el descomunal
contraste que voy a referiros. Hallábame en mí cuarto con \emph{El
Sargentico} y a meterme en la cama me disponía, cuando sonaron
golpecitos en la puerta. Fugaz presagio cruzó por mi cerebro. El sonido
seco de la madera me delataba los nudillos de una persona conocida.
¿Sería \emph{Chilivistra}?\ldots{} Sí, sí; era ella, ¡Dios!\ldots{}
Apenas pronuncié yo el \emph{adelante}, abriose la puerta y penetró de
rondón la señora mística y destornillada. Venía bien arregladita, con el
hábito de los Dolores. En su bello rostro notábase, fresco y reciente,
un discreto aliño de colorete y polvos.

«Pero mujer, ¿qué es esto?---exclamé indicándole un sillón cojitranco
.---¿Qué buscas, qué quieres, cómo has venido aquí?» Y ella, serena y
flemática, me contestó: «Desde lejos he seguido tus pasos, sabiendo día
por día y hora por hora dónde estabas. Razón tuve de tu alojamiento en
cuanto llegamos aquí, a eso de las diez. En esta misma posada buscamos
albergue. Tú no te enteraste porque habías ido al Ayuntamiento a ver el
cadáver del pobrecito Concha.»

---Según eso, no has venido sola---exclamé yo, aterrado ante la idea de
habérmelas con el elegante caballero, Administrador de Rentas de
Vitoria.

---Solita hubiera venido---afirmó Silvestra,---sin más compañía que mi
anhelo de verte. Pero traigo conmigo dos personas respetables que,
compadecidas de mis infortunios, no han querido separarse de mí en todo
el viaje, y me seguirán, según dicen, hasta donde yo vaya. Una de estas
buenas almas es el Capellán de las Brígidas. La otra, una señora mayor
con quien hice conocimiento en el trayecto de Vitoria a La Guardia. Es
dama muy principal, de finísimo trato y mucho saber. Conversamos,
intimamos y nos hicimos muy amigas.

Oyendo a la voluntariosa mujer me maravillaba de los enredos e
imaginarias historias que se traía. Mi estupefacción llegó al colmo
cuando me dijo, para darme pormenores de sus compañeros de viaje: «El
Capellán de monjas, para que te enteres, es el padre Carapucheta, que
como recordarás, estaba de Rector en el Oratorio del Olivar. La dama es
una matrona de regia estirpe\ldots{} No te rías\ldots{} que a ti te
conoce mucho y te llama su muñeco. Su nombre es\ldots{} ¿no lo
adivinas?\ldots{} \emph{Doña Mariana.»}

Este nombre retumbó en mi cerebro como el eco de un cañonazo\ldots{} Se
nublaron mis ojos, no sabía lo que me pasaba. «Tú---dije a Silvestra,
poniendo mis manos trémulas junto a su rostro,---o padeces un mal que te
sugiere los absurdos más desatinados, o posees una imaginación que deja
tamañitos a todos los inventores de fábulas, a todos los poetas del
mundo. Si esa \emph{Doña Mariana} no es engendro de tu caletre
enfermizo, quiero verla ahora mismo. Pronto, pronto.»

Grave y serena se levantó \emph{Chilivistra}, y cogiéndome la mano, me
dijo: «Pues ven a verla. Bien cerca la tienes. Dos puertas más allá, en
este mismo pasillo. Ven, Tito, ven.»

Momentos después, mis ojos, asustados de su propia visión, distinguieron
la imagen o la persona de \emph{Mariclío} en una estancia mal alumbrada,
anchurosa, con las paredes cubiertas de viejos cuadros al óleo
ennegrecidos por el tiempo. En un sofá de dos cabeceras y respaldo de
crines, modelo antiquísimo que sólo se ve ya en alguna fonda de pueblo,
estaba la excelsa Madre, apoyada en una de las cabeceras, en actitud de
tristeza y cansancio. Adelanteme hacia ella con timidez y
respeto\ldots{}

Las primeras palabras articuladas por sus labios augustos determinaron
súbitamente en mí la transformación de lo interno y lo externo, de todo
cuanto yo llevaba en mi espíritu y de lo que mis sentidos podían
apreciar. La estancia creció desmesuradamente, la figura olímpica se
agigantaba, y su voz llegó a mis oídos como lejana música. Mi turbación
no me permitió retener el justo sentido de aquella música. Creo que me
dijo: «Lo que has visto de esta guerra estúpida yo también lo vi\ldots{}
La Fatalidad, ley que viene de muy alto, impidió al gran soldado dar un
golpe decisivo\ldots{} No creas que puedan concluir estas luchas de otro
modo que por conciertos y cambalaches como los de Vergara\ldots{} Tu
pobre España gemirá, por largos años, bajo la pesadumbre del despotismo
que llaman ilustrado, enfermedad obscura y honda, con la cual los
pueblos viven muriendo\ldots{} y se mueven, gritan y discursean,
atacados de lo que llaman \emph{epilepsia larvada\ldots{}} Debajo de
esta dolencia se esconde la mortal \emph{tuberculosis\ldots»} Si tales
no fueron sus expresiones textuales, no creo equivocarme respecto al
sentido de ellas.

Desde que oí a la Señora subió de punto el desvarío de mis pensamientos.
Se me olvidó el nombre del pueblo donde me encontraba. «¿Pero dónde
estás, Tito?»---me pregunté\ldots{} Vi a \emph{Chilivistra} arrastrando
por los polvorosos ladrillos de la inmensa habitación la cola negra de
un vestido como los que usan las damas en la Corte. Me senté a distancia
de la Madre en una banqueta de nogal lustroso. Creí advertir que el sofá
de antiguo modelo no estaba próximo a la pared, y que por aquel hueco
discurrían las figuras descendidas de los cuadros viejos, tomando las
negras apariencias de \emph{Doña Gramática} y \emph{Doña Caligrafía}.

Transcurrió un lapso de tiempo, que ignoro si fue de minutos o de horas.
Silvestra se llegó a mí, diciéndome: «Quiero que conozcas a mi segundo
acompañante, el bendito Capellán padre Carapucheta.» Ausentose un
momento, y reapareció trayendo de la mano a un sujeto esmirriado y
larguirucho, vestido de luenga sotana. ¡Dios, Jehová, Lucifer! El hombre
que hacía reverencias frente a mí era el mismísimo Ido del Sagrario.
«¿Pero es usted don José?»---dije o creí decir yo. Y él, dilatando su
boca en larga sonrisa, habló en su habitual estilo: «\emph{Francamente,
naturalmente}, señor don Tito, no podía venir a estas tierras sin
disfrazarme\ldots{} Sabrá Vuecencia que al llevar a mi hija Rosita, el
mes pasado, a la feria de Huete, que es el pueblo de Nicanora, me fue
robada en Fuentidueña de Tajo por la partida carlista que manda el
cabecilla Santés. Desesperado salí a recuperarla. Dijéronme que su
raptor se la llevó a Navarra, y aquí me han dicho que ahora podré
encontrarla en tierras de Guadalajara o de Cuenca. Ayúdeme usía en mi
empresa y Dios le dará el Reino de los Cielos.»

Al oír estos desatinos, me llevé las manos a la cabeza creyendo que de
ella se me escapaba la razón y todo el sentido de la realidad. Salí de
la estancia como alma que lleva el diablo, gritando: «¡Favor,
socorro!\ldots» Dando tropezones y metiéndome en diferentes cuartos
llegué por fin al mío, donde me encontré frente a un hombre escueto, con
chaleco de pana y \emph{zorongo}. Cogiéndole de los brazos le zarandeé
mientras le decía: «¿Qué hace usted aquí?\ldots{} ¿Quién es
usted?\ldots{} ¿Dónde estoy?»

Turbado me contestó el buen hombre: «Señor, ¿qué le pasa? Soy \emph{El
Sargentico}. ¿No me conoce ya?\ldots{} De aquí salió usted despierto y
vuelve dormido.»

\hypertarget{xxi}{%
\chapter{XXI}\label{xxi}}

Con solícitos cuidados, mezclando en su lenguaje la expresión seria con
la festiva, mi buen espolique se esforzaba en serenarme. Hízome tender
en la cama, y sentado junto a mí apuró razones y cuchufletas para
traerme a la percepción de la realidad. Yo le dije: «Quedamos en que tú
eres \emph{El Sargentico}. Bien: \emph{El Sargentico}. Sobre eso ya no
hay duda. Dime ahora cómo se llama este maldito pueblo donde estoy, pues
mi memoria es esta noche como una jaula rota de la que se escapan todos
los pájaros.» Al oír el nombre de Tafalla, repetido tres veces por mi
espolique, agarré el vocablo y me lo metí en la casilla más honda de mi
cerebro.

«Ya me vuelve poco a poco el sentido---dije incorporándome en el
camastro.---Tafalla es esta ciudad, y a ella hemos traído un muerto que
se llama\ldots{} ¡ah, ya me acuerdo!\ldots{} el General Concha\ldots{} Y
ahora, Fermín, contéstame a otra pregunta. Pero has de prometerme, por
la salvación de tu alma, decirme la verdad. Vamos a ver, ¿no crees tú
como yo que estamos en una casa encantada?\ldots»

---Como encantada por achaque de brujería o maleficio, no lo creo,
señor---replicó mi espolique.---Ahora, si achacamos a encantamento el
golpe de gente, el rebullicio, el entrar y salir de oficiales, curas,
mujeres de toda laya\ldots{} con perdón\ldots{} todos pidiendo de comer,
comiendo el que puede, éstos borrachos por el mosto, aquéllos por el
meneo de los naipes\ldots{} si es así, la casa de Irucheta está dada,
como quien dice, a todos los demonios.

Con la grata conversación de \emph{El Sargentico}, mi ánimo iba entrando
en su normalidad. Sentí sueño, me metí en la cama, y cuando mi espolique
quiso retirarse le ordené que se quedase a dormir en mi cuarto. Yo tenía
miedo de que se repitieran las morbosas aberraciones que me atormentaron
antes de media noche. En un sofá de enea arregló lindamente su cama mi
escudero con dos mantas y un maletín que convirtió en almohada. Dormí
algunos ratos. En mis instantes de desvelo agradábame oír a los serenos
cantando las horas.

La del alba sería cuando hirió mis oídos una música dulcísima, un coro
armónicamente concertado con voces agudas y graves, tan hermosas por
timbre como por su cabal afinación, música deliciosa, solemne y mística,
que a mi parecer pasaba por la calle cual bandada de angélicos cantores
que al término de la noche se retiraban de la Tierra al Cielo.
Embelesado por aquel divino cántico, en cuyas vocalizaciones distinguí
el nombre y alabanzas de la Virgen María, me incorporé en el lecho y
afiné mi oído para que no se me escapase ni un acento de tan
incomparable salmodia.

«¿Qué es esto que oigo?»---pregunté a Fermín, notando que remuzgaba
desperezándose.

---Señor---me contestó al momento.---¿No sabe que estamos en la tierra
de los cantores? Todo navarro nace músico antes que carlista. Eso que
oye es \emph{el alba}, como decimos por acá, un canticio \emph{mucho
precioso} que los serenos echan al retirarse, alabando a la Virgen
Santísima. Sereno hay aquí que cuando suelta la \emph{melodia} da quince
y raya a los tiples de las iglesias\ldots{} ¡Ay, señor, si hubiera usted
oído a un chico del Roncal que vino a Pamplona poco tiempo ha!\ldots{}
¡Aquello sí que era voz! Por gracia cantó algunas mañanas con los
serenos, y los vecinos salían en paños menores a los balcones para oírle
más a gusto. Voz de tenor tan fina y bien timbrada diz que no se ha oído
jamás, como no sea en los coros que festejan al Padre Eterno. Por toda
Navarra se corre que han venido unos maestros de Madrid para llevarle a
cantar óperas en el Teatro Real.

Ya entraba la luz solar en la habitación cuando dije a mi espolique:
«Mientras yo me levanto vete callandito a la cocina, manda que me
aderecen la riquísima esencia de castañas que aquí llaman café, y me la
traes con abundante leche bien caliente para desayunarme. Para ti pides
el chorizo y panazo que te gusta. En cuantico que metamos ese lastre en
el cuerpo recogemos nuestros bártulos, bajamos de puntillas sin que
nadie nos vea, pagamos la cuenta, ensillamos el jaco y salimos pitando
de esta condenada Tafalla.»

Largo rato empleó \emph{El Sargentico} en dar cumplimiento a mi encargo,
y cuando me ponía delante el cocimiento de achicorias y la leche aguada,
me dijo tranquilamente: «Bueno, señor: nos escapamos de tapadillo sin
que nadie nos vea. Muy bien. Y ahora le pregunto yo: ¿a dónde vamos?»

La pregunta del viejo navarro me dejó suspenso. ¿A dónde iríamos? El
problema era grave. Hallábame perplejo y atontado, discurriendo a qué
punto del globo terrestre debíamos encaminar nuestros pasos, cuando un
súbito estremecimiento como sacudida de terremoto me hizo saltar en la
silla. Mas no fue temblor del suelo propiamente sino dos tremendos
golpes en la puerta, los cuales, por la dureza de la percusión, debieron
de ser dados con nudillos de piedra. «¡Ay! ---grité.---No abras,
\emph{Sargentico\ldots{}} Sí, sí; abre, que si no, puede que nos
derriben la puerta.»

Franqueada la estancia vi en el umbral una mujer de espigada estatura,
vestida de luengos paños negros que caían hasta sus pies con pliegues
estatuarios. La blancura de su rostro era blancura de alabastro, y su
voz, como articulada por una boca de piedra, heló mi sangre cuando me
dijo: «La señora doña Silvestra y el padre Capellán han ido a la iglesia
de Santa María y San Pedro. Allí está también la soberana Madre. De su
parte vengo a decir al señor don Tito, que le espera sin demora en aquel
lugar: Clío necesita dar órdenes a su gentil muñeco.»

Al decir la última palabra se apartó para darme paso. Yo alargué mi mano
y toqué la suya: era de mármol\ldots{} Temblé de frío y de
pavura\ldots{} Miré al \emph{Sargentico} y vi que se santiguaba\ldots{}
«No temas---le dije tratando de sobreponerme a la turbación.---La Señora
que me llama es mi Madre, es también la tuya, porque tú, Fermín, antes
de estar a mi servicio y desde que estás en él, si no has escrito la
Historia la has hecho. Todos hemos sido y somos modeladores de la vida
de los pueblos.»

Salimos, apoyado el uno en el otro, pues ambos flaqueábamos de las
piernas\ldots{} En la calle, cuando dije a Fermín que me guiara a la
iglesia de Santa María y San Pedro, me sentí otra vez navegante en el
piélago de las cosas suprasensibles. «Mejor---pensé avivando el
paso.---Bien venido sea el mundo quimérico. Bendita sea la sinrazón que
es casi siempre el molde de la razón.»

Lo primero que vi al entrar en la iglesia y llegamos a una de las
capillas, fue un delicioso absurdo que en pocas palabras refiero\ldots{}
¡Ido del Sagrario estaba acabando de decir misa, con casulla encarnada!
Al pronto dudé. Pero cuando se volvió de cara a los fieles para decir el
\emph{ite, misa est}, reconocí sus inequívocas facciones. Al retirarse
el oficiante hacia la sacristía, calado el bonete y llevando en sus
manos el sagrado cáliz, no pude reprimir las ganas de soltarle una
chirigota. «Vaya, don José---le dije ,---que sea enhorabuena: esto es
mejor que ir a la compra.»

Vi a \emph{Chilivistra} surgir de un grupo de mujeres arrodilladas, y
cuando iba hacia ella, una mano blanda me tocó en el brazo. Era la
Madre, que me dijo con acento jovial: «Ven aquí, perdulario; ahora no te
me escapas. Salgamos al pórtico y hablaremos.» Se me presentaba
\emph{Mariclío} en la forma más humana, ajustada estrictamente al tipo
de señora principal, como tantas otras que vemos en el mundo físico. No
advertí en ella ni el menor asomo de figura olímpica ni de fantástica
evocación pagana. Su rostro y porte eran los de una matrona hermosa,
aunque algo madura. Llevaba un trajecito de merino y su mantilla negra;
en la mano el libro de Jenofonte, \emph{Agesilao}, impreso en griego,
que yo pude ojear cuando \emph{Clío} me visitó en la fonda de Cartagena.

Al salir al pórtico me llevó la Madre a uno de los poyos más distantes
de la puerta, donde charlamos tranquilamente en el lenguaje más opuesto
al que suelen usar las almas del otro mundo. «Esta vez, como
siempre---me dijo,---has de cumplir fielmente mis órdenes. Forzoso es
seguir los pasos de una guerra, que juzgo hermanando dos calificativos
tan distintos y antitéticos como lo son de \emph{infantil} y
\emph{sangrienta}. Creyérase, mi querido Tito, que estos niños grandes
se matan por el gusto de la destrucción, y que el fin sin fin de las
batallas, encuentros y emboscadas, no es otro que disminuir la población
hispana. Vuestros políticos y vuestros guerreros estiman como el mal el
crecimiento de la raza. Hay que matar, matar sin tregua para que se
acorte el número de los españoles que viven y comen\ldots{} Has visto,
en sus diferentes fases, la guerra en el Norte. Conviene que la veas en
el reino de Valencia y términos fronterizos de Castilla. Vete, pues, yo
te lo mando, en compañía del buen Capellán padre Carapucheta y de la
desdichada señora a quien sus conterráneos dan el gracioso nombre de
\emph{Chilivistra}.»

Como yo, sin oponerme a sus mandatos, indicara que las genialidades de
Silvestra me amargaban la vida, la excelsa matrona rebatió mis
escrúpulos con estas sendas razones: «Has de persuadirte, hijo mío, de
que en el carácter borrascoso y tornadizo de tu \emph{Chilivistra}
tienes un perfecto símbolo de la vida española en el aspecto político, y
estoy por decir que en el militar. Tan pronto es cariñosa y tierna como
altiva y marimandona. El amor la dulcifica hoy, y mañana la endurece el
orgullo. Inventa con lozana imaginación fábulas absurdas y acaba por
creerlas. Se finge deshonesta sin fundamento real de sus mentirosos
pecados. En ella habrás observado que al fuego del sentimentalismo
sustituye rápidamente el hielo de los negocios menudos, todo ello sin
criterio fijo, sin noción alguna de la realidad. En su desconcertada
cabeza es un mito el Administrador de Rentas de Vitoria; mito es también
ese marido errante, y por fin, personaje de leyenda es el hijo que
busca.»

Asombrado escuché el admirable juicio que en cortas razones hizo
\emph{Clío} de la histérica dama, y acabó de maravillarme con esta
discreta síntesis: «Fíjate bien, hijo mío, y verás que con el sistema
puramente \emph{Chilivistril}, y conforme al voluble proceso mental de
tu amiga, gobiernan a España las manadas de hombres que alternan en las
poltronas o butacas del Estado, ahora con este nombre, ahora con el
otro. También ellos invocan el sentimentalismo patriótico cuando les
conviene, o se entregan a los espasmos del despotismo cuando no hallan
salida por la vía patriótica, o sea la vía liberal. También ellos
inventan historias para domar las fieras oleadas de la opinión y acaban
por creer lo que engendró su propia fantasía. Tus gobernantes son
creadores de mitos, y mostrándolos al pueblo andan a ciegas sin saber lo
que quieren ni a dónde van\ldots{} Resígnate, pues, a llevar contigo
este emblema de la vida nacional en la cristalización que llamamos
política militante. \emph{Chilivistra} será para ti lección viva, que
hora tras hora te mostrará los capitales defectos de tu patria, para que
aprendas a precaverte contra ellos con la mira de que algún día seas
llamado a gobernar la Nación.»

El talento de la Madre, con ser divino y de tan extraordinarias luces
adornado, no acabó de llevarme al convencimiento. Pero, sin dejar salir
de mis labios la menor objeción, declaré que obedecería ciegamente sus
mandatos. Donosa y risueña me dijo la Señora que en todo tiempo no me
inspiraría conducta y acciones que no fueran para mi provecho, y con
dulzura materna me encareció que desechase toda sensación de miedo
cuando ella creyese necesario llamarme a su presencia. Respondile que la
noche anterior me había sobrecogido el verme de improviso y sin
preparación alguna frente a tan excelsa divinidad, y que asimismo me
turbé horriblemente aquella mañana cuando recibí sus órdenes por la
mensajera más clásica y más helénica que vi en mi vida: una estatua de
mármol. «¡Pero, hijo del alma---exclamó la celeste Musa, soltando una
deliciosa risa que también me pareció helénica,---si el recado para que
vinieras aquí te lo mandé con la criada de la fonda!»

En esto, llegaron al pórtico Silvestra y el enigmático sujeto en quien
se fundían las dos personalidades del cura Carapucheta y del filósofo
simple Ido del Sagrario. Reunidos los cuatro, \emph{Mariclío} se mostró
impaciente y nos incitó a partir sin demora. En mis manos puso una
carterita que contenía, según me dijo, cuanto dinero pudiera yo
necesitar para un largo viaje. Antes de que preguntase a dónde íbamos,
afirmó que \emph{Chilivistra} y el señor Capellán marcarían nuestro
derrotero. Preparado tenía un buen coche con cuatro poderosos caballos,
que podríamos dejar cuando se nos presentase coyuntura de recorrer
largos trayectos en ferrocarril.

Antes de emprender tan aventurada correría, no debía yo olvidar a mi
buen espolique Fermín, ni al espejo de las cabalgaduras, el gallardo y
sufrido \emph{Babieca}. Pero la Madre, que todo lo había prevenido,
declaró que a su cuidado quedaban \emph{El Sargentico} y mi corcel,
agregando que ella guardaría y conservaría con toda solicitud al hombre
y al bruto, para que yo los recobrase en el punto y hora en que tan
dulces prendas me fueran necesarias. Llamé al escudero fiel, que a corta
distancia nos oía, y con pocas palabras le enteré del acuerdo. Quedó muy
complacido de servir, por plazo más o menos largo, a la más alta Señora
que en estos reinos existe.

En fin, lectores de mi alma, que no sé si llamar severos o socarrones,
sabed que me llevaron a donde esperaba el coche, que en él metieron los
equipajes de los tres viajeros, que por un callejón cercano vi que se
retiraba \emph{Mariclío} entre dos estatuas de mármol vestidas con
negras y ajustadas túnicas, que al \emph{Sargentico} se le humedecieron
los ojos al despedirme, y que a mis oídos llegó lastimero relincho de mi
\emph{Babieca}, encerrado en una cuadra próxima. ¡Adelante con la
Fábula, adelante con la Historia! El coche partió a escape por la margen
del río Cidacos. ¡Arre, caballitos, arre hacia lo desconocido, hacia las
alturas, hacia los abismos, hacia el ensueño!\ldots{}

\hypertarget{xxii}{%
\chapter{XXII}\label{xxii}}

Como mi pobre cabeza tardó horas y horas en recobrarse de aquel vértigo,
no me es fácil determinar el lugar y momento en que cambiamos el coche
por el ferrocarril. Sí recuerdo que al anochecer íbamos en un tren
mixto, de cuya dirección no pude enterarme hasta que Silvestra dijo que
estábamos cerca de Las Casetas. Poco antes de esto, tras penosa lucha
entre mi razón y mi fantasía, llegué al convencimiento de que no llevaba
traje sacerdotal aquel don José, que en boca de Silvestra era el padre
Carapucheta y en la mía el señor Ido del Sagrario.

En la estación que empalmaba la línea de Castejón con la de Madrid a
Zaragoza, bajamos a restaurar nuestras fuerzas con el comistraje que dan
las fondas ferroviarias, y entre una sopa aguanosa y un pollo más duro
que la pata de un santo deliberamos sobre la ruta que nos convenía
seguir. Opinó \emph{Chilivistra} que debíamos continuar en tren hasta
Calatayud, y de allí internarnos por Daroca hacia la provincia de
Teruel. El don José, cuya delgadez era ya transparente, sostuvo la
conveniencia de llegarnos por el ferrocarril hasta Guadalajara, donde él
tenía que tomar lenguas acerca del asunto que a tales trotes le llevaba.
Yo, Proteo Liviano, mensajero de los Dioses, envolviéndome en una
serenidad majestuosa les dije que mi opinión era no tener ninguna, y que
me dejaría llevar a donde la dama gordita y el caballero flaco
determinasen, ora fuese a las delicias del Paraíso Terrenal, ora fuese
al mismísimo Infierno.

De la deliberación de mis dos compañeros de viaje resultó que haríamos
una paradita en Calatayud. Paradita fue que en la ciudad aragonesa que
los antiguos llamaron \emph{Bílbilis}, patria del poeta latino Marcial,
estuvimos tres días. Ello sucedió porque nos metimos en una fonda con
ánimo de pasar la noche, y apenas viose Silvestra bajo techo se puso
tierna, indolente, mimosa, aquejada de esa insana languidez que sólo se
cura con los melindres afectivos. Estábamos en la faceta de los
arrumacos pasionales. Ya vendría la contraria. ¡Dios!

Respondí a los arrullos de mi amiga por mantener la paz en nuestra
errante comunidad; yo no tenía prisa en cerrar aquel paréntesis de
descanso, ni el bueno de don José mostrábase impaciente: pasaba todo el
día recorriendo calles y visitando conventos\ldots{} Al tercer día de
nuestra parada le cogí a solas en su estancia y así le dije: «Ya mi
cabeza está despejada y no le vale a usted su disfraz de capellán ni
toda esa monserga que se trae. Usted es mi patrón, el gran filósofo Ido
del Sagrario, sujeto que con ninguna otra criatura humana puede
confundirse.»

---Sí, señor: soy el que Vuecencia dice y no puedo ser otro---me
contestó Ido un tanto lacrimoso.---Pero, \emph{francamente,
naturalmente}, ¿qué he de hacer yo si esa doña Silvestra se ha empeñado
en que soy el padre capellán don José Carapucheta?\ldots{} Veréis,
Ilustrísimo Señor: fui a Vitoria buscándole las vueltas a la pobre hija
que me robaron, y me encontré a \emph{doña Chilivistra}. Esta
señora\ldots{} ya sabe usted que está loca perdida\ldots{} me metió en
el enredo de vestirme de cura para poder penetrar con seguridad en el
riñón de Navarra\ldots{} En el riñón entramos y del riñón salimos. Luego
se nos apareció esa \emph{madama Clío}, sabedora de todo lo que ha
pasado en el mundo y de lo que ha de pasar, y gracias a la supradicha
\emph{madama}, que mil años viva, me veo junto \emph{al hombre del gran
poder}, quien seguramente me llevará a donde encuentre lo que busco.

---Sí, sí, no tenga usted duda: rescataremos a Rosita---dije yo
pavoneándome al recobrar mi papel de consolador de todos los afligidos.

---Pues bien, Ilustrísimo Señor. Si ahora vamos Vuecencia y yo a
\emph{doña Chilivistrilla}, y le decimos que yo no soy el padre
Carapucheta sino el marido de Nicanora, verá Vuecencia cómo le tiembla
el labio y nos pega a los dos.

---No le diremos nada; descuide don José. Y si para mantenerla en su
engaño fuese menester que dijera usted misa en cualquiera de los pueblos
por donde hemos de pasar, la dice usted, yo le ayudo, ella la oye, y
\emph{pax Christi}.

\emph{---Amén\ldots{}} Ahora hablemos de otra cosa. Si esa señora se
obstina en ir al Maestrazgo, no cuenten conmigo. He pasado estos días
enterándome de las cosas de la guerra, y sé que toda esa parte de Teruel
y Albarracín es un volcán. \emph{Francamente, naturalmente}, no he
venido yo al mundo para que me fusile un Cucala, un Bonet, u otro de
esos bárbaros matarifes.

---Estamos conformes. ¿A dónde quiere usted que vayamos?

---A donde dije en la estación de Las Casetas. A Guadalajara,
Ilustrísimo Señor.

---Pues allá iremos. Yo convenceré a doña Silvestra.

Al día siguiente habríamos llegado a la ciudad que goza fama de ser el
emporio de los bizcochos borrachos, si a mi Silvestra no se le hubiera
metido en la chola hacer otra paradita en Alhama. Seguía la racha
voluptuosa. Ya me iba yo cansando de paraditas, mimos y empalagos de
sentimentalismo dulzón. Y gracias que en todas las estaciones siguientes
no propuso más que otras dos paradas, una en Medinaceli para ver el
sepulcro de Almanzor, otra en Sigüenza porque había hecho promesa de
ofrecer sus pías devociones a la gloriosa mártir Santa Librada\ldots{}
Con estas lentitudes, ya corrían los primeros días del mes de Julio
cuando entramos en la capital de la Alcarria.

Apenas instalados en la posada de donde parten las diligencias para
Brihuega y Pastrana, olvidó \emph{Chilivistra} su terca obstinación de
visitar el Maestrazgo, país entonces erizado de peligros que en su magín
enfermo se revestían de formas románticas. Ilusionada por nuevas ideas
imaginó que sería muy divertido dar un vistazo al país donde se cría la
exquisita miel y a los verdes oteros poblados de aromáticas
hierbas\ldots{} A todas éstas, el pobre Ido andaba desatinado por la
población, donde no le faltaban amistades y conocimientos. Díjome una
tarde que había tenido noticias desconsoladoras; mas para confirmarlas
era preciso que fuéramos a Huete.

A \emph{Chilivistra} no le pareció bien abandonar la región melífera.
Antojósele además tomar las aguas de La Isabela, en Sacedón, que según
decían eran excelentes para conservar la tersura del cutis. En estas
disputas acerca del punto a donde debíamos ir pasaron dos días más. Por
fin determiné yo alquilar un buen coche para irnos por el camino de
Pastrana hacia la provincia de Cuenca, después de asegurar a Silvestra
que cuando despachásemos un asunto particular del señor Capellán la
llevaríamos a zambullirse en las aguas de La Isabela.

De mala gana emprendió la vizcaína el viaje, y por el camino nos daba la
tabarra volviendo su enojo contra el padre Carapucheta, de quien decía
que iba siempre huroneando los conventos de monjas, con las cuales a
hurtadillas se refocilaba. Oía con resignada humildad estas cosas el
bueno de Ido, cuya inquietud y zozobra se mostraban en lo escuálido del
rostro y en el crecimiento de la nuez.

Rodando por desiguales caminos llegamos a Huete avanzada la mañana de un
luminoso día de Julio, y don José, apenas nos quitamos el polvo en el
parador de \emph{Santa Clara}, encaminose al monasterio del mismo
nombre, situado a corta distancia de nuestro alojamiento. Más de dos
horas permaneció el manso filósofo en la casa monjil, conferenciando con
una tal Sor Inés de la Transverberación, prima carnal de Nicanora.

En el largo tiempo que pasamos esperando a Ido, noté que a
\emph{Chilivistra} le tembliqueaba el labio. Ya venía la racha de la
impertinencia borrascosa. «Bonito papel estamos haciendo---me
dijo---tapándole los vicios a este capellán que parece una mosquita
muerta y es un tenorio de monjas. Opino que debemos dejarle aquí,
marchándonos nosotros hacia La Isabela, donde encontraré el remedio para
estos granitos que me han salido en las piernas. Míralos, Tito, y te
convencerás de que me son precisas aquellas aguas, que instaló Fernando
VII para pulimentar la epidermis de su segunda mujer, la Reina doña
Isabel de Braganza.»

Hice cuanto pude para contener y amansar a Silvestra con blandas
razones. Llegó por fin el buen Ido, consternado, y llevándome aparte
discretamente me dijo: «Ilustrísimo Señor; ya sé a ciencia cierta que mi
adorada Rosita está en Cuenca, en una casa de esas que llaman\ldots{}
con perdón\ldots{} mancebías públicas, y yo llamo templos del
escándalo.»

---Pues vámonos allá, don José---repuse yo,---y salvaremos de la infamia
a esa sacerdotisa de Venus.

No necesito decir los artificios amorosos que puse en juego, halagos que
prodigué y patrañas que discurrí, para convencer a \emph{Chilivistra} de
que debíamos ir a Cuenca. Con todo, momentos hubo, a poco de arrancar el
coche, en que don José y yo estuvimos a dos dedos de ser abofeteados por
el basilisco; poco faltó para que sus blancas y afiladas uñas se
clavaran en mi rostro. La lucha duró hasta que el sueño y la fatiga
rindieron a la fierecilla, andados ya dos tercios del camino. Nocturno
fue aquel viaje y fecundo en molestias de todo género. Ya era más de
media noche cuando entramos en Cuenca. Nuestros pobres huesos y nuestros
desmayados espíritus tuvieron descanso en la mejor fonda de la
Carretería, parte llana de la ciudad.

Al siguiente día, 12 de Julio, fecha que no se me olvidará mientras
viva, el molimiento de nuestros cuerpos nos retuvo \emph{en las ociosas
lanas} más tiempo de lo que acostumbrábamos. Levantose Silvestra de mal
talante, que manifestaba con agrias y descomedidas voces, y agarrando
sus libros de rezos y su rosario requirió mi compañía para ir
inmediatamente a la Catedral, pues quería prosternarse ante el sepulcro
del bendito San Julián, Obispo de Cuenca.

Salimos los tres y nos dirigimos por la Carretería hasta una vetusta
puente sobre el río llamado Huécar, la cual une la ciudad vieja con los
arrabales. Como poseo un gran sentido topográfico, andando me enteraba
de la estructura de aquella ciudad celtíbera, visigoda, arábiga o no sé
qué, asentada en varios montículos rocosos. El conjunto del viejo
caserío escalonado en diferentes anfiteatros, donde al parecer los
cimientos de unas casas pisaban las techumbres de las otras, era de lo
más pintoresco que yo había visto en mi vida. Pasado el puente entramos
en una calle que, según me dijo Ido, se nombraba de \emph{Las Cocheras}.
Allí nos separamos; el filósofo torció a la derecha en busca de las
casas públicas y pecaminosas, donde creía encontrar a su desdichada
hija. \emph{Chilivistra} y yo, por la empinada y tortuosa ruta que nos
señaló don José, subimos hasta la Catedral.

Aquel día estaba mi basilisco en la plenitud de sus vesánicas
impertinencias. Por la menor cosa reñíamos. Si tropezaba yo en un
pedrusco (y hay que ver, señores, lo que eran aquellos empedrados, los
partidos losetones y los peldaños puntiagudos), se ponía furiosa y me
increpaba de esta manera: «Hoy estás cargantísimo. No se puede ir
contigo a ninguna parte\ldots{} Claro, ¡como no te dejo ir con el
bigardo del Capellán Carapucheta a jugar con las monjitas!\ldots{} A mí
no me toques, no me des la mano, que yo sola sé andar muy bien. No tengo
las piernas de trapo como tú.»

El interior de la Catedral me impresionó grandemente por la majestad y
elegancia de sus líneas ojivales, diluidas en un doble misterio de
silencio y obscuridad. El presbiterio y el ábside me parecieron
espléndidos, las verjas magníficas. Silvestra oyó dos o tres misas en
diferentes capillas, y luego estuvo arrodillada largo rato ante el altar
de San Julián, un armatoste greco-romano del estilo más antipático y
pedantesco. Beatas vejanconas no cesaban de llegarse a los mármoles del
sepulcro para besuquearlos y llenarlos de babas. Apenas se apartó del
altar mi basilisco para marcharnos, adelantose a darle agua bendita un
hombre de buena estatura, vestido con decorosa modestia, de negra barba,
pelo rizoso, facciones de varonil belleza y edad como de cuarenta o
cuarenta y cinco años. Al acercarme yo, le oí decir: «¿No me reconoce
usted, Silvestra?» Y como ella dudara observándole, él prosiguió: «Soy
primo de Delfina Gay, y en su casa nos hemos visto algunas veces, ¿no se
acuerda? Mi nombre es Avelino Palomeque.»

---¡Ah!, ya, ya, Palomeque---dijo Silvestra agradeciéndole con su más
delicada sonrisa.---¿Es usted de aquí?

---No, señora; yo nací en Toledo. Pero estoy en Cuenca desde muy niño y
en ella tengo mis negocios: dos fábricas de harinas y los molinos de San
Antón.

Salimos los tres. El gaznápiro de Palomeque iba junto a Silvestra,
dándole conversación, y a mí ni me saludó ni me hacía caso. Le pagaba yo
este desaire con la moneda de mi desprecio. Mirándole bien recordé
haberle visto en la casa de Delfina y en la tienda de ataúdes. Era un
carlistón rabioso, fanático, muy cerrado de mollera. Al llegar a una
calle, que luego supe se llamaba de \emph{Caballeros}, tan pendiente que
por ella había que andar a gatas, se paró el cerril carcunda y dijo
estas palabras, volviendo su rostro hacia mí como para que yo me
enterase bien:

«No pasarán dos días, y casi estoy por decir que no pasará ni uno, sin
que entren en Cuenca las tropas del Ejército Real del Centro, mandadas
por Sus Altezas los Serenísimos Infantes don Alfonso y doña María de las
Nieves. Creo que no ha de hacer resistencia este pueblo donde hay pocos
liberales, y esos pocos tontos de remate\ldots{} Si usted teme el fuego
y las balas, póngase en salvo hoy mismo, señora doña Silvestra. Puede
usted refugiarse en mi casa, donde estará más segura que en ninguna
parte. Soy viudo y vivo con mi madre, mi hermana y una hija mía de
catorce años.»

Luego seguimos bajando hasta la plaza de San Vicente. Palomeque invitó a
\emph{Chilivistra} a comer en su casa aquel día, anunciándole que iría a
buscarla a la fonda. El basilisco, con no poca sorpresa mía, aceptó
diciendo al carcunda que se arreglaría deprisa y corriendo para no
faltar a la hora.

Solos otra vez Silvestra y yo, nos dirigimos a la fonda por la puerta
que llaman del Postigo. Íbamos a escape, yo silencioso, ella punzándome
con sus acres intemperancias. «Aprende, tonto---me dijo.---Ese caballero
sí que es fino y galante. Tú, en cambio, eres un avefría y no sabes
tratar con damas.» Poco después de las doce llegó Palomeque a nuestro
alojamiento. Silvestra, bien apañadita de ropa y pergeñada de lindos
accesorios, sin omitir ninguno de los retoques de su bella faz, se fue
con él, dejándome en una soledad deliciosa.

Cuando Ido no había vuelto de sus diligencias, me lancé solo por las
calles de la ciudad baja, después de comer. Por un momento se me ocurrió
volver a la Catedral para pedirle a San Julián que me concediera el
inmenso favor de librarme para siempre de la fémina mortificante y
tornadiza. Pero me detuvo el extraordinario movimiento que notaba en las
calles: iban y venían hombres y mujeres en actitud de recelo y alarma.
Acerqueme a un grupo y no tardé en conocer la causa de tal agitación.
Del pueblo de La Cierva, distante unas cuatro leguas de Cuenca, había
llegado una mujer con la noticia de que allí y en Pajarón estaban los
carlistas: la mar de batallones, con unos llamados \emph{zubabos} que
parecían fieras, y el don Alfonso y la doña Blanca. En otro grupo oí que
de Palomera, distante sólo una legua, acababan de llegar emisarios que
también anunciaban la presencia de las bárbaras legiones.

Antes de amanecer caería sobre Cuenca la turba desmandada, feroz y
hambrienta, y se haría dueña de la ciudad riscosa si las peñas y los
corazones no le oponían una brava defensa.

\hypertarget{xxiii}{%
\chapter{XXIII}\label{xxiii}}

Recluido en el cuarto de la fonda, pasé la noche muy agitado por el
tumultuoso ruido que de la calle venía. Además, me inquietaba que
Silvestra no hubiera vuelto a mi lado, no porque me hiciera falta su
presencia, sino por el temor de que le hubiera ocurrido algún desavío.
Al manso filósofo le esperé hasta la madrugada; mas tampoco vino a la
mansión hospederil. Pensé que había encontrado a su hija, o que las
diabólicas sacerdotisas venustas le retenían en sus nefandos cubículos.
Al amanecer, las cornetas tocaron diana cerca y lejos, las unas en el
interior de la ciudad, las otras en el campo, ocupado ya por los
carlistas. Me asomé un momento a la ventana de mi cuarto, y vi en las
crestas de los cerros humazo de fusilería. Poco después empezó el
tiroteo en los términos cercanos. Dijéronme que los sitiadores atacaban
la Puerta del Castillo, y que ya eran dueños de un barrio del mismo
nombre, situado extramuros de la ciudad.

Bajé al comedor, donde el patrón y otros que con él estaban me dieron
noticias desconsoladoras. Las fuerzas que habían de defender a Cuenca
eran harto débiles: cuatro compañías de la Reserva de \emph{Toledo}, un
escuadrón de Lanceros del Regimiento de \emph{España}, otro de
Carabineros, algunos Guardias civiles, y dos centenares de Voluntarios,
gente por punto general poco aguerrida. Las fortificaciones se reducían
a unas verjas de hierro, arrancadas de las iglesias para ponerlas en las
entradas de la ciudad vieja, y a unos cuantos remiendos echados de
cualquier manera en la vetusta muralla. Cuatro cañones con insuficiente
servicio de artilleros eran las únicas piezas disponibles para tener a
raya al enemigo.

El fuego siguió muy nutrido durante la mañana. Poco antes de las once,
los vecinos de los arrabales, creyéndose poco seguros en aquella parte
de la ciudad, empezaron a trasladarse a toda prisa a la ciudad alta. Mi
patrón y su mujer, personas sencillas y afables, se empeñaron en
llevarme consigo. «Caballero---me dijo el fondista,---aquí no puede
usted quedarse, porque esto está muy malo. Véngase con nosotros. Allá,
en los altos de la Plaza de San Nicolás, tenemos una casita en paraje
resguardado de los zambombazos que atizan esos perros. Coja usted su
ropa y los efectos de valor; nosotros salvaremos lo que podamos. Bueno
que se lleve el diablo nuestros intereses, pero la vida no queremos
perderla\ldots{} ¡Ay, caballero: lo peor para la pobre Cuenca es que
tenemos el enemigo en casa! Muchos vecinos, muchas familias de acá son
carcundas hasta los tuétanos. Conque hágase cargo\ldots»

Por el puente de la Puerta de Valencia me llevaron a un barrio de calles
pinas, angostas y obscuras. Entramos en una casa de no sé cuántos pisos:
la escalera no tenía fin. En un desván lleno de pobretería de ambos
sexos hallé albergue que parecía seguro de las balas, mas no lo era de
insectos y alimañas molestas. En aquel camaranchón traté inútilmente de
conciliar el sueño. Pasada la infernal noche, decidí cambiar de
alojamiento, y bajé a otros pisos donde encontré mejor compañía,
personas amables que me dieron pan y vino para sostener mis fuerzas.
Entre los allí refugiados había un chico de tipo gitanesco, vivaracho y
más listo que el hambre, el cual salía y entraba a cada momento,
trayéndome noticias de lo que ocurría.

Por aquel galopín supe que se habían apoderado los sitiadores de la
Carretería y calles inmediatas, saqueando casas y tiendas con infernal
estrépito. Supe también que los carlistas quisieron parlamentar junto al
Instituto; pero el Brigadier don José de la Iglesia, Gobernador Militar
de la Plaza, hombre tan chiquitín como bravo, les mandó a escardar
cebollinos\ldots{} Mientras el chiquillo andaba recorriendo los sitios
donde más empeñada era la lucha, mi patrón, dolorido y suspirante, me
dijo: «Caballero, nos quedamos sin agua. Esos cafres han cortado el
acueducto en el caserío de la Cueva del Fraile.» La patrona, llorando,
agregó: «¡Ay, Virgen Santísima, mañana no habrá ya pan en Cuenca! El
poco que amasaron hoy se lo arrebata la gente en la calle, y los pobres
que están batiéndose no tienen qué comer.»

Por la tarde, volvió despavorido el chicuelo contándonos que había un
fuego horroroso en la cuesta de Tarros, Matadero, Jardín de las
Carteras, Retiro, San Miguel y las Angustias, con la mar de muertos y
heridos. Una vieja que vino después nos dijo que los Voluntarios, con el
cañón que habían puesto en una de las ventanas del Instituto, estaban
abrasando a los carcas. Otra vieja, con las sayas en la cabeza,
compareció ante nosotros y nos largó un relato terrorífico del fuego que
hacían los carlistas desde las casas contiguas a las puertas del
Postigo, Valencia y convento de la Concepción. Los pobres carabineros,
soldados y voluntarios que defendían aquellos lugares caían como moscas.

La noche fue pavorosa. Los insectos y la fetidez de las habitaciones
atestadas de gente expulsáronme de la casa. Bajé a la calle, prefiriendo
que me matase una bala a morir de asfixia y asco. Tirado en el suelo,
entre un ciego, dos lisiados, un sin fin de mujeres, y rapaces medio
desnudos, me enteré de que los caribes que llamaban Zuavos habían
intentado vadear el Huécar, siendo rechazados por unos cuantos Lanceros.
Las llamas de los incendios daban a la ciudad un aspecto de siniestra
desolación.

El hambre, el miedo y el cansancio me obligaron a meterme en el zaguán
de una casa, y arrimándome a un bulto que debía de ser un durmiente
envuelto en mantas, descabecé algunos sueños. Al amanecer, noté que el
tiroteo había disminuido considerablemente\ldots{} Dijéronme que los
carlistas desmayaban por la tenaz resistencia del pueblo en el día
anterior.

A media mañana, advertí grande animación en la ciudad. Corría la noticia
de que se aproximaba una columna de tropas del Gobierno mandada por un
tal señor Calleja. «¡Ay, Dios mío---exclamaba todo el mundo,---que venga
pronto ese Calleja!» Contagiado yo de estas públicas alegrías, y
sintiendo los horrores del hambre, trepé por los empinados escalones de
una calleja angosta, en busca de un alma caritativa que me diera un
pedazo de pan. Torciendo a mano derecha, vi venir hacia mí un esqueleto
que me estrechó en sus brazos. ¡Por San Julián bendito! El esqueleto
cuyos huesos chocaron con los míos era don José Ido del Sagrario.

«¡Ay, don José de mi alma!---exclamé con grande alegría;---¿está usted
muerto?»

---Por milagro no estoy muerto---me contestó Ido.---Sepa Vuecencia que
una bala me atravesó de parte a parte.

---A ver, a ver; enséñeme esa tremenda herida.

---No es de cuidado. Mire, ha sido en el chaquetón. El proyectil lo pasó
de parte a parte\ldots{} ¡Ay, don Tito, toda la noche buscándole! No ha
sido mala suerte encontrarle ahora para poder decirle\ldots{}

---Cuénteme, don José; ¿ha encontrado a la niña?

---Sí señor. Estuvo algunos días en una casa de picaronas; pero ya
¡gracias a Dios!, ha ido a parar a lugar más honesto, aunque no del todo
limpio. ¡Ah, señor, déjeme usted que suspire!

---Yo también suspiro, don José, pero de hambre.

---¿Hambre Vuecencia, Ilustrísimo Señor? Pues aquí tengo yo pedazos de
pan para Usía. Cómalo, que es bastante bueno.

Vi el cielo abierto. Me abalancé a los mendrugos, y para comerlos con
más comodidad me senté en un escalón, en medio del arroyo. Lo mismo hizo
Ido, y en aquel momento se nos acercaron unos pobres perros que olieron
el pan. No tuvimos más remedio que darles algo de lo que nos sobraba.

«Ya que este corto desayuno me aclara un poco las entendederas---dije al
filósofo,---prosiga el cuento de la infeliz Rosita.»

---Pues nada: que hace días está al servicio de un señor Canónigo, muy
apersonado y muy galán, que la tiene en su casa en calidad de doncella
para todo y con honores de sobrina. Allí he pasado yo toda la noche bien
resguardado de esta horrenda trifulca, y de allí salí a buscar a
Vuecencia para llevármele conmigo.

---¿A casa del Canónigo?\ldots{} ¡Sí, hombre, vamos! Allí estaremos bien
seguros, porque supongo que el amo de Rosita será carcunda neto.

---Sí que lo es, pero buena persona y muy torero, con perdón. Está loco
por la niña\ldots{} Vamos, vamos\ldots{} Pero ¡ay de mí!, buscando a
Vuecencia me he perdido en el laberinto de estas rinconadas y
costanillas, y no sé por dónde volver allá.

En esto, oímos que de la parte baja venía, con gran clamor de gente,
estruendo de cataclismo. Unos ancianos que subían nos dijeron que, en la
calle de la Moneda, los bravos defensores arrojaban petróleo con la
bomba de incendios del Municipio sobre las casas de la calle de los
Tintes, ocupadas por los carlistas. No pudiendo realizar su intento,
lanzaban a mano el líquido inflamable contenido en botellas. Huyendo de
la quema seguimos calle arriba, acelerando el paso. Don José, casi sin
resuello, me dijo: «¿No sabe, don Tito, que ayer tuvieron los carlistas
una gran pérdida? El cabecilla Segarra quedó muerto de un balazo junto
al convento de la Concepción, al atacar la Puerta de Valencia.»

---¿Segarra? Pues en el Infierno nos espere muchos años\ldots{} Vamos,
vamos a ver si podemos dar con la casa del Canónigo. Preguntaremos al
primero que pase para que nos oriente. ¿Cómo se llama ese señor?

Detúvose Ido perplejo, y llevándose un dedo a la frente me dijo: «¡Ay,
señor don Tito!, el apellido del Canónigo es de tal manera enrevesado y
estrambótico, que no sé si lo podré recordar ahora. Ayer, cuando él me
lo dijo, lo apunté en un papel, y toda la noche lo estuve repitiendo,
sílaba por sílaba, para ver si me lo clavaba en la memoria\ldots{}
Espere Vuecencia un poco\ldots{} déjeme pensarlo\ldots{} Ya tengo
algunas sílabas, pero otras me faltan\ldots{} Calma, calma\ldots»

Mediano rato aguardé a que terminase su trabajo mental el cuitado
filósofo. Luego, con semblante risueño, me dijo: «Ya, ya tengo las
sílabas todas. Ahora falta el acento\ldots{} Espérese otro poco,
Ilustrísimo Señor\ldots{} Tengo que arrimarme a la pared para poderlo
decir seguido\ldots{} y he de agarrarme la nuez, vea Vuecencia, la nuez,
que se me quiere escapar cuando pongo el acento\ldots{} Allá va. El
Canónigo que ahora es tío de Rosita se llama de apellido
Pagasaunturdua.»

\hypertarget{xxiv}{%
\chapter{XXIV}\label{xxiv}}

---Después de pronunciar ese nombre---dije yo---es preciso tomar alguna
cosa, por ejemplo, una copita de Jerez. Vamos a ver si ese bendito
Canónigo nos la da.

---Excelentísimo Señor---replicó Ido,---llevando por única guía ese
nombracho no llegaremos nunca. El tío de mi niña hace poco tiempo que ha
venido a esta Catedral desde la de Calahorra, y apenas se le conoce.
Además, señor, no hay un solo conquense que sepa entender ni pronunciar
el trabalenguas de ese apellido.

Llegamos a una plazoleta en la que Ido reconoció que había confundido la
torre de la Catedral con la de \emph{Mangana}, y cuando discutíamos la
dirección que debíamos seguir para enmendar nuestro rumbo, nos vimos
envueltos en un tumulto de gente que nos llevó consigo como barredera
humana al grito de \emph{¡Abajo todo el mundo!} \emph{¡A las Puertas, al
Instituto, que vienen los nuestros!} \emph{¡Ya está ahí Calleja!}
\emph{¡Viva Calleja!} Imposible resistir al torbellino patriótico.
Corriendo, más bien rodando, descendimos por las calles de guijas
puntiagudas. A mi lado se puso, chillando desaforadamente, el chiquillo
gitanesco y vivaracho que me había servido de informante histórico en
los primeros encontronazos entre conquenses y carlistas. «Quieren
entrar---me dijo---por la calle de la Moneda. Allí hay fuerte quemazón.
Pero no saben ellos quién es Calleja. ¡Viva, viva Calleja!»

Fuimos a parar cerca del Instituto, y allí nos encontramos a nuestro
fondista y a un sin fin de mujeres llorosas, que se disputaban los
corruscos de pan\ldots{} No sé el tiempo que duró aquella situación
equívoca en que alternaban los gritos de entusiasmo con las expresiones
de desaliento. Por fin corrió entre la muchedumbre ansiosa esta
desoladora noticia: «El que viene no es Calleja ¡maldita sea su alma!,
sino un cura guerrillero que llaman \emph{el de Flix}, con dos
batallones de fieras desbocadas\ldots{} ¡Perdición, ruina,
muerte!\ldots»

Esta triste realidad alentó a los carlistas residentes en Cuenca.
Propalaron por todas partes que los sitiadores entraban ya en la ciudad,
sembrando el desaliento, y muchos defensores se retiraron de sus
puestos, convencidos de que era inútil toda resistencia. Sin saber cómo,
nos encontramos Ido y yo en la miserable casa donde pasé la primera
noche de asedio, y en uno de sus aposentos nos guarecimos, esperando la
suerte que nuestro adverso Destino nos deparara. Allí supimos por
algunos Voluntarios que los defensores que ocupaban el Jardín de las
Carteras se habían retirado y la facción era ya dueña de algunas casas
de la calle de la Moneda.

La última página de la tenaz resistencia fue gloriosamente escrita por
el Gobernador Militar, Brigadier don José de la Iglesia, que levantando
barricadas disputó palmo a palmo la ciudad a las salvajes hordas
realistas. En esta postrera jornada pereció heroicamente el Teniente
Coronel de la Reserva de \emph{Toledo} don Francisco de la Peña. En
tanto, el Brigadier La Iglesia, sereno en medio del peligro, al frente
de cuarenta hombres, se retiraba lentamente mandando hacer fuego de
trecho en trecho. Al llegar a la parte más empinada de la calle de San
Pedro, agotados todos los recursos y siendo la retirada imposible, hizo
señal de parlamento. Los carlistas, que estaban a pocos metros,
destacaron un pelotón mandado por un jefe. La Iglesia se desciñó la
espada, y entregándola al cabecilla, puso término definitivo al esfuerzo
gigante de los humildes y beneméritos defensores de Cuenca.

Desde aquel momento cambió con súbito giro el panorama histórico,
trocándose el honrado choque de las armas rivales en feroz
desbordamiento de los vencedores, que hollaron con cínica barbarie las
leyes de la Guerra y los elementales principios de Humanidad. Contaré
los horrores, crímenes y vergüenzas de las jornadas de Cuenca en los
días 15, 16 y 17 de Julio, con toda la fidelidad que mi oficio me
impone; contaré lo que vieron mis ojos espantados y lo que, visto por
otros ojos, fue transmitido del alma de las víctimas y de sus allegados
al alma dolorida de este humilde narrador. Ante la brutalidad de los
hechos que fluctúan vagamente entre lo verdadero y lo inverosímil
evitaré la mentira y la hipérbole, y no recargaré de negras tintas las
perversidades de los hombres, ni aun cuando éstos, más que hombres,
parezcan demonios.

Al penetrar en la ciudad las manadas realistas, fueron víctimas de su
desenfreno las propias familias de los vencedores. Diose el caso de que
algunos facciosos nacidos en Cuenca oyesen de labios de sus madres, al
abrazarlas, súplicas implorando respeto para sus vidas y haciendas. Pero
tales ansias traían aquellos bárbaros de celebrar su victoria con la
saciedad de todos los apetitos, aun los más infames, que nada
respetaron. Entraban en las casas, lo mismo por las puertas que por las
ventanas, forzaban los muebles, sacaban ropa, dinero, alhajas, y luego
porfiaban entre sí para repartirse el fruto del pillaje. Lo mismo
expoliaron las casas liberales que las carlistas; no hicieron
diferencias de clases ni de ideas, ni se acordaron para nada de la
Religión que figuraba en su execrable bandera.

En una desdichada iglesia, cuyo nombre no recuerdo, afanaron con avara
rapidez un soberbio pectoral, dos mantos de terciopelo de San Juan, y
una corona, rosario y diadema de la Virgen del Puente. En los casinos
rompieron los espejos, las mesas y sillas, hartándose de licores, cuyas
botellas arrojaban a la calle después de vaciarlas. En el Instituto
destruyeron el Gabinete de Física y el de Historia Natural, lanzando por
las ventanas los aparatos y las colecciones zoológicas. Al ver la
máquina eléctrica llegó a su máximum el ansia de destrucción, y mientras
la pulverizaban decían: \emph{¡Duro, duro con esto, que sirve para
mandar partes al Gobierno!}

Se les veía correr de calle en calle y de casa en casa, dando alaridos
de salvaje alegría. Algunos se desnudaron públicamente para vestirse la
ropa blanca y los trajes que habían robado. Después de vestidos, dejaban
en medio del arroyo los guiñapos llenos de porquería y miseria. Aunque
uniformados, los \emph{Zuavos de los Príncipes} presentaban el aspecto
más siniestro y repugnante por la desenvoltura cínica de sus maneras y
la grosería de sus vociferaciones, en ronca mixtura de italiano y
francés. Con hambre atrasada devoraban embutidos, lonchas de jamón y
cuanto pudieron atrapar. Por toda la ciudad retumbaron destemplados
toques de corneta y estas estridentes voces: \emph{¡No hay para nadie
cuartel!}

De los \emph{Zuavos} y de los que no eran \emph{Zuavos} huían las
mujeres, lo mismo jóvenes lozanas que viejas tembliconas, corriendo a
refugiarse en los sótanos más hondos o en los más altos desvanes. Aun
allí eran perseguidas, pues aquellas bestias lujuriosas no sólo habían
perdido la vergüenza sino el sentimiento de la hermosura, de la gracia y
de la juventud\ldots{} Los facciosos no se limitaron a saciar sus
groseros instintos, y movidos de criminal saña política, perseguían como
perros rabiosos a los \emph{cipayos}, que así llamaban a los liberales,
y a los que habían contribuido con su denuedo a la defensa de la
población.

Voy a referir a mis horrorizados lectores el trágico fin del Comandante
don Enrique de Escobar y Valdeolivas, que se hallaba en situación de
reemplazo, recluido en su domicilio por larga enfermedad. Creyeron los
carlistas que aquel \emph{cipayo} había tomado parte en la defensa, y
asaltaron su casa, en la calle de Cordoneros, subiendo atropelladamente
hasta las habitaciones altas, donde el infeliz señor yacía en el lecho,
asistido por su madre. Al verse rodeado de aquellas fieras que le
insultaban profiriendo las amenazas más atroces, el desdichado enfermo
perdió el conocimiento. La madre lloró, imploró, y no pudiendo ablandar
los corazones petrificados por la incultura y el fanatismo, se abrazó a
su hijo intentando en vano librarle de las acometidas de tales
monstruos. Sobre el cuerpo de la pobre mujer llovieron golpes terribles.
El Comandante fue cosido a bayonetazos, y cuando ya se le escapaba la
vida, arrancáronle de los brazos maternales y lo arrojaron por el
balcón.

El cuerpo chocó contra las piedras, y yacía exánime en medio del arroyo,
cuando apareció en la calle abigarrada muchedumbre, a cuya cabeza venía
una mujer a caballo, como amazona de circo, radiante de fatuidad,
decidida y altanera. Era la tristemente famosa Princesa doña María de
las Nieves, esposa de don Alfonso de Borbón. Los que la vieron venir
pensaron que desviaría su caballo para no pisar el cuerpo expirante.
Pero la terrible capitana de bandidos no se inmutó, y sin dar señales de
ninguna emoción ante aquel espectáculo dejó que el animal pisotease a un
honrado caballero moribundo.

\hypertarget{xxv}{%
\chapter{XXV}\label{xxv}}

Siguió la cruel amazona su sangriento camino hacia la Correduría. Era de
corta estatura, flaca, rubia, de azules ojos: su belleza, completamente
apócrifa, consistía tan sólo en la marcialidad de su apostura y en su
destreza hípica, cualidades de marimacho, no de mujer. En su rostro vi
un mirar ceñudo y una rígida contracción de la boca que indicaban la
sequedad del corazón confundida con la brutal soberbia. Llevaba boina
roja con borlón de oro, traje negro de montar, altas botas de charol, en
la mano un latiguillo que le servía de bastón de mando, y en el cinto un
revólver. Tras ella iba el marido, que sólo brillaba por su
insignificancia junto a la marimandona. Llevaba boina encarnada con
áureo borlón y dormán de Caballería. Seguía la caterva de jinetes,
algunos con distintivos de oficiales, otros con escolta, todos de
aspecto bárbaro y provocativo.

No sé a dónde iban en aquel instante. Pero, esclavo de mi obligación, he
de referir las escenas más patéticas del drama conquense, y para ello
haré uso del don de ubicuidad que, con otras atribuciones, me concede en
casos tales mi divina Madre \emph{Clío}. Sabed, pues, que aquella mañana
presentose ante la Catedral el aparatoso y ridículo cortejo de la
Generala doña Nieves de Borbón, de Braganza o de los demonios coronados.
Apeose la tal de un salto y entró en la basílica seguida del marido y de
los jefes que componían su abigarrado séquito. Junto a ella se coló en
el sagrado recinto un perro de presa que era su inseparable compañero.
Ya se habían dado las órdenes para que el Obispo saliese a recibirla y
le cantase el indispensable \emph{Tedéum} por la feliz entrada del
Ejército Real en la histórica ciudad de Cuenca.

He aquí, lectores míos amadísimos y cristianísimos, al venerable Prelado
señor Payá y Rico, plantado en el trascoro con todo su Clero, para
recibir ceremoniosamente a la que representaba el poder
majestático{[}2{]} impuesto por la fuerza bruta. Con evangélica humildad
acompañaron el Obispo y Clero Capitular a los regios figurones,
llevándolos al presbiterio, donde tomaron asiento en los sillones
preparados para el caso. El \emph{Tedéum} fue breve, llevado a paso de
carga, a estilo militar. Berrearon los cantores de mala gana, y el alto
Clero, con excepción del Obispo, hizo gala de la pompa litúrgica y de su
fanático servilismo.

Terminada la ceremonia con su canticio bostezante, acompañado de sonoros
golpes de órgano, los Príncipes \emph{de la sangre} se aposentaron en el
Palacio del Obispo, próximo al templo diocesano. Ignoro si la ocupación
de la morada episcopal fue por galante obsequio del señor Payá y Rico, o
por \emph{motu proprio} de la desenvuelta doña Nieves, que a sus
indudables dotes de mando unía la frescura y desahogo que a las personas
vulgares da la falsa conciencia del derecho divino. Su temple arbitrario
se manifestaba lo mismo en la llaneza para incautarse del solar ajeno,
que en la fea costumbre de tutear a las personas de más alta posición y
jerarquía. Apenas instalada en el Palacio la trashumante Corte, se
vieron llenas de uniformes las anchurosas estancias; el arrastrar de
sables y el militar bullicio sustituyeron al murmullo sigiloso de una
mansión eclesiástica.

En el salón de honor, decorado con un soberbio crucifijo, recibieron los
Príncipes comisión de señoras, comisión de notables, que eran lo más
granadito de la carcundería conquense. Allí dictó la despótica doña
Blanca los fieros bandos que causaron terror al sufrido vecindario. En
el primero se ordenaba que los habitantes de la ciudad, sin distinción
de clases, acudieran a demoler las fortificaciones, llevando ellos
mismos los útiles y herramientas necesarios. En el segundo se disponía
que acudieran las mujeres y señoras con vasijas llenas de agua a sofocar
el fuego del Gobierno civil, incendiado por los carlistas. El tercero,
inspirado por Judas, mandaba que todos los Voluntarios defensores de
Cuenca se presentasen en los claustros de la Catedral, advirtiendo que
de no hacerlo así serían fusilados donde quiera que se les encontrara.
Los tres bandos se fijaron en las esquinas o fueron publicados por
pregón, y decían que sus disposiciones habían de cumplirse \emph{bajo
pérdida de la vida}.

Obedientes a las draconianas órdenes de la que algunos llamaron \emph{el
Atila con faldas}, acudieron con palas y picos los pobres de chaqueta y
los señores de levita a desbaratar las obras de fortificación. Y como a
todos les iba en ello la pelleja, también corrieron a sofocar el fuego
las menestralas y las señoras, transportando el agua en cántaros,
barreños y pericos. Los Voluntarios defensores de la Plaza, entendiendo
que serían indultados si hacían acto de arrepentimiento en el sagrado
recinto de la Catedral, allá fueron cual ovejas sumisas y, con más
paciencia que el amigo Job, esperaron el fallo benigno de la serenísima
tirana.

¿Benignidad dijisteis? Espérense un poco, caballeros. Apenas estuvieron
los voluntarios reunidos en los claustros de la basílica, llegó una
cuadrilla de \emph{Zuavos} que les maniató por parejas; sin pérdida de
tiempo los condujeron a los sótanos del Palacio episcopal, y allí
quedaron encerrados cual rebaño destinado al sacrificio.

En tanto, la soldadesca vencedora, harta de comistrajes y de vino, harta
de volubles placeres, mas nunca saciada ni satisfecha en sus brutales
instintos, continuaba la cacería y exterminio de \emph{cipayos}. Pedro
Díaz Escamilla, maestro alpargatero de la Casa de Beneficencia,
voluntario que peleó en la calle de la Moneda, retirose herido,
escondiéndose en un desván de su casa. Allí lo encontraron los
carlistas, y después de rematarlo a tiros y bayonetazos le rompieron el
cráneo con las culatas de los fusiles, haciendo saltar en pedazos la
masa encefálica. A la viuda de este infeliz la martirizaron cruelmente
pinchándola en la espalda, y a una muchachita hija del muerto le dieron
a beber tila con pólvora \emph{para que se le pasara el susto}.

A un pobre vendedor de frutas, \emph{Anico el de la Ventosa}, a quien
acusaban de haber matado a dos \emph{Zuavos}, lleváronle a rastras por
las calles con infernal gritería, y después de asestarle innúmeros
bayonetazos, acabaron con él, junto al cuartel de San Francisco,
quemándole la cara con petróleo. Un humilde dependiente municipal fue
capturado cuando regresaba de llevar un parte del Ayuntamiento al
Brigadier Villalaín. Cediendo a instigaciones de un carlista conquense,
aquel desventurado fue conducido en las puntas de las bayonetas por la
Correduría, y en su sangre mojaron los asesinos la suela de las
alpargatas \emph{para reforzarla}. Junto a la Puerta del Postigo asesinó
la soldadesca a un cartero, de quien dijo una mujer que había dejado de
entregar algunas cartas a los carlistas del pueblo. La agonía de este
desgraciado fue horrenda, pues su delatora se obstinaba en hacerle comer
pan y pepino.

Por soplo de gentes malignas, que nunca faltan en casos tales, supieron
los vándalos del \emph{Dios, Patria y Rey}, que en una casa del Pósito
se escondía un \emph{cipayo} llamado Vicente Cornago, enfermo de viruela
negra. Allá marcharon en tropel los asesinos, decididos a librar de
penas al virulento. La pobre madre del enfermo creyó que mostrándoles el
cuerpo de éste, cubierto de pústulas, les convencería de la verdad de la
dolencia. Los menos feroces quedaron perplejos; mas otros, que sin duda
eran fieras en figura humana, insistieron en asegurar que el
\emph{cipayo} era un enfermo de conveniencia y que aquellas costras
serían pintadas. La embriaguez les enloquecía. Tras una espantable
escena en que la madre trató de salvar la vida de su hijo, abrazándole
con desesperado esfuerzo, se consumó el crimen odioso, entre salvajes
gritos y carcajadas infernales de aquellos caribes.

Más horrores contaría; pero temo que mis buenos leyentes aparten sus
ojos de estas páginas, bárbaramente ensangrentadas. Por mi gusto pondría
siempre en ellas la miel de la Historia, aderezándola sabiamente con las
hieles amargas que en todo tiempo afluyen de las humanas acciones. Mas
tengo que rendirme a las brutalidades de una raza, que en sus accesos de
locura suicida se divierte rasgando sus propias venas para morir de
anemia.

Diré tan sólo que a la mujer de un pobre zapatero, asesinado en la calle
del Agua, dieron el pañuelo de la víctima empapado en su propia sangre,
caliente todavía. A la esposa de un humilde agente de Orden público le
ofrecieron el sable con que acababan de cercenar el cuello de su marido.
No satisfechos los facciosos con ser asesinos y ladrones, fueron también
incendiarios, y a más del Gobierno civil pegaron fuego a la Diputación
provincial, a la Plaza de Toros y a otros edificios. Con enormes
lavativas lanzaban petróleo a los pisos altos; con regaderas empapaban
de líquido inflamable las plantas bajas. El inmenso ruedo de la Plaza de
Toros, del que surgían llamas gigantescas, era como el cráter de un
volcán.

Como infernal apoteosis de aquella fiesta de barbarie, clavaron los
vándalos banderillas de fuego a los caballos heridos o enfermos que,
locos de dolor, corrían por la ciudad, entre el chisporroteo y las
detonaciones de la pólvora que abrasaba sus carnes.

\hypertarget{xxvi}{%
\chapter{XXVI}\label{xxvi}}

Mi privilegio de ubicuidad permitiome presenciar las pomposas audiencias
que daba doña Nieves a los Jefes de su mesnada de matachines: salían
éstos llevándose el aplauso y albricias de su Generala por los
asesinatos y desvergüenzas con que castigaban al pueblo infeliz. En
esto, anunciaron la llegada de una Comisión del Ayuntamiento que iba,
con toda sumisión y protestas de fidelidad, a impetrar de Sus Altezas
clemencia para los vencidos. Como medida preventiva, metieron a los
comisionados en las habitaciones bajas donde estaban las cuerdas de
Voluntarios presos. No quise dejar de ver a los que representaban el
organismo municipal, algunos del antiguo Ayuntamiento, otros de la nueva
hornada carlista. En todos vi caras afligidas y largas, y admiré las
arrugadas levitas que habían sacado del fondo de los cofres para
presentarse ante las reales personas, así como las chisteras abolladas y
peinadas a contrapelo en las precipitaciones que la etiqueta les
imponía.

\emph{Francamente, naturalmente}, diré con mi amigo Ido que me
acompañaba por las escaleras y pasillos de la casa episcopal, me dieron
lástima los señores concejales tratados como perros, y aun el propio don
Avelino Palomeque, concejal de nuevo cuño, me fue menos antipático, por
verle en tan humillante situación. No pensaba yo hablarle, pero él se
dirigió a mí con menos arrogancia de lo que yo esperaba, diciéndome
estas palabras: «No pase usted pena por doña Silvestra, que está bien
segura en mi casa, al lado de mi madre. Si los excelsos Príncipes
acceden a lo que les pediremos, todo se arreglará.»

---Quédese \emph{Chilivistra} al lado de su señora madre---contesté yo
cumplidamente,---que allí estará como en la gloria. Y si la nobilísima
doña María de las Nieves la toma bajo su protección, miel sobre
hojuelas. Silvestra es una malva como usted habrá visto, un carácter
angelical, dulcísimo. Para mí será muy grato que permanezca en la
honesta y sagrada casa de usted hasta que Dios fuere servido de poner
término a los males que a todos nos afligen.

Díjome entonces el estirado señor Palomeque que si yo gozaba, como
parecía, de algún predicamento cerca de la brava doña Nieves y de su
augusto esposo, les hiciese presente la conveniencia de que fuera pronto
recibida la Comisión municipal que ansiaba ofrecerles sus respetos. Sin
negar yo mi supuesta influencia, respondí que hablaría de buen grado a
los Serenísimos Infantes, procurando llevar a feliz término aquellas
diferencias, y añadí que esperaran sentaditos a que de arriba viniera la
orden de ser recibidos en audiencia solemne.

Volví con Ido del Sagrario al piso principal, y lo primero que vi fue el
venerable Obispo sentado en el banco del portero, aguardando ser
admitido a la presencia de doña Nieves. Diferentes personas había en la
antesala, y entre ellas\ldots{} no sé si por testimonio de mis ojos o de
mi exaltada imaginación\ldots{} creí distinguir la faz de
\emph{Mariclío} en un grupo de señoras que hablaban con Payá y Rico,
lastimándose de la humillación que sufría. Estoy bien seguro de haber
oído de labios del Prelado estas tristes palabras: «Ayer me pedían
ustedes su protección: hoy la necesito yo.» Puse toda mi alma en
cerciorarme de si era verdad la presencia de \emph{Mariclío}, mas no
pude obtener la certidumbre que buscaba porque el buen Ido me cogió de
un brazo, y llevándome al cercano pasillo donde aguardaban varios
clérigos en actitud expectante, me dijo: «Véngase acá, Ilustrísimo
Señor, que quiero presentarle al Canónigo Pagasaunturdua. Este buen
señor desea conocer a Vuecencia.»

Presentado al Canónigo, nos estrechamos las manos con familiaridad
cortesana. Era un clerizonte guapo, joven y rollizo: su desenvoltura de
lenguaje y ademanes revelaban el gusto del buen vivir y el menosprecio
de las trabas y preocupaciones que entorpecen la existencia. Después de
los saludos campechanos, quedamos en que honraría yo su casa aquella
misma tarde para tomar juntos una copita de Jerez y fumar un buen
habano.

Al volver a la antesala vi que entraba una caterva de vándalos,
arrastrando los sables y metiendo mucha bulla. Entre denuestos y
amenazas decían que la canalla \emph{cipaya} trataba de asesinar a los
Príncipes, y que para castigar su intento sería conveniente acabar con
ella. De estas inauditas barbaridades resultó que Sus Altezas dieron
orden de despedir a la Comisión municipal, mandándola que se largara con
viento fresco. Poco después fue admitido en audiencia el reverendo
Prelado, y al gozar yo el extraordinario privilegio de presenciarla
reconocí la proximidad de mi excelsa Madre, que por interés de ella y
honor mío se dignaba ponerme en directo contacto con las verdades
netamente históricas.

Vi a doña Nieves en pie frente a una mesa forrada de damasco. Rodeaban a
la Infanta su insignificante esposo y unos cuantos bigardos de su
cuadrilla: Monet, Grollo, Freixá, Villalaín, el cura de Flix y otros. La
Generala vestía un traje de amazona, cuya falda recogía con la mano
izquierda; en la derecha empuñaba un latiguillo que era como el cetro de
su realeza, lo mismo a caballo que a pie. El perro de presa no faltaba
en aquel acto solemne, vigilante al lado de su ama. Con la boina roja
encasquetada, los cabellos rubios mal recogidos en un voluminoso moño,
el cuerpecillo tieso, la mirada fría, el rostro avinagrado, condensando
en sus duras facciones toda la energía de un alma dominadora y salvaje,
aguardó la entrada del Obispo.

El venerable Payá se adelantó con sereno continente, y anticipando sus
finas reverencias, rogó a la Infanta que perdonase la vida a los
Voluntarios presos y que pusiera término a los actos de inhumana
crueldad, tan contrarios a la Religión que el Rey don Carlos ostentaba
en su bandera.

«Ya he dicho a las señoras---contestó colérica y nerviosa la terrible
mujer---que mis soldados necesitan un poco de expansión, después de los
trabajos y privaciones que han sufrido.» Y tras esto, atreviéndose a
tutear a persona tan venerable, investida de la autoridad evangélica,
esgrimió el látigo para imprimir acento y vigor a estas infames
palabras: «Da gracias a Dios porque no hacemos contigo lo mismo que con
todos esos miserables.»

Aguantó el Obispo con firme ánimo la rociada y dijo, tarde ya pero aún a
tiempo, lo que debió decir a los Príncipes cuando entraron en Cuenca
pidiéndole que les cantara un \emph{Tedéum}. Allá va el verdadero
\emph{Tedéum} y la sagrada voz evangélica de un Prelado que sabe su
obligación: «Señora: con esa conducta ni se conquistan tronos en la
tierra ni coronas para el cielo. Adiós, adiós.»

Dio media vuelta el buen Payá, y retirose de la sala sin hacer la menor
reverencia.

\hypertarget{xxvii}{%
\chapter{XXVII}\label{xxvii}}

Permitidme ahora, lectores muy católicos y muy amantes de nuestra
patria, que os dé una opinión sincera y humana de la nefasta María de
las Nieves, opinión que, sin excluir las execraciones que merece al
mostrarse por primera vez en la candente arena de aquel torneo político
y militar, contendrá las alabanzas que le corresponden como el modelo
más extraordinario de fuerza y energía dentro del tipo femenino.

Al ponerse con su esposo al frente del Ejército Real del Centro, doña
Nieves fue el alma de la facción; se impuso a todos los cabezas y
cabecillas; erigiose en Generalísima incuestionable; llegó a ser muy
pronto la primera estratega, la primera autoridad táctica de sus
cuadrillas, a las que disciplinó y gobernó dándoles apariencias de
hueste organizada. Compartía con sus soldados las inclemencias del cielo
y las fatigas de las penosas jornadas; compartía también con ellos los
piojos, la bazofia, los mendrugos de pan, la dureza de los lechos de
piedra en las sierras ásperas, la humedad y desamparo en las desoladas
llanuras.

De este modo les llevó a la conquista de Teruel, tan difícil y cruenta
que hubo de levantar el asedio y salir en busca de otras arriesgadas
aventuras. Con su infatigable tropa, ella, que no conocía tampoco el
cansancio, compartió la rabia de no haber podido ganar a Teruel, y en
terrible avalancha cayeron sobre la pobre Cuenca, donde alcanzaron la
gloria (que gloria fue para ellos) de plantar por primera vez en la
capital de una provincia española el pendón del Carlismo.

Cuando se tuvo en Cuenca conocimiento de la entrevista de \emph{doña
Blanca} con el señor Obispo, antes referida, dijeron algunos: \emph{esa
mujer es una hiena}. Pues yo os digo que será todo lo hiena que se
quiera en determinada ocasión; pero me permito enmendar la frase de este
modo: \emph{esa mujer\ldots{}} \emph{es un hombre}, el primer hombre del
absolutismo, desde los tiempos de don Carlos María Isidro hasta la edad
presente. En los días del asedio de Cuenca, cuando los Infantes tenían
su Cuartel General en un lugar apacible de la Hoz del Huécar, la
Generala, que todo lo disponía y ordenaba como experto caudillo, viendo
que la ciudad no se rendía tan pronto como ella deseara, llamó a
Villalaín y le dijo: «Necesito que las tropas reales tomen al momento la
ciudad. Apelo a tu bravura y no creo hacerlo en vano. Ve y tómala, yo te
lo mando. Si en el término de una hora no se cumplen mis órdenes,
fusilarás al jefe u oficial que flaquee en el cumplimiento de su deber.»

Chispazos del genio de Atila y del Tamerlán iluminaban el cerebro de
aquella hembra temeraria y cruel, negación de su sexo. Desde el momento
en que Cuenca cayó en poder de \emph{las honradas masas}, la doña Nieves
les permitió todas las brutalidades, crímenes, atropellos y vandálicas
libertades que se han descrito, porque sabía que de este modo se captaba
para siempre la voluntad y sumisión de aquellos forajidos.
Consintiéndoles la saciedad de sus apetitos, les adiestraba para
continuar peleando por ella y allanando los caminos por donde corría
desenfrenada la feroz ambición del marimacho más genial que ha tenido
España.

Aquella misma tarde, don José y yo volvimos la espalda a los horrores
trágicos para penetrar en la mansión apacible del Canónigo don Plotino
Pagasaunturdua. Abrionos la puerta Rosita, no sin precauciones,
descorriendo cerrojos y quitando trancas de hierro. El buen Capitular no
había llegado aún: estaba acompañando al señor Obispo y dándole ánimos
para soportar la tribulación que sufría. ¡Por Júpiter Capitolino y por
la divina Cytherea, que me gustó Rosita! Estaba muy linda, tan limpia y
bien apañada de ropa y aliños del rostro, que daban ganas de comérsela.
Por hacer tiempo a que llegara su amo nos llevó al aposento alto en que
moraba, en el cual admiré el buen arreglo y la comedida elegancia de un
vivir modesto y dichoso.

Antes de repetir en mi presencia lo que a su padre había dicho respecto
a su nuevo estado, quiso mostrarnos a los dos los diferentes regalitos
que le había hecho su tío putativo: unos zapatitos de charol muy monos,
todavía no estrenados; un vestido de merino negro muy honesto y
apersonado, para ir a la iglesia; otro de percal sin colorines, pero
adornado con mucho Chispazos del genio de Atila y del Tamerlán
iluminaban el cerebro de aquél; varias alhajillas de poco precio, de oro
fino, que no llamaban la atención ni por sus dimensiones ni por su
riqueza; medias negras de semiseda; zapatillas de abrigo para dentro de
casa; peines y avíos de tocador; un rosario hecho con huesos de
aceitunas del Huerto de las Olivas donde oró el Señor en Jerusalén; y,
por último, un devocionario monísimo, con sus tapas de nácar y broche
dorado para cerrarlo. Viendo y admirando estas cosas advertí en el
rostro de Ido del Sagrario una mezcla singular de alegría y tristeza.

Cuando Rosita, un poco cohibida y vergonzosa, empezó a contarme las
razones que tenía para no abandonar aquella casa, llamó a la puerta el
Canónigo. La muchacha bajó, abrió, y poco después estábamos los cuatro
en la sala donde el buen sacerdote recibía sus visitas. Desde el primer
momento nos mostró don Plotino su llaneza y amabilidad campechana. No
necesitó pedir el Jerez, pues Rosita se apresuró a traerlo, acompañado
de bizcochos y de unos puros, no de primera, pero bastante aceptables.
Como supondréis, la conversación recayó al instante en el asunto de
actualidad que excitaba los ánimos en Cuenca. Sirviéndonos Jerez y
excitándonos a no ser parcos en la bebida, el desahogado cura señor
Pagasaunturdua nos dijo:

«Es preciso confesar que esa buena señora nos ha hecho un flaco servicio
con venirse acá mandando las tropas de don Carlos. Quedárase la doña
Nieves en Albarracín o en cualquier otra parte de los Estados del
Centro, y no hubiéramos tenido aquí los desmanes y atropellos que
ustedes han visto. ¡El demonio con la señora esa!\ldots{} ¿Se enteraron
ustedes del trato que dio esta mañana al señor Obispo?»

---Sí que nos enteramos, señor don Plotino---repliqué yo.---Si usted me
lo permite, le diré que ese trato y otro peor lo tenían ustedes bien
merecido por haber salido a recibirla con palio y largarle luego el
\emph{Tedéum} con órgano, cantorrio y toda la pesca. ¿Por qué el señor
Payá, cuando la vio entrar en la Catedral, no mandó al perrero que la
pusiera en la calle?

---Eso no podíamos hacerlo, señor don Tito de mi alma---repuso
Pagasaunturdua con humildad risueña, tras de la cual asomaban puntas y
ribetes de ironía.---El señor Obispo y el Cabildo cumplieron su deber.
La Iglesia está siempre en su puesto, y no podía negarse a rendir
honores a los Serenísimos Príncipes, hermanos del Católico Rey, nuestro
Señor\ldots{} Comprendo lo que quiere usted decirme; tiene usted razón;
déjeme concluir\ldots{} Esta tarasca nos ha puesto en una gravísima
tirantez de relaciones con el pueblo en que vivimos, y no sé en qué
parará esto. En fin; creo que se van mañana tempranito. Dios vaya con
ellos; la Virgen les acompañe\ldots{} y que no vuelvan a parecer por
acá.

---¡Desgraciado el pueblo en que caigan ahora esos serenísimos
diablos!---exclamó Ido elevando los ojos al techo y atizándose otra
copa.

Haciendo lo mismo, el Canónigo pasó a tratar de un asunto muy
interesante: «Pues no se van con las manos vacías---nos dijo.---Como
contribución de guerra, he aquí que arramblan con todos los fondos
públicos y particulares que hay en Cuenca. Verán ustedes: a los vecinos
les han sacado cerca de 800.000 reales; de la Caja provincial han
sustraído bonos del Tesoro, libramientos y metálico, por valor de 90.000
pesetas mal contadas; de la Delegación del Banco de España, casi
100.000; de Tesorería, en pagarés de bienes nacionales y metálico, más
de medio millón.»

Lo restante de nuestro coloquio no merece mención. Al despedirnos del
bondadoso don Plotino Pagasaunturdua, le preguntamos si podríamos contar
con su ayuda para salir de Cuenca sanos y salvos. Él, con gallardía
protectora, nos dijo: «No teman nada; si los Serenísimos se van mañana,
como dicen, yo les respondo a ustedes de que podrán salir tranquilos,
sin que nadie les moleste, ya se vayan por Huete, ya por San Clemente y
Villarrobledo\ldots{} Conque, adiós, señores, y descansar, que buena
falta nos hace a todos\ldots{} Usted, don José, no ponga esa cara triste
ni haga pucheros: su hija está en mi casa como en la gloria. ¿Verdad,
Rosita, que no quieres volver a Madrid?\ldots{} Repito que su hija de
usted, señor de Ido, al venir a esta su casa, ha pasado del Infierno a
la Bienaventuranza\ldots{} ¿Verdad, Rosita?\ldots{} ¡Ay, señor Sagrario!
Si usted la hubiera visto donde estuvo, no lloraría de verla aquí. Al
contrario, bailaría de gusto.»

---¡Ah, sí, señor!\ldots{} Pero ya ve usted\ldots{} un
padre\ldots---rezongó el filósofo lagrimeando.

---Rosita está muy contenta. Vea usted esa cara---dijo el clérigo,
acompañándonos hasta la puerta.---¡Y ahora que la voy a llevar de
viaje!\ldots{} En cuanto llegue Agosto tomo una licencia y me voy a
Lequeitio, mi pueblo, para que Rosa respire las brisas del Cantábrico.

\hypertarget{xxviii}{%
\chapter{XXVIII}\label{xxviii}}

Camino de nuestro albergue (que era una cabrería de la calle de Pilares,
donde pasamos la noche anterior con sosiego y buena compañía), iba yo
consolando al buen Ido, lo que no me fue difícil, pues la fácil teoría
del mal menor vino muy a pelo para el caso de la deshonra de Rosita.
¿Qué mejor solución podía esperar el desolado padre que ver a la niña
reposando a la sombra de una protección tan benéfica como la de don
Plotino? Obra fue de los hados\ldots{} estoy por decir que de la Divina
Providencia. Por lo que el propio Ido me contara cuando llegamos a
Huete, sabía yo los horribles temporales que había corrido la niña,
desde que la raptaron en Fuentidueña de Tajo hasta que fue a caer en las
inmundas mancebías. El cómo pasó Rosita de tal ignominia a las
paternales manos del Pagasaunturdua, ni don José lo sabía, ni en
averiguarlo teníamos interés. Nos contentábamos con celebrarlo y ver en
ello una divina intriga tramada por los ángeles del cielo. Así debía
decirlo el filósofo simple a su esposa Nicanora cuando le diera cuenta
de la paz y tranquilidad que la niña disfrutaba. Era cosa de que toda la
familia festejara el suceso alabando al Señor y encomendándose a la
Santísima Virgen.

A nuestro cansancio debimos un profundo y dilatado sueño, entre cabras y
honrados vecinos de Cuenca. Al despertar, avanzada ya la mañana, oímos
gran trompeteo y bullanga: los forajidos se iban, con su condenada doña
Nieves a la cabeza. Marchaban hacia Levante, llevándose prisioneros a
los soldados del Ejército, a los Voluntarios liberales y a gran número
de contribuyentes, personas de arraigo y posición. ¡Pobrecitos, buena
les esperaba! ¡Infeliz Cuenca, infeliz España!

Decididos mi amigo y yo a poner pies en polvorosa, nos abocamos con
nuestro protector don Plotino, el cual ya nos tenía preparada fácil
salida en los carros de unos madereros que por San Clemente iban a
Villarrobledo. Nos despedimos de Rosita, y en la tierna escena advertí
que las lágrimas de Ido del Sagrario eran más de alegría que de
pesadumbre. La sobrina del Canónigo dio a su papá un imperdible de oro
muy lindo para que lo entregase como recuerdo de la tierna hija a la
nunca olvidada madre. ¡Adiós, Rosita; adiós, don Plotino, trasunto de la
Providencia; adiós, Catedral, Obispo, vecindario cadavérico; adiós,
Cuenca moribunda y trágica, aún envuelta en humo, en vapores de sangre,
en ambiente de tristeza y desolación!

No quisimos partir sin informarnos del paradero de Silvestra. Mandamos
un recado a la casa del concejal señor Palomeque; mas como este señor no
nos diera ninguna respuesta, creímos perdida a la voluntariosa y
antojadiza dama, de cuya reaparición daré noticias a mis buenos lectores
en posteriores páginas, que ya no caben en este libro. No saldré de la
patria de San Julián sin deciros que recobramos parte de nuestro
equipaje y que momentos antes de partir vimos entrar por Carretería
tropas a caballo, vanguardia de una columna mandada por el General Soria
Santa Cruz, que el Gobierno envió el día 13 en auxilio de Cuenca.
Entraban con extraordinarias precauciones, cuando ya en Cuenca no había
ni un voluntario de la facción. ¡A buenas horas mangas verdes!

Salimos en la gratísima compañía de los madereros. Y no te digo adiós,
lector pío, benévolo, buen católico y amante del orden social; no te
digo adiós sino \emph{hasta luego}, pues la deuda que tengo contigo de
referirte lo de \emph{Sagunto}, aplazada queda por apremios del tiempo y
del espacio, superiores a la voluntad de vuestro leal y asendereado
Tito. Otorgadme el respiro que os pido, y pronto me encontraréis camino
de Sagunto, acompañado de las figuras representativas de la
Restauración, \emph{Chilivistra}, \emph{Leona la Brava} y otras no menos
interesantes personas que se aprestan a bailar conmigo y con vosotros en
la nueva contradanza histórica.

\flushright{Santander-Madrid.—Agosto-Noviembre de 1911.}

~

\bigskip
\bigskip
\begin{center}
\textsc{fin de cartago a sagunto}
\end{center}

\end{document}
