\PassOptionsToPackage{unicode=true}{hyperref} % options for packages loaded elsewhere
\PassOptionsToPackage{hyphens}{url}
%
\documentclass[oneside,14pt,spanish,]{extbook} % cjns1989 - 27112019 - added the oneside option: so that the text jumps left & right when reading on a tablet/ereader
\usepackage{lmodern}
\usepackage{amssymb,amsmath}
\usepackage{ifxetex,ifluatex}
\usepackage{fixltx2e} % provides \textsubscript
\ifnum 0\ifxetex 1\fi\ifluatex 1\fi=0 % if pdftex
  \usepackage[T1]{fontenc}
  \usepackage[utf8]{inputenc}
  \usepackage{textcomp} % provides euro and other symbols
\else % if luatex or xelatex
  \usepackage{unicode-math}
  \defaultfontfeatures{Ligatures=TeX,Scale=MatchLowercase}
%   \setmainfont[]{EBGaramond-Regular}
    \setmainfont[Numbers={OldStyle,Proportional}]{EBGaramond-Regular}      % cjns1989 - 20191129 - old style numbers 
\fi
% use upquote if available, for straight quotes in verbatim environments
\IfFileExists{upquote.sty}{\usepackage{upquote}}{}
% use microtype if available
\IfFileExists{microtype.sty}{%
\usepackage[]{microtype}
\UseMicrotypeSet[protrusion]{basicmath} % disable protrusion for tt fonts
}{}
\usepackage{hyperref}
\hypersetup{
            pdftitle={AMADEO I},
            pdfauthor={Benito Pérez Galdós},
            pdfborder={0 0 0},
            breaklinks=true}
\urlstyle{same}  % don't use monospace font for urls
\usepackage[papersize={4.80 in, 6.40  in},left=.5 in,right=.5 in]{geometry}
\setlength{\emergencystretch}{3em}  % prevent overfull lines
\providecommand{\tightlist}{%
  \setlength{\itemsep}{0pt}\setlength{\parskip}{0pt}}
\setcounter{secnumdepth}{0}

% set default figure placement to htbp
\makeatletter
\def\fps@figure{htbp}
\makeatother

\usepackage{ragged2e}
\usepackage{epigraph}
\renewcommand{\textflush}{flushepinormal}

\usepackage{indentfirst}

\usepackage{fancyhdr}
\pagestyle{fancy}
\fancyhf{}
\fancyhead[R]{\thepage}
\renewcommand{\headrulewidth}{0pt}
\usepackage{quoting}
\usepackage{ragged2e}

\newlength\mylen
\settowidth\mylen{……………….}

\usepackage{stackengine}
\usepackage{graphicx}
\def\asterism{\par\vspace{1em}{\centering\scalebox{.9}{%
  \stackon[-0.6pt]{\bfseries*~*}{\bfseries*}}\par}\vspace{.8em}\par}

 \usepackage{titlesec}
 \titleformat{\chapter}[display]
  {\normalfont\bfseries\filcenter}{}{0pt}{\Large}
 \titleformat{\section}[display]
  {\normalfont\bfseries\filcenter}{}{0pt}{\Large}
 \titleformat{\subsection}[display]
  {\normalfont\bfseries\filcenter}{}{0pt}{\Large}

\setcounter{secnumdepth}{1}
\ifnum 0\ifxetex 1\fi\ifluatex 1\fi=0 % if pdftex
  \usepackage[shorthands=off,main=spanish]{babel}
\else
  % load polyglossia as late as possible as it *could* call bidi if RTL lang (e.g. Hebrew or Arabic)
%   \usepackage{polyglossia}
%   \setmainlanguage[]{spanish}
%   \usepackage[french]{babel} % cjns1989 - 1.43 version of polyglossia on this system does not allow disabling the autospacing feature
\fi

\title{AMADEO I}
\author{Benito Pérez Galdós}
\date{}

\begin{document}
\maketitle

\hypertarget{i}{%
\chapter{I}\label{i}}

El 2 de Enero de 1871 vimos entrar en los Madriles al Monarca
constitucional elegido por las Cortes, Amadeo de Saboya, hijo del
llamado \emph{re galantuomo}, Víctor Manuel II, Soberano de la nueva
Italia. En las calles, alfombradas de nieve, se agolpaba el pueblo,
ansioso de ver al príncipe italiano, de cuyo liberalismo y
caballerosidad se hacían lenguas los amigos de Prim, que le habían
buscado y traído para felicidad de estos abatidos reinos. Como los
españoles no habíamos visto, en lo que iba de siglo, Rey ni Roque a la
moderna, más arrimados a la Libertad que al feo absolutismo, ardíamos en
curiosidad por ver el cariz, el gesto, la prestancia del que nos mandaba
Italia en reemplazo de los en buen hora despedidos Borbones.

Entró don Amadeo a caballo, con brillante escolta, y su persona despertó
simpatías en el pueblo\ldots{} Varios amigos, de quienes hablaré luego,
nos situamos en la esquina de la calle del Turco, palacio de Valmediano,
orilla baja del Congreso, y le vimos muy a gusto desde que apareció por
el Prado y embocó el repecho que llaman Plaza de las Cortes. Saludaba
con graciosa novedad, extendiendo ceremoniosamente el brazo al quitarse
el sombrero. Uno de los amigos que me acompañaban aseguró que aquel era
el saludo masónico en su expresión castiza, y sólo por este detalle vio
en el Rey entrante una esperanza de la Patria.

A todos pareció don Amadeo gallardo, y animoso hasta la temeridad. Y que
el hombre tenía los riñones bien puestos y un cuajo formidable, se
demuestra con decir que de una monarquía juvenil le traían a reinar en
una vieja monarquía, devastada por la feroz lucha secular entre dos
familias coronadas. Verdad es que España se sacudió a entrambas como
pudo; pero una y otra dejaron en los repliegues del suelo cantidad de
huevecillos que el calor y las pasiones de los hombres cluecos, aquí tan
abundantes, habrían de empollar más tarde o más temprano. Venía el buen
príncipe de un país en que el pueblo y sus reyes recíprocamente se
amaban, y entraba en este, recocido en el hervor de las opiniones,
amante tan sólo de irisados ideales, o de vagas incógnitas que sólo
podría despejar el tiempo.

Y por si no estuviera bien probado el valor del \emph{chico de Saboya},
la fatalidad le sometió a mayor prueba. Al llegar a Cartagena, diéronle,
para hacer boca, la noticia del asesinato y muerte de Prim, que le había
traído a reinar en este manicomio. Mostrose apenado y sereno el príncipe
al recibir este jicarazo\ldots{} Su arribo a España en momentos
trágicos, no carecía de romana grandeza. La Historia, que aún no tenía
nada que decir del nuevo Rey, señaló aquel primer paso, puesta la mano
en el esforzado corazón del hijo de Víctor Manuel.

En el trayecto por ferrocarril desde Cartagena a Madrid no llegaron a
don Amadeo calurosas demostraciones populares. Diéronle la bienvenida
caciques inveterados en la adulación, y alcaldes de Real orden que lo
mismo habrían festejado al Moro Muza si el Gobierno se lo mandase. Llegó
a Madrid la Majestad saboyana, y de la estación fue al santuario de
Atocha, donde visitó a Prim muerto y amortajado de uniforme entre
hachones; y cuando el Rey, con mudo estupor y recogimiento, contemplaba
el embalsamado cadáver, este le dijo: «Aprende de mí la inseguridad de
las grandezas humanas. Vienes a reinar en España traído por Prim. Pues
aquí tienes a tu Prim\ldots{} Ya no soy más que un nombre, un despojo
mortuorio, un tema para que algún sabio cuente lo que hice y lo que no
he podido hacer. Creíste encontrar un hombre, y sólo soy una
leyenda\ldots{} una ráfaga de gloria, un frío mármol quizás y una
biografía\ldots{} Arréglate como puedas, hijo. Consulta el corazón del
pueblo, y al son de los latidos de este pon los del tuyo. Para poseer el
arte de reinar, aprende bien antes la ciudadanía. El buen Rey sale del
mejor ciudadano\ldots»

Oído esto, o pensado (es un suponer), don Amadeo hizo su oficial entrada
en la Villa y Corte con la arrogancia caballeresca que le captó la
querencia y agrado de los madrileños. Después de jurar en las Cortes,
siguió su camino, entre soldados y apretada muchedumbre, prodigando el
quita y pon del tricornio, que mi amigo llamaba \emph{saludo masónico}.
Los que gozamos de aquel lindo espectáculo éramos cinco: Córdoba y
López, federal exaltado y escritor valiente; Emigdio Santamaría, furioso
propagandista republicano; Mateo Nuevo, otro que tal, revolucionario de
acción, que a la idea consagraba toda su actividad y toda su pecunia:
los dos restantes, inferiores sin duda en edad, saber y gobierno, nos
habíamos conocido y tratado en una casa de huéspedes donde juntos
hacíamos vida estudiantil. Él era \emph{guanche} y yo \emph{celtíbero},
quiere decir que él nació en una isla de las que llaman adyacentes, yo
en la falda de los Montes de Oca, tierra de \emph{los Pelendones}; él
despuntaba por la literatura; no sé si en aquellas calendas había dado
al público algún libro; años adelante lanzó más de uno, de materia y
finalidad patrióticas, contando guerras, disturbios y casos públicos y
particulares que vienen a ser como toques o bosquejos fugaces del
carácter nacional. A mí también me da el naipe, por las letras; pero
carezco de la perseverancia que a mi amigo le sobra. Ambos, en la época
que llamaré \emph{amadeísta}, matábamos el tiempo y engañábamos las
ilusiones haciendo periodismo, excelente aprendizaje para mayores
empresas. Y no digo más por ahora, reservándome, con permiso del
bondadoso lector, el nombre de mi amigo y el mío.

Visto el paso del Rey, divagamos por las calles, recogiendo de las bocas
y de las caras de la muchedumbre la impresión del suceso, y debo
declarar honradamente que el príncipe italiano, traído a ocupar el trono
vacío de los Borbones, había entrado en la capital del Reino con
\emph{buena sombra}. Las mujeres encomiaban al Rey forastero por su
garbo y su valor sereno, y los hombres, en general, le veían como una
esperanza engarzada en una novedad. Lo nuevo lleva siempre ventaja sobre
lo gastado y caduco. La medicina desconocida consuela al enfermo, ya que
no le cure, y el cambio de amo trae algún alivio a los que sufren
miseria y esclavitud.

Los amigos que desde la tribuna de periodistas del Congreso presenciaron
la sesión solemnísima de las Constituyentes cuentan que el nuevo Rey,
bien plantado, la derecha mano sobre el corazón, pronunció con voz
entera el \emph{Sí juro}, sanción elemental de su investidura y primer
aliento de su reinado. Respondiole con fervientes aclamaciones la
turbamulta que llenaba el salón, voces que fueron ¡ay!, el estertor de
las Constituyentes, pues con aquel hálito expiraron y se desvanecieron
en la Historia, dejando tras sí un rastro glorioso. En el propio
instante feneció también la discreta Regencia ejercida por Serrano desde
que la Democracia se hizo monárquica por el voto de los más, hasta que
el Principio se hizo carne en la persona del hijo de Víctor
Manuel\ldots{}

Al salir del Congreso, el Rey alteró la carrera y ordenamiento de su
marcha triunfal, volviendo al Prado para dirigirse a Buenavista. No
quería entrar en su casa sin visitar a la viuda de Prim, Condesa de Reus
y Marquesa de los Castillejos, doña Francisca Agüero. La visita fue
breve y patética, según nos contó Ricardo Muñiz en la misma tarde del
día 2. Don Amadeo besó la mano de la desolada señora y abrazó a los
huérfanos. Ni él pudo hablar largo por su escaso dominio de la lengua
castellana, ni la viuda tampoco, porque la intensidad de su dolor le
entorpecía la palabra\ldots{} De Buenavista subió el Rey por la calle de
Alcalá, saludando y saludado con afectuosa cortesía.

Buenos observadores éramos para saber apreciar el momento político por
el adorno de los balcones de la carrera. Las irreductibles formas de
opinión hablaron aquel día claramente, aquí con las profusas percalinas,
allá con la ausencia de toda clase de trapos manifestantes de una idea.
Un amigo muy despierto, de filiación \emph{moderada}, Juanito Valero de
Tornos, nos hizo notar que los palacios de Medinaceli y Villahermosa en
lo más bajo de la plaza de las Cortes, no habían colgado sus elegantes
reposteros. También faltaban los tapices en la casa de Miraflores,
Carrera de San Jerónimo, y en la de Oñate, calle Mayor. El veto del
alfonsismo era, pues, terminante. Yo me permití decir a nuestro amigo
que más significativo que aquel veto era el de los federales, bien
manifiesto en innumerables balcones desnudos, y él respondió burlándose:
«Poco significa la opinión de la cofradía \emph{sinalagmática},
\emph{conmutativa}, \emph{bilateral}, que muerto Prim, ya no podéis
tocar pito ni flauta.» Uno de los nuestros le dijo: «Tocaremos lo que
nos acomode, y vosotros el cuerno.» Y el otro replicó: «Sí, sí, el
cuerno de Hernani.»

Vuelvo un poquito atrás para referir que los cinco amigotes agrupados el
2 de Enero de 1871 para ver entrar a don Amadeo, formamos la misma piña
el día anterior, domingo 1 de Enero, en las rampas aún no concluidas del
palacio de Buenavista, para ver salir y pasar tristemente el féretro de
Prim. También aquel día cubrían el suelo cuajarones de nieve. El sol se
ocultaba entre nubes pardas, ceñudas. ¡Oh luctuoso día, el más triste
que yo había visto desde que mis ojos pudieron observar la corriente de
la Historia viva! Pasó el coche en que iba el General cuando le
dispararon los tiros en la calle del Turco, rotos los vidrios, enlutados
los faroles, enlutado el cochero; detrás la carroza fúnebre, lenta como
el barquichuelo de Aqueronte. Vi a los que llevaban las cintas por el
lado en que yo estaba: eran el General Contreras, don Manuel Silvela y
don Vicente Rodríguez. Seguía la cabecera del duelo: General Serrano,
don Salustiano Olózaga, un obispo, don Nicolás Rivero, Moreno
Benítez\ldots{} Ulloa, Ruiz Zorrilla, que se habían adelantado al Rey
para llegar al entierro del grande hombre, y detrás la revuelta
turbamulta, diputados y políticos de todas marcas y abolengo. Recuerdo
haber visto a Castelar, a Pi y Margall, a García Ruiz, Sánchez Ruano,
Becerra\ldots{} Era un desfile de caras que constituían la iconografía
política de aquel tiempo\ldots{} figuras del montón complejo, algunas de
las cuales entraron en la Historia, y otras se quedaron fuera mirando a
una puerta que se llama \emph{del Olvido}\ldots{} En marcha se puso la
tétrica procesión, Prado abajo, en dirección del santuario de Atocha.
Lloraba el día, lloraban los árboles desnudos, lloraba la muchedumbre
negra, silenciosa, con el solo rumor de sus pisadas. Así fue llevado al
sepulcro el hombre que ejerció en España durante veintisiete meses una
blanda dictadura, poniendo frenos a la revolución y creando una
monarquía democrática como artificio de transición, o \emph{modus
vivendi} hasta que llegara la plenitud de los tiempos.

El mismo día, tempranito, habíamos ido los cinco a los funerales
masónicos que se hicieron al General en la basílica de Atocha. Aunque yo
y mi amigo de hospedaje y periodismo no teníamos vela en aquel entierro,
nos agarramos a los faldones de Nuevo, Córdoba y Santamaría, para
colarnos en el sacro recinto y en la capilla que los atrevidos masones
convirtieron por un buen rato en logia o \emph{taller}. Nunca vi cosa
semejante, alarde atrevidísimo de licencia cultural. En los tiempos que
corren, aquel acto habría sido la más escandalosa de las profanaciones,
merecedora de los tizonazos del Infierno. Yacía el cadáver del héroe de
los Castillejos en una capilla de las primeras a mano izquierda,
descubierto en su caja bronceada. De la otra parte del templo venía el
tintín de campanillas, señal de misa, y se oían pisadas y carraspeo de
viejas. Los masones, que eran unos treinta, pertenecientes al Gran
Oriente Nacional de España, dieron comienzo a la ceremonia, sin que
nadie les estorbara en los diferentes pasos y manipulaciones de su
extraño rito.

Descripción del funeral. Lo primero fue hacer tres \emph{viajes}
alrededor de la caja, formados uno tras otro. El primero y segundo
\emph{viajes} iban dirigidos por los dos primeros \emph{Vigilantes} de
la Orden; en el tercero iba de guía el \emph{Gran Maestre}
(Gr.\textsuperscript{\textbf{.}}. Mae.\textsuperscript{\textbf{.}}. de
la Ord.\textsuperscript{\textbf{.}}.). Al paso arrojaban sobre el
cadáver hojas de acacia. Luego, el propio \emph{Gran Maestre} dio tres
golpes de \emph{mallete} (un mazo de madera) sobre la helada frente de
Prim, llamándole por su nombre simbólico: \emph{Caballero Rosa Cruz},
\emph{Grado} 18. A cada llamamiento, los masones, mirándose con gravedad
patética, exclamaban: «¡No responde!» Después formaron la \emph{cadena
mística}, dándose las manos en derredor del muerto. El \emph{Vigilante}
declamó con voz sepulcral esta fórmula: \emph{La cadena se ha roto}.
\emph{Falta el hermano Prim}, \emph{Caballero Rosa Cruz}. \emph{Gr}. 18.
A continuación el \emph{Gran Maestre} pronunció un breve discurso
apologético, y luego leyó un \emph{balaustre}. Así llaman a las
comunicaciones o documentos que las logias de diferentes países se
cruzan entre sí para restablecer la fraternidad universal. El
\emph{balaustre} era de la masonería italiana, que ponía bajo la
salvaguardia de los Hermanos del \emph{Grande Oriente Español} la
persona de Amadeo de Saboya, encargándoles encarecidamente que velaran
por el nuevo Rey, y le protegieran de la maldad y asechanzas de todo
género.

({\textsc{Nota}}. Luego resultó, según me dijo Santamaría, que el
\emph{balaustre} era falso, y que Amadeo no figuraba en la masonería de
su país, ni pisó jamás las \emph{cámaras}, \emph{logias} o
\emph{talleres}. Superchería fue de un español amante de la casa de
Saboya. Con tal ardid logró un efecto de propaganda previsora, muy
eficaz en la ocasión crítica de aquella traída de un rey para fundar
dinastía en país turbulento y alocado.)

Observé que en la última parte del ceremonial, cuando los \emph{Hijos de
la Viuda} estaban en la plenitud de su abstracción litúrgica, asomaron
en la entrada de la capilla dos o tres viejas y algunos inválidos que
habían despachado sus misas. Con más curiosidad que espanto miraron y
oyeron los arrumacos y el vocerío masónicos. Debieron de pensar que
aquellos señores rezaban por sus muertos en una forma y estilo
extravagantes; mas no veían gran malicia en ello\ldots{} Sotanas de
curas y sacristanes no vimos que a la capilla se acercaran, lo que
demostraba excesiva tolerancia, o vista muy gorda de la superior
clerecía de Atocha\ldots{} Tolerancia hubo de una parte; pero la otra
incurrió en el pecado de indiscreción, porque algún periódico describió
la ceremonia con todos sus pelos y perendengues, sin omitir las hojas de
acacia. Consecuencia de esta simplicidad periodística fue la destitución
del Rector de la basílica, don Leopoldo Briones, varón docto y un tanto
hereje, según oí decir; liberal sin careta, muy dado al libre pensar y a
la libre crítica de personas y cosas eclesiásticas.

\hypertarget{ii}{%
\chapter{II}\label{ii}}

Volviendo al punto inicial de este relato, diré que a media tarde del 2
de Enero nos dispersamos los cinco ciudadanos que habíamos presenciado
juntos la entrada del nuevo Rey. Mi amigo el canario se fue con Córdoba
López a la casa de pupilos donde moraban (Olivo, 9); Santamaría se unió
a la trinca de Félix La Llave, Patricio Calleja y Nicolás Calvo,
conspiradores de oficio, y se encaminaron los cuatro al domicilio del
último (Olmo, 30), donde tenían su \emph{sanhedrín}. Yo me fui con Mateo
Nuevo a su casa (Montera, 11), donde se agazapaba la redacción de un
ardiente periodiquillo, \emph{El Tribunal del Pueblo}. Ayudábale yo a
escribirlo, y no miento al decir que las parrafadas más libres y
frenéticas eran de un servidor de ustedes. Sorprendíanos a Mateo y a mí
la aurora del nuevo día enjaretando artículos y sueltos, o hablando de
la revolución que a juicio de él se incubaba sigilosamente, y pronto
saldría del cascarón cantando la Marsellesa.

Era Mateo Nuevo un hombre ingenuo y exaltado. Su fe revolucionaria, a
prueba de desengaños, le inspiraba la persistente acción y el ciego
impulso hacia los fines que creía tener al alcance de la mano. Los dedos
tocaban los fines, y estos huían alejándose en una atmósfera de azul y
dorado ensueño. Su casa era un tubo de largo pasillo y habitaciones
lóbregas que empezaba en la calle de la Montera y acababa en la de los
Negros, rebautizada con el nombre de \emph{Tetuán}. En esta parte estaba
la redacción, y allí teníamos nuestro club y mentidero, con asistencia
de amigos locuaces, adorantes de un dogma bellísimo, dispuestos a dar
toda su saliva y en último caso su sangre por traerlo a los altares de
la realidad. Las noches largas de invierno se nos hacían cortas, y
deslizaban sus horas entre el correr de nuestras charlas, ora utópicas,
ora proyectistas, pues en el delirio de la conversación imaginábamos
lindas leyes concisas que no esperaban más que el triunfo material para
colmar a España de felicidad y contento. El desperezo matutino del
próximo mercado del Carmen y \emph{el ronco} son de la taberna y
carbonería que caían bajo los balcones por la calle de los Negros, nos
traían a la razón y al sueño. Ya era virtud el descanso. Cada mochuelo
se iba a su albergue, y yo a mi \emph{cueva}, que así la llamaba por ser
en la calle de los Leones.

Mi trato constante con Mateo Nuevo y otros románticos de la política,
constructores clandestinos de una España feliz, me puso en condiciones
de descubrir algunos tapadijos revolucionarios y rasgar velos de
conspiración, cosa muy grata a los que anhelamos libertad que nos
despabile y mudanza que nos mejore. Con mi destreza en atar cabos, y
algo que se le salía de la boca al bueno de don Mateo, vine a saber que
existía en Madrid un organismo designado con el resonante título de
\emph{Junta Suprema del Consejo de la Federación Española}. Lo presidía
don Francisco García López, diputado constituyente, estirado de palabra
y de ropa, y fueron Vicepresidentes los hermanos Pierrad, y después don
Juan Contreras. Mateo Nuevo figuraba como Vocal, y también Córdoba López
y Emigdio Santamaría.

Tuve luego conocimiento de otros, y de los que componían las juntas de
distrito, que irán saliendo conforme los reclame el desarrollo
histórico. Reuníase a veces la \emph{Junta Suprema} en la casa de mi
amigo Nuevo. Por variar de sitio se congregaron alguna vez en el
\emph{taller} de Nicolás Calvo (Olmo, 30); andando días, los olfateos de
la policía les movieron a recatarse más, y la guarida revolucionaria
fue\ldots{} lo diré aunque no me lo crean\ldots{} fue un convento de
monjas.

Ello era en la plaza de Jesús esquina a las Huertas, y ocurría cuando ya
llevaba largos días en Madrid el Rey saboyano. Emigdio Santamaría, que
era el mismo demonio, me reveló, cuando llegamos a unirnos con mayor
confianza, que él había sido el catequizador de las monjas para que
facilitaran un salón de planta baja donde se reuniera la \emph{Junta
Suprema}. Mas no supo o no quiso explicarme el porqué de tal tolerancia
en personas de ideas tan contrarias a las nuestras. He dado en pensar
que como la conjura iba contra un Rey excomulgado, creían aquellas
mujeres simplísimas que ayudando a la Federación Española, laboraban
santamente en servicio de Dios. Misterios de la conciencia, misterios de
la política, ¿quién os entiende, quién os deslinda, quién os baraja?

Perdóneme el piadoso público la falta de método que habrá notado en mis
escritos, los cuales aparecen reñidos con el orden cronológico. Este
defecto mío radica en el fondo de mi naturaleza, y sin darme cuenta de
ello refiero los acontecimientos invirtiendo su lugar en el tiempo. Si
nunca me ha entrado en el cerebro la aritmética, tampoco hice migas con
la cronología, y sin pensarlo refiero lo de hoy antes que lo de ayer, y
la consecuencia antes que el antecedente\ldots{} Va siempre por delante
lo que hiere mi imaginación con más viveza\ldots{} Al franquearme
contigo, noble y cachazudo lector, presumo que desearás conocerme, saber
quién soy, de dónde he salido, y el cómo y por qué de mi metimiento, de
mi colaboración en estas historias. Por de pronto diré que soy un hombre
chiquitín de cuerpo, grande de espíritu y dotado de amplia percepción
para ver y apreciar las cosas del mundo. Reservo por ahora mi verdadero
nombre, y entre los diferentes motes que suelo usar en mi labor
periodística, escojo el más adecuado, que es también el más breve:
\emph{Tito}.

Si queréis saber algo de mi ascendencia os diré que es un extraordinario
ciempiés o cienramas. Por mi padre tengo sangre de los Pipaones y
Landázuris de Álava, absolutistas hasta la rabia, y sangre de los
Torrijos y Porlieres, mártires de la Libertad. Mi madre me ha
transmitido sangre de verdugos como González Moreno y Calomarde, sangre
de Zurbanos, y aun la de fieros demagogos, ateos y masones. Mi abolengo
es, pues, de una variedad harto jocosa. Yo, con paciencia y saliva,
quiero decir tinta, he reconstruido mi árbol, y en él tengo señoras
linajudas, títulos de Castilla, que casi se dan la mano con logreros y
mercachifles de baja estofa; tengo un obispo católico, un cura
protestante, una madre abadesa, dos gitanos, una moza del partido, un
caballero del hábito de Santiago y varios que lo fueron de
industria\ldots{} Soy, pues, un queso de múltiples y variadas leches.
Debo declarar que de la heterogeneidad de mis fundamentos genealógicos
he salido yo tan complejo, que a menudo me siento diferente de mí mismo.

En la época de este mi cuento amadeísta había cumplido yo los veintitrés
años; pero declaraba veinticinco por el afán de hacerme más hombre, y
atenuar la poca estimación en que, a mi parecer, se me tenía por mi
rostro aniñado, casi lampiño, y mi corta estatura. Temeroso de que se
dudara de mi eficacia varonil, yo aumentaba mi humanidad agregándome
años, y mi talla usando descomunales tacones\ldots{} Han pasado desde
entonces algunos lustros: rugoso y lleno de canas, ya no me cargo años,
sino que me descargo de ellos, y ni a tiros me hacen pasar de los
cincuenta y nueve. La estatura es la que no ha cambiado, ¡ay de
mí!\ldots{} Suspiro, señores míos, porque este defecto de mi pequeñez ha
sido y es la mayor amargura de mi vida. A la menguada talla debo
atribuir todas mis desgracias, el fracaso de mis tentativas literarias y
el estancamiento de mis ambiciones\ldots{} Mi defecto era simplemente la
pequeñez, pues no padecía ninguna deformidad: al contrario, mi rostro
era correcto, mi cuerpo bien repartido de miembros y de notoria
esbeltez, mi temperamento de gran viveza y acometividad, compensación
que la Naturaleza suele dar a los chiquitines, casi enanos. Completo mi
retrato asegurando con toda veracidad que en los días a que me refiero
hice la mar de conquistas, como verá el que me leyere.

Una de las más rápidas y felices la intenté y llevé a venturoso término
en Palacio, en la época de interinidad, poco antes de que las Cortes
eligieran Rey a don Amadeo de Saboya. ¿Quién era ella? Pues una mujer
picotera y bien armada de carnes, planchadora desde los tiempos de doña
Isabel, esposa de un portero, que tuvo bastante habilidad y cuquería
para empalmar el último reinado borbónico con el primero de la dinastía
italiana. Vivían marido y mujer en una modesta habitación del piso más
alto, y les protegía el intendente interino don José Abascal. A Palacio
iba yo para visitar a un primo de mi madre, don José Folgueras, empleado
en las oficinas. Recorriendo las alturas, topé con María de las Nieves.
Pronto hallé un pretexto para entrar en su casa. Ello fue que se me hizo
un tremendo desgarrón en la capa y ella me ofreció el remedio de aguja y
hebra de seda. Era bajita y frescachona. Sin encomendarme a Dios ni al
Diablo le planteé la cuestión de confianza. A mi primer exabrupto
contestó con risas y fingidos desdenes; al segundo advertí que le había
caído en gracia; al tercero fue la vencida, y quedamos amigos. El
marido, Quintín González, que se pasaba gran parte de la tarde y prima
noche trajinando en la reventa de billetes de teatro, era un buenazo,
corpulento como un buey y confiado como un borrego de Dios.

No duró mucho tiempo aquel lío. En Febrero del 71 fui una tarde a
Palacio, por visitar a Nieves, sin otro fin que preparar un delicado
rompimiento, pues ya me había deparado el Cielo conquista mejor. Apenas
pude ver a Nieves un instante: toda la servidumbre estaba muy afanada en
disponer las habitaciones para la Reina doña María Victoria, que no
tardaría en venir a estos reinos. El Marqués de Dragonetti, caballero
rubio y de buena presencia, ayudante, secretario y amigo de Amadeo I, se
multiplicaba en la organización de los servicios palatinos, y en equipar
con arte pintoresco la servidumbre. A los porteros vistió de colorado
rabioso. Cuando en la puerta del Príncipe topé con mi candoroso y
coronado amigo Quintín González, vestido en tal guisa y armado de una
cachiporra, no pude contener la risa. Bromeamos un rato. Díjome que a su
mujer le gustaba lo colorado. Era Nieves muy fantasiosa y algo torera. A
él no le hacía maldita gracia el traje, porque ya la gente tomaba en
broma las libreas rojas de los porteros, y dentro y fuera de Palacio les
llamaban los \emph{langostas}. «Mala cosa es---dijo moviendo el
testuz---que empecemos ya con el mote, el chistecito y la guasita. Yo le
diría al Rey, si tuviera confianza: Mire, señor, si los españoles le
atacan con discursos, injurias y aun con armas blancas o de fuego,
manténgase tieso; pero si vienen con chafalditas y remoquetes, a puede
ir preparando el petate.»

Mi siguiente conquista fue romántica, pasión que venía rezagada, no de
los tiempos de \emph{Don Álvaro} y \emph{El Trovador}, sino de otros más
próximos en que privó el sentimentalismo baboso de \emph{Flor de un día}
y de \emph{Borrascas del corazón}. La mujer soñada se me apareció en el
anfiteatro del Teatro del Príncipe, viendo, en función de tarde,
\emph{Los Polvos de la Madre Celestina}, obra de risa en que Mariano
Fernández derrochaba su inagotable gracejo. ¡Ay!, aquellos polvos me
trajeron pronto a los lodos de mi amorosa demencia. La joven que me
trastornó era, como yo, chiquitina, de bellas facciones y cuerpo
primorosamente formado. A esta igualdad o armonía de nuestra naturaleza
visible se debió quizás la repentina inclinación de ambos, y el fogonazo
de amor que no tardó en producir voraz incendio. El nombre de la menuda
divinidad era Obdulia, de exquisito sabor romántico, y su talle y rostro
componían la más encantadora muñeca que en bazares de juguetes se ha
podido ver. Iba en compañía de otra mujer, de más edad y complexión
hombruna, y desbordada entre ellas y yo la confianza, supe que la
pequeña servía y la grande había servido en la casa de una empingorotada
señora, la Marquesa de Navalcarazo.

En el primer acto de \emph{Los Polvos}, hicimos Obdulia y yo nuestra
presentación respectiva; en el segundo declaramos la mutua simpatía, y
en el tercero afirmamos enfáticamente que habíamos nacido el uno para el
otro. Romeo y Julieta no se dieron más prisa. Fue casualidad picante o
simbólica que la compañera de Obdulia se llamara Celestina, y
confirmaron el nombre sus astutos requerimientos. A la salida de
\emph{Los Polvos} las acompañé, y en el tránsito desde el teatro a la
calle del Sacramento, repetimos nuestros gorjeos amorosos, añadiéndoles
ya planes y horarios para nuestras futuras entrevistas. Celestina Tirado
nos dio facilidades de tiempo y lugar que me colmaron de gratitud.

Aventura tan novelesca me pareció cuento de hadas. Fue Obdulia encanto y
alegría de mi existencia, y yo con mi labia y fáciles recursos de
expresión, la trastorné y enloquecí. Mi muñeca dejaba traslucir
constantemente el romanticismo azucarado y tonto que llevaba en su alma.
A lo mejor salía diciendo con canturria: \emph{Si oyes contar de un
náufrago la historia}---\emph{ya que en la tierra hasta el amor se
olvida}\ldots{} y lo demás de que no me acuerdo. Cuando yo le
preguntaba, suponiéndome náufrago, si me olvidaría, contestaba poniendo
la mano sobre el corazón: \emph{Aquí}---\emph{vivirás mientras yo viva}.
A pesar de estas ardientes ternuras, tuve que darle palabra de
casamiento para continuar nuestros amores. Cada día me requería con más
empeño a legalizar su situación. Mostrábase celosa guardiana de los
buenos principios y de la corrección legal\ldots{} En verdad, la melaza
romántica no se avenía con las asperezas del deber social y católico;
pero yo entraba por todo, y cuando mi Obdulia salía con la tecla del
matrimonio, yo le aseguraba que en cuanto me mandaran los
papeles\ldots{} pim\ldots{} a casarnos.

Llegó un día en que mi muñeca, sin apagar sus poéticos fulgores,
mostraba un admirable sentido práctico. «He confesado a mi señora---me
dijo poniéndose muy seria---que tengo un novio, a quien quiero de
veras\ldots{} novio con buen fin, que si otra cosa le dijera se pondría
furiosa; que a nosotras las criadas no nos consienten gallos tapados,
por más que veamos a nuestras señoras enredadas con este o el otro
caballero, que a lo mejor es el más íntimo del marido\ldots{} Pues bien:
sabedora de estas relaciones, me aseguró que si vamos por el camino de
la decencia y la religión, nos protegerá. ¿Te vas enterando? Sabrás que
la Marquesa de Navalcarazo es muy lista, que ha leído y lee libros en
francés de mucha sabiduría, y que en política vale más de lo que pesa. A
un cura de cuello y medias moradas, que suele comer en casa, le oí decir
que las personas más sabias de España son ese Cánovas y mi
señora\ldots{} Bueno: pues me dijo ayer que este Rey que han traído
tendrá que tomar el tole dentro de unos meses, porque en esta tierra no
puede cuajar rey extranjero. Y no le vale que sea, como dicen, honrado y
caballero. Con eso y la excomunión que tiene encima su padre el Rey de
Italia, saldrá pronto de aquí con viento fresco. En seguida vendrá esa
cosa que llaman la Restauración, que es como decir Alfonsito, el niño de
doña Isabel, y ese día mandarán los que hoy se llaman alfonsinos. ¿Te
vas enterando? Pues en cuanto eso venga, si para entonces estamos
casados, tendrás un destino de doce mil reales, y de catorce mil si
quieres servir en provincias mejor que en Madrid\ldots{} Mi señora es
cumplidora fiel de su palabra. Del empleo no dudes, que ello es pan
comido, en cuanto este pobre don Amadeo se aburra y salga pitando,
despedido por los tiros de los federales y los desprecios de la
aristocracia. \emph{Si oyes contar de un náufrago la historia}\ldots{}
Si ves que Amadeo se embarca\ldots{} ya sabes\ldots{} destino al canto.»

\hypertarget{iii}{%
\chapter{III}\label{iii}}

Y siguió diciendo mi muñeca, o lo dijo otro día: «Sabrás que en casa se
reúnen a tomar té las señoras alfonsinas. Van la Monteorgaz y la
Campo-Fresco. Esta tiene, según dicen, la contrata de los chistes,
porque los hace tan graciosos, que dan risa para todo el año. Es muy
salada, no se asusta de lo verde ni de lo colorado; cuenta sus
historias, y a lo mejor te suelta una barbaridad que canta el credo. Esa
fue la que, hablando de su hijo, se dejó decir que le había llevado en
su vientre como se lleva una solitaria. También van la Belvís de la
Jara, la Villares de Tajo, la Villaverdeja y la de Yébenes. Esta, que
según cuentan, es más \emph{nea que Dios}, toma las cosas de política
por el lado de la religión. Dice que este Rey es masón y nos ha traído
acá el Infierno\ldots{} Pues allí se están picoteando toda la tarde, y
por la noche van algunas de ellas y muchos señores: uno que le llaman
Orovio, el Marqués de Molins, este\ldots{} ¿cómo se llama?, Iranzo, y
otros que tú conocerás\ldots{} En fin, que no paran de hablar mal de
este pobre Rey\ldots{} Que si no piensa más que en divertirse\ldots; que
si sale a la calle como un cualquiera, encanallando la majestad; que si
todas las noches se va de picos pardos con su ayudante italiano; que si
le han visto en tales o cuales casas\ldots{} ¡Jesús, qué cosas dicen!»

Hablome otra tarde Obdulia de su familia. Era natural de Villaviciosa de
Odón, donde su madre moraba. En Madrid tenía un tío muy bestia, que
diferentes veces la requirió de amores con mal fin; pero ella no se daba
a partido. Temía que cuando el tal tío tan \emph{tío} se enterara de
nuestras relaciones y del proyecto matrimonial nos dificultara la boda
de acuerdo con la madre. ¡Ay!, lo que nos enfadó esta idea no hay para
qué decirlo. Hicimos juramento de vencer cuantos obstáculos se nos
opusieran. Antes que renunciar al casorio, haríamos cuanto aconsejasen
el romanticismo y el clasicismo más desenfrenados. Huiríamos, nos
mataríamos con pistola o veneno si alguno intentaba cortarnos la fuga.
Acordado esto solemnemente, volvíamos a nuestras conversaciones. Obdulia
me dijo:

«No sabes cómo andan ahora de alborotadas las señoras alfonsinas con la
llegada de la Reina, que parece está ya en camino. ¡Y cómo la muerden!
Lo menos que dicen de ella es que es \emph{una buena mujer}, sin hábitos
de reina. No pasa de \emph{señora de un comandante}, lo más \emph{de un
teniente coronel}. Es algo instruida, como que ha estudiado para
maestra. Su título es \emph{de la Cisterna}. El título no puede ser más
profundo. Fama de virtuosa sí que tiene. Gusta más de vivir recogida,
que en las bullangas de la Corte. Eso no se puede criticar, digo yo,
pero tampoco es razón para que venga aquí \emph{a por una corona}. Una
reina debe ser ante todo reina. La de Yébenes dijo: «No nos oponemos a
que sea virtuosa; eso nunca. Las virtuosas reinan en sus casas. Oí que
esa buena señora da el pecho a sus niños y a los niños de sus criadas.
Lo mismo puede ser esto afectación que pobreza de espíritu.»

Yo advertí a Obdulia que la guerra de damas estaba prevista, porque
cuando acudían a cumplimentar a don Amadeo las entidades decorativas del
Estado, la Diputación de la Grandeza se abstuvo, salvo dos o tres
familias. La aristocracia está de uñas\ldots{} De doña María Victoria se
sabe que es una gran señora, y que viene a honrar la Corte y Trono de
España. Dilo así a tu ama\ldots{}

«¡Qué tonto! ¿Cómo quieres que le diga yo eso a mi señora? ¡Buena se
pondría!\ldots{} ¡Bonito genio tiene estos días para que se le vaya con
bromas! Sabrás que\ldots{} Esto te lo digo con reserva\ldots{} No salgas
por ahí contándolo a tus amigos\ldots{} Sabrás que está con un humor de
mil demonios porque el suyo parece que anda distraído\ldots{} dícese que
con la Tordesillas\ldots{} Cuando yo entré en la casa, ya mi señora
\emph{hablaba} con el Marqués de Uclés\ldots{} Todo Madrid le conoce por
Manolo Uclés. Pues ahora están de monos\ldots{} A mi señora no hay quien
la aguante, de la celera que tiene. Y ya no es una niña\ldots{} Los
cuarenta y pico no hay quien se los alivie\ldots{} Y ya no te digo más;
no se te vaya la lengua con tus amigos\ldots»

---Nada importará que cuente lo que sabe todo el mundo---repliqué
yo.---Esas historias son en Madrid comidilla fiambre, que pasa de boca
en boca sin que nos parezca muy gustosa. Los paladares piden manjares
fuertes, Obdulia.

---Llamas tú manjares fuertes al escándalo gordo, a las
revoluciones\ldots{} Hazme el favor de no andar tan metido con los
federalotes, gentecilla fulastre\ldots{} Ya sabes que tienes que hacerte
alfonsino, poquito a poco para que no chillen tus amigos. Si no te
conviertes, será difícil que nos casemos\ldots{} Y ahora que me acuerdo:
mi señora me preguntó ayer si mi novio es católico. Yo le respondí que
sí, que eres muy católico.

---Sí, sí; tan católico por lo menos como Manolo Uclés, que es grande
amigo de Nocedal, y da dinero para el \emph{Pensamiento Español}, donde
escribe Gabino Tejado\ldots{} Si a pesar de ser yo catoliquísimo no nos
dejan casar, nos suicidaremos, ¿verdad, gacela mía?

---Eso habrá que pensarlo\ldots{} Cierto que es bonito morirse de amor,
como aquellos de Teruel, o matarse una con el humo de un braserillo,
como leí en una novela \emph{de por entregas}. Pero la muerte más
simpática es la de la dama de \emph{Espinas de una flor}, que se va
quedando muertecita en un sillón; y allí sale un cura vestido de calzón
corto, que le dice al oír la campana: es \emph{un alma que
divisa}---\emph{las palmeras de Sión}. Para mí, que esas palmeras son el
cementerio. A mí me gusta pasearme por un cementerio, y ver los nichos,
las lápidas del suelo, y pensar que debajo de ellas están descansando
tan tranquilos los enamorados\ldots{} En fin, chico, a ver si vienen de
una vez tus papeles, que los míos encargados los tengo al secretario del
Ayuntamiento de mi pueblo, sin que lo sepa mi madre\ldots{} Me corre
mucha prisa, no sea que\ldots{} ¡Ay! Es cosa fea el salir una en estado
interesante, cuando menos se piensa, y no poder ocultarlo, y que le
digan a una que no es católica por no haberse casado antes de\ldots{}

Yo procuré tranquilizarla, persuadiéndola de la pronta venida de los
papeles, que ya estaban de camino. Pero los papeles no podían venir, ni
yo los había encargado. Vino en cambio un grave suceso que torció de
súbito la corriente histórica de mi vida, llevándola por torrenteras
dramáticas. Veréis lo que pasó. Llegado el día de la entrada de nuestra
Soberana, doña María Victoria, me planté en el Prado, por donde la
comitiva había de pasar, dispuesto a referir el acto para nuestro
periódico, conforme a las indicaciones de Mateo Nuevo, quien me ordenó
que hablase de la señora Reina con respeto, pero sin entusiasmo. Yo
debía decir que doña María Victoria era atrozmente virtuosa; pero que no
lograría captarse el amor de los españoles, que ya no querían cuentas
con reyes, y menos si son extranjeros.

Vi la regia procesión palatina entre filas de tropas y grandes masas de
gentío curioso. Pensaba decir en mi crónica que en las caras del pueblo
se \emph{combinaba la curiosidad con la indiferencia}, y que el
sentimiento general era de lástima más que de simpatía. En esto no decía
verdad. Oí comentarios en extremo favorables. Las mujeres, sobre todo,
contemplaban a la Reina con alegría, y con cierta confianza la
saludaban, cual si en ella vieran la más alta de sus iguales. No sé si
me explico bien. Al paso de la ilustre dama, se discutía su hermosura.
Algunos la ensalzaban con exceso; otros la deprimían con esta crítica
pesimista, que es la miel más grata en bocas españolas. Yo, dejando a un
lado la reseña \emph{oficial} escrita para mi periódico, daré a los
beneméritos lectores de estas páginas la veraz impresión de un honrado
testigo.

Era doña María Victoria de buena presencia y más que regulares carnes,
que propendían a la gordura. En su rostro advertí perfil y rasgos
napoleónicos, la sonrisa franca, el mirar entre melancólico y asustado.
Creyérase que la dignidad real era en su pensamiento cosa prestada o
postiza, y que a nosotros venía, no a ejercer un cargo, sino a
desempeñar un papel. En estas ideas me afirmé después, cuando la esposa
de Amadeo convertía la realeza, que le dieron entonada y rígida, en cosa
blanda y doméstica. Al verla pasar en el coche de gala, a la derecha del
Rey, que no paraba en repartir a un lado y otro su garboso saludo,
comprendí que doña María Victoria sería muy querida de las mujeres
humildes, y admirada de las de clase intermedia, que pueden ser llamadas
señoras sin llegar a damas. Estas brillaron en la recepción de Palacio
con todo el fulgor de su ausencia, bien campaneada por los periódicos
moderados, alfonsinos y carlistas. La gente adinerada se hizo notar
también por sus desdenes. \emph{El Imparcial} señaló las casas donde no
lucían colgaduras, y aludió claramente a Manzanedo, hablando de un
palacio que debía ostentar en los florones de su escudo \emph{Tabaco
Virginia o Kentucky}, y \emph{algunas motas de ébano}, representativas
de la compra y venta de negros en Cuba.

En la Puerta del Sol hubo apreturas y algún calor de vivas y aplausos al
paso de los Reyes; en la plaza de la Armería mayor tumulto, por el
gentío que esperaba el desfile de la tropa. Salieron las saboyanas
Majestades al balcón, y el pueblo desempeñó muy bien la parte de coro
que le corresponde en estas partituras. Las músicas militares enardecían
a las muchedumbres, y estas, a su vez, estimulaban con sus gritos al
fervor de los inocentes soldados\ldots{} Hallábame yo muy entretenido
con aquel espectáculo pintoresco, cuando me sentí tocado en el hombro.
Volvíme, y vi a un hombrejo zanquilargo, feo, encopetado con sebosa
chistera que fue de moda el año 41. Con señas y medias palabras me dijo
que le siguiera para hablar conmigo dos palabritas, y me fui tras él,
rompiendo no sin dificultad por el primer resquicio que nos ofreció la
multitud. Fuera ya del arco de la Armería y encontrándonos en sitio más
desahogado, el tal, ceñudo y con áspera voz, me dijo: «Usted no me
conoce.»

---Sí, señor---le respondí.---Usted es don José Malrecado, que sirve en
la Policía.

---No soy Malrecado ni Buenrecado, ni permito que usted se burle de mí.

---Dispense: no me burlo---dije, observando su ropa negra y raída, con
las trencillas del chaleco y levitín deshilachadas, el rostro sudoroso,
el bigote lacio, los ojos de carnero moribundo.

Y él, mirándome con amenaza y cogiéndome el brazo con garra de
cernícalo, soltó la voz a estas ásperas razones: «Yo soy Aquilino de la
Hinojosa\ldots{} Veo que se asusta. Es natural. Por mi nombre se entera
de que soy tío de Obdulia por parte de madre.»

---Aunque lo fuera usted también por parte de padre no me
asustaría---respondí, sacando del pecho toda mi entereza,---pues nada
tengo que ver con usted, ni me importa un bledo que sea usted tío de la
Osa Mayor o del Espíritu Santo.

---¿Bromitas tenemos?---replicó el tío, tambaleándose en su
soberbia.---Le he buscado para decirle que no se casará usted con
Obdulia\ldots{} que aquí estoy yo para impedir que siga trastornándole
el seso a esa buena chica. Entiéndalo, y me ponga en el caso de hacer
con usted una barbaridad.

---Pues le participo que me casaré con Obdulia cuando me dé la gana, y
sepa que me descargo en usted y en su pastelera madre.

Hizo ademán de echarme al cuello sus manos; pero yo, que chiquitín y
todo soy una fiera cuando tocan a mi dignidad, invoqué a mis tacones
para que aumentaran media cuarta, y haciendo como que requería un arma
en mi bolsillo, le solté esta rociada: «Si usted me provoca, no tendré
inconveniente en sacarle al aire el bandullo, so tío.»

---Poco a poco---gruñó el estafermo echándose atrás.---No hemos de armar
escándalo entre tanta gente. Si usted no tiene vergüenza, yo la tengo.
Tiempo y lugar habrá para ver quién puede más.

Diciendo esto sacó del bolsillo una tarjeta sucia, en la que leí:
\emph{Aquilino de la Hinojosa}, \emph{afinador de pianos}.
\emph{Cuchilleros}, 3. Yo me arranqué a decirle con mayor coraje: «Iré a
buscarle a usted y le afinaré el entendimiento.» A lo que, ya en
retirada discreta, respondió: «No me busque en mi casa, donde tampoco
quiero escándalos. Me encontrará todas las tardes en el \emph{Casino
Conservador}\ldots{} Abur\ldots{} Nos veremos, caballero miniatura.»

---Cuadrúpedo, nos veremos.

Nada me sulfuraba tanto como que me llamaran chiquitín. El
\emph{miniatura} me sonó como la injuria más grosera y soez\ldots{}
Viendo al tío gandul alejarse hacia los Consejos, hice juramento mental
de romperle la crisma o el hueso palomo donde y cuando le
cogiera\ldots{} La inesperada emergencia de aquel gaznápiro fue la mueca
repugnante con que el Destino me anunciaba una reata de infortunios: al
siguiente día me tocaba entrevista con Obdulia, y Obdulia no fue.
Busquela en la calle del Sacramento, paseando desde las Monjas de este
nombre a la plazuela del Cordón, y el eclipse de mi linda muñeca en la
calle como en nuestro nido me colmó de amargura y despecho. El jicarazo
lo recibí aquella misma noche en mi casa por una carta que me llevó
Celestina. ¡Oh ansiedad, oh enigma fatídico! ¿Qué diría la carta? Pues
la carta, con el lenguaje burlón de sus garabatos, esto decía:

«Apreciable \emph{Mico} (apodo familiar inventado por su cariño): Tengo
que decirte con sentimiento que ya no puedo casarme contigo, porque he
sabido que no eres católico. Mi señora la Marquesa y mi madre, que ha
venido ayer, son muy católicas, y las dos me mandan renegar de ti. ¡Ay,
\emph{Mico} mío, qué pena! ¿Pero qué quieres que yo haga? Dejar de ver a
Dios por ti y condenarme, no puede ser. Si me muero por esta pena, que
me entierren en un cementerio bonito, con muchas flores\ldots{} y que me
dé sombra una palmera de Sión. Yo le pediré a Dios en la otra vida que
te arrepientas y en seguida te mueras, para que allá estemos juntos mi
\emph{Mico} y yo.

»Supe que no eres católico porque me contaron que estuviste en la
reunión de los federales en el Teatro de la Alhambra, y allí dijeron mil
herejías ese \emph{Pío Margallo}, el \emph{Castelar}, el \emph{don Roque
de Barcia}, \emph{don Marcos de Albaida}, y tú te subiste a una silla y
soltastes el mayor sacrilegio, diciendo que no estabas seguro de que hay
Dios, ni ángeles ni Virgen\ldots{} que adorabas al Demonio y que te
\emph{descomías} en los santos\ldots{} ¡Qué cosas, qué pena! No puedo
ser más larga. Ya no vuelvas a verme ni a escribirme\ldots{} De ti se
despide hasta la eternidad la que llorando te aborrece y verte no
desea.---{\textsc{Obdulia}}.»

Estrujando la carta en el puño dije a Celestina que aquello me parecía
una estúpida farsa. La letra era de Obdulia; pero no el sentido ni la
intención de la carta. Algún mal intencionado la obligó a escribirla,
dictando quizás parte de ella. En el párrafo tocante a mi supuesto
discurso en la reunión de la Alhambra, vi bien a las claras la malvada
inspiración directa del siniestro mastín que había querido morderme en
la plaza de la Armería. Cierto que estuve en la reunión de los federales
y que pronuncié algunas palabras; mas no fueron para meterme con Dios ni
ensuciarme en las imágenes de santos. Celestina, dejándome ver su blanca
dentadura, se reía de mi furor y de las vulgares perfidias que lo
motivaban. Confesome que la familia de la muñeca no aprobaba sus
relaciones conmigo; querían casarla con un hombre de más fuste y
estatura. Lo de estimar los maridos por la alzada levantó en mí una
borrasca de indignación. Díjome también que Obdulia me tenía ley, y
vacilando entre el amor y la obediencia, se hallaba la pobre \emph{como
una borrica entre dos piensos}.

Sospechando que la señora Marquesa de Navalcarazo pudo ser causante de
mi desventura, interrogué a Celestina, la cual, soltando de nuevo su
reír frescachón, me dijo: «La \emph{señá} Marquesa es muy católica, eso
sí, pero no se mete en los líos de sus criadas, ni se cuida de lo que
ellas hacen o dejan de hacer con sus novios. La Marquesa no piensa más
que en \emph{el suyo}\ldots{} Por cierto que ya se ha reconciliado con
el caballero de Uclés\ldots{} El galán ha vuelto arrepentido cantando
\emph{la mea culpa}. La señora le ha perdonado, y tan creída está de que
por sus oraciones ha vuelto el caballero, que ayer, en acción de
gracias, confesó y comulgó, y a las monjas del Sacramento llevó de
limosna un buen puñadito de monedas de cinco duros. Protege de largo a
la Comunidad. Es beata de ley, socorre a los necesitados, y como tiene
más dinero que pesa, a todos atiende: da para el culto, da para que se
casen los amancebados, da para los pobres de su casa y de la casa de
Uclés, y siempre le queda un buen pico para mandárselo al pobrecito
Papa, que está preso, como usted sabe, en su propio palacio convertido
en cárcel por esos malditos italianos\ldots{} ¡Ay, Jesús!»

\hypertarget{iv}{%
\chapter{IV}\label{iv}}

«¡Cómo está la sociedad!---exclamé yo.---¿Cuándo se vio pisto igual? ¿Es
que Dios y Luzbel han llegado a un arreglo? Civilización de España,
¿quién te entiende? ¿Somos un país europeo, o aquel \emph{País de las
monas} descrito por un inglés de cuyo nombre no me acuerdo?» Viéndome
tan triste, la bondadosa Celestina me administró estas palabras de
consuelo: «Confórmese con lo sucedido, y no crea que se acaba el mundo
porque se le va una novia. Mujeres hay muchas, y yo, si quiere, le
proporcionaré una mejor que esa sosaina de Obdulita. Si sus negocios
andan mal, y la pluma no le da para vivir, arrímese a lo católico, pues
lo que es dinero no encontrará fuera del catolicismo. Si no tiene valor
para meterse de hoz y de coz en el alfonsismo, no hable mal del hijo de
su madre, ni le ponga motes feos, como el que le aplican ahora los que
no le quieren, ni le saque a relucir al padre ni a la madre\ldots{} Siga
el consejo mío, que es consejo de persona que conoce como nadie el
tecleo de este Madrid y su gente. Tenga juicio y \emph{pupila}; váyase
\emph{desapartando} de los federales, familia tronada que no da más que
palabrería sin jugo\ldots{} No se meta con el Altísimo ni con el Papa,
escriba para el Gobierno, y saque un buen destino, que si usted pega de
firme a los que mandan, de ellos saldrá el amansarle con un cacho de
turrón.»

Aquella mujer ruda era una sabia de tomo y lomo, y yo la estimaba y
agradecía sus consejos, sin tener en cuenta su ruin oficio, del cual
dijo Cervantes que era muy necesario en la república. Debo declarar que
antes de oír los sesudos consejos de Celestina ya había pensado yo en
gestionar una colocación. Todos los españoles adquirimos con el
nacimiento el derecho a que el Estado nos mantenga, o por lo menos nos
dé \emph{para ayuda de un cocido}. Los valedores a quienes acudí fueron
Llano y Persi, amigo de Sagasta, y Ramos Calderón, íntimo de Rivero y de
Martos. Ambos me las prometieron muy felices; pero\ldots{} había que
aguardar a que pasara el periodo electoral\ldots{} Pasó el funesto
periodo, y por cierto que el bueno de Práxedes manejó los cubiletes con
arte maestro para traer mayoría; mas no pudo impedir que la coalición de
carlistas y republicanos, diabólico himeneo, trajera setenta o más
diputados.

No sé si mis lectores tendrán interés en conocer el Ministerio de
conciliación, presidido por el Duque de la Torre. Eran los de siempre,
ni mejores ni más malos que los anteriores y subsiguientes. ¿Qué hacían?
Ir viviendo, ir trazando una Historia tediosa y sin relieve, sobre cuyas
páginas, escritas con menos tinta que saliva pasaban pronto las aguas
del olvido. Si no recuerdo mal, Martos se encargó del \emph{Foreign
Office}, Ulloa regentaba la Gracia y la Justicia, Sagasta era el gallito
de Gobernación, Moret tomó las riendas del Fisco, y Beránger el timón de
la Marina. Paréceme que Ruiz Zorrilla ocupó la poltrona de Fomento y
Ayala la de Ultramar.

Más que el quita y pon de ministros, os interesa sin duda mi asunto
personal, que a mi parecer también era histórico. Pues a ello voy. No
tenía yo sosiego hasta que pudiese acometer y apabullar al ruin, al
sucio, negro y desvergonzado aguilucho que me privó de las gracias de
Obdulia, Aquilino de la Hinojosa. Designado el día de mi venganza, me
calcé las botas de tacón más alto que en aquellas décadas poseía, cogí
un roten nudoso que parecía la maza de Hércules, y me fui derecho al
Casino Moderado de la calle de Atocha, donde esperaba medir mi fiereza
con la barbarie soez del tío más tío del mundo.

Pronto comprendí que iba mal encaminado, porque al Círculo de la calle
de Atocha no concurrían más que moderadotes de ropa limpia y elevada
representación pública, como el señor Carramolino, el señor Moyano, el
señor Collantes, el Conde de Cheste y otros tales. Mejor orientado, me
dirigí a un casinejo de reciente fundación, abierto en la calle de
Jacometrezo con el mote de \emph{Círculo popular}\ldots{} no sé si
\emph{conservador} o \emph{alfonsino}, y apenas entré en la obscura,
deslucida y puerca antesala, oí la voz del cernícalo graznando en
estridente disputa con otros pajarracos de la fauna reaccionaria. Con un
mozo que pasaba llevando servicio de café en abolladas cafeteras, mandé
recado a mi enemigo\ldots{} Una visita\ldots{} un señor que deseaba
decirle dos palabras\ldots{}

Los vocablos con que se inició la visita fueron más de dos, seguidos de
réplica insolente y de un garrotazo que descargué con delicioso coraje
sobre la cabeza del tío, la cual sin el resguardo del sombrero habría
quedado rota. Em como hucha de barro que yo quería cascar para sacarle
la calderilla, digo, los sesos\ldots{} Al vocerío de Hinojosa y el
traqueteo de los palos acudieron de una parte el mozo y conserje, de
otra los compinches de mi enemigo. Unos me sujetaban, otros corrían al
socorro del tío\ldots{} Ya he dicho que soy un hombre terrible, y que me
crezco al castigo convirtiéndome de chico en grande por la fiereza de mi
embestida y la arrogancia de mis actitudes. Con presteza increíble me
sacudí de los que intentaban acorralarme, y seguí el vapuleo contra todo
el que por delante me caía. El número al fin pudo más que el ardimiento
feroz. Uno salió al balcón gritando: ¡guardias, guardias!; otro a la
próxima escalera reclamando el auxilio de los vecinos. Pude, tras ruda
pelea, batirme en retirada solo contra tantos, y gané la escalera. A no
ser yo quien soy, habría bajado rodando; pero no perdí pie\ldots{}
Felizmente, acudió en mi ayuda un amigo que a punto subía presuroso,
alarmado del estruendo.

Era Telesforo del Portillo, en los viejos anales conocido con el apodo
de \emph{Sebo}, criado que fue del Marqués de Beramendi, después
policía, funcionario de Gobernación, y al cabo cesante cuando ya le
indicaban para secretario de un gobierno de provincia. Provino su
desgracia de habérsele descubierto concomitancias con el Marqués de
Bedmar, el de Uclés y otros acreditados alfonsinos. Su esposa, Fabiana
Jaime, ex-criada de la Campo Fresco, tenía parentesco con mi madre, de
donde vino mi amistad con \emph{Sebo}, y las consideraciones que me
guardaba, estimándome más que como periodista como pariente. En cuanto
me vio, púsose de mi parte, diciendo con aplomo policiaco: «Paz,
caballeros. Ténganse a la autoridad, que todo ello será por mala
inteligencia. Vengan explicaciones leales de una parte y otra. Conmigo
no valen \emph{soterfugios}. Silencio, digo, y envainen los insultos.
Este joven es de mi familia, y será el primero en retirar sus palabras.»
Algún trabajo le costó al ilustre \emph{Sebo} imponerse, y en cuanto
hubo sosegado las encrespadas olas, lo primero que hizo fue sacarme del
remolino, escaleras abajo, recomendándome, como había hecho más de una
vez, que pusiera frenos a mi fiereza indómita. Aunque yo había quedado
airoso, por ser uno contra tantos, llevaba en mi cabeza tremendos
chichones, y mataduras dolorosas en distintas partes de mi cuerpo
garboso y pequeñín.

Acompañome Telesforo a mi casa de la calle de los Leones, llevándome
antes a una botica, donde fue el león asistido de apósitos y tafetanes.
Y véase ahora cómo se empalman y enraciman los males con los bienes en
esta vida humana, complejidad eterna de llanto y de risa, de ansias
coléricas y expansiones de júbilo. ¿Qué creéis que en mi casa encontré
al volver a ella con bizmas y parches? Pues la credencial que meses
antes había solicitado de Llano y Persi. Bendije la risueña credencial;
bendije al Destino y a Dios, inspiradores del próvido Sagasta, sin
acordarme de que dos días antes habíale disparado un dardo periodístico
hablando de su tupé, de su frescura y otras zarandajas. ¡Cosas de la
vida! La vida es pasión, contrastes, fuga veloz de corazones duros, de
corazones tiernos, toma y daca de arañazos y caricias. Y el mundo
marcha\ldots{} y el sol sale todos los días. Vivid, humanos, en la dulce
alternativa del odiar y el querer.

Mi primer pensamiento al verme colocado fue ocultar mi felicidad a Mateo
Nuevo, a Santamaría y demás amigos políticos. Luego lo pensé mejor y
abominé del tapujo, que, además de ser inútil, me habría colocado en el
listín de traidores o siquiera sospechosos. Franqueado con mis amigos,
que conocían la distancia que la Fatalidad había puesto entre mi boca y
el pan, alentáronme a envainar mi dignidad, previa declaración de que
sería más federal hoy que ayer, y mañana más que hoy\ldots{} El mundo
marchaba y yo con él derechamente a mi bienestar, porque para colmo de
ventura, me dijo Llano y Persi que yo no tenía que ir a la oficina más
que a cobrar, el primero de cada mes.

Encerrado permanecí en mi leonera esperando a que fueran menos visibles
en mi cara los achuchones de la reciente trifulca. Apenas puse el pie en
la calle fui a ver a Llano y Persi, el cual me dijo que deseaba llevarme
a la redacción de \emph{La Iberia}. Quedé perplejo. No quería disgustar
a Llano, uno de los hombres más nobles y generosos que he conocido,
ardiente liberal y patriota desinteresado; no me agradaba ser redactor
de un periódico rabiosamente ministerial, un cuerpo anquilosado de la
opinión que sólo a la defensiva funcionaba, desaborido y sermonario, sin
\emph{vis} política, ni gracia ni literatura. En tal indecisión pedí a
mi buen amigo plazo de tres días para decidirme\ldots{} Y aconteció que
en aquella semana se acumularon sobre mí, como aluvión de un Destino
caprichoso, multitud de sucesos raros y sorprendentes.

Entre aquellos halagos de una fatalidad benigna, menciono la visita del
amigo citado por mí en las primeras páginas de esta relación, el
excelente chico isleño con quien trabé amistad en la casa de huéspedes
donde vivimos desde el 66 hasta el 70\ldots{} No vino el tal a mi casa
por visita de cumplido, ni por ociosa charla; vino a proponerme que
fuese a trabajar con él en \emph{El} Debate, fundado a principios del
año por José Luis Albareda. La verdad, me sedujo la proposición, por el
modernismo y buen tono de aquel periódico, y con esto y una sola
consulta con la almohada, quedé libre de mis dudas y me desligué del
pendiente compromiso con Llano y Persi\ldots{} No poco se holgó el
isleño de mi resolución, y al día siguiente nos fuimos gozosos al pisito
bajo de Trajineros, donde estaba \emph{El Debate}, y en otro cuarto del
mismo piso tuve el gusto de hablar con Albareda, a quien yo no conocía
más que de vista y fama.

Por las Once mil Vírgenes, que me fue muy simpático el caballero
andaluz. Hombre más salado no he visto, y si en la primera visita me
cautivó por su gracejo, cuando el trato afinó mi conocimiento, le admiré
por su talento macho y por la viveza con que percibía y atrapaba las
ideas políticas culminantes en cada día, y la claridad con que veía la
fase de razón de esa idea, la fase de oportunidad y la fase de
peligro\ldots{} Inspirado por José Luis, que así le llamaban sus
íntimos, escribía yo de todo: teatros, vida social, política. El
fundador leía nuestros artículos, y si le gustaban nos elogiaba
desaforadamente. Cuando, según él, lo hacíamos mal, nos trataba como
perros.

Prevínome el isleño contra las hipérboles de Albareda. «Ni cuando te
pone en los cuernos de la luna te envanezcas, ni demasiado te aflijas
cuando te trata a zapatazos.» Un día que escribí muy a su gusto una
croniquilla de salones elegantes, alfonsinos y católicos, me dijo así:
«Tiene usted más talento que Dios.» Al día siguiente le desagradó un
suelto político, y al entrar en su alcoba, oí que decía, por mí: «A ese
judío enano le voy a dar cien patadas.» Su enojo pasaba como el humo y
se desvanecía en donosas ocurrencias. Nos quería, y le queríamos. Para
mí era el periodista ideal. Cuando nos llamaba para sugerirnos alguna
idea, con igual confianza nos recibía en su alcoba, recién dormida la
mañana, que en la próxima pieza donde le veíamos bañarse en pelota,
tomar ducha por regadera, y hacer luego su \emph{toilette} de persona
pulcra y elegante, todo esto hablando de lo humano y lo divino con
singular donaire ceceoso, apuntando la idea política o el juicio picante
de cosas y personas.

Era nuestro inspirador y Mecenas partidario ferviente de la
Conciliación, y apoyaba con su periódico el primer ministerio de don
Amadeo, armadijo de unionistas y radicales. Creía el buen andaluz que se
hundiría el mundo en cuanto los dos concertados puntales de la situación
se cayeran cada uno por su lado. Y si esto creía el maestro, o si no
creyéndolo lo afirmaba, de su caletre al nuestro lo transmitía por
razones de puro arte político. Yo no pensaba como él en lo tocante a la
Conciliación, que infecunda me parecía, pues cada una de las dos partes
a la otra estorbaba para toda labor eficaz. Pero me guardaba de
manifestarlo a mi jefe, que me habría soltado el chorro saladísimo de su
verbosidad andaluza. Yo pensaba en ello y me decía: «Algún motivo tendrá
este hombre para patrocinar con tanto ardor la forzada coyunda de los
dos partidos, y para fundar un periódico con el fin exclusivo de
sostenerla.» \emph{El Debate} araba la tierra política sin lograr la
derechura del surco, porque ni el buey unionista ni el buey radical se
avenían a tirar del arado con igualdad. ¿Romperían el yugo?

Lo rompieron, sí señor, bastantes días después de entrar yo en \emph{El
Debate}; pero antes de referir esto, traeré a colada otras materias para
no disgustar a los devotos de la exacta cronología. De asuntos privados,
confundidos con los públicos hablaré, para que resulte la verdadera
Historia, la cual nos aburriría si a ratos no la descalzáramos del
coturno para ponerle las zapatillas. ¡Cuántas veces nos ha dado la
explicación de los sucesos más trascendentales, en paños menores y
arrastrando las chancletas! Y vais a verlo.

\hypertarget{v}{%
\chapter{V}\label{v}}

Sabréis, amigos, que mi conquista de aquellos días (que no consigno por
orden numérico porque he perdido la cuenta) me deparó una moza bravía y
algo hombruna, morena y agitanada, más alta que yo en cuarta y media,
gallardísima, de ojos bonitos y más bonitos morros, la cual me juró amor
eterno y fidelidad, siempre que yo le mantuviese el pico y con decencia
la vistiera, sin interrupciones de ayuno y desnudez. Trájome Celestina
aquella hermosa bestia, diciéndome que era su prima, y yo le di el
gobierno de mi casa y la soberanía de mi persona. Vivíamos felices.
Felipa, que así se llamaba, natural de las Peñas de San Pedro, era una
fuerte trabajadora en los menesteres más duros de la vida doméstica;
lavaba la ropa y los suelos y toda la casa con verdadero frenesí;
guisaba con abuso de especias y picantes, y hablaba con estridor de
gritos y libérrimo vocabulario\ldots{}

Naturalmente, mis relaciones con Felipa trajéronme nuevas amistades y
trato con personas del propio jaez. Conocí a otra mujer, muy bonita por
cierto, pelo rojo, figura delicada. Aunque el tipo, lenguaje y modales
de Lucrecia (¿nombre verdadero o postizo?) eran tan distintos de los de
Felipa, tratábanse las dos mujeres con familiar intimidad\ldots{} Tras
Lucrecia compareció en nuestras tertulias un hombre ordinario,
disfrazado de elegante, estrenando ropa, mal carado, y hablador verboso,
insubstancial y cínico de asuntos que no entendía.

De esta sociedad, llamémosla así, que a mi albergue acudía, pasamos a
otras, yendo Felipa y yo a tertulias amenas en casas donde conocí y
reconocí caras bonitas y feas, y encontré amigos entre sujetos que veía
por vez primera. No se crea que era la mansión de Celestina ni otra
semejante. Algo se celestineaba allí, es cierto, por bajo cuerda, y más
que algo se le tiraba de la oreja al amigo Jorge; el tono general era de
semi-decencia o \emph{medio-mundo}, y algo de \emph{armas al hombro}.
Útiles enseñanzas de la vida y del mundo adquirí en aquel extraño
beaterio. Oyendo aquí y adivinando allá, vine a comprender que mi Felipa
había sido criada de Lucrecia, y que el fachoso cortejo de esta,
adornado con gruesos brillantes en pechera y sortijas, era jugador de
profesión, y poseía en Madrid pingües chirlatas. Otras muchas rarezas vi
y observé, que no cuento a mis buenos lectores porque quiero irme
derecho al asunto de más interés. Una mujer entró allí, la \emph{Tía
Clío}, con mantón y delantal, arrastrando gastadas pantuflas en
chancleta. Mirándola en tal guisa y desgaire, tardé un rato en
reconocerla, y me dije: «yo he visto a esta vieja en alguna parte.»

Y en el mismo instante se destacó del grupo principal de la tertulia un
señor inflado, calvo y herpético que me llevó aparte para reanudar
conmigo una conversación entablada la noche anterior. Aquel sujeto
llevaba frac, no por llevarlo allí, sino porque de allí se iba al Teatro
Real. «En \emph{El Debate} está usted muy bien---me dijo.---José Luis es
listo, bien relacionado, y sabe mirar por los que le sirven, y abrirles
camino para las buenas posiciones políticas. Un sueldecito regular no le
faltará a usted\ldots{} El periódico está bien hecho: me gusta
mucho\ldots{} Y vivirá: su vida está asegurada para largo tiempo. Hay
dinero, amigo, hay dinero a granel. ¿Sabe usted de dónde vienen los
monises?\ldots{} Pues vienen de Cuba\ldots{} ¿Por qué abre tanto esa
boca? De Cuba, sí, señor. ¿Pero usted cree que hay en España dinero que
no venga de la perla de las Antillas?\ldots{} ¿Qué\ldots{} lo niega
usted?»

---No señor, no niego ni afirmo nada: oigo.

---Pues oiga usted más. El dinero lo mandan los ricos hacendados de la
Isla para crear aquí una opinión favorable a sus intereses. Considere
usted, joven, lo que son los intereses en aquel país tan rico, y tan
desatendido por estos Gobiernos. Los buenos españoles de allí quieren
que no se precipite el Gobierno en echarles reformas y reformas. Sobran
aquí sabios, oradores, y el buen sentido se cotiza muy bajo. Quieren los
buenos españoles que si se ha de quitar la esclavitud, nos contentemos
ahora con \emph{el vientre libre}, dejando lo demás para mejores
tiempos. Si así no se hace, peligrará la riqueza, la propiedad, y los
ingenios serán pronto montones de ruinas\ldots{} Para meter estas ideas
en las cabezas alocadas de acá, los hacendados desean tener aquí órganos
de la opinión sensata\ldots{} Hacen ellos su cuestación, reúnen una
porrada de miles de pesos y la mandan acá. Ahora viene el dinero a las
manos de don Manuel Calvo, que está en Madrid. ¿No le conoce usted? Vive
en casa de Lhardy. Es la única persona que Lhardy aposenta en su
casa\ldots{} De las manos de Calvo pasa el dinero a las de don Adelardo
Ayala, que lo distribuye\ldots{} porque no es sólo \emph{El Debate} el
que cobra por defender la buena causa. ¡No he visto yo pocos fajos de
billetes pasar de las manos de don Manuel a las de don Adelardo! ¿Qué,
se asusta, Tito? ¿Es usted de los españoles pacatos que tiemblan y se
descomponen cuando oyen hablar de gruesas cantidades?

---No me asusto, señor---le dije;---me asombro y casi me indigno de que
se suponga a mi jefe capaz de\ldots{}

---¡Ay qué gracia!---exclamó el herpético rompiendo en franca
risa.---¡Pero si Albareda no pierde con ello ni un átomo de su honradez;
si esto es lo más lícito, lo más meritorio, lo más\ldots! Albareda es un
amable filósofo, que se adelanta a su época. Si a él le conviene tener
un periódico defensor de su política, ¿qué mal hay en recibir auxilio de
un grupo de buenos españoles que miran por su patria? Me consta que el
dinero pasa por las manos de Albareda sin que nada se pegue en ellas.

Aquel hombre, que, según dijo, venía de comer en Lhardy, hablaba con
salpicaduras de saliva y un galopar tumultuoso de los conceptos. Creí
advertir en su lenguaje los efectos de un mediano exceso en la bebida.
Sin venir a cuento, sacó un largo puro habano, diciéndome: «Tome este
tabaco. Es de los de regalía.» En seguida me dio otro, y cuando yo creía
que tomaba aliento para seguir despotricando, se levantó, dejándome con
mis observaciones atravesadas en la boca\ldots{} Le vi acercarse a las
que llamaremos damas por no saber qué nombre darles, y se fue no sé por
qué puerta\ldots{} Acerqueme entonces a la \emph{Tía Clío} con avidez
para interrogarla, y me volvió la espalda, volteando su anchuroso
cuerpo, pobremente vestido\ldots{} Y al instante, sin decirme una
palabra, sin dejar tras de sí otro rumor que el de sus chancletas sobre
la gastada esterilla, desapareció. Mis ojos la buscaban; buscándola la
perdieron de vista. En medio de la sala quedeme perplejo y
apenado\ldots{} Cogí de un brazo a Felipa y le dije: «Ven, vámonos de
aquí, mujer, que en esta casa hay duendes.»

Me guardé bien de contar a don José Luis lo que había visto y oído, tal
vez soñado. Tratando en largos días al maestro y a sus amigos, llegué a
la certidumbre de que \emph{El Debate}, como otros periódicos de Madrid,
vivía de la savia cubana. Esta pasaba por las manos de Albareda sin que
en ellas quedaran ni partículas del precioso metal. Todo era poco para
el cuerpo y el alma de la publicación (imprenta, papel, redactores). El
hombre que sostenía con fatigas y el apoyo de sus amigos \emph{La
Revista de España}, fue un grande y desinteresado propulsor de la
cultura de este país. Fue el más aristócrata de los periodistas y el más
elegante de los políticos. Las campañas que él inspiraba llevaron
siempre el sello de distinción exquisita. En contacto constante con la
gente linajuda se mantuvo fiel a los ideales de la soberanía de la
Nación; era conservador a la inglesa y predicador del
\emph{self-government}. Esta fórmula y los motes de los dos partidos,
fundamento y piezas principales de la máquina política, los \emph{torys}
y los \emph{wighs}, no se apartan de su boca andaluza\ldots{} Y viviendo
entre millonarios siempre fue pobre, y en la pobreza se deslizó su vida,
que muchos tenían por ociosa y era muy activa. Mujeriego, taurófilo y
deportista, tenía tiempo para todo, hasta para demostrar con hechos que
el talento fecundiza la misma frivolidad, y de ello sacan frutos
preciosos la razón y el ingenio.

A propósito de ingenios quiero hablar del conocimiento que en \emph{El
Debate} hice con varios sujetos que lúcidamente han figurado en las
Letras y en el Periodismo. Los que más vivos conservo en mi memoria son
Rodríguez Correa y Ferreras\ldots{} ¡Alto!\ldots{} Déjenme volver atrás.
Necesito el desorden; la estricta cronología pugna con mi temperamento
voluble y mis nervios azogados. Atención. Cuando llegamos a casa
pregunté a Felipa quién era el señor obeso y calvo, de frac, que me
había llevado aparte para hablarme a solas. Díjome que era un mozo de
café o de fonda, que se fue a La Habana y de allá volvió dándoselas de
ricachón, o siéndolo de verdad. De la \emph{Tía Clío}, por cuya
procedencia y oficio le pregunté, díjome lo que a la letra copio: «Es
una vieja medio loca que en el piso bajo tiene una tienda de muebles,
armas y papelorios antiguos. Lejos de aquí la hemos visto vestida de
señora con borceguíes de tacón dorado, y aquí se nos presenta hecha un
pingajo, con chinelas que dice fueron de una tal doña Urraca. Charlotea
de trifulcas que pasaron y de las que están pasando, y es una criticona
que no hace más que gruñir. Se va como viene, sin saludar a nadie y
diciendo no más que: «Hasta ahora.» Y el ahora quiere decir
\emph{siempre}.»

Hablábamos de esto medio dormidos ella y yo, por lo cual quedó en mi
cerebro aquella conversación como cosa de incierta realidad, tocando en
la frontera de lo mentiroso y fantástico\ldots{} Y a los pocos días caí
enfermo de una fiebrecilla que empezó leve, y por descuidarla hubo de
parar en tifoidea, que a mí me postró por más de tres semanas, y a
Felipa dio mucho que hacer y que sentir. La pobre mujer, creyendo que me
las liaba, forcejeó con la muerte, y mientras esta tiraba de mí para
llevarme al otro barrio, mi coima tiraba con verdadera furia para
dejarme aquí.

¡Qué días de sufrimiento y qué noches de angustia! El único amigo que me
acompañaba y a ratos hacía de enfermero auxiliar de Felipa, era el
isleño por cuya mediación afectuosa entré yo en \emph{El Debate}. No se
concretaba su auxilio a las palabras consoladoras y a la dulce compañía,
sino que, a las veces, con su corto peculio cuidaba de proveer el vacío
portamonedas de Felipa\ldots{} En la soporífera largura de mis horas de
fiebre me acosaban las visiones de la \emph{Tía Clío} y del hombre
herpético que me contó la leyenda de los dineros de Cuba\ldots{} Al fin,
restablecida poquito a poco la normalidad en mi caletre, entré en
convalecencia, fui tomando fuerzas, curé, y una tarde, cuando ya podía
valerme y saborear la lectura y la conversación, hablé de este modo a mi
buen camarada el isleño: «Por mucho que yo viva y prospere, no podré
pagarte lo que en esta ocasión, la más crítica de mi vida, has hecho por
mí.» Y él me respondió: «Quién sabe si algún día me presentaré yo a
cobrarte esta deuda, y tú, con buena memoria, te apresures a pagarme.»

Corrió el tiempo arrastrando sucesos públicos y privados; se fue don
Amadeo; salió por escotillón la República, feneció esta, dejando el paso
a la Restauración\ldots{} Reinó Alfonso XII; pasó a mejor vida. Tuvimos
Regencia larga; se fueron de paseo las Colonias y entraron a comer
manadas de frailes y monjas\ldots{} El niño Alfonso XIII fue hombre;
reinó, casó\ldots{} Vino lo que vino: agitación de partidos, inquietud
social, prurito de libertad, alerta de republicanos, guerra con moros,
semanas de fuego y sangre\ldots{}

Pues en tan largo estirón de la Historia, el hombre chiquitín que os
habla vio caer sobre sí un diluvio de calamidades. Pasó miserias, sufrió
persecuciones; trabajó sin descanso, repartiendo su voluntad entre las
tareas de pluma y la conquista de mujeres, únicas empresas en que le
favoreció la fortuna. Errante anduvo de un hemisferio a otro; fue
empleado en Cuba, empleado en Filipinas, periodista que jamás obtuvo
recompensa, escritor que no llegó a conocer el galardón de la fama.
Siempre obscuro y desconsiderado, en sus retornos de América y Oceanía
vivió pobre en Madrid, vegetó en diversos pueblos y poblachos de
provincia. En el curso de esta odisea, alguna vez topó con su amigo el
isleño; se cumplimentaron y departieron sobre la buena o mala suerte de
cada uno. Pero llegó un día en que la conversación fue más larga y de
mayor substancia, como a continuación se verá.

En la Puerta del Sol nos encontramos a los treinta y siete años justos
del día en que tomó el portante don Amadeo de Saboya. ¡Treinta y siete
años! Muy pronto se dice; mucho se tardaría en contar lo que pasó bajo
las chinelas o el coturno de la \emph{Tía Clío} en trece mil quinientos
cinco días. Yo, lejos de aumentar, había menguado de talla; los pelos
que me quedaban eran hebras de plata, y rostro y cuerpo mostraban
lastimosamente los zarandeos del tiempo. Mi amigo no llevaba mal sus
años maduros, y su rostro alegre y su decir reposado me declaraban mayor
contento de la vida que el que yo tenía. Hablamos de trabajos y
publicaciones; díjele yo que había leído las suyas, y él, replicándome
que algo le quedaba por hacer, saltó con esta idea que a las pocas
palabras se convirtió en proposición:

«Una promesa indiscreta oblígame a escribir algo de aquel reinadillo de
don Amadeo, que sólo duró dos años y treinta y nueve días. Tú y yo vimos
y entendimos lo que pasó y lo que dejó de pasar entonces. Tu memoria es
excelente; sabes contar con amenidad los sucesos públicos. Hazme ese
libro, y con ello quedará saldada la deuda de caridad que tienes
conmigo. Puedes observar el método que quieras, ateniéndote a la
cronología en lo culminante y zafándote de ella en los casos privados,
aunque estos a veces llegan al fondo de la verdad más que llegan los
públicos. Puedes entreverar entre col y col la lechuga de tus
conquistas; ya sé que han sido innumerables, algunas acometidas y
consumadas con temerario atrevimiento y dramáticos peligros\ldots{} Por
este trabajo te pagaré lo que dio Cervantes al morisco aljamiado,
traductor de los cartapacios de Cide Hamete Benengeli, dos arrobas de
pasas y dos fanegas de trigo, o su equivalente en moneda, añadiendo el
gasto de papel, tinta y tabaco en los pocos días que tardes en rematar
la obra\ldots{} Dime pronto si aceptas, para cerrar trato contigo, o
buscar otro plumífero con quien pueda entenderme para sacar al mundo la
vaga historia de Amadeo I.»

Vacilé un instante, mirando al cielo y a los tranvías que de un lado a
otro pasaban, y acepté, y con un apretón de manos sellamos nuestro
compromiso.

\hypertarget{vi}{%
\chapter{VI}\label{vi}}

Y ya que sabéis la razón de que yo escribiese lo que estáis leyendo,
añadiré para mayor claridad de este negocio, que el isleño me autorizó a
contar la Historia como testigo de ella, figurándome en algunos pasajes,
no sólo como presenciador, sino como lo que en literatura llamamos héroe
o protagonista. A mi observación de que yo tendía por temperamento y
volubilidad natural a la mudanza de opinión, y a variar mi carácter y
estilo conforme a la ocasión y lugar en que la fatalidad me ponía,
contestó que esto no le importaba, y que la variedad de mis posturas o
disfraces daría más encanto a la obra.

Dadas estas explicaciones, continúo mi cuento. En pleno verano del 71 se
despegó con el calor la Conciliación, retirándose cada parte por su lado
con ganas de pelea. No habían hecho nada. Al soltar sus cuellos del
yugo, la emprendieron a cornadas unos contra otros: «Ya ve usted, mi
querido don José Luis---dije al maestro,---lo mal que resulta el
intentar que gobiernen juntos los que de su separación y diferencia
sacarían la fuerza eficaz que pone en marcha la máquina del sistema. Ya
que tan enamorado está usted del turno inglés, hágase la prueba de que
gobiernen ahora los \emph{wighs} con su programa y planes de reforma, y
que los señores \emph{torys} aguarden con paciencia su vez.»

Pero Albareda no daba su brazo a torcer. Hombre agudísimo, que por
imposiciones de la Fatalidad tenía compromiso de abogar por el
contubernio, desmintiendo su \emph{dilettantismo} anglómano, sacaba
razones de su fértil ingenio, y me apabullaba con sofismas deliciosos.
Seguía yo defendiendo con mi fácil pluma el desbaratado armadijo,
tratando de recoger los pedazos para volver a pegarlos con la cola de
mis artículos. Pero por mi cuenta digo que los \emph{torys} de acá eran
la mayor calamidad del Reino. De cepa unionista moderada, llevaban en la
masa de la sangre los vicios y las malas mañas de la rancia política y
de la Administración apolillada. Con necia fatuidad aseguraban que ellos
solos poseían el secreto de regir a la Nación, y que sin ellos todo era
desorden y merienda de negros. Conocía yo a un señor, inveterado
unionista del 63 y 64, y siempre que nos encontrábamos largábame un
sermón, contrastando la omnisciencia de los suyos con la ineptitud de la
gente nueva. La síntesis era esta: «Nada, nada, amigo; es cuestión de
camisa limpia\ldots» Según aquel inmenso congrio, la clave del gobierno
de España estaba en manos de las lavanderas y planchadoras.

Divorciados el Ayer y el Mañana, matrimonio de conveniencia, entró a
formar Gobierno el Mañana, don Manuel Ruiz Zorrilla, el más valiente y
entero de los hombres de la Revolución, popular cual ninguno por mirar
de frente a los intereses del pueblo, voluntad firme, corazón que ardía
en el amor romántico de una España redimida. Sus compañeros de Gabinete,
llamándose demócratas, gastaban pecheras tan blancas y lustrosas como
las de los palaciegos mejor almidonados. No era cuestión de camisas
limpias, sino de cerebros lavados de roña y telarañas.

Un poquito atrás, caballeros. Se me olvidó decir que en los tenebrosos y
amargos días de mi enfermedad fue la apertura de Cortes, y en el acto
solemne leyó don Amadeo el acostumbrado discurso, como todos los del
ritual, enfático y pedantesco, henchido de vanas promesas y preñado de
hiperbólicas esperanzas. En boca del Rey puso el Gobierno parrafillos en
que este pudo vanagloriarse con sincera bravura de su liberalismo, como
de su respeto a la voluntad de la Nación. Con entusiasmo loco recibió el
anfiteatro estas lindas canciones, que trascendieron pronto a las calles
y el corazón de los adictos\ldots{} Presidente de las Cortes fue Olózaga
por votación no muy nutrida. Ciento diez papeletas le colaron en las
urnas. La oposición era tremenda; entre federales, carlistas, moderados
netos, alfonsinos de solemnidad o vergonzantes, formaban una falange de
complejos rencores que iban a una contra el Gobierno, el Rey y el Verbo
divino.

Adelante. Reanudo el hilo cronológico para deciros que Ruiz Zorrilla
trajo a la política oxígeno abundante y frescura de reformas por las que
suspiraba el envejecido ser de la Patria. Entró don Manuel con singular
arranque a matar las rutinas; crujía la \emph{Gaceta} del empuje, y el
radicalismo se estrenó con un sonoro triunfo. De aquel Gobierno se dijo
que era una \emph{República con Rey}. ¡Lástima que no hubiera sido
cierto, y que no durara lo bastante para que se consolidase la utopía y
se hiciera verdad de carne y hueso! Los Ministros que don Manuel asoció
a su obra tuvieron éxitos redondos desde los primeros días. Don Servando
Ruiz Gómez realizó brillantemente una emisión de 220 millones en un
papel que yo no he poseído nunca, y que llaman \emph{Billetes del
Tesoro}, y un empréstito de 150 millones; Montero Ríos dio un buen tajo
al presupuesto eclesiástico; el tan modesto como entendido don Santiago
Diego Madrazo ordenó las cosas de Fomento, y Mosquera intentó lo mismo
con las antillanas, que eran más duras de pelar.

El verano apoyó con su calor esta vehemencia del zorrillismo, y todos
íbamos viviendo\ldots{} digo mal, yo no vivía, porque no daba un paso
sin pisar horrendas dificultades, por los desniveles de mi hacienda, que
ya me llevaban a la bancarrota inevitable. Así como los Estados, en sus
conflictos pecuniarios, acuden a los grandes financieros del mundo, yo,
en mis apuros (secuela de mi enfermedad y otros excesos), llamaba a las
puertas de la \emph{Casa Rostchild}, a las de la \emph{Casa Lafitte}. Mi
sueldo y lo que yo ganaba en \emph{El Debate} hablando pestes del
radicalismo, barajando los \emph{torys} con los \emph{wighs}, o bien
preconizando como heroica medicina de España el \emph{self-government},
todo esto y algo más se lo llevaba la \emph{Casa Rostchild}, un roñoso
prestamista de la plazuela del Alamillo, que en diferentes crisis
metálicas me había facilitado algunos millones o puñados de
maravedises\ldots{} Ahogado ya, puse mis paralelas a otras opulentas
casas judaicas, y como estas me mandaran a escardar cebollinos, fui y
qué hice, contratar un empréstito de diez duros, a corto plazo, con
\emph{Baring Brothers} de la \emph{City} (en Madrid, callejón de San
Cristóbal); mas no habiendo podido cumplir, me dieron un escándalo, y a
la escandalera se agregó la \emph{Casa Rostchild}, y entre todas
aquellas casas me dejaron, como quien dice, en cueros vivos; buena moda
para verano.

A estos males se sumaron otros, que por ser de calidad afectiva dolían y
amargaban más, y fue que Felipa empezó a mostrarse displicente y a
renegar de mi estado financiero. Aunque me adoraba, según decía, no se
sentía con fuerzas para vivir del aire como los camaleones, y en sus
actos y aun en la palabra, notaba yo el propósito de poner entre mi
descarnada pobreza y su gallarda persona la distancia que impone el
instinto de conservación. A cada momento, por un daca o por un toma, nos
peleábamos\ldots{} El regaño gordo vino al cabo, y la vi recoger su ropa
para marcharse a vida menos ruin. Como yo observara que alguna prenda de
su uso dejaba en casa, pensé que preparaba un artificio para
volver\ldots{} Al verla salir, tomé una actitud de dignidad severa, sin
desplegar los labios ni alterar mi adusto entrecejo\ldots{}

Al día siguiente supe que se había hospedado en una casa donde la
honestidad no tiene su asiento\ldots{} Como yo esperaba y temía,
volvió\ldots{} Burla burlando nos enredamos en reconvenciones, \emph{más
eres tú} y \emph{que torna, que vira}\ldots{} Con furia un tanto
grotesca Felipa me cogió de improviso doblándome por la cintura en la
disposición de darme lo que llaman en Cuba un boca-abajo, y con la palma
de su mano dura me arreó tal azotina en semejante parte, y luego tales
estrujones en la espalda y cabeza, que olvidé mi condición varonil para
chillar como un niño. Concluyó el castigo poniéndome en pie y
zarandeándome. «Aunque me voy, pizca de hombre---me dijo cogiendo la
puerta,---no creas que te dejo campar solo\ldots{} ¡Qué sería de este
pobre Tito sin mis azo\ldots{} titos!\ldots»

Al siguiente día recibí por un mozo de cuerda un paquete conteniendo
entre papeles un terno de lanilla de los que en \emph{El Águila} valen
cinco o seis duros. No era nuevo, pero sí en buen uso, comprado a una
prendera, o en el Rastro. Debió de pertenecer a un niño de catorce años,
y a mí me venía como si me lo hubieran hecho por medida. En un bolsillo
del chaleco encontré dos pesetas envueltas en un papel. La procedencia
del regalo ninguna duda me ofrecía. Antes que el mozo me diera las señas
de la donante, reconocí a Felipa, que era una bestia muy
delicada\ldots{}

Pues, señor, me endilgué al instante mi trajecito, que me caía muy bien,
y salí a la calle gustoso de exhibir en ella mi persona, recluida por
falta de vestimenta\ldots{} Y bien podría mi buena sombra depararme una
conquistilla que me consolara de tantos infortunios\ldots{} Después de
pasear un rato por las aceras, caldeadas del sol, volví a casa, donde
reparé mi organismo con el frugal comistraje que me aderezaba la
portera. Fuime después al Café Oriental, y me arrimé a la tertulia de
don Santos la Hoz, Roque Barcia, Rispa Perpiñá y otros desinteresados
patriotas. Sólo estaba el primero, y con él me explayé hablando de la
situación y poniendo la persona de Zorrilla sobre el cuerno de la luna.

Ya sabéis que don Santos la Hoz era un curita que condenó a garrote vil
sus hábitos, metiéndose de lleno en la vida laica y en el torbellino de
la política, primero progresista, después republicana. Mezquino de
cuerpo, ahilado de rostro, en el cual dejó crecer patillas y un lacio
bigote; suelto de nervios y más suelto de palabra, don Santos ponía en
la política toda la honrada vehemencia que su alma no pudo encontrar en
la vida eclesiástica\ldots{} Había cambiado de tema, de norte y de
ideales; pero su estilo era el mismo, y en los clubs tenía dejo y tonos
de predicador; en el café, delante del licor negro y humeante, movía las
manos y miraba al vaso como un grave sacerdote que está diciendo misa.

«Esto va muy bien---me dijo mirando a un periódico que al lado tenía,
como si estuviera leyendo la Epístola.---Si don Manuel sigue por el
camino que ha emprendido, la democracia forzosamente ahogará la
Monarquía, y don Amadeo tendrá que volverse a su tierra diciendo:
`Españoles, habéis demostrado que merecéis la República\ldots{}'. La
benevolencia se impone. Pi Margall, Castelar y Barcia, que forman el
Directorio, dirán a las masas en el manifiesto que preparan: `¿Hemos de
tratar con igual rigor a los que nos dan condiciones de vida y de
progreso, y a los que pugnan por quitárnoslas?'. En fin, yo estoy
contento. Esto marcha\ldots{} Claro es que Sagasta y el Duque pondrán en
el camino de don Manuel chinitas y peñascos\ldots{} pero, amigo,
\emph{todo lo vence amor o la pata de cabra}, todo lo vence el principio
sacrosanto de libertad, ese rayo de Dios, esa palanca, esa
panacea\ldots»

Nos burlamos luego de los carlistas, diciéndoles ante el mármol de la
mesa del café: «Venid, echaos de una vez al campo\ldots{} Así os
aniquilaremos más pronto.» Nos reímos de las damas católico-alfonsinas.
Ya podéis guardar en vinagre o en alcohol a vuestro niño. La Patria le
rechaza (frase de Castelar), \emph{como el mar arroja a la playa los
cadáveres}\ldots{} Y dicho esto, nos quedamos tan frescos, con permiso
del calor que nos abrasaba. Don Santos pagó mi café, y yo me fui a la
calle\ldots{} ¡Oh calle, única delicia y recreo del hombre tronado!

El verano se me presentaba fosco y aterrador. Casi todos los amigos que
podían aliviar mi penuria, habían echado a correr. Para mayor desdicha,
la inacción veraniega metió a \emph{El Debate} en el pantano de las
economías, y a mí me tocó el ser uno de los licenciados hasta otoño. El
isleño se fue a Santander, Albareda a tomar los baños de Dax, y yo no
tenía santo a quien poner una vela\ldots{} Ferreras y Correa, ¡ay de
mí!, también levantaron el vuelo. Lleneme de paciencia, y me vestí de la
coraza del estoicismo. Hallaba consuelo en mi fatalismo musulmán, el
cual en aquella triste ocasión me decía: «Está escrito que por
desconocida senda te vendrán satisfacciones y venturas\ldots»

En largos y calurosos días esperé, mirando a la esfinge del Mañana. Por
pasar el rato escribía gratis en \emph{La Igualdad} y en \emph{La
Ilustración Republicana Federal}. Tenía esta su redacción en la Plaza de
la Cebada, 11, y la dirigía Rodríguez Solís. En la lista de los
colaboradores figuraba todo el santoral republicano, con los pontífices
a la cabeza; pero los más constantes eran Roque Barcia, Roberto Robert,
Ramón Cala y otros de vago y hoy olvidado nombre. Tanto como me
encantaban Robert y su acerada sátira, me entristecía don Roque con su
literatura bíblica y orientalesca en rengloncitos de este jaez: «Avanza,
hombre loco, y dime: ¿cuál es tu sino?\ldots» y el hombre loco y pálido
responde: «Mi sino es llorar hoy el Pasado, que no quiere volver y
vuelve.---Retírate, Pasado, y no olvides llevarte tu manto de
tinieblas.---Adiós, hijos del día; la luz en que vivía me daña. Adiós.»
¡Y había lectores, entre ellos mi portera, que se deleitaban con estas
cosas!

En \emph{La Ilustración Republicana Federal} me aclimataba yo más que en
\emph{La Igualdad}, pues aunque en ninguno de los dos periódicos ganaba
un real, en el primero tenía de director al bueno y cristianísimo
Rodríguez Solís, que solía convidarme a comer en su modesta casa,
llenándome el buche para un par de días. A las veces, llevábame Roberto
Robert a \emph{Lhardy}, un espléndido bodegón que radica en los sótanos
de la Plaza Mayor, y tiene su entrada suntuosa por Cuchilleros, en lo
más bajo de la Escalerilla. Dábannos allí cocido, judías u otro plato
suculento; y amenizábamos el festín con el dulce murmurar, comentando la
vida social o política. Recuerdo que en aquel \emph{Lhardy} apuramos una
tarde el tema candente de las \emph{Cacerías de Riofrío}. No se hablaba
de otra cosa. Persiguiendo venados con el Rey, Serrano conspiraba para
derribar a Zorrilla, al mes de subir este al poder. No sería verdad;
pero el público, ávido siempre de novedades, se hartaba de aquella
comidilla\ldots{} Las cacerías fueron y son los más seguros vedados para
matar las grandes reses políticas.

Pero don Manuel seguía tan terne, sin que le alcanzaran los tiros, si
acaso los hubo, ni cuidarse de ellos. Por aquel tiempo, si no me falla
la memoria, visitó a su hermano el Príncipe Humberto, heredero de la
corona de Italia. Estuvo en La Granja, en Madrid y en Toledo y Sevilla.
Al despedirle, nuestro Presidente del Consejo oyó de labios del huésped
ilustre estas palabras de felicitación, que recordaba siempre con
orgullo: «Deseo para mi hermano y su dinastía diez años de gobierno
radical.»

Grabada con letras de oro quedó en mi memoria esta frase, porque la oí
de la boca dulce y colorada de una dama, de una mujer\ldots{}
que\ldots{} Leed, os lo suplico, leed a renglón seguido mi nueva
conquista.

\hypertarget{vii}{%
\chapter{VII}\label{vii}}

Doña María de la Cabeza Ventosa de San José, a quien respetuosamente
inscribo con el número \emph{tantos} en mi amoroso Registro, era una
dama fresca y agraciada, de negros ojos, risueña boca, lucidas carnes,
poseedora de dos tiendas de telas, una en la calle de Toledo y otra en
la Concepción Jerónima, donde habitualmente residía. No diré que fuese
una cabal hermosura; pero sí que tenía lo que llamamos un gancho
fisionómico, un garabato facial, un mirar pillín y un fruncimiento de la
boquita que a todos cautivaba, y con tal gancho a mí me pescó el alma,
inspirándome una pasión que no vacilo en llamar volcánica.

¿Cómo la conocí? Pues los vaivenes de mi miseria me llevaron de nuevo
hacia Córdoba y López, y Mateo Nuevo, que quiso arreglar mi complicada
cuenta con la \emph{Casa Rostchild} de \emph{Alamillo Square}. Algo se
aflojó con aquellas gestiones el dogal que me apretaba el pescuezo;
respiré un poco, y por derivaciones naturales hice conocimiento con un
vejete gracioso y pío, que llamaban Plácido Estupiñá, corredor de
dependientes de comercio, el cual me exhortó a dejar la pluma por la
vara de medir, y la literatura por la contabilidad mercantil. Intercedió
noblemente con las \emph{opulentas casas de banca} para que me dieran
mayor respiro, y llevándome de tienda en tienda, di con mi persona en la
de doña María de la Cabeza, que precisamente, ¡oh felicísima
casualidad!, necesitaba un \emph{chico que supiera llevar cuentas}.
¡Cielos divinos!, aquel chico fui yo. ¿Era sueño, era realidad? Estupiñá
fue el alado mensajero de la Providencia que me llevó del abismo de la
desesperación al pináculo de mi ventura.

Del gusto que me dio el verme admitido por doña Cabeza y aposentado en
su propia casa, me puse muy malo, me entró fiebre, atacome la tos ferina
con quebranto de todo el cuerpo. Me metieron en cama; mi admirable
patrona y \emph{principala} me llevaba calditos, infusiones, alguna
golosina para llamar el apetito, apelando a las friegas para desvanecer
los dolores erráticos. Mi gratitud hízome ver en la señora un ser
divino, quizás la propia esposa de San Isidro Labrador, Santa María de
la Cabeza, cuyo glorioso nombre llevaba. ¡Vive Dios, que antes que el
nombre las igualaba y confundía la santidad!\ldots{} Cuando me dieron de
alta y me levanté, poniéndome la ropa limpia, lavada en mi nueva casa,
me sentí inundado de una luz celestial y abrasado en fuego de
inspiración. El alma se me quería salir por ojos y boca para ofrecerse
con sublime rendimiento a doña Cabeza, como galardón de sus divinas
bondades e infinita misericordia.

Yo soy un hombre que no sabe disimular sus sentimientos. Soy todo un
torrente para la sinceridad, y un águila para poner en ejecución, sin
perder instantes, lo que me dicta mi conciencia. Consecuente conmigo, me
arranqué, como suele decirse, de una vez, y le solté a mi doña Cabeza
una declaración de amor tan coruscante y ardorosa, que la buena señora
se quedó asustadica, vacilante entre la risa y el asombro. Notando yo
que no era la dama tan fácil al asedio, avivé el fuego de mi oratoria,
echando en él llamaradas de locura, sutilezas de poesía, y conceptos que
doña Cabeza oía quizás por primera vez en su vida\ldots{} Y el efecto se
produjo al fin. Al través de los espesos vapores que, a mi parecer,
levantaba mi apasionado lirismo, observé que el rostro de doña Cabeza se
ponía muy serio, que en su boca graciosa expiraba la última risa, que
aparecían después unos pucheritos muy monos\ldots{} y que la interesante
señora, enmudecida por la emoción, me mandaba callar\ldots{} ¡Ay, qué
pillo!

Aunque doña Cabeza me dijo aquella tarde que se vería en el caso de
despedirme de su casa, en tal forma lo dijo, y con tal mimo de
\emph{quiero} y \emph{no quiero}, que me tuve por vencedor. Debo
declarar que mi pasión era sincera, y que mi protectora se hacía dueña
de todo mi ser. ¿Había encontrado mi felicidad y la solución de los
graves problemas de mi vida? Tal vez\ldots{} A los tres días de aquella
mi flamígera declaración, desesperado vuelo de un alma que huye del
vacío, aseguré y celebré mi triunfo. Loco de orgullo juré amor eterno,
fidelidad hasta la muerte. Y cuando a este culminante fin llegaba, un
desengaño enfrió mi entusiasmo. María de la Cabeza no era viuda, como
presumí viéndola vestir de alivio. Por ella supe que su viudez consistía
en vivir separada de su esposo, un perdido criminal, con méritos
bastantes para ir a presidio. En Madrid andaba el tal: su mujer le
pasaba un duro diario, y de vez en cuando le pagaba las trampas; pero
antes muriera que admitirle a su lado. La riqueza, las tiendas y alguna
finca rústica eran de ella. No refiero lo que Cabeza me contó del engaño
y disparate de su casamiento, porque no añade ni quita interés a esta
verídica historia.

Si me afligió por un lado el saber que mi dama no estaba capacitada para
segundas nupcias, me agradó mucho conocer su abolengo liberal, rancio y
clarísimo, como esas aristocracias cargadas de blasones. Mi señora era
nieta, por parte de madre, del gran don Benigno Cordero, espejo de
milicianos, que inmortalizó su nombre en el Arco de Boteros, hoy \emph{7
de Julio}; sobrina, en segundo grado, de Calvo Asensio, y en tercer
grado, de don José Abascal. Parentesco lejano tenía con Mariana Pineda,
y cercano con don Vicente Rodríguez y don Juan León Moncasí. Su padre,
don Lucas Ventosa, fue uno de los más leales amigos de Espartero, íntimo
de don Evaristo San Miguel y de don Ramón de Calatrava. En su casa, y en
la de sus padres, Cabeza se pasó parte de la vida bordando banderas para
los batallones de milicianos. Era la encarnación del ideal progresista,
y en sus dos tiendas se refugiaron una y mil veces los cabildeos
electorales y aun los tapujos revolucionarios. Toda esta tradición
cálida y candorosa se fue acumulando en la cabeza de mi doña Cabeza, tan
entusiasta de Prim, que lloró tres días cuando le mataron. Muerto el
héroe, la idolatría de mi dama vino a condensarse en el único santo que,
a su parecer, representaba las glorias del \emph{Progreso}, don Manuel
Ruiz Zorrilla.

Yo también me volví radical como el mismo don Manuel, o como su
trompetero Ángel Fernández de los Ríos. Fuera de esto, yo estaba en la
gloria, bien comido, bien bebido, admirablemente apañado de ropa, y
satisfecho en cuantas necesidades y estímulos constituyen la vida
espiritual y fisiológica. El marido de Cabeza, Serafín de San José, no
me inquietaba gran cosa. Alguna vez me tocó despacharle con tres pesetas
o un duro, sacados del cajón; era un cínico silencioso que a su
degradación ponía máscara de prudencia. Más me inquietaban algunos
parientes de Cabeza que se retraían de visitarla, reprobando así
discretamente su irregular trato conmigo. Y mayor zozobra que el despego
de los primos y agnados me causó la insistencia con que paseaba la calle
un sujeto alto y zancudo, de color cetrino, barba negra, nariz tajante,
con lentes que daban no poca impertinencia a su mirar fisgón, bien
vestido, la chistera un poco ladeada. Advertí un día que al pasar le
saludó Perico Luna, que solía tertuliar en mi tienda.

Interrogué al amigo, que así me dijo: «Es un tal Alberique, amigo de
Madoz, empleado que fue en \emph{La Peninsular}. No tiene hoy más oficio
ni más beneficio que pintar la mona y hacer el oso.» Por algo más que se
escapó a la discreción de Luna, y otro poco que me indicó Roberto
Robert, sospeché que aquel tipo había sido mi antecesor en los blandos
afectos de mi señora doña Cabeza. No necesité saber más para decidirme a
espantar al enojoso estafermo. Elegida la ocasión más favorable, salí a
la calle una mañana, y me encaré con el cargante individuo. A quemarropa
le di el quién vive en la forma que cuento, y no es jactancia:
«Caballero, quiero saber qué se le ha perdido a usted en esta parte de
la calle, y qué motivos tiene para montarnos la guardia. Si es policía,
dígalo y se le dará una propineja para que no moleste tanto.»

---Señor enano de esta venta---me replicó zumbón, ajustándose los lentes
en la nariz huesuda y poniéndose en facha,---yo estoy en mi derecho
cogiéndome parte de la calle o la calle entera, y usted váyase a medir
percales, y déjeme en paz.

---Si usted me insulta, le diré que voy a coger la vara para medirle a
usted las costillas.

---Antes me insultó usted a mí llamándome policía y ofreciéndome
propina\ldots{} Si usted no fuera tan chiquitín le pediría cuenta de sus
ridículas arrogancias. Conozco su nombre y condición. Por si usted no
sabe quién soy y cómo las gasto, ahí le dejo mi tarjeta. Como usted no
trae tacones altos, y ha salido en zapatillas, tengo que inclinarme para
que la tarjeta pueda llegar a sus manos.

Tomé la tarjeta, y leí: \emph{Modesto Alberique}, \emph{representante de
la Sociedad Belga Constructora de cierres mecánicos}. \emph{Esgrima},
\emph{3.} Y viéndole partir con aire jaquetón, le dije con el
pensamiento: «Ya te daré yo a ti cierres mecánicos, farsante.» Volví a
mi tienda, y nada dije a Cabeza, que estaba en el principal, en manos de
su peinadora. Era tan firme mi resolución de mandarle los padrinos al
infatuado virote que me ultrajó groseramente, que no pasó la tarde sin
pensar en los amigos que debía escoger para función tan delicada.
Andando en esto, supe que mi rival era un poco espadachín, o que de ello
presumía. Mejor que mejor. El lance había de ser duro. Mi amor propio no
consentía otra solución que matar a mi contrario, y quedar yo airoso y
arrogante, cantando el \emph{quiquiriquí} en mi gallinero.

En las tertulias de mi tienda menudeaban los noticiones y las profecías
políticas. Oigan lo que me dijo aquella tarde, o la siguiente, un amigo
nuestro, inveterado progresista semi-fósil: «Parece que se conspira de
lo lindo. ¿Qué hay de La Granja? Pues hay\ldots» Diciendo esto mostraba
un fajo de periódicos, entre los cuales vi \emph{El Imparcial}, \emph{El
Debate} y \emph{La Política}. El corresponsal del periódico del señor
Mantilla contaba que la Reina María Victoria había salido como escapada
del Real Sitio, llevándose a su marido\ldots{} Hay más: «El Brigadier
Palacios, Comandante General del Real Sitio de San Ildefonso\ldots{}
¡oído a la caja!, arrestó al joven Díaz Moreu, oficial de Marina,
ayudante de Su Majestad.» ¿Por qué creerán ustedes? \emph{Porque siguió
demasiado cerca a don Amadeo}. Pero \emph{El Imparcial} trae otra
versión. Oigan: \emph{La causa del arresto del ayudante fue que este
saltó una zanja con más presteza que el Brigadier Palacios}. ¿Quieren
decirme ustedes qué significa esto de Reina fugada, y de arrestos y
zanjas? Pues el corresponsal de \emph{La Política} salta otra vez con la
cuenta de cuarenta y ocho reales que no ha sido abonada al dueño del
Hotel Europa de La Granja, el señor \emph{Davide Macchino}. ¿Qué es
esto? ¿Quién me compra un lío? Hame dado en la nariz, señores, olor de
barraganía\ldots{} Estas cosas tan raras y esta cuenta sin pagar, y el
Rey que escapa con la Reina, ¿no os señalan un rastro? Seguid el rastro,
seguid la pista, y encontraréis una res que dicen es hermosa, yo no la
he visto\ldots{} la \emph{dama de las patillas}.

Tomó entonces la palabra don Francisco Bringas, otro de los asiduos a mi
tienda, varón calmoso y sesudo, colocado recientemente por Zorrilla en
una modesta plaza de Fomento. Asegurándose las gafas sobre la nariz,
aquel hombre, que llamaban \emph{Monsieur Thiers} por la perfecta
semejanza de su rostro y talle con los del celebérrimo político francés,
nos dijo que no era de buenos españoles sacar a colación a \emph{la de
las patillas}, ni dar aire a los malignos rumores, de que se apacienta
el vulgo ignaro. El Monarca que nos regía, por obra de los 191 votos o
por lo que fuere, se menoscababa en su alta dignidad, traído y llevado
en lenguas de la gente ociosa. «Yo serví lealmente a doña
Isabel---añadió,---y mientras comí su pan, jamás permití que en mi
presencia se dijeran las atrocidades que corrían acerca de ella\ldots{}
Ahora, después de larga cesantía, debo un humilde destino a don Manuel,
colocación que viene encabezada con el nombre del Rey. Pues yo, fiel a
mis principios, no digo ni escucho ninguna cuchufleta en mengua del Jefe
del Estado. ¿Qué más? Ayer me vino Rosalía con el cuento de la señora
patilluda, y yo le dije: `Rosalía, hazme el favor de callarte la boca'.
Por mi decoro de funcionario público, por respeto al primer Magistrado
de la Nación, oigo esas \emph{anedoctas} como fábula indecente. Y punto
final.»

---Tiene razón don Francisco---me dijo Cabeza interviniendo en el
coloquio con la bondad juiciosa que era el mayor encanto mío.---Sí,
amigo Bringas, fuera cuentos que bien pueden ser falsos testimonios.
¿Qué nos importa que Su Majestad tenga un devaneo, y que la tal gaste
patillas o barba corrida? No demos aire a las habladurías, y menos ahora
que tenemos el \emph{progreso} en el poder. ¡Y que está el Rey poco
contento, vaya! Por lo que he contado a ustedes de las palabritas del
don Humberto al despedirse, comprenderán que hay don Manuel para
rato\ldots{} lo que digo: ¡don Manuel para rato!

Al anochecer desfilaron los amigos, y antes de cenar di un salto al
Casino Federal, para conferenciar con mis padrinos, hombres inflexibles
en materias de honor: Córdoba y López, Ramón Cala\ldots{} Pasaron tres
días; el feroz Alberique no se daba prisa para designar padrinos. Los
míos iban en su busca; no le hallaban nunca en su casa. Temimos que se
lo tragara la tierra. Pero del centro de ella le habría sacado yo para
vapulearle públicamente y pregonar su cobardía. Por fin dio la cara, y
se concertó el duelo en las condiciones que imponía la gravedad del
caso. Y en los días que precedieron al terrible lance, mi señora doña
Cabeza mostró deseos de que yo escribiese en \emph{Las Novedades},
ensalzando hasta las nubes a don Manuel, y declarándome radical
monárquico, bajo el manso poder de don Amadeo I. Claro es que yo no
podía negarme a tan dulces requerimientos. Escribí, pues, sin esfuerzo,
hinchados panegíricos de la política radical, y el bueno de don Manuel
se asfixiaba seguramente con las nubes de oloroso incienso que yo
arrojaba sobre él. Llevado y traído por fatal corriente misteriosa, yo
era el campeón de todas las causas. En corto tiempo enaltecí con mi
fácil pluma el federalismo intransigente, el federalismo templado, la
monarquía conservadora de Serrano y Sagasta, y la monarquía democrática
de Ruiz Zorrilla. Era yo, pues, un caso peregrino de proteísmo; y ved,
amigos, cómo esta mi voluble constitución mental venía consagrada desde
mi nacimiento y bautismo por mi nombre y cognomen. Yo me llamo, sabedlo
ya, \emph{Proteo Liviano}, de donde saqué el \emph{Tito Livio} usado en
mis primeros escritos, y el \emph{Tito} a secas que hoy merece mi
preferencia por lo picante y diminuto.

Escribí, como digo, furiosos alegatos ministeriales para dar gusto a la
gobernadora de mi existencia. Pero en lo más recio de mi campaña, vino
el trueno gordo; las intrigas del Real Sitio dieron su fruto, y Ruiz
Zorrilla con todo su radicalismo reformista se desplomó con estrépito. Y
he aquí que aparecieron en el tablado, por el foro derecha, Serrano y
Sagasta tapándose el rostro con el antifaz del Ministerio
Malcampo-Candau.

\hypertarget{viii}{%
\chapter{VIII}\label{viii}}

Un poquito atrás. No se me vaya a quedar en el tintero mi épico lance
con Alberique, más interesante, a mi juicio, que aquella cáfila de
hombres que iban y venían, y aquellas menudencias del vivir nacional,
que el Tiempo y la \emph{Tía Clío} arrojan en el polvoriento rincón de
la trastienda, donde toda antigüedad inútil tiene su sepulcro.

Acordaron los padrinos que el duelo fuese a pistola: la desigualdad de
talla entre mi enemigo y yo imposibilitaba el uso del arma blanca. Los
padrinos de mi contrario, Felipe Ducazcal y el teniente Luque, de quien
hablaré después, propusieron el sable, arma en que Alberique se creía
fuerte; pero al fin cedieron a la razón, que era la pistola. Llevamos de
médico a un chico de San Carlos que en aquellos días recibió la
Licenciatura. El lugar donde habíamos de tirar a matarnos era un jardín
o huerta en las cercanías de las Ventas del Espíritu Santo.

Las ocho de la mañana serían cuando llegamos al terreno los dos rivales,
con nuestros respectivos apoderados. Alberique iba muy estirado de
guantes, vestido de negro, el sombrero muy encasquetado para que no se
lo arrebatase el viento que del Oeste soplaba. Por no cansar, suprimo
los pormenores. Partido el campo y cargadas a conciencia las pistolas,
nos pusimos frente a frente. Sin ninguna jactancia, debo hacer constar
que yo estaba sereno ante la faz del drama, como lo estoy en el momento
de referirlo. Yo he nacido para las ocasiones críticas, para los actos
que se desarrollan en raudos minutos, decisivos entre la vida y la
muerte. Tocó a mi rival disparar primero. No me acertó. Disparé
yo\ldots{} Nada\ldots{} En su segundo disparo, Alberique afinó la
puntería. Yo dije: «¿Sí? Pues ahora verás.» No era yo tirador; afiné con
toda calma\ldots, ¡pim!, le metí la bala en el costado derecho\ldots{}
¡Alto!\ldots{} La herida de Alberique era de pronóstico reservado.
Terminó el lance. No me presté a reconciliaciones ni saluditos, y me
retiré con tranquilidad augusta.

O mucho me equivocaba yo, o todos los que se cruzaron con mi coche en la
carretera de Aragón me miraban con respeto admirativo, quizás, quizás
con respeto medroso. En mi casa me declaré a Cabeza, refiriéndole con
terroríficos detalles el lance y sus antecedentes y motivos. Oyome
atenta sin mostrarse demasiado orgullosa de mi serena valentía, y contra
lo que yo esperaba, me salió con esta desentonada cantinela: «Has hecho
mal, Proteo, en tomar las cosas tan por lo caballeresco, porque ese
majadero de Alberique es casado\ldots, casado y con cinco hijos.
Figúrate que se muere de la herida. Pues tú le has matado, y por tu
quijotismo quedarán huérfanas esas pobres criaturas\ldots{} Todo por el
honor. ¡Dichoso honor, que sólo existe en las lenguas de los que no lo
tienen! Dime, Proteo querido, ¿dónde tienes tú el honor? ¿Lo has traído
tú a casa, o estaba aquí ya cuando llegaste?\ldots{} Hazme el favor de
no hablarme a mí de esas pamplinas. No hay más ley que el amor, el
trabajo, la libertad y el progreso, y todo lo demás es \emph{verso} y
tonterías. ¡Ah!, se me olvidaba: también es ley de vida la buena
contabilidad y el arreglo de los negocios, y respetar el tuyo y mío.
Como me llamo Cabeza, que esto creo y no creeré otra cosa si mil años
vivo.»

Quedeme de una pieza oyendo estas razones, y ellas habrían bastado a
quitarme el sosiego, si Cabeza no me mostrara su cariño y confianza en
terreno que no era el ideológico. Adelante: Como decía, cayó Zorrilla
cuando se le creía más seguro. El terremoto político que llamamos
Crisis, se produjo por la elección de Presidente de la Cámara. El
candidato ministerial, Rivero, obtuvo 110 votos, y a Sagasta, candidato
de los unionistas, progresistas templados y carcundas, le votaron 123
padres de la Patria. Esta se quedó turulata viendo que por corta
diferencia de votos se cambiaba el Gobierno. Pero tal era el sistema,
mal traducido del inglés, tal la bastarda imitación de aquel
\emph{self-government} con que Albareda y yo andábamos a vueltas en
\emph{El Debate}\ldots{} Malos ratos debió de pasar el Rey con este
\emph{self-desbarajuste}.

¡Sorpresa, escándalo, furor! La Tertulia Progresista se echó a la calle
con un pendón morado. Salieron los estudiantes de Farmacia y San Carlos
a ventilar su ardorosa juventud, fatigada de la estrechez y disciplina
de las aulas. Madrid ardió en alborotos, vocerío de vivas y mueras.
Restallaban de boca en boca los dicterios contra Sagasta, y hasta las
verduleras designaban a las fracciones políticas contrarias al
Radicalismo con los viles apodos usuales: \emph{fronterizos},
\emph{cangrejos}, \emph{calamares}, \emph{palomos},
\emph{tomadores}\ldots{} Mi Cabeza me mandaba que fuese a meter ruido en
las manifestaciones, y a enfoguetar los ánimos con mi briosa elocuencia.

Obediente a mi dulce tirana, acudí al bullicio, y entre la turbamulta
encontré a muchos federales que se agregaban al progresismo radical,
para hinchar el coraje público y armar camorra con los agentes de la
autoridad. Ramón Cala me aseguró que antes de dos meses tendríamos la
Federal con todo su complejo tinglado de pactos y cantones; Rodríguez
Solís comentó el retraimiento cada día más significado de la sangre azul
y del dinero amarillo. Las únicas damas de alcurnia que iban a Palacio y
acompañaban a la Reina, más por lástima y respeto que por adhesión
verdadera, eran las Duquesas de Fernán-Núñez y de Tetuán, la Condesa de
Almina y otras poquitas más. Y Luis Blanc opinó cándidamente que la
Grandeza, con la sorda y persistente conspiración del desaire, nos
estaba haciendo el caldo gordo a los republicanos. Yo, que si en letra
de molde, por dar gusto al dedo, falsifico donosamente la verdad, soy
esclavo de ella cuando hablo con mis amigos, les dije que nosotros
éramos los que hacíamos el caldo gordo a las elegantísimas damas
alfonsainas y catolicoides, ayudando a convertir en palabras vacías los
tres rotundos \emph{jamases} del General Prim.

La implacable cronología, de la cual quiero hacerme esclavo, me lleva en
los primeros días del Ministerio Malcampo a referir una nueva y
peregrina conquista\ldots; digo mal, porque en realidad no fui yo
conquistador, sino conquistado. Ved qué cosa más rara. Una tarde,
terminado el trajín de la tienda (que fue, por más señas, harto
engorroso: recibir el género de invierno, anotar precios según factura,
precios de venta al vareo), salí a desentumecerme y proveer de aire
fresco mis pulmones, y cuando pasaba junto al callejón de la Concepción
Jerónima, salió de este una muchacha, que puso en mi mano una cartita y
apretó a correr. Pronto la perdí de vista. «Aventura tenemos» pensé yo;
y antes de que abriera la esquelita, comprendí, por el color del papel y
el perfume que de él se desprendía, que era carta de fémina. No creí
prudente leerla en mi calle, y seguí hasta la plaza del Progreso, donde
satisfice mi curiosidad. Ved la carta, que me sorprendió tanto por su
contenido como por su excelente escritura y ortografía, mejor que las
que gastan las mujeres bonitas\ldots{} y aun las feas.

«Caballero: Reciba usted la entusiasta felicitación de una señora
desconocida para usted\ldots{} Sentime ¡ay!, inundada de alegría cuando
supe que había castigado al infame y presumido Alberique, y mi júbilo
habría sido completo si hubiera usted dirigido su puntería al costado
izquierdo en vez del derecho, para que quedase partido aquel corazón
donde jamás anidó un sentimiento noble\ldots{} He sabido con
satisfacción que se agrava la herida de ese bigardo insolente. Lo
celebro con toda el alma. Yo soy así, implacable con los que me han
ofendido. Sé querer; no sé perdonar.

»En usted veo al hombre honrado que, cuando el caso llega, sabe proceder
con vigor y arranque, comprometiendo su vida. Mis plácemes y vítores
entusiastas al héroe. ¡Arriba los hombres de ánimo grande y corta
estatura!\ldots{} Cuando me han enterado de que el héroe es chiquitín de
talla, he sentido por usted admiración más viva. Séame lícito decir que
de niña jugué con muñecas más tiempo del que mi crecimiento permitía;
que de mujer me agradan todas las variedades de muñecos. Entre lo
pequeño y lo grande hay una escala de gratas sensaciones. Ya sabe usted
que \emph{per troppo variar Natura è bella}.

»Y no digo más por hoy. Deseo conocerle, mas no es ocasión. La ocasión
llegará\ldots{} En tanto, valiente caballero, admita los sinceros
plácemes de su amiga---\emph{Graziella}.»

Leí por tres o cuatro veces la carta, y ni con veinte lecturas habría
salido de mi confusión. Por la gramática no parecía carta de mujer.
¿Sería obra de algún amigo maleante? No\ldots{} La corrección gramatical
y la ortografía revelaban quizá las manos y pensamiento de mujer
neurótica, de superficial cultura. No desconocía yo la suma
extravagancia mezclada con el sumo donaire que constituyen el ser de
algunas almas del reino femenino, entendimientos desequilibrados que
fluctúan entre la sutileza del ingenio y los desvaríos de una razón
desmandada. Por su nombre y la cita italiana, la tal declarábase
compatriota del Dante. Nueva confusión mía mezclada de ardiente
curiosidad. ¿Por qué me dejaba, como quien dice, a media miel, revelando
su nombre y guardándose la dirección de su casa? ¡Pues de saberlo, no
iría yo poco contento a darle las gracias y rendirme a su fineza y
bondad!\ldots{} Rompí la carta en los pedacitos más chicos que pude
obtener, cuidando mucho de que alguno de ellos no se me quedase pegado a
la ropa, porque\ldots{}

Ya lo comprenderéis. Cabeza era muy celosa, y además mujer de grandísimo
talento. Por algo se llamaba Cabeza. No ignoraba mis aficiones al bello
sexo. Mi fama de galanteador afortunado le quitaba el sueño, y a mí me
ocasionó sofoquinas. En sus ataques agudos de celera, mi dama se
levantaba de puntillas, a media noche, para registrar mi ropa, buscando
alguna carta que su encendida imaginación sospechaba y temía. Y cuando
entraba yo en casa de dar un paseíto o evacuar alguna diligencia
mercantil, me olía las solapas, la corbata, el cuello, buscando algún
aroma que delatase mi supuesta infidelidad. La tarde de marras, al
llegar a la tienda después de rotos y aventados los pedacitos de la
carta de Graziella, me asaltó el temor de que el papelejo hubiese dejado
en mis dedos algún resto de su intensa fragancia. Subí corriendo a
lavarme las manos, mas ni aun con esto estuve tranquilo, ni vencer pude
el terror que me causaban los ojos inquisitivos de Cabeza y el venteo de
sus narices.

Advertí en los siguientes días a Cabeza más pensativa y fisgona que
nunca lo estuvo. Parecíame que su mirada, al fijarse en mis ojos, los
atravesaba para sorprender los pensamientos míos replegados dentro del
cerebro. Y en este no habría encontrado más que una infidelidad
puramente mental. Yo pensaba en la italiana. Su imagen revoloteaba
dentro de mi caletre como un insecto alado que cambiara de luz y colores
a cada instante. Por las noches, mi cara mitad me tenía prisionero en
casa, no permitiéndome ni quince minutos de expansión en el café
Oriental o en el de las Columnas, donde yo encontraba los amigos de mi
mayor aprecio. Vedme, pues, forzado a soportar la insípida tertulia
casera, formada por dos viejas regañonas, que se dormían cuando no
jugaban a la brisca, y de tres o cuatro sujetos soporíferos, entre ellos
un primo de Rojo Arias, que no hacía más que hablar pestes de Sagasta y
de los amigos de este, Abascal, Muñiz, don Zoilo Pérez, y un inspector
de arbitrios municipales, que proponía como única solución política
\emph{la traída de Espartero}.

El Ministerio Malcampo-Candau seguía pasando el rato con un enredoso
debate parlamentario sobre \emph{La Internacional}. Pero el interés
político no estaba en el Congreso, sino fuera de él, en los
conciliábulos y recíprocas embajadas de los dos feroces bandos que se
disputaban la primacía. Rompieron en terrible pelea zorrillescos y
sagastorros. Cada uno de los jefes de estas dos revoltosas taifas dio al
país su manifiesto. Leílos yo, y la verdad, no encontré gran diferencia
entre una y otra soflama. No era obra de romanos concordarlos y hacer de
los dos uno solo, que fuera cimiento en que fundar honrosas y duraderas
paces\ldots{} Los padres de las criaturas, que parecían mellizas,
Zorrilla y Sagasta, se avinieron a nombrar un Jurado o comisión de
arbitraje que examinara los dos manifiestos, y desarmándolos y
volviéndolos a armar en un solo cuerpo de doctrina y conducta, creara el
progresismo único y de una sola pieza, amplio terreno dogmático en que
pudieran vivir y comer todos los caballeros de la orden setembrina. ¡Qué
cosa más sencilla, ¡vive Dios!, y qué facilísima dificultad!

Apoderados de don Práxedes fueron Calatrava, el Marqués de Perales y don
Cipriano Montesinos; de Zorrilla, Fernández de los Ríos y Moya (don
Javier). A estos, por si eran pocos a discutir, se unieron luego otros
cuantos, que no me tomo el trabajo de citar, pues para lo que hicieron
vale más dejarlos recostaditos en el almohadón del olvido\ldots{}
Conque, manos a la obra, caballeros. Un día se reunían aquí, otro allá,
y vengan consultas, vengan ponencias, vengan\ldots{} Y no sigo, pues me
urge decir que cuando comenzaban los finos dedos de los señores jurados
a tejer aquella tela de \emph{Pentecostés} (como decía un General de la
época queriendo decir \emph{Penélope),} recibí segunda carta de la
italiana, más perfumada y más pequeña que la primera. Diómela la misma
criadita en el mismo sitio, y yo, poseído de zozobra, escapé a leerla lo
más lejos posible, y no pareciéndome bastante segura la distancia de la
plaza del Progreso, fui a dar con mi cuerpo y mi epístola
olorosa\ldots{} más abajo de Antón Martín.

¡Oh, \emph{Tito}, afortunado mortal! ¡La incógnita dama te indicaba
calle y número\ldots{} y hora para recibirte! Aventura tan bonita y
novelesca no se presentó jamás a ningún nacido. Esto pensaba yo cuando
me acercaba, tímido y dudoso amante, a la gruta en que la diosa se
ocultaba. La misma duda aumentaba el encanto de amor. ¿Sería Graziella
una hermosa ninfa, o un culebrón espantable? Pronto había de verlo.

\hypertarget{ix}{%
\chapter{IX}\label{ix}}

Ni culebrón repugnante ni hermosura radiosa. La llamada Graziella,
italiana o española, debiera ser clasificada en el tipo vulgar de la
escala femenina, si no le dieran valor estético las llamaradas de sus
ojuelos negros, su graciosa movilidad de ardilla, y el libre chorro de
su lenguaje atrevido y pintoresco\ldots{} En mi primera visita, que hubo
de ser corta, como simple acto informativo, de puro reconocimiento, no
pude adquirir la identificación completa de mi nueva conquista, nombre,
familia, lugar de nacimiento. Diome en la nariz que el nombre de
Graziella era postizo, la nacionalidad dudosa, la mujer un misterio, una
cifra obscura de interpretación imposible. La gruta de tan singular
ninfa estaba en barrio muy distante del mío, allá por Monteleón o
Maravillas. El interior era reducido y pulcro: pocas y bien arregladas
estancias, gabinete coquetón y alcoba rosada. Sorprendiome el adorno de
paredes, donde descollaban panderetas pintadas entre láminas de Santos y
Vírgenes de distintas advocaciones, Pilar, Desamparados, Sagrario y
Paloma. En peana y entre flores vi a San Antonio, el frailecito amable,
indulgente patrono de las enamoradas. En la heteróclita casa vi a la
mozuela que me llevara las cartitas, y mujerona que se escurría por los
pasillos sin otro rumor que el de toses y carraspera. Era un anchuroso
bulto de vieja, o una elefanta en dos pies cubierta de refajos\ldots{}

En nuestra conversación inicial, la enigmática hembra puso algo de
sordina en su expresivo parlar de amores y en su liviano propósito de
entenderse conmigo. «Ya ves, Tito---me dijo con donaire,---que la
franqueza es mi Norte y mi Sur, mi Este y mi \emph{Aquel}. Si te dijera
que soy honrada, te echarías a reír. Tráeme una honradez que me dé de
comer, y tendrás que santiguarte al entrar en mi casa. Yo he admirado en
ti al caballero valiente, vengador de la virtud ultrajada. Eres chico y
grande\ldots{} Me gustaste por tu hazaña, y más me gustas ahora que te
conozco\ldots{} Pero entendámonos. Tú eres pobre. A mí no me hace
maldita gracia la pobreza\ldots{} No soy hermosa; pero no soy
pava\ldots{} Soy de esas feas que dan la desazón y revuelven medio
mundo\ldots{} Como no quiero perjudicarte, lo primero que te digo es que
no dejes a tu tendera lozana y rica\ldots{} La engañas un tantico, y
nada más. Yo no engaño\ldots{} Vivo en libertad\ldots{} protegida por la
Corte Celestial\ldots{} Entre los santos que cuelgan de estas paredes,
hay uno, que no se ve, y es mi \emph{Santo Gusto}\ldots{} Por el reverso
de los santirulicos, andan mis diablillos, quiero decir, mis rencores y
malos quereres\ldots{} Has de saber que uno de mis mayores odios ha sido
ese ladrón de Alberique\ldots{} Algún día te contaré la trastada que me
hizo, y que no pagará con cien vidas.»

Tras una pausa grave, siguió así: «Ya me irás conociendo; soy voluble,
caprichosa y un demonio de travesura\ldots{} Tengo una virtud, digo,
muchas virtudes\ldots{} Vas a saberlas: 1.ª, que el que me la hace me la
paga; 2.ª, que todo lo que digan de mí me sale por una friolera; 3.ª,
que soy larga en tomar dinero, y más larga todavía para darlo al que lo
necesita\ldots{} Si tú hicieras comedias y quisieras sacarme en una,
deberías titularla: \emph{La deshonrada más honrada}.»

Volví a mi casa un poco aturdido. Pensando en mi aventura, hice
propósito de proceder con cautela. No me convenía dejar lo cierto por lo
dudoso, ni sacrificar lo positivo a lo de puro pasatiempo y fantasía.
Tuve la suerte de que mi señora Cabeza no estuviese aquel día tocada de
celera, y sacudiéndome el perfume, salí pronto de mi cuidado. Al día
siguiente tuve ocupaciones en casa; pero al otro, que fue viernes, me
entendí con un amigo progresista radical para que me escribiese
llamándome a una entrevista con Zorrilla, que quería encargarme un
trabajo de pluma urgentísimo. Con este sutil engaño, en que fácilmente
cayó mi Cabeza (que si en amores era la misma suspicacia, en política
tenía tragaderas para cuanto se le quisiera echar), me fui a la gruta,
donde pasé toda la tarde con la endiablada ninfa, recreándome con su
grácil salero, y disfrutando en su compañía variedad de esparcimientos,
algunos, créanmelo, del orden espiritual\ldots{}

Del ingenio y del libertinaje de la diabólica italiana (me aseguró aquel
día que era hija de un cardenal) saqué no pocas enseñanzas para mi
estudio y conocimiento del mundo. Ratos pasé de alegría, ratos de
confusión y perplejidad. Si mi huéspeda empezó la tarde con dulce
temple, luego le sobrevino de súbito la racha de las diabluras, y me
fastidió de medio a medio al acercarse la hora de separarnos. «Tito,
\emph{mio caro},---me dijo cuando me disponía para la retirada.---Me ha
picado la tarántula, y esta noche quiero darte un bromazo\ldots{} y otro
a tu doña Cabeza.»

---¿Qué dices, Graziella?

---No pongas esa cara de tonto. Esta noche no vas a tu casa. Yo lo he
determinado así. ¿No me has dicho que soy una ninfa hechicera? Pues
prepárate a pasar la noche en mi gruta.

---Graziella, por San Antonio bendito, que te custodia, no gastes bromas
trágicas.

---Aquí estaremos los dos divirtiéndonos con la idea de lo que ha de
rabiar doña Cabeza. ¿No me has dicho que es celosa y que te huele la
ropa y te registra los bolsillos? Pues yo detesto a las personas
celosas, y me divierto aplicándoles al corazón un hierro encendido al
rojo. Yo soy así.

Protesté indignado\ldots{} Pero Graziella, con infernal risa, me dijo
que me había escondido botas, ropa y sombrero, y que estaba cautivo, sin
que por ningún medio pudiera evitarlo. Omito, por no fatigar a mis
lectores, los gritos que proferí, ahora coléricos, ahora suplicantes;
las vueltas que di por toda la casa, descalzo y en mangas de camisa,
buscando mi ropa; los extremos de ira y desesperación; los ruegos y
amenazas; el último recurso de mi desesperación, que fue lanzarme
escaleras abajo, escaleras arriba, llamando al portero, a los vecinos
para que me sacaran de aquel aprieto. ¿Dónde estaba la policía, dónde el
alcalde de barrio, dónde el sereno que ampararan a un honrado cliente de
la nefanda Antarés, diosa del quinto Infierno?

Nada me valió. Con risueña frescura Graziella contemplaba mi
sufrimiento; la muchacha reía, y la vieja elefanta deforme y
carraspienta se mofaba también de mí.

Dieron las ocho, las nueve, y cuando sonaron las diez me rendí\ldots{}
«Ya no te atreverías a ir a tu casa si yo te soltara---me dijo la
hechicera,---porque Cabeza te sacaría los ojos. Vale más que esta noche
prepares aquí tranquilamente el lindo embuste con que podrás aplacarla
mañana. ¿No le diste el pego con una fingida carta de Zorrilla,
llamándote para escribir con él un papelón político? Pues date prisa:
escríbelo aquí. Yo te ayudaré.» Esta donosa superchería me consoló un
tanto. Audaz era la idea; pero no despreciable para soslayar el peligro
y gravedad de mi situación. En esto pusieron la mesa para cenar. Cuatro
cubiertos vi: sin duda comíamos juntos las criadas, Graziella y yo. ¡Oh,
burlesca democracia y confusión de clases! La cena fue substanciosa:
estofado y frituras, hojaldres y polvorones, todo ello ingerido con el
estímulo de un vino blanco, excitante y traicionero, que a los pocos
tragos me puso perdido de la cabeza, alterándome la justa percepción de
las cosas. Advertí que Graziella tragaba como si no hubiera comido en
tres días, y que la vieja elefanta, sin dar paz a los dientes, rezongaba
conceptos ininteligibles. El recuerdo más claro de aquella noche fue
que, después de cenar, me cogieron en vilo las tres mujeres, y con gran
chacota y fiesta me arrojaron sobre la cama como un fardo insensible.

¡Noche de fiebre, de un girar vertiginoso en torno de mi propio
pensamiento! La primera sensación de la mañana siguiente fue que una de
las tres, no sé cuál, me llevó en brazos a la salita que comunicaba con
el gabinete. Yo me sentía más chiquitín; no pesaba ni abultaba más que
un nene de cinco años. Desgreñada, pálida y pitañosa, Graziella me
sirvió café con leche y tostadas. Me entoné con el brebaje
caliente\ldots{} Junto a la butaca donde mi menguada persona yacía,
pusieron un velador con papel en cuartillas, tintero y pluma, y la ninfa
me dijo: «Aquí tienes los avíos de escribir. Tómalo con calma.
Fácilmente podrás enjaretar el \emph{turri-burri}, que supones dictado
por ese don Manuel, para dársela con queso a tu cara mitad. ¡Pobre
Cabeza\ldots{} destornillada! Dará gusto verla con el adorno de la
vistosa cornamenta que le has puesto. Siento que mi peinadora no sea la
suya. Yo le diría: «Cuando arregle a esa señora, lleve serrucho en vez
de peine. ¡Ay, Tito mío chiquitín!\ldots{} Eres lindo y perverso: así me
gustas.»

En esto, entró la matrona corpulenta trayéndome de la calle todos los
periódicos del día y de la noche anterior: \emph{Iberia},
\emph{Correspondencia}, \emph{Novedades}, \emph{Eco de España},
\emph{Tiempo}, \emph{Pensamiento Español}, \emph{Universal},
\emph{Discusión} y alguno más. «Ahí tienes hilaza---me dijo
Graziella.---Ya puedes hilar y tejer cuanto quieras.» Viendo salir a la
vieja pregunté su nombre, condición y empleo que en la casa tenía, a lo
que respondió mi tirana: «Es la tía \emph{Mariclío}, comercianta de
antigüedades y papeles viejos, que ha venido a menos. Yo le doy
albergue, y me hace servicios menudos y recados. Tú la conoces: no te
hagas de nuevas\ldots{} No se ha podido averiguar la edad que tiene. Hay
quien asegura que nació un poquito después del principio del mundo. No
siempre está en el mal pergenio en que ahora la ves. Si en tales o
cuales días viene a menos, en otros sube a más, y se pone unas botas al
modo de borceguíes de cuero carmesí, con tacones dorados, y de
gordiflona y ordinaria se te vuelve esbelta y elegante\ldots{} Sabe más
de lo que parece, y cuando escribe lo hace con primor. Llámala para que
te ayude, y te dará buena cuenta de lo mucho que ha visto, y te
alumbrará las entendederas para que sepas ver lo que ahora pasa.»

Oí estas advertencias de la diablesa como si sus palabras fueran rum rum
de mis propios oídos. Yo no estaba en mis cabales. Sospeché que aún me
duraba el efecto del vinazo ardiente que aquellas hechiceras, brujas o
lo que fuesen, me dieron en la cena de la noche anterior. Fuese
Graziella, reclamada por su peinadora, y yo me puse a leer
periódicos\ldots{} Largo tiempo, a mi parecer, invertí en la lectura,
que fue irregular y nerviosa, saltando de uno en otro papel, y fijándome
en todos antes que en ninguno de ellos. ¿Qué decían? Que si el Jurado
encontraba la fórmula, que si la fórmula resbalaba cual anguila en las
manos de aquellos respetables majaderos\ldots{} De pronto vi a la vieja
sentada frente a mí. No supe cuándo ni por dónde entró. Apoyaba sus
robustos brazos en el velador, y me acariciaba con su mirada
complaciente. Sus cabellos, que antes me parecieron blancos, tenían
irisaciones y reflejos que en las ondas del rizado tan pronto eran oro
como plata. Su rostro se había tornado apacible, tirando a hermoso, y el
volumen de su cuerpo quedaba reducido a las proporciones de una mujer de
medianas carnes.

Antes de que yo le hablara, acercó sus dedos al rimero de periódicos, y
con voz que de ronca se había trocado en blanda, me dijo: «Pobre Tito,
si para sortear la furia de tu mujer engañada has de fingir un alegato
dictado por el bueno de Zorrilla, puedes empezar diciendo que los del
Jurado no acabarán de encontrar la fórmula de avenencia hasta el momento
preciso en que suenen las trompetas del Juicio final. De estos hombres
que ponen en la mediocridad el límite más alto de sus ambiciones, nada
puede esperarse. Ya ves. Empezaron por decir que no veían gran
diferencia entre los dos manifiestos. Se les dice: `A ver, a ver.
Reducid las dos monsergas a una sola', y empiezan a quitar o poner esta
o la otra palabra, y aquí doy un toque, allá otro toque.»

---Ya, ya\ldots{} Y luego vienen las consultas\ldots{} «¿Qué les
parece?\ldots» «Nos parece---responden de allá---que ahora debe
atenuarse aquel verbo, y poner aquí un adjetivo de más color.»

---«Está bien,» dicen los otros\ldots---prosiguió Mariclío
zumbona.---«Pero antes conviene discutir la cuestión previa, para fijar
la forma y manera de proceder en este negocio.» Y en la cuestión previa
se pasan días y días, noches y noches.

---Llegan al artículo de \emph{La Internacional}\ldots{} ¡Ah!, es
indispensable poner algún freno a ese monstruo disolvente.

---Sí, sí\ldots{} Pero ¡ah!, no toquemos a los derechos individuales,
inalienables\ldots{} Sistema preventivo\ldots{} No, no,
represivo\ldots{} Pues hagamos un bello maridaje de lo represivo y de lo
preventivo\ldots{}

---Viene la cuestión de Cuba. ¡Ah!, ante todo la \emph{integridad del
territorio}\ldots{} Cuestión elemental, cuestión previa.

---Pero ¡ah!, las reformas se imponen\ldots{} No puede España permanecer
divorciada de la opinión universal.

---Sí, sí\ldots{} reformas, aire nuevo\ldots{} Pero ¡ah!, alentemos la
abnegación y el patriotismo de los Voluntarios de Cuba, salvaguardia del
honor de España, y de la integridad, etc.

---Por encima de todo, los derechos ilegislables, por ser naturales,
inherentes a la personalidad humana\ldots{} Pero ¡ah!, medios ha de
tener siempre el Gobierno para castigar, sin salirse de la Constitución,
todo acto político de carácter inmoral o delictivo\ldots{}

---Otra cuestión a debatir: \emph{La Internacional}, ¿es moral o
inmoral? Que sí, que no\ldots{} Por fin, tras largas disputas enredosas,
declaraban que entre el programa de Sagasta y el de Zorrilla no había un
comino de diferencia\ldots{} Pero ¡ah!\ldots{}

Rompimos en franca risa los dos, mirándonos sin pestañear. Y ella fue la
primera que convirtió las notas picantes de su risa en palabras donosas.

---¡Ay, Tito, no sé cómo me río hablando de estas cosas que son, ¡vive
Dios!, tan tristes! ¡Que un país, donde hay sin fin de hombres que
discurren con juicio, y sienten en sí mismos y en conjunto el malestar
hondo de la Patria; que una Nación europea y cristiana esté en manos de
esta cuadrilla de politicajos por oficio y rutinas abogaciles, hombres
de menguada ambición, mil veces más dañinos que los ambiciosos de alto
vuelo! Si algo pudiera contra ellos, los barrería como barro esta sala,
regándolos antes para no levantar polvo, y mezclados con serrín los
metería en su más adecuado sumidero, que es el eterno olvido.

---Pues anda, anda\ldots{} En este periódico veo que después de inútiles
conferencias, alambicando palabras, y evacuando consultas\ldots{}
¡ridículas diplomacias!, salimos con que todos se sacrifican\ldots{} No
hay avenencia\ldots{} ¡Ah!, yo me sacrifico\ldots{} No quiero ser
obstáculo\ldots{} Y salta otro por allí sacrificándose\ldots{}

---Sacrifiquémonos. Eso dicen cuando se ven cogidos en la última maraña
de sus enredos\ldots{} Si creen que debe sustituirse en el manifiesto la
palabra \emph{pitos} por la palabra \emph{flautas}, hágase en buen hora;
pero ¡ah!, mi dignidad no me permite\ldots{}

---Y por allí salta otro diciendo que su \emph{Credo} es tal o cual
cosa, y que no puede quitar ni una tilde de su \emph{Credo}. ¡Valientes
\emph{Credos}, valientes \emph{Salves} las que rezan estos farsantes!
Riámonos de su indigna dignidad y de sus interesados sacrificios. Si no
se avienen a vivir juntos en una sola Iglesia con un solo \emph{Credo} y
un solo \emph{Gloria patri}, es porque en caso de avenencia sólo serían
ministros las cabezas más visibles\ldots; mientras que dividiéndose en
hatillos o cofradías de corto personal, irían todos entrando en el
comedero, y hasta los gatos serían ministrables. La ambición de estos
hombres raquíticos y de cortas luces se limita, como ves, a la vanidad
de ser ministros, sin otros fines que darse tono, repartir empleos, y
que la señora y los niños paseen en coche galonado. Ello les dura poco
tiempo, y salen del Gobierno en completa virginidad política. Lo más que
han hecho es \emph{estudiar} los asuntos que allí se quedan para que los
\emph{estudie} el sucesor. Esta caterva de \emph{estudiantes} debiera
ser mandada, ¡voto a Sanes!, al Limbo de las eternas vacaciones\ldots{}

Esto dijo la vieja \emph{Mariclío}, a quien diputé por persona sagaz y
de mundana picardía. Salió para entrar de nuevo, y durante su ausencia
me visitó Graziella en un intermedio de sus abluciones. Aún le faltaban
toques de afeite y compostura, y el pelo lo traía suelto\ldots{} La
peinadora, que podía pasar por hombre público, según lo que charlaba y
peroraba, lucía en el cercano gabinete la soltura de su lengua. La tía
\emph{Mariclío} volvió a mí con un libro viejo, que abrió sobre el
velador sentándose en postura de escribir. «Aquí voy yo anotando\ldots{}
Mira, mira---me dijo risueña, escribiendo con un estilete que a cada
momento se llevaba a la boca para mojarlo con su saliva.---Obligada
estoy por mi Destino a mencionar todo lo que hace esta gentezuela; pero
escribo sus nombres con una saliva especial que me dio mi padre para
estos casos.»

---¿Qué casos?

---Esta saliva tiene una virtud preciosa. Lo que con ella escribo se lee
hoy, se lee mañana; pero luego se borra y no llega a la posteridad.

\hypertarget{x}{%
\chapter{X}\label{x}}

Ignoro cómo tracé, con rápido mover de pluma, lo que suponía dictado por
don Manuel Ruiz Zorrilla; pero hablando en conciencia, no puedo afirmar
sino que me lo dictaron los mismos demonios. En mi escrito, que no tiene
principio ni fin, ensalcé el radicalismo puro, única receta para sacar a
esta Nación de su atonía y somnolencia mortíferas. Si don Manuel se
sentía con redaños para obra tan grande, bastárale plantarse en firme, y
dar grandes voces diciendo: «Cortes y Rey, caterva de políticos
intrigantes y ociosos: Convocad a la Nación con verdad y honradez, y
ella os dará un criterio de gobierno. ¿No queréis hacerlo? ¿Teméis que
os manden a todos al corral? Pues aquí estoy yo para esa
hombrada\ldots{} ¿Que yo tampoco me atrevo? Pues al corral con
vosotros\ldots{} Venga un hombre, un tiazo que hable poco y sepa sacar
la voluntad nacional de las teorías pedantescas a la realidad
viva\ldots{} O perecemos como nación, o hay que rehacerla desde el
cimiento. Justicia, Ejército, Administración, Trabajo, Igualdad ante la
ley, Libertad de la conciencia. Que todo sea nuevo, de flamante material
y hechura\ldots{} Que todo sea tan sólidamente construido que no podamos
volver atrás, y que si cuatro carcundas o cinco sacristanes intentasen
remover las viejas ruinas, sean hechos polvo, y el polvo aventado por
los espacios infinitos\ldots» Estos y otros disparates escribí con mano
febril, dejándome arrastrar de mi ardiente imaginación, y de mi odio a
las repugnantes rutinas y ficciones que forman el entramado político y
social de nuestra existencia.

Tres, cinco, seis pliegos emborroné, cual máquina que obedece a un
impulso extraño y superior. En mi delirio llegué a trazar planes y
programas de orden jurídico, financiero, social: Presupuestos,
Organización de tribunales, Mecanismo electoral, Espectáculos públicos,
Relaciones entre el Municipio, la Provincia y el Estado; Ley de Servicio
militar, Catastro, Minas, Código de Comercio, y mil y mil disposiciones
que en surtidor inagotable salían de mi cabeza\ldots{} Y en los pasajes
más afluentes de mi inspiración metía paréntesis imperativos: «Don
Manuel, ánimo; don Manuel, atrévase; don Manuel, ahora o nunca\ldots» La
presencia de Graziella, ya peinadita y acicalada, contuvo un tanto la
velocidad de mi rotación cerebral. Leyó algo de lo que yo escribía; lo
alabó sin entenderlo, y yo le dije: «Espérate un poco, ninfa hechicera.
Déjame acabar. Aún me falta lo de Culto y Clero, Instrucción
Pública\ldots; ahí es nada\ldots{} Receta contra frailes y
monjas\ldots{}

---Con toda esa monserga que llevas a tu casa, doña Cabeza quedará
desenojada. El toque está en que sortees la primera embestida de la
fiera celosa\ldots{}

---Déjame acabar. Pongo la última razón: «Don Manuel de mi alma: o sois
el salvador de España, o quedaréis perdido en el montón gregario, donde
se os pondrá un cencerro y pastaréis tranquilamente en el
presupuesto\ldots»

Concluido mi trabajo, me sentí satisfecho, y hasta cierto punto conforme
con la esclavitud que la hechicera me imponía. Ya me inquietaban menos
los temores y el deseo de volver a mi casa. Hallábame un si es no es
alelado, como si obraran en mi voluntad los efectos de un licor o
esencia de extraordinaria virtud aplanante. A ratos dormía, y en mi
sueño me asaltaban visiones placenteras, me arrullaron lejanos cantos
eróticos de ninfas, entre cuyas voces distinguí la de Graziella con
agudas notas humorísticas\ldots{} Desperté, y halleme solo en la casa,
la puerta cerrada con llave\ldots{} Entraron luego la italiana y su
criadita, que me traían dulces, cigarros y más botellas de aquel
delicioso y somnífero vino que me apagaba la voluntad, y me encendía la
imaginación con ardores resplandecientes\ldots{} No pedí a mis
carceleras que me devolviesen la libertad. Dulce pereza me familiarizaba
con la atmósfera tibia y perfumada de aquel presidio. Pasó todo el día
sin que me aliviara de la holganza, y vi llegar la noche sin que me
asustase la idea de pasarla blandamente en la serena gruta.

En mi segunda noche, no vi a \emph{Mariclío}. Pregunté por ella, y
dijéronme que había ido a la Academia de la Historia (calle de León),
donde cobra la menguada pensioncilla de que vive. En aquella casa
venerable, suele entretenerse ayudando al conserje en el barrido de la
biblioteca y en quitar el polvo a los estantes. Si le anochece en esta
faena, suele quedarse a dormir en la portería, y por la mañana le
cepilla la ropa al gran don Marcelino, por quien siente ardoroso cariño
maternal\ldots{} Prosigo contando que yo dormitaba, y Graziella, junto a
mí, escribía cartas en el velador. Y a cada renglón que trazaba se
interrumpía para celebrar con risas lo que había puesto en el papel.

«Estás en ascuas---me dijo---viéndome escribir y reír juntamente. Es que
cuando estoy aburrida, me entretengo escribiendo anónimos\ldots{}
Verás\ldots{} escribo a las damas católicas y alfonsinas, que andan en
intriga contra el pobre don Amadeo y su mujer\ldots{} En mis cartas
figuro que soy también católica, y que para traer al Alfonsito ofrezco
todo el \emph{parné} que tengo\ldots{} En esta he firmado la
\emph{Marquesa del Congosto}, y en esta otra la \emph{Condesa de Pata
del Cid}\ldots{} No creas, algunas las pongo con tan lindo artificio que
no parecen de burlas. Otra voy a poner diciendo que a mis tes viene
todita la crema de Loeches. Me divierto la mar. Les digo que cuenten
conmigo para todo, y que vino a verme Zorrilla para ofrecerme la plaza
de Camarera de doña María Victoria, y yo le respondí: «Para ese cargo
pongo a su disposición cualquiera de mis criadas\ldots» Voy a escribir
otra en que me planto título de Duquesa, y digo que en mi palacio se han
reunido ayer el Obispo de la diócesis y el Clero castrense, Sor
Patrocinio, el fiel de fechos y dos generales invictos, manifestando
todos a una que están decididos a pronunciarse por Alfonso y a dar el
grito un día de estos, con la fresca\ldots»

Leyendo y comentando los disparates con que amenizaba sus ratos de
ociosidad, me entretuvo la diablesa toda la prima noche\ldots{} Me
maravillaba que, en largas horas de mi permanencia en la gruta, no fuera
esta visitada de hombres\ldots{} A mis dudas contestó, poniéndose un
poquito seria, lo que literalmente copio: «Aquí no vienen hombres,
Tito\ldots{} Porque has entrado tú, no vayas a creer\ldots{} que esta
casa es un tranvía para el Infierno\ldots{} Infierno no digamos\ldots{}
En fin, lo que sea. Yo vivo amparada por un señor, por un
caballero\ldots, te lo diré claro, por un sacerdote que podría ser mi
padre\ldots, y por su comportamiento conmigo lo es. Créelo, Tito, aunque
lo oigas de estos labios míos, que te parecerán mentirosos; puedes
creerme que persona como esa no existe en el mundo, y que si entre
tantas virtudes no tuviera la flaqueza de quererme, sería un verdadero
santo, mejor que muchos que se han encaramado en los altares. El nombre
no te lo diré; lo venero y guardo por respeto\ldots{} Es bueno para
todos; es humano, caritativo, y no se asusta de nada. En su oficio de
cuidar de las almas cumple como el primero\ldots{} Reprende todos los
vicios; pero hay uno en que a mi buen cura le falta valor para
incomodarse\ldots, y abre la mano\ldots{} Lo que él me ha dicho mil
veces: «Por esta debilidad, que es imperio de la carne, no se va al
Infierno. Se va por la crueldad, por el no socorrer a nuestros
semejantes cuando están necesitados, por levantar falsos testimonios,
por la usura, la ira y la soberbia.»

---Me dejas atónito, Graziella. ¿Y cómo has encontrado ese mirlo blanco,
ese espejo de los caballeros, más digno cuanto más tonsurado?

---¡Ay, no fue poca mi suerte al dar con él! Perdida andaba yo, cuando
una casualidad me deparó su conocimiento. Hará de esto diez años. Me
recogió y amparó\ldots{} Prendose de mí; le cautivaba el fenómeno de
que, siendo yo lo que era tuviese el poquito de ilustración que se me
pegó en Italia. Él también estuvo en Italia. Era familiar de un cardenal
español, y fue con él al cónclave en que eligieron Papa a Pío IX. Cuenta
cosas muy interesantes del cónclave y de fuera de él. En Roma perdió la
fe\ldots{} Ya sabes: \emph{Roma vedutta, fede perdutta}.

---¿Y no quieres decirme\ldots?

---No, no, Tito; el nombre no me lo preguntes\ldots{} Es persona muy
conocida, muy apreciada en Madrid\ldots{} Puedo alabarle, puedo contarte
lo bueno que es\ldots; pero la boca se me cierra al querer pronunciar su
nombre. Si algún día lo sabes, te lo callas, guárdate de decir que es mi
protector, y que viene a verme una o dos veces por semana\ldots{} Antes
venía más a menudo; ya no puede\ldots{} Está viejo, achacoso\ldots; las
piernas le flaquean\ldots{} Ya no dice la misa todos los días\ldots{}
Sale poco de su casa\ldots{} Y ninguna falta le hace trabajar en el
oficio de cura, porque es rico. Tiene fincas allá por Toledo, y dinero
en el Banco.

A propósito de la riqueza del santo varón, dije a la ninfa que bien
podría contar con una parte en la herencia, si no había sobrinos o amas
con mayor derecho, y Graziella me aseguró que no tenía pizca de ambición
en lo tocante a intereses. De aquí derivó la conversación hacia el
terreno moral, y no pude ocultar a la moza mi extrañeza de que no
guardara fidelidad a un protector tan generoso y bueno. Delicada era la
cuestión. Graziella supo sortearla con sutil razonamiento y
gracia\ldots{} Harto sabía el caballero sacerdote que su protegida era
de la piel del diablo, alocada fantasía y temperamento inflamable.
Tolerante y filósofo, no había de exigir que\ldots{} Sin manifestarlo
claramente, dio a entender a su amiga que podría tomarse una libertad
relativa\ldots, evitando todo escándalo. Mil veces le había dicho que no
era pecado\ldots{} sino en tanto cuanto\ldots{} Ni Graziella encontraba
la fórmula racional para cohonestar sus pasatiempos licenciosos, ni yo
podía dar mi conformidad a tan absurda ética.

Con nuevos pormenores adornó la ninfa su peregrino cuento. La razón de
su odio al farsante Alberique era que este malvado, furioso porque ella
desatendió sus requebrajos, cometió la villanía de abochornar
públicamente al cura, una mañana, a la salida de la parroquia de San
Marcos. De aquí provino el entusiasmo y alegría de ella cuando supo que
yo le había metido una bala en el cuerpo, y el felicitarme y hacerme por
escrito amorosas carantoñas, llamándome valiente caballero y un poquito
héroe\ldots{} Otro detalle: el buen presbítero era muy aficionado a los
estudios históricos; poseía copiosa biblioteca, y mataba sus largos
ocios escribiendo una obra de mucha miga, titulada: \emph{Historia del
Clero Mozárabe en la diócesis de Toledo}. «Y para que te enteres,
Tito---añadió Graziella poniendo toda el alma en sus ojos,---aquellos
benditos clérigos no eran solteros, y todos tenían sus lindas
barraganas. De su gran obra, ya lleva el señor publicados tres
tomos\ldots; la venerable \emph{Mariclío} que has visto en casa, sabe de
estas cosas más que yo. Ella te contará\ldots»

Esto y algo más que hablamos completó en mi mente la figura extraña de
la hechicera, que en su gruta me alojó tres días. Al tercero salí, más
que por impulso mío, por un suave empujón de ella, que así me dijo: «Ya
es hora, Tito, de que vuelvas a tu casa. Anda; muéstrale a tu consorte
el programa pistonudo que te ha dictado don Manuel. Tres días y dos
noches te ha tenido en su casa sin dejarte salir\ldots{} Entiendo yo que
al verte llegar, Cabeza te recibirá de uñas, bramando improperios y
rugiendo amenazas. Pero en cuanto se entere de lo que rezan esos
papeles, se irá trocando de frenética en razonable, y de dura en tierna.
Si así no fuere, aplícale unos cuantos bastonazos en las partes blandas,
con que la sacudas bien el polvo sin hacerle daño. Verás qué pronto se
le aplaca el genio\ldots{} Anda, hijo mío; no lo pienses más. Ánimo, y a
la \emph{Cabeza}.»

Salí de la gruta con flojera de piernas y desmayo de mi corazón, y en
todo el largo trayecto desde aquel lejano barrio al mío, fui pensando en
la catástrofe que esperaba y temía. Al dar en mi calle los primeros
pasos me detuve a pensar si no me convendría más volverme atrás y
emprender definitiva y veloz carrera en sentido contrario. La imagen de
María de la Cabeza Ventosa de San José se me ofrecía en el pensamiento
como la de una espantable hidra\ldots{} Por fin, anteponiendo a todo mi
dignidad de varón, avancé hacia el peligro y me metí en la
tienda\ldots{} Las caras de los dependientes me dieron la impresión de
estupor, de miedo y lástima\ldots{} Yo les dije: «¿Qué hay de nuevo por
aquí\ldots?» Y como no me contestaran, quedándose ante mí cual estatuas
de hielo entre percales y lanillas, les dije otra vez: «¿Qué hay por
aquí\ldots? ¿Y de ventas qué tal?» El mayor de ellos respondió: «Así,
así\ldots{} ¿Y a usted cómo le ha ido por esos mundos?»

---¿Qué mundos ni qué carneros?\ldots{} ¿Cabeza no está?

---Creo que ha salido. Suba usted y le dirán\ldots{}

Subí medio muerto de sobresalto. Salió a recibirme Jesusa, la criada
vieja que a Cabeza servía desde tiempo inmemorial. No esperó a escuchar
el metal de mi cortada voz para decirme: «Cabeza no está. Se ha ido a
casa de su tía doña Florencia.»

---¿Pero no vendrá pronto?

---No\ldots{} Pase usted por aquí. Tengo que darle un recado.

Llevome a la que había sido mi habitación, y con seca voz me dijo
señalando mi baúl: «Aquí tiene usted su ropa\ldots, lo mismo la nueva
que los pingajos que trajo acá\ldots{} Puede usted retirarse. Cabeza me
ha dicho que le diga\ldots, vamos\ldots, que no volverá a su
casa\ldots{} hasta que usted no se haya ido, llevándose su ropa.»

---¡Jesús, Jesusa!---exclamé yo.---Eso no puede ser\ldots{} Necesito
explicar a Cabeza\ldots{} ¿Ve usted estos papeles que traigo? Pues aquí
está la explicación\ldots{} Don Manuel Ruiz Zorrilla\ldots{} ya sabe
Cabeza que\ldots{}

---Cabeza no sabe nada de eso. Por don Ignacio Rojo Arias mandó recado
al señor Ruiz Zorrilla preguntándole\ldots{} Total, que ni ese señor le
llamó a usted, ni usted ha parecido por la casa de él, y que todo es
inventorio de usted. Ya se lo dije yo a Cabeza: «¡Ay, Cabeza, Cabeza;
ten cuidado con esa sabandija que has metido en tu casa!»

---Pero yo necesito explicar, dar mis razones\ldots{} Este papel\ldots{}
¡Oh! Me harán pedazos antes que retirarme sin que Cabeza se entere
de\ldots{} Dígame, Jesusa: ¿las personas que aquí suelen venir han
hablado desfavorablemente de mí\ldots, han supuesto que yo\ldots?

---Algo han hablado, por mi fe. «Es mucho Tito este,» decía el señor de
Bringas. Según don Roque Barcia, usted se había perdido en los
laberintos federales. Ni don Mateo Nuevo, ni don Roberto Robert, ni
ningún otro dieron razón. Y todos a una decían: «Perdido está entre
faldas\ldots» ¡Ah!, se me olvidaba\ldots{} Llegó ayer una carta\ldots{}
La firmaba una Marquesa\ldots{} A ver si me acuerdo\ldots{} \emph{La
Marquesa de Pata del Cid}\ldots{} Decía que el señor Tito se había
puesto al servicio de las damas católicas y alfonsinas, y que con ellas
pasaba el día y la noche\ldots{} Ya se vio que era broma. Pero detrás de
las bromas salen las verdades\ldots{} Conque a despejar pronto\ldots{}
Cabeza no vuelve a su casa, ya se lo he dicho, ¡caramba!, hasta que
usted no se haya perdido de vista.

---Pues, ea, me voy al otro mundo---dije avergonzado de la ultrajante
despedida.---Me llevo mis papeles\ldots{} Ella se lo pierde. A ver, a
ver, Jesusa: llame usted a un mozo de cuerda, para que me lleve el baúl.
Y diga usted a Cabeza que la perdono. Ella se lo pierde\ldots{} Ella es
la reacción; yo soy el progreso; pero \emph{el progreso
indefinido}\ldots{} No lo digo yo. Lo dice Ruiz Zorrilla en estas
páginas que han de ser inmortales\ldots{} Ea\ldots{} con Dios\ldots{}
Abur\ldots{} Conservarse. ¡Oh, qué país! Al español honrado no se le
hace justicia hasta que se muere\ldots{} Pues venga la muerte, y tras de
la muerte vendrá la justicia, vendrá la apoteosis.

Y así, empleando los tonos patéticos al emprender mi forzosa retirada,
salí de aquella casa, donde mi vida tormentosa gozó algunos días de
regularidad placentera. Mandé el baúl a la portería de mi antigua casa
de la calle de Los Leones, y me lancé a una divagación callejera, dando
libre vuelo a mi desolado pensamiento. ¿Dónde me guarecería? Felizmente
tenía cinco duros que me había echado en el bolsillo al salir para mi
aventura loca. Por una noche y un día, podría creerme potentado. En el
café de Las Columnas, me convidé a comer una tortilla y un bistec,
seguidos de café con leche\ldots{} Abreviaré mi relato, diciendo que
aquella noche me dio albergue Mateo Nuevo, mi consecuente y bondadoso
amigo, y que al siguiente día, mis pasos se fueron solos, por
inconsciente magnetismo, hacia el barrio de Maravillas, donde tenía su
encantadora gruta la diablesa causante de la soledad en que me
veía\ldots{}

Entré en la calle, y como a primera vista no reconociera la casa, fui
mirando los números, y atontado anduve de abajo arriba sin encontrar el
16 que buscaba. Aturdido pregunté a una mujer que parecía portera: «¿No
es este el 16?» Respondiome que era el 14\ldots{} «El siguiente será
16,» dije yo; y la maldita vieja, que me miraba con sorna, tomándome por
demente o borracho, pronunció estas fatídicas expresiones: «El número
que sigue es el 18. En la calle no hay 16. Lo hubo. ¿Ve usted esa valla
de madera que sigue? Pues en ese solar estuvo el 16\ldots, si usted no
manda otra cosa\ldots» No pude menos de hacer una juiciosa observación:
«Anoche debió de ser derribada la casa, porque ayer estuve yo en ella. Y
si así no es, habré yo confundido el número. Dígame, ¿no vive en este 14
una señora que llaman doña Graziella? Si no es aquí, será en el 18.»
Oído esto, la portera me dio la contestación más inconveniente y soez:
«¿Sabe lo que le digo? Que si viene usted dormido, aquí tengo yo el palo
de la escoba para despertarle\ldots{} Y váyase pronto a que le den el
amoníaco.»

En mi confusión y azoramiento al ver desaparecida o tragada por la
tierra la gruta de la maga, me retiré sin saber por dónde iba. El
incierto rumbo de mis pasos me llevó a la calle de Fuencarral; por esta
me metí en la de San Mateo, y al promedio de ella vi que hacia mí venía
una persona\ldots, un hombre, en quien creí reconocer a uno de mis
amigos más queridos. Dudé; desconfiaba de mis ojos, que en tales días
padecían quizás la dolencia de ver visiones. Avanzaba el sujeto\ldots{}
Su talla y andar, su rostro, su larga perilla rubia no podían engañarme.
Era él, era él. Cuando a mí llegó con los brazos abiertos, mis dudas se
extinguieron en este grito de alegría: ¡Estévanez\ldots{} Nicolás
Estévanez!

\hypertarget{xi}{%
\chapter{XI}\label{xi}}

Bastante más joven que él era yo, y por la edad, como por el respeto,
solía llamarle \emph{don Nicolás}. Él me devolvía la fineza llamándome
burlonamente \emph{don Tito}. Abrazados todavía me dijo que acababa de
llegar de Cuba, por vía muy larga y tortuosa\ldots{} ¡Qué viaje, qué
fatigas! Aún llevaba el pantalón blanco de hilo que usan los militares
antillanos. Con él salió de la Habana, con él andaba en Madrid por no
tener otro. ¡Y estábamos en pleno invierno! Por sólo este detalle, me
movió a grande admiración la sublime pobreza del héroe\ldots{} Así le
llamo, porque por tal le tuve y le tengo.

«Yo no poseo más que cincuenta reales mal contados, don Nicolás---le
dije;---pero con esa suma, le convido: almorzaremos juntos.» Aceptó, y
nos fuimos en busca de un cafetín. Por el camino y dentro del local
modesto donde almorzamos, me explicó los motivos de su inesperada vuelta
de Cuba, cuando le suponíamos allá bregando con los insurrectos\ldots{}
Hallábase en Madrid de reemplazo a fines del 71. No deseaba la situación
activa, porque en ella se habría visto en el caso duro de tener que
combatir a los republicanos. Puesto en el dilema de faltar a sus deberes
o a sus arraigadas creencias, pensó en abandonar la carrera
militar\ldots{} Sus modestas ambiciones se verían colmadas con un
destino civil. ¿Cuál? Desde niño soñaba con desempeñar plaza de torrero
en un faro. Era su ilusión \emph{vivir entre las olas, con los pies en
tierra, gozando la inefable ventura de no tener vecinos}.

Ignoro si había llegado Estévanez a pretender la plaza de torrero, que
era su ensueño. Soñando vivía cuando se pensó en destinarle a un
regimiento, y aquí vino el conflicto: o mandar soldados, cuya misión
entonces no era otra que pegar a los republicanos, o abandonar la
carrera. No teniendo otro medio de vivir que su paga de capitán, salió
del paso pidiendo el traslado a Cuba con el propio empleo. Otros iban
con ascenso; él no aspiró a tal gollería. Embarcó en Octubre; llegó el 2
de Noviembre, día de los Difuntos; se presentó a las autoridades; no se
le dio ocupación activa, ni en guarnición ni en campaña. Su único
trabajo era pasearse en la acera del \emph{Louvre}, y charlar con los
amigos en el café, del mismo nombre.

Ocurrió en el curso de aquel mes que se alborotaron los Voluntarios por
no sé qué broma, ligereza o travesura de los estudiantes de Medicina.
Contaba don Nicolás que no dio importancia al suceso, y que cuando oyó
en el café que se había formado consejo de guerra para juzgar a los
estudiantes, creyó que era también ligereza o broma de la infatuada
tropa de Voluntarios\ldots{} Una tarde, al entrar en el café, lo
encontró casi vacío. En las calzadas y paseos próximos no se veía un
alma. ¿Qué ocurría? Pues nada\ldots{} «¿Pero qué ocurre?» preguntó a un
mozo del café.

---¿Qué ha de ocurrir? \emph{Que los están fusilando}.

---¿A quién?

---A los estudiantes.

Contándolo, el rostro de Estévanez se transfiguraba\ldots{} parecía
otro\ldots{} «Nunca, ni antes ni después---me dijo,---en ninguno de los
trances por que he pasado en mi vida, he perdido tan por completo mi
aplomo. Grité, me descompuse, pensé en mis hijos, creyendo que también
me los fusilaban\ldots{} No sé lo que me pasó\ldots{} Ahora mismo no
puedo explicármelo.» El horror de la brutal tragedia, la indignación, la
idea del oprobio que caería sobre España y su Ejército por tal acto de
barbarie, le pusieron en un estado congestivo, privándole de
conocimiento. Fue menester sangrarle. Amigos cariñosos le llevaron a su
casa\ldots{} En una noche de insomnio y horribles pesadillas,
atormentado por la idea y visión de que le arrancaban de cuajo el alma y
con ella los sentimientos más arraigados, Estévanez pasó por todas las
formas de la demencia; y cuando esta fue declinando hacia la serenidad,
surgió la inquebrantable resolución de abandonar la Isla.

Hombre de tal temple, enardecido desde sus años juveniles en la devoción
de la Humanidad, de que se derivan las ansias de Libertad y Progreso, no
podía vivir en aquel campo de fieras discordias: por un lado los
enemigos de la Patria, por otro los que, llamándose hijos de ella, la
deshonraban con sus violencias y crueldades; allí la soberanía del honor
militar; aquí el imperio de las ideas\ldots{} Imposible residir en Cuba
sin tirar el uniforme o tirarse al mar\ldots{}

¿Pero cómo volver a España? Amigos fieles facilitaron a don Nicolás la
salida de aquel cráter: se solicitó del Capitán General licencia y
pasaporte para la Península, y conseguido esto, ya sólo faltaba esperar
la salida del primer vapor. Pero a Estévanez se le hacían siglos las
semanas, los días\ldots{} Ansioso de partir, como si en ello le fuera la
vida, tomó pasaje en una goleta llamada \emph{Estrella}, que salía para
Nueva Orleans con cargamento de madera\ldots{} El relato que me hizo el
hombre de su viaje en aquel barcucho, ponía los pelos de punta. Fue un
viaje de incidentes y trabajos que recordaban la primitiva navegación en
los mares de América.

Zarpó la goleta al anochecer, y a las pocas horas se inició en su bodega
un incendio. Echaron el bote al agua, y en él se embarcaron
precipitadamente tripulación y pasajeros. Estos eran dos: don Nicolás y
un chino. El capitán de la goleta, un yanki de mala catadura, les puso a
remar, y al fulgor de las llamas que devoraban el barco, emprendió el
bote la penosa navegación por un mar nada tranquilo. Sospechaba mi amigo
que el incendio no había sido casual: capitán y tripulantes dieron fuego
al barco con un fin de piratería. Provocaban un siniestro para estafar a
la Compañía de Seguros\ldots{} Esto sospechó Estévanez. Confirmaron su
presunción las maneras y actitud del capitán y marineros.

Rema que te rema, los dos infelices pasajeros veían cercano el momento
de ser asesinados o arrojados al mar. Parecía novela de navegación por
aguas de piratas o caribes. El miedo que pasaron fue tal que a otro que
Estévanez le habría durado toda la vida. Así transcurrió la noche, y en
tan horrorosa incertidumbre llegaron los náufragos al nuevo día.
Felizmente encontraron un vapor yanki que los recogió y los llevó a Cabo
Haitiano. De Cabo Haitiano partió mi amigo a Santomas, y allí,
descansado de tan hondas angustias, no pensó más que en dar realidad
legal a la situación que se había creado. Al abandonar la Isla de Cuba,
devolvía resueltamente a la Nación la espada que esta puso en sus manos.
En cuanto pisó tierra de Santomas, fue al Consulado de España, y entregó
al Cónsul un pliego en que solicitaba del Rey la licencia absoluta.

«Lo hice con pena---me dijo grave y melancólico.---Yo no tenía más
carrera que la militar: era capitán del 59, con el grado de comandante;
pero me había persuadido al fin de que no se puede pertenecer a la
milicia cuando se antepone la propia conciencia a todas las leyes, a
todas las ordenanzas, a todos los prejuicios de profesión y de
escuela\ldots» Siguió refiriéndome que por hallarse muy escaso de
dineros, tomó pasaje de tercera en un vapor francés, que a Europa venía
con escala en Santander. Recaló el vapor en el puerto cantábrico en día
de furioso temporal del Noroeste, y suprimida la escala, siguió a
Saint-Nazaire. Desembarcó don Nicolás, y con los pantalones blancos de
la Habana, en pleno invierno, y la misma ropa veraniega estuvo en
Nantes\ldots{} Prosiguiendo en ferrocarril su odisea, pasó la frontera y
se plantó en Madrid.

Esta breve y pálida referencia no puede dar a mis lectores idea, ni
siquiera remota, de la precisión, elocuencia y donaire con que el héroe,
que tal nombre debo aplicarle, relataba su dramático viaje de las
Antillas a España, y las tremendas causas que lo motivaron, y el
admirable tesón cívico que vigorizaba su alma generosa. Oyéndole,
saboreaba yo una gallarda página histórica, que él solo puede y debe
escribir, como su propio creador o cosechero.

Del cafetín fuimos, corriendo calles, a la busca y captura de amigos de
él y míos, y por el camino le enteré de las extrañas cosas que aquí
pasaban. Se maravilló y enojó de que los republicanos estuvieran
divididos en Intransigentes y Benévolos, y me dijo que por esta castiza
propensión al divorcio, estábamos tan lejos del advenimiento de la
República. No había en España voluntades más que para discutir, para
levantar barreras de palabras entre los entendimientos, y recelos y
celeras entre los corazones\ldots{} Puedo afirmar con plena convicción
que de cuantos amigos tenía yo, ninguno me cautivaba como aquel hombre
inflexible y \emph{de una vez}, dicho sea vulgarmente.

Perdónenme ahora si me acuso de una nueva licencia cronológica\ldots{}
Caigo en la cuenta de que mi destornillado caletre ha invertido los
hechos, pues mi encuentro con Estévanez fue bastantes días después de mi
violenta salida de la casa de Cabeza, y de la misteriosa desaparición de
la gruta (número 16 de cierta calle) en que visité a la ninfa graciosa y
endemoniada. Se me apareció el gran republicano ya bien entrado Enero
del 72, y lo compruebo con un dato político. Hablamos don Nicolás y yo
del Ministerio Sagasta, y precisamente en aquellos días don Práxedes
derribó con un simple codazo al Gobierno de Malcampo para subirse al
pescante y coger las anheladas riendas.

Sagasta era otra vez el gallo de nuestro corral político, y con su
arrogante cresta o tupé, su \emph{quiquiriquí} tribunicio y el irisado
plumaje de su simpatía personal, dominaría las olas que socavaban el
trono de Amadeo I. Del caído Ministerio conservó a Malcampo y a Angulo,
y completó el retablo con estas figuras: De Blas, Groizard, Topete y
Gaminde.

En los propios días, ¡oh lector mío bonachón!, esa misteriosa fuerza de
los hechos menudos que llamaré \emph{onda social}, me apartó del trato y
compañía de Nicolás Estévanez para llevarme a la vera de mis antiguos
camaradas de \emph{El Debate}. ¿Fue caso providencial, o una nueva
virazón de mi voluble destino? Pues una noche, dadas ya las once, me
encontré a Ramón Correa que del Príncipe venía muy embozado en su
capita. Del teatro solía ir a sus tertulias de gente de tono, y después
se zambullía en el Casino hasta el amanecer. Me paró; hablamos con
expresiva confianza; quejose de mi retraimiento\ldots{} «¿Pero dónde te
metes, Titillo? Ya sabes que te queremos\ldots{} Vete por mi casa\ldots»
Le prometí visitarle, y él puntualizó la cita, diciéndome: «Vete pronto.
Ya sabes\ldots; a la \emph{hora a que me levanto}. Abur. ¡Qué flaco
estás!»

\emph{La hora a que me levanto} era, en el reloj de la vida de Correa,
las siete de la tarde. Hombre más nocturno no he visto nunca. Vivía en
un pisito bajo de la calle de Claudio Coello. Retirábase al despuntar el
día. Despertaba de doce a una; se incorporaba, y sus criadas le servían
un buen almuerzo en una mesilla de patas muy cortas, construida \emph{ad
hoc} para formar un plano sólido sobre las telas del rebozo. Después de
bien almorzado, seguía durmiendo hasta las seis y media o las siete. Era
la hora de recibir a los amigos, y lavándose y vistiéndose charlaba con
ellos hasta que salía para la casa rica en que había de comer. Tal era
el vivir de Ramón Correa, que se pasaba meses y años sin conocer al sol
más que de oídas. En la noche social resplandecía la luciérnaga de su
grande ingenio. Por ser Correa cubano, debo decir \emph{cucuyo}. De
noche brillaba más que de día, y hablando más que escribiendo, pues la
indolencia ponía diques a su talento para mostrarse en la literatura
escrita. Su gracia, su exquisito gusto literario y su inmenso saber de
cosas mundanas corrían sin tasa en los raudales de la conversación.

Desde que iniciamos la nuestra, todo lo que me dijo mi amigo, acabado de
salir de la cama, iba encaminado a catequizarme para que me hiciese
sagastino. Con burlas y razones quería convencerme de mi estulticia, y
alabó a don Práxedes y al Duque de la Torre, presentándolos como los
únicos hombres que podían traer a España la paz, el bienestar y la
cultura. Era Correa un espíritu liberal metido en la armadura de un
eclecticismo elegante y conservador, como Albareda y demás políticos
procedentes de \emph{El Contemporáneo}. Con el buen gusto y la pasta de
un positivismo del mejor tono adornaba sus argumentos. Pero con todo su
donaire y amenidad no lograba convencerme.

«Mire usted, amigo Correa---le dije.---Yo, bien lo saben Albareda y
Ferreras, escribo fácilmente, ajustándome a las ideas que se me piden.
Escribo en republicano, escribo en conservador y hasta en neo si fuera
menester. Pero esto es, como si dijéramos, producción inconsciente de mi
ser, un chorro con variados criterios, que brota de mí sin más valor que
el de un juego de palabras. Dentro de mí quedan mis convicciones
inalterables. Si se me piden parrafadas anónimas, dispuesto estoy a
darlas; pero si me quieren afiliar públicamente al sagastismo, o como se
le llame, no accederé nunca, aunque usted me ofrezca posiciones,
destinos y jamón con chorreras. Vendo por un pedazo de pan mis tiradas
de prosa política; mis ideas no las vendo por ningún tesoro.» Sin
pensarlo me ponía yo en la cuerda paradójica en que él con gracioso
balancín sabía moverse y bailar.

«Todos guardamos en nuestra alma, querido Tito, un depósito grande o
chico de convicciones, que vienen a ser nuestro equipaje para el siglo
que viene. Pero no cambiemos de siglo antes de tiempo. La vida presente
nos tira del faldón cuando queremos lanzarnos hacia un lindo porvenir, y
nos dice: `Detente, amigo, y no corras hacia las fechas de 1910 ó 1915,
que aún están vacías'. Tiéntate el estómago, y tu estómago te dirá:
`Estoy como caño de órgano. Échenme algo pronto, que si no, me muero y
te mueres'.»

De broma en broma fui a parar a mi grave profesión de fe política,
diciéndole que yo no quería cuentas con Sagasta, el cual era el
escepticismo, el aplazamiento, el ya se verá, y yo aceptaba de lleno el
programa de don Manuel Ruiz Zorrilla, la reforma inmediata, radical,
concluyente\ldots{} Libertad de cultos, Enseñanza totalmente laica,
Derechos inalienables, imprescriptibles; Igualdad social, Reparto
equitativo del bienestar humano, Supresión del voto de castidad,
Desamortización de conciencias, Ejército cívico, Autonomía municipal y
provincial. Fuera títulos de nobleza; fuera cruces y calvarios\ldots{}
No más pena de muerte; no más quintas; no más frailes, no más gandules
presupuestívoros; no más colmenas para zánganos administrativos\ldots{}
En mi exaltación, me dejé decir aturdidamente que tal programa me lo
había dictado \emph{el propio cosechero}, y en mi poder lo tenía para
darle publicidad\ldots{} Mirábame Correa con asombro, poniéndose las
gafas, después de lavarse\ldots{} Dudó de que yo estuviera en mis
cabales; soltó la risa\ldots{} Volví yo entonces de mi fugaz desvarío, y
sujetando la burra que se me quería escapar, rectifiqué. No me lo había
dictado Zorrilla\ldots{} Obra mía fue la nueva Constitución, en noche
fantástica, hospedado en la gruta de una hechicera Circe, barragana de
un cura loco.

Contagiado el gracioso cubano de los escapes flamígeros de mi
pensamiento, aseguró que él iba más allá, y que dentro de un par de
siglos levantaría la simpática bandera de la supresión de todo gobierno
que es como decir \emph{anarquía}. La entidad Gobierno es la negación de
la paz pública\ldots{} Y de aquí, con gradaciones airosas, iba a parar a
este dilema: O yo me afiliaba públicamente en el \emph{Sagastismo}, o se
me ofrecería celda gratuita en Leganés, ya que no se habían creado aún
los \emph{tonticomios} que reclama el considerable aumento de la
necedad\ldots{} Una vez que endilgó su frac, como feliz comensal de casa
grande, salimos juntos, y por la calle repitió sus bondadosos
requerimientos para redimirme de la obscuridad y solitaria pobreza en
que yo vivía. Díjome al despedirme que si él no lograba convencerme lo
haría Ferreras, que también me distinguía y honraba con su
afecto\ldots{}

A buen paso me fui a mi domicilio, que a la sazón era una casa de
huéspedes, calle del Amor de Dios, de mediano trato y no muy lucido
aspecto, donde en días de penuria grande me metí, por los motivos y
circunstancias que a renglón seguido contaré. La horripilante situación
de mi erario me lanzó nuevamente a la busca y captura de la \emph{Casa
Rostchild}, la cual, echando los bofes, encontré reencarnada en un varón
seco, duro, agrio, que se llamaba don Francisco Torquemada y vivía en la
calle de San Blas, zona baja de Atocha. Enorme cantidad de saliva gasté,
y sin fin de escalones subí para conseguir de aquel perro algún alivio
de mi necesidad. Pidiome garantía del Banco de España, o la firma de
Manzanedo, y cuando ya llegaba yo a los extremos de la ira, llegó él a
los de la piedad, y salí de su casa contento, aunque desplumado para una
fecha no lejana. Al despedirme quiso mostrarme su protección
recomendándome una casa de huéspedes buena, limpia y económica. Acepté
por hallarme a la sazón muy mal alojado, y por dar gusto a Torquemada.
Sin duda la casa de pupilos era suya, o de algún cliente con quien iba a
la parte.

Mi patrona era una pobre mujer derrengada y envejecida por el trabajo,
con la carga de cuatro hijos y la impedimenta de un marido que no le
servía para nada, en el orden de la industria huesperil. Llamábase
Nicanora, y Rosita la mayor de sus niñas, que era muy mona y algo
bachillera. El esposo, don José Ido del Sagrario, había sido maestro de
escuela. Aquejado de cierta frialdad del cerebro, hubo de abandonar el
noble oficio de desasnar chicos; mas no con el descanso pudo recobrar la
salud, ni siquiera un mediano \emph{gobierno} de su máquina muscular y
nerviosa. Quedó, pues, en situación de esqueleto vestido de fláccidas
carnes; no resistía ningún trabajo fuerte, físico ni mental; ocupábase
tan sólo en repartir entregas de una Casa Editorial, reduciéndose a un
corto callejero, y en hacer recados a los huéspedes, que eran conmigo
tres estudiantes de San Carlos. El trato de Ido me agradaba; era hombre
que no carecía de luces, aunque solían brillar tan sólo por ráfagas
intercadentes, lívidas llamaradas de alcohol. Tristeza y goce me
causaban a la par mis conversaciones con aquel hombre inocente y bueno,
cerebro que yo comparaba a la celda de una cárcel, en que hubiera estado
preso un filósofo. Este se había fugado dejando en las paredes efluvios
de su espíritu.

A poco de entrar en la casa de doña Nicanora, tuve amores con una
princesa\ldots{} Déjenme explicar. Era una tiple que había estrenado en
los Jardines del Retiro el airoso papel de la \emph{Princesa Colibrí},
farsa medio lírica, medio bailable. Por la interpretación libérrima y
desahogada de aquel personaje mímico y cantable, quedole entre el vulgo
teatral el mote de \emph{La Princesa}. Su nombre auténtico era Pepa
Hermosilla, sobrina carnal de dos guapísimas hembras de la generación
pasada, \emph{las Hermosillas}, comúnmente llamadas \emph{las Zorreras},
por ser hijas de un fabricante de zorros. Vierais en Pepa una mozuela
linda y desfachatada, bailarina más terrestre que aérea, tiple ligera,
ligerísima.

\hypertarget{xii}{%
\chapter{XII}\label{xii}}

Sí; tan ligera, que la conocí antes de media noche en el escenario, y a
la madrugada estábamos ya casados requetecivilmente\ldots{} No debería
yo contar estas cosas; pero allá van para descargar mi conciencia,
mostrando a mis lectores la locura de aquellos años juveniles. Confieso
mis pecados con la mira saludable de que en ellos se vea la procedencia
de mis fieros quebrantos y desdichas, y de ello tome ejemplo la juventud
para que se aparte de los caminos que no conducen a la moral\ldots{}
Pues, señor, llevaba yo media semana en las alegrías de \emph{príncipe
consorte}, cuando una tarde me encontré en la Plaza de Matute con
aquella Lucrecia de quien ya hice mención, bonita y vaporosa rubia
bermeja amiga de Felipa\ldots, la que conocí asociada a un jugador de
oficio que llevaba la pechera y los dedos cuajados de brillantes. Al
jugador le había salido la mala, y se lo llevaron los demonios. Lucrecia
se me presentó desolada. La compadecí, le prodigué los consuelos que mi
alma generosa me sugería, y por último, observando que su pena no tenía
más alivio que el contármela a mí, decidime a protegerla; hablamos, nos
entendimos, y punto concluido.

Mi doble juego de amor fue descubierto a los pocos días por las dos
apasionadas hembras, a quienes yo engañaba y entretenía con toda clase
de sutilezas o equilibrios. El resultado fue que estalló el conflicto
una mañana\ldots{} Encontrándose en la calle de Santa Isabel se
acometieron, se arañaron, se dijeron cuanto dos bravas mujeres pueden
decirse en caso tal, y se arrancaron recíprocamente mechones de sus
respectivas cabelleras, negra la una, rojiza la otra. El culpable de
aquella mujeril trifulca, que los periódicos narraron como un caso de
risa y festejo, fue el bendito chiflado don José Ido, a quien entregué
dos cartas, una para cada cual, y el desventurado filósofo las trabucó
y\ldots{} Ya comprendéis lo demás\ldots{} Cuando enterado de la zaragata
increpé al mensajero por su descuido, me respondió con fría y angelical
serenidad: «Francamente, naturalmente, yo pensé, señor don Tito, que
usted, en vez de regañarme, me agradecería la equivocación, porque así,
enzarzadas la una con la otra, se ve usted libre de las dos, y quedará
en franquía para mejor arreglo con una sola.»

No dejé de apreciar en su justo valor esta sutil filosofía; pero, ¡ay!,
del lance mujeriego no me resultó el beneficio que el candoroso Ido
presumía, sino todo lo contrario\ldots{} Sucedió que cuando se hallaban
Lucrecia y Pepita en lo más recio de su pelea, acudió a separarlas y a
poner paz una señora que con su criada venía de hacer la compra en el
mercado de los Tres Peces\ldots{} Logró el armisticio entre ellas; oyó
las razones de cada cual, y con humanitaria diligencia vino a mí para
gestionar avenencia y concordia con una de ellas, ya que con las dos no
podía ser. Y cómo se arreglaría la desconocida señora en su arbitraje,
que de las sucesivas conferencias resultó que llegué a un \emph{modus
vivendi} con las dos separadamente, y luego me entendí con la mediadora,
que era mujer agradable, viuda en buena edad y de no poca sal en la
mollera\ldots{} Yo no sé qué tengo, señores que me leéis, no sé qué
tengo\ldots{} Lo mismo es hablar yo con una mujer, que esta se pone
tierna y no tarda en enloquecer por mí\ldots{} No sé lo que tengo,
repito, no sé\ldots{}

De lo que acabo de referir, salió, como podréis suponer, mayor
desventura mía, y el trabajo hercúleo de tener que triplicarme con
diarias fatigas y combinaciones. La más amada de las tres era la que fue
mediadora. Trataba yo de que fuera la única; pero tales dificultades y
trapisondas me salieron al paso en mi tentativa de moralidad, que hube
de seguir bailando, como decía el otro, \emph{en el triple trapecio de
Trípoli}, hasta que la desdichada derivación de tales hechos dio su
funesto resultado\ldots{} Antes de que pasaran dos semanas de este
horrible trajín, Lucrecia fue asesinada por el empresario de timbas que
había sido su amante, y aunque no me alcanzaba ni alcanzarme podía
culpabilidad en el crimen, por el lugar y ocasión en que fue perpetrado,
no me libré del espanto y consternación propios del trágico suceso.
Pocos días después descubrió la \emph{princesa} mi triple juego, y
alborotada se plantó en mi casa, y cual furiosa rabanera, vertió sobre
Nicanora y el pobre Ido las más groseras injurias. Lo que me dijo a mí
me está escociendo todavía\ldots{} Y por último, compasivo lector, mi
\emph{tercera}, que yo tenía por primera, no pudo menos de abrir sus
enamorados ojos a estos escándalos, y me despidió de su trato, ya que no
de su corazón, derramando lágrimas amarguísimas.

Era una viuda tierna, bastante supersticiosa, tirando a mística.
Llamábase Delfina. Su padre fue un excelente confitero que tuvo gran
parroquia en Madrid. Su marido fundó y disfrutó la más elegante
Funeraria de esta Corte, industria que la viuda traspasó, mediante
\emph{conquibus}, al que había sido primer dependiente del fundador. Con
este provecho y lo que heredó de su padre, Delfina disfrutaba de un buen
pasar; vivía holgadamente, y daba socorros a parientes pobres, suyos y
de su marido\ldots{} Entendía yo que aquellas granjerías tan diferentes
en forma y fondo habían dejado en la infancia y juventud de la buena
señora la impresión de las cosas familiares adheridas a la existencia.
Por esto decía de ella mi amigo Roberto Robert que era \emph{dulce y
tétrica}\ldots, y que en su carácter veía un ataúd lleno de yemas y
tocino del cielo.

Algo de verdad había en estas paradojas. Mi amiga era suave y
borrascosa; con sólo minutos de diferencia mordía y acariciaba.
Ferviente devota de San José, a quien pedía todo lo que anhelaba, creía
mil profanos disparates. Cuando en misa sacaba el cura casulla verde (lo
que sólo en contados días se ve), doña Delfina se llenaba de terror, y
de la iglesia salía persuadida de la proximidad de grandes daños y
calamidades. Creía en el mal de ojo y en las recetas para impedir sus
terribles efectos, y era fuerte en fórmulas cabalísticas para conseguir
de la Santísima Trinidad la pronta cura de tercianas y cuartanas.

Refiriendo a mi persona estas extravagancias, diré que la viuda me
quería y me apartaba de su trato; tan pronto era la benigna divinidad
que por mí se interesaba, como la fiera sacerdotisa que arrojaba sobre
mí siniestros augurios y maldiciones\ldots{} Termino el retrato con
estas noticias que, si por el momento no interesan, podrán tener algún
valor en lo que más adelante relataré. Delfina Gil era natural de un
pueblo próximo al que tuvo el honor de verme nacer. A no pocas personas
de mi familia conocía, y huroneando en el pasado sacaba remotos
entronques de sus antecesores con el claro linaje de los Livianos.

Adelante con mi cuento. Las resultas de la referida borrasca mujeril, y
la extraña doblez del carácter de Delfina, mi benéfica protectora por un
lado, por otro mi fiscal implacable, me llevaron a un estado de intensa
melancolía. Vagaba yo mañana y tarde por los barrios extremos y las
afueras de Madrid, hablando a solas, o pronunciando discursos férvidos
ante la soledad agreste. El casual encuentro con algunos amigos me sacó
del pozo de mis meditaciones, llevándome a la política, que es eficaz
medicina de tristezas. El trajín de las opiniones propias y ajenas, que
en mil casos no nos llegan a lo hondo del ser, nos restablece a una
normalidad vividera, y al suave pasar de las horas y los días\ldots{}
Sin saber cómo llegué a verme metido en el hervor de la campaña
electoral. Corría Febrero, Marzo le siguió en aquel afán; yo, avispado o
embrutecido, que esto no lo sé, por la propaganda, me metí más en ella.
No era que yo pretendiese la diputación; pero amigos míos pedían sus
votos al pueblo, y quise poner en la lucha todos mis esfuerzos,
interesándome particularmente por Nicolás Estévanez, que presentaba su
candidatura en uno de los distritos de Madrid.

En aquellos días de ciego furor sectario, quedó formada la magna
Coalición o piña electoral para derrotar al Gobierno. Componían la
\emph{Junta Mixta}, o si se quiere, el pisto manchego, tres individuos
por cada uno de los cuatro partidos de oposición: por el carlismo tres
neos hidrófobos; por el alfonsismo tres reverendos caballeros de los de
alba camisa, únicos poseedores de lo que se llama \emph{dotes de
gobierno}, esto es, planchado con brillo; por los radicales tres añejos
progresistas, y por los republicanos los más culminantes del partido.
Omito los nombres para no contribuir a que llegue a la generación
venidera el fuerte olor del vinagre en que se hizo esta ensalada o
gazpacho\ldots{}

Menudeaban las reuniones, las prédicas y las asambleas. Yo fui a las que
celebraron los republicanos en el teatro de la Alhambra, y sin hacerme
de rogar, por impulso instintivo y comezón declamatoria, en todas
hablé\ldots{} Me oían con vivo interés, me aplaudían a rabiar. Luego, mi
ardor y los aplausos me llevaron a la exageración de mi énfasis, a
emplear argumentos retorcidos y dislocados y a burlarme de la lógica.
Una noche defendí el contubernio electoral, y a la siguiente lo combatí
con saña\ldots{} Sin saber cómo, se me salían del pensamiento a la boca
las ideas de aquel fantástico programa que supuse dictado por Ruiz
Zorrilla en la hechizada gruta de Graziella. Todas las zarandajas de mi
Credo radicalísimo iban cayendo de mis labios sobre el auditorio, como
lenguas de fuego sobre el montón de combustible. Una noche, a la salida,
Santamaría y Luis Blanc me dijeron: «Chico, no hables más. Te exaltas
demasiado. Procura serenar tu entendimiento.»

Estas suaves reprimendas de mis amigos, y otras más agrias de algún
primate de los que ocupaban la mesa, conminándome con no concederme la
palabra si seguía por aquel camino, me redujeron un triste silencio.
Salíame yo por las tardes a los barrios del Sur y de allí a las afueras,
y dondequiera que veía un grupo de seis o siete personas, me detenía y
les predicaba\ldots{} No tardé en encontrar prosélitos; llevaba tras de
mí una pandilla de hombres y mujeres que me incitaban a que les
arengase, y yo, diciendo para mí \emph{aquí que no peco}, soltaba el
surtidor de mi desordenada oratoria. No ponía ningún freno a mis ideas,
y lo menos que les decía era que el mejor Gobierno era el
no-gobierno\ldots{} Cuando a mi casa me retiraba fatigado y ronco, y en
la soledad de mi cuarto con fría reflexión pensaba en mis discursos, me
asaltaba la sospecha de que en mi cerebro había ocurrido alguna
conmoción, que desmontara o por lo menos sacara de sus quicios las
piezas del mecanismo pensante. Y cavilando más en esto cada noche sobre
el agasajo de las almohadas, creí dar con la razón de tales sinrazones.
Si en efecto yo iba camino de la demencia o de la chifladura, la causa
no podía ser otra que el desequilibrio en que estaba mi ser por la
interrupción de mis conquistas y de los dulces efectos de ellas, o sea,
el trato con el bello sexo.

Firme en esta tesis, me propuse volver a las amenidades amorosas. Sí,
sí; el amor es la vida, y además la razón, y el perfecto funcionar
armónico de nervios, sangre, masa encefálica, estómago, pulmones,
etc\ldots{} ¿Qué hice? Visitar a Delfina Gil y abordarla bruscamente con
arrumacos sentimentales, suaves arrullos, miradas incendiarias, y sobre
todo ello puse las \emph{florituras} y \emph{fermatas} de un vocabulario
de seducción que, dicho sea sin falsa modestia, sé manejar como
nadie\ldots{} Pues Delfina no me hizo caso. Hallábase en un estado de
espíritu incompatible con mis malvadas pretensiones. Sufría el ataque de
virtud furiosa y empedernida, que solía durarle diez o doce días y a
veces meses enteros. Seria y desdeñosa, me dijo que llamase a otra
puerta, y al verme salir, me retuvo para echarme esta suave
\emph{indirecta del padre Cobos}: «Estás mal de la cabeza, pobre Tito.
He notado el desorden de tus razonamientos. Tus amigos se alarman oyendo
los disparates que dices en los \emph{metingues}. Será preciso aislarte,
tenerte en encierro y observación hasta que entres en caja. Escribiré a
tu familia, enterándola de tu mal. Allá dispondrán si vienen a buscarte
y te llevan al pueblo, que sería lo más acertado, o me autorizan para
ponerte en cura.» Yo me reí\ldots{} «Adiós, adiós\ldots»

Al pie de la letra tomé el \emph{llama a otra puerta}, y de la calle de
la Magdalena me fui tan campante a la de Tabernillas. Sabía que en
aquellos barrios moraba mi antigua socia Felipa, que aún me guardaba
ley, demostrándomelo en repetidas ocasiones con recaditos de amistad y
aun con menudos obsequios\ldots{} Busca buscando, la encontré en la
calle del Águila, más negra y agitanada que antes, por efecto del
negocio de carbón a que se dedicaba en compañía de un hombre robusto,
tiznado y carbonífero, llamado Bernabé Díaz. A mis halagos contestó
Felipa que no contara con ella para nada contrario a la fidelidad que a
su Bernabé debía. Hallábase, pues, en pleno periodo de virtud; era
feliz, trabajaba de sol a sol, y no cambiaba su actual vida de activa
tranquilidad por otra de escándalo y deshonor. Pregunté si se casaría
con Bernabé, y me dijo: «En eso andamos. Las damas catoliconas nos están
trabajando el casorio. Yo lo deseo. Me espanta la idea de llegar a vieja
sin tener un arrimo y vivir en ley\ldots»

Ya me iba cargando tanta virtud\ldots{} ¿Por ventura tendría yo que
hacerme también virtuoso para recobrar mi equilibrio?\ldots{} De la
carbonería pasé a la taberna próxima, donde tuve la satisfacción de
encontrarme a mi amigo y casi pariente, \emph{Sebo} por mal nombre,
rodeado de toscos ciudadanos, entre los cuales estaba el tal Bernabé,
presunto esposo de Felipa. Trataban de la elección por aquel distrito
(Latina), el más republicano de Madrid. \emph{Sebo}, agente electoral de
la Coalición, recomendaba la candidatura de Estévanez, que era predicar
a convencidos, pues en aquel barrio pobre, liberal y entusiasta, gozaba
don Nicolás de gran predicamento. Metí yo al instante mi cuarto a
espadas en la reunión, haciendo del candidato el más fogoso panegírico
que aquellos hombres inocentes habían oído. Y fue grande mi satisfacción
oyendo lo que a la salida de la tasca me dijo Telesforo: «Mi antiguo
señor, el Marqués de Beramendi, me ha mandado que apriete de firme para
sacar a Estévanez, pues aunque no le trata ni le ha visto nunca, le
tiene en gran estima por su honrada convicción, y por lo derecho y firme
que va camino del Progreso, sin mirar atrás.»

Desde aquel día, me metí en el trajín electoral, y tuve la dicha de oír
de los autorizados labios de don Nicolás, en las reuniones del teatrito
de la calle de Las Aguas, parrafadas y apóstrofes tan tremendos como los
que a mí me valieron poco menos que la excomunión de la Asamblea del
partido\ldots{} Si a mí me tuvieron por loco, no lo estaba menos
Estévanez, y esto me consolaba. O ser revolucionario de verdad, o no
serlo. Si nuestra sociedad reclamaba, con su hondo malestar, renovación
completa, nada se haría si no demolíamos el vetusto y apuntalado
edificio para reconstruirlo con nuevos planos, nuevos materiales y
arquitectos nuevos. Sacáramos estos de la nada, no del personal
existente\ldots{} Antes de crear un nuevo mundo, hiciéramos un delicioso
caos.

No canso a mis lectores refiriendo al detalle una campaña electoral en
que apenas hubo pelea, por la excelente disposición del popular distrito
y el arranque del candidato. Sin gastar una peseta le sacamos, con 8.000
votos de ventaja sobre el contrincante sagastino. Los electores eran
gente sencilla, proletaria, que no ambicionaba destinos ni prebendas,
voz y voluntad auténticas del pueblo soberano. La Coalición triunfó en
Madrid, con dos republicanos, Estévanez \emph{(Latina)} y Galiana
\emph{(Hospital)}; cuatro radicales, Montero Ríos \emph{(Palacio)}, Ruiz
Zorrilla \emph{(Centro)}, Martos \emph{(Congreso)} y Becerra
\emph{(Audiencia)}; el único ministerial que tuvo acta fue el General
Beránger \emph{(Hospicio)}. En provincias, los amaños de Sagasta dieron
a este una mayoría gregaria; mas no pudo ahogar el empuje de las
minorías. Sólo el carlismo trajo treinta y cinco puntos\ldots{} Y estos
sí que eran puntos negros.

Seguí en relaciones de cordial amistad con Estévanez, que no se
envanecía de su triunfo, ni creía que en el futuro Congreso pudieran
hacerse campañas eficaces para la idea republicana. En nuestras charlas,
tuve el gusto de oír de su boca las apreciaciones más exactas de la
realidad política en aquellos días. La revolución estaba muerta por
haber perdido en gran parte la savia progresista que le dieron los
trabajos del 67 y el triunfo del 68. Los alfonsinos habían ganado
terreno con la traída de un Rey extranjero; contaban a la sazón con lo
más florido de la oficialidad del Ejército. Todo cuanto veíamos despedía
olor a muerto. Los Gobiernos de don Amadeo no salían de la norma y pauta
somníferas de los Gobiernos anteriores a la Revolución. Los vicios se
petrificaban, y las virtudes cívicas no pasaban de las bocas a los
corazones. Administración, Hacienda, Instrucción Pública, permanecían en
el mismo estado de quietismo y pereza oriental. No salía un hombre que
alzara dos dedos sobre la talla corriente. Hacía falta un bárbaro, como
Pizarro, que sin saber leer ni escribir, creó un mundo hispano en la
falda de los Andes.

Estas ideas me cautivaron. Sí, hacía falta un bárbaro que creara otro
mundo hispano. Pero aquel bárbaro no era yo, que poseía regular cultura,
sabía escribir, y echaba sin ton ni son discursos elocuentes\ldots{}
Hacía falta un mudo, que hablara con los hechos y con la piqueta,
demoliendo los viejos muros, sin pedir permiso a las letras de molde; un
mudo, sí, que entendiera de cirugía política, y supiera leer lo escrito
con caracteres de fuego en el alma de la Nación\ldots{} Debajo del
pesimismo de mi gran amigo, latía, como es de ley en todo ser superior,
un fuerte optimismo. No desconfiaba de la idea, sino de los hombres que
en el telar político, llamándose ministeriales o de oposición, tejían la
misma tela frágil y descolorida, tan fea y tan mala por el derecho como
por el revés\ldots{} En suma, que la oposición republicana, aliándose
con los \emph{Nocedales} y \emph{Barzanallanas}, se contagiaba de esa
legalidad indigesta que siempre resulta infecunda, y cándidamente hacía
el juego a sus naturales enemigos. Los arañaba; pero no supo darles,
como debía, muerte y sepultura\ldots{} Mientras más lecciones de estas
cosas me daba mi amigo, más me enamoraba su carácter. Lo que aún tengo
que decir de él quédese en remojo todavía, pues me urge contar un suceso
de importancia, que a mi ver cae dentro de la fase humorística de la
Historia. Sígame, si gusta, el benigno lector desde este capítulo al que
inmediatamente le sigue.

\hypertarget{xiii}{%
\chapter{XIII}\label{xiii}}

No cesaba yo de interrogarme así: «¿Estaré un poco demente, o siquier
tocado de tenaces manías, la manía de mi proteísmo, que consiste en
escribir con distintos criterios y aparente convicción, la manía de mi
esencial criterio inmanente, de tendencias atrozmente revolucionarias?»
Y otra cosa pregunto a los que me leen y a mí mismo: «¿Todo lo que
cuento es real, o los ensueños se me escapan del cerebro a la pluma y de
la pluma al papel? ¿Las amorosas conquistas que me sirven de trama para
la urdimbre histórica, son verdaderas o imaginarias? ¿Creo en ellas
porque las imagino, y las escribo porque las creo?\ldots» Mientras con
ayuda de mis indulgentes lectores dilucido estos puntos, seguiré
contando\ldots{} A ver si me acuerdo\ldots{} Ya, ya he cogido el
hilo\ldots{} Pues Felipa, después de repetida por décima vez la
proclamación dogmática de su virtud, me aconsejó que viese a Celestina
Tirado, y a sus buenas disposiciones me encomendara.

Pero\ldots{} el demonio lo hacía\ldots, encontreme a Celestina también
atacada de monomanía virtuosa, y en vías de abandonar su vil industria,
dándose de baja en el escalafón del Infierno. Tenía una hija, criada en
el campo, ya grandecita. Celestina la llevó consigo, sedienta de cariño
maternal, que apenas había gustado en su vida liosa. Enterose de ello la
Marquesa de Navalcarazo, y queriendo apartar a la pobre niña de todo
influjo maléfico, obligó a la madre a ponerla bajo la guardia y custodia
de unas monjitas de la calle de San Leonardo. Accedió Celestina, movida
de un vago prurito de corrección espiritual, y las mañanas pasaba en la
iglesita del convento, o en la fronteriza parroquia de San Marcos,
entretenida en rezos y otros actos de devoción. Hablando de esto, me
confesó que hasta las oraciones más elementales, \emph{Credo} y
\emph{Padrenuestro}, se le habían olvidado, y en aquella ocasión las
aprendía de nuevo, sintiéndose volver a sus años infantiles.

En estos contactos con la vida eclesiástica, la antes pecadora, y
después reformada Celestina, echose también su director espiritual, y
tuvo la suerte de topar con un sacerdote ejemplarísimo, llamado don
Hilario de la Peña. Hablando de él la pícara convertida no agotaba el
filón de las alabanzas. Tales cosas me dijo, que me entraron vivas ganas
de conocer al bendito clérigo. Y una mañana, en que mis divagaciones
callejeras me llevaron a la de San Leonardo, me deparó mi suerte el
encuentro de Celestina, que del convento salía con su reverendo amigo y
capellán don Hilario, y ambos iban hacia la parroquia de San Marcos.
Presentome la pícara como periodista y cultivador de las Letras, y
apenas hablé diez palabras con el buen señor le diputé por hombre bueno,
tolerante, y de no común cultura.

Metiose Celestina en la parroquia, y yo seguí con el cura hasta la
puerta de su casa. Era viejo, de gran talla y al parecer gotoso.
Aliviaba su cojera con un grueso bastón. Lucio y carilleno, pareciome
hombre que se había dado buena vida. Su afable sonrisa y sus ojuelos
vivarachos delataban el amplio conocimiento del mundo y el hábito de la
preciosa indulgencia. Mostrose complacido de hablar con un escritor, y
juzgándome con benevolencia cortés, por desconocer mi escasa valía, me
reveló que él también plumeaba, por pasar el rato, y sin pretender el
galardón de la fama. «Soy aficionado a los estudios históricos---dijo
con modestia,---y he consagrado mis ocios a escribir la \emph{Historia
del Clero Mozárabe en Toledo}, de la cual llevo ya publicados tres
tomos. Es obra de pura erudición, árida, como centón de documentos.» Mi
cortesía correspondió a la suya, diciéndole que conocía parte de los
tres tomos publicados, y haciendo del contenido de ellos un ardiente
elogio. Al darme las gracias advertí en él un amable escepticismo. No
creía en mi entusiasmo por su obra\ldots{} Con recíprocos plácemes y
cumplimientos nos separamos, pidiéndole yo la venia para visitarle, pues
me honraría mucho su trato y buena amistad.

Y no pasaron tres días sin que me personara en la casa del cura. Me
recibió en su biblioteca, que era copiosa y algo desordenada, como toda
biblioteca en que se trabaja. De lo que habló don Hilario, saqué en
limpio que era rico, que por no abandonar en absoluto su ministerio
religioso, desempeñaba la capellanía de las monjas vecinas. Algún
trabajo le daba el delicado gobierno de las conciencias de aquellas
santas señoras, que por no tener nada que hacer, inventaban pecadillos,
y apuraban la paciencia del confesor para lavarlos y restablecer su
inmaculada pureza\ldots{} Deseaba el señor Peña ocasión para zafarse del
enfadoso lavatorio y planchado de las monjiles conciencias\ldots{}
También me dijo que le amargaba el sentimiento de no poder terminar su
obra. Herido de la gota y otros desgastes del organismo, sólo contaba ya
con un par de años de vida, o poco más\ldots{}

La persona del venerable clérigo trajo a mi cabeza espantosa confusión.
Antes de tratarle, tenía yo noticia de él (ignorando el nombre) y de su
magna \emph{Historia del Clero Mozárabe}. Intentaba yo por mañana y
tarde descifrar aquel enigma, y desvanecer mi perplejidad. No sé cuántas
veces me llegué a la calleja, entre Monteleón y Maravillas, y con ojos
inquietos buscaba el 16 de marras, sin perder la esperanza de que la
casa de aquel número hubiera salido de las entrañas de la tierra. Pero
lejos de ver que esta devolvía lo que se tragara en días ya lejanos, mi
barullo mental aumentó con sucesos más contrarios a la lógica y al
sentido común.

Acudiendo una mañana de Abril a mi tercera visita, encontré a don
Hilario en la calle, yendo yo por la de los Reyes. Nos paramos, y
después de los recíprocos saludos, me dijo: «Tengo que ir a Palacio. Si
no tiene usted que hacer acompáñeme, y por el camino le contaré el
porqué de ir yo a la Casa Grande, novedad para mí extraordinaria, pues
sólo una vez estuve en ella, cuando a doña Isabel le dio por hacerme
obispo, y yo rehusé. No recuerdo la fecha. Ello fue cuando Pío IX
concedió a doña Isabel la \emph{Rosa de Oro}. Vamos, hijo.» Andando,
siguió así: «Pues esta buena señora, doña María Victoria, sale ahora con
que quiere nombrarme capellán de ese Asilo que ha fundado para las
lavanderas\ldots{} Ello habrá sido idea del Conde de Rius, intendente de
Palacio, y gran amigo mío. Usted le conocerá: es yerno de Olózaga, que
también me honra con su amistad. Sea de quien fuere la iniciativa de mi
designación, voy a decir que nombren a otro. Yo declino ese honor, yo no
sirvo para nada. Busquen para las lavanderas un clérigo mozo. Yo no
estoy ya para ninguna función que reclame el vigor juvenil\ldots»

Charlando con voluble intercadencia de veras y bromas llegamos a Palacio
y entramos en la Intendencia, que está, como sabéis, en la planta baja,
plaza de la Armería. En una antesala nos detuvimos; salió el intendente,
Conde de Rius, a quien yo sólo conocía de vista; el cura me presentó a
él como un amigo que le acompañaba en clase de rodrigón o
\emph{lazarillo de su cojera}, y pasaron los dos al despacho próximo,
donde a mi parecer trataron de la Capellanía de Lavanderas. Quedeme solo
en aquel aposento, donde no veía más que estantes llenos de legajos, y
algunos cuadrotes deslucidos del tiempo y del humo del gas, y que
representaban edificios o campiñas de los Sitios Reales. A poco de estar
sumido en tal soledad, sentí hormiguilla en brazos y piernas, y zumbar
de mis oídos, cual si a ellos llegaran las ondas de un lejano son de
bronces vibrantes. Convirtiéronse aquellos sonidos en voz humana,
demasiado dulce para ser de hombre, demasiado grave para ser de mujer.
Volví la cabeza\ldots{} y vi que por escondida puertecilla entraba un
bulto\ldots{} Mi primera impresión fue de una señora gorda y
ajamonada\ldots{} Al acercarse a mí se volvió esbelta sin gran merma de
sus carnes lúcidas. Vestía elegante traje negro de seda, a la última
moda\ldots{} ¡Ay, Dios mío! Que me llevaran los demonios si no era la
\emph{Mariclío}, con sin fin de años menos de los que representaba
cuando anteriormente la vi, y muy apersonada y peripuesta.

«Hola, Tito---me dijo con graciosa confianza, arrastrando un pesado
sillón para sentarse frente a mí.---¿No me habías conocido? Vengo ahora
un poquito transformada. Yo me pongo más fea o más bonita según los
lugares por donde paso y las diligencias que traigo entre manos. Estamos
en lo que los periodistas llamáis \emph{el regio alcázar}, y cuando aquí
entro, procuro adecentar mi facha y traje por si me sale en estas
alturas del Estado algo decoroso que pueda llevar a mis archivos.»

Diciendo esto, alargó hacia mí uno de sus pies, con la mayor
desenvoltura, sin cuidado de que yo le viera la pantorrilla. Calzaba en
aquel pie un lindo borceguí colorado, con tacón de plata. Y viéndome
suspenso, sin saber qué hacer con el precioso y bien engalanado pie, me
dijo risueña: «Parece que estás tonto. Haz el favor de descalzarme.
¿Tanto te asusta una vieja compuesta? No es el coturno lo que ves; es un
zapatón de media gala. Me lo he puesto para venir a esta casa, y ya me
pesa. No lo merecen\ldots» Le quité el borceguí con todo el respeto que
me inspiraba, y al instante sacó, no sé de dónde, una blanda zapatilla,
que por su propia mano se calzó sin esperar mi auxilio. Antes de repetir
la operación en el otro pie, levantose muy ligera, y dio paseos airosos
por la estancia, un pie con medio coturno y el otro con zapatilla.
Esgrimiendo la que le quedaba en la mano, decía: «Con este escarpín
azotaría yo las posaderas de los desgraciados y ridículos hombres que
arriba he visto. Pide a tu Patria que tenga un arranque y los mande a
donde fue mi amigo el reverendo \emph{Padre Padilla}.»

Dicho esto, volvió a sentarse; la descalcé y calcé del otro pie, y
quedose meditabunda un mediano rato, mientras yo discutía mentalmente
con mis ojos sobre la realidad o ficción de lo que veían, y les acusaba
de burlarme con alucinaciones infantiles\ldots{} Y ellos me contestaban
que no era culpa suya, sino de doña \emph{María Clío}, hechicera y
juguetona. Esta terminó sus meditaciones diciendo: «Mal andan allá
arriba. Ministros y Rey han rivalizado en torpezas. Al Rey le disculpo.
Sagastinos y zorrillistas le traen mareado con sus necias enemistades
por un quítame esas pajas. Los 191 votos que dieron la corona a la casa
de Saboya, ¿qué se hicieron? Hanse dividido en dos bandos; viven
tirándose a la cabeza todos los trastos de la Constitución. Como don
Amadeo no se imponga a esta tropa, ya puede preparar sus
equipajes\ldots{} Figúrate, hijo mío, que los llamados constitucionales
se dividen a su vez, y por la combinación de generales andan también a
repelones\ldots{} El sábado, día de Consejo en Palacio, se presenta
Sagasta en la Cámara Real, y dice al Rey que no se celebrará Consejo,
porque no hay asuntos de qué tratar. No le valen al camerano sus
marrullerías, y Amadeo, con acento más firme del que suele usar, le
contesta: \emph{Si el Gobierno no tiene hoy nada que decirme, yo tengo
cosas muy serias de que hablar al Gobierno. Cite usted ahora mismo, y
aquí quedo esperando}\ldots{}

---Ya sé lo demás, señora mía---repliqué yo.---Lo traen los periódicos.

---Cada periódico cuenta el caso a su modo, y con el aderezo y salsa que
cada bandería suele gastar en sus guisos. Óyelo de mi boca, que no
miente. Mi único guiso es la verdad\ldots{} Azorados reuniéronse los
ministros en Consejo, y ante ellos desenvainó el Monarca un papel que
leyó con buena entonación. El documento era declamatorio y enfático,
como los que escribías tú en \emph{El Debate}, recomendando el
específico de la Conciliación. No admitía el Rey nuevas disidencias, ni
que el partido llamado \emph{Constitucional} se partiera en mitades, que
en la política general resultaban cuarterones. La intención expresada en
el papelito era buena, el modo de señalar y el estilo vulgarotes a no
poder más\ldots{} Los ministros fueron desde aquel momento pintorescos
personajes de ópera cómica. ¿Dimitían o continuaban después de rascarse
las partes de sus cuerpos azotadas por el papelito? De sus reflexiones
resultó que debían quedarse, con ligero cambio de personas. No hay cosa
más desagradable que dejar vacías las poltronas para que otros las
ocupen\ldots{} La gran escena cómica de hoy en la Cámara regia y piezas
inmediatas es de tal modo bochornosa, que me he quitado los coturnos por
zafarme de la obligación de contarla. Para dar noticia de lo que hoy he
visto, heme puesto estos borceguíes traídos y viejos\ldots{} Figúrate
que los sagastinos y unionistas han arreglado su guisote de crisis con
salsa de \emph{calamares}, y hoy se han presentado a jurar.

---Juran y perjuran poniendo su mano al revés sobre un falso Evangelio.

---¡Anda, que del indecoroso plantón que les dio el Rey, se acordarán
mientras vivan! Yo le dije a Sagasta: «¿No te sientes humillado? ¿Eres
un cochero que viene a pedir plaza en las Caballerizas?» Y él,
rascándose la barba, me contestó: «Paciencia, \emph{madre Clío}; este
oficio pide mucho aguante y resignación por arrobas. La política es
valle de bilis.» Dos horas les tuvo Amadeo en la antecámara. A lo mejor
salía Dragonetti con recaditos: «Dice Su Majestad que si traen el
programa.» Y el riojano de amarillo rostro y boca rasgada, respondía:
«El programa no lo traemos; pero\ldots{} se traerá. El amigo Colmenares
lo está confeccionando\ldots» \emph{Confeccionando}, como si fuera un
pastel o una torta de dulce\ldots{} Vuelve Dragonetti con dulzura
oficiosa, y dice: «Que si no traen el programa no juran.» Yo disimulaba
mi enojo hablando de teatros con la Marquesa de Constantina. El hombre
del tupé bajó al Ministerio de Estado con De Blas, y los que allí
quedaron se miraban asustados de su paciencia ovejuna. Me acerqué al
Ministro primerizo, y le dije: «Simpático \emph{pollo antequerano},
parece que estás triste. Te ha tocado un estreno de mala sombra.» Y él
desplegando su boca y mostrando su blanco dentamen, se sacudió así la
broma: «Madre, la buena sombra la traigo yo conmigo\ldots{} Sea usted
benigna, y dejaré memoria de mí.»

»Volvió de Estado Sagasta, con el tupé más crecido y la color más
biliosa. Traía también el programa que enseñó a los amigos. Como
reapareciese Dragonetti con nuevas chinchorrerías, Práxedes le dijo:
«Aquí está, aquí está el programa. Mañana lo verá Su Majestad en la
\emph{Gaceta}. Le hemos dado forma de Circular a los Gobernadores. Se
les dice que este Ministerio es estrictamente compacto, que somos el
progresismo histórico, firme columna de la Monarquía; y al propio tiempo
les encarecemos la más exquisita legalidad en las elecciones. Legalidad
ahora y siempre, para que el sufragio sea la exacta expresión de la
voluntad del país\ldots» Amén. Pasaron a la Cámara Real; hicieron
arrumacos de juramento\ldots{}

»Yo lo vi; hice cuanto pude para ponerme seria. Di una vuelta en
derredor de todos; pasé delante del Rey casi tocándole las narices, y ni
él ni sus desaprensivos secretarios me vieron. Fuertemente dirigidos
hacia los senos de su egoísmo tenían los ojos del alma, y los del cuerpo
estaban ciegos. «Si me vierais, hijos del aire---les dije,---no seríais
lo que sois.» Bajé corriendo a quitarme el calzado, que torpemente llevé
a las alturas. No merecen los de arriba mis tacones de plata\ldots{} Y
ahora, buen Tito, acompáñame. Quiero espaciarme, alegrar mi pobre
espíritu ansioso de verdad. Vámonos a la Fuente de la Teja, y allí
veremos a los soldados bailando con las criadas. Aquello, en su
humildad, es más noble que esto. De allí puede salir algo grande, de
aquí no. Iremos también a ver a los chicos jugando al toro o a la tropa,
en la Virgen del Puerto. De allí saldrán hombres de poder, ciudadanos,
trabajadores, mártires, héroes. Aquello es la sal y el fuego de la
vida\ldots{} Aquí no hay más que hombres de humo que burla burlando
asfixian a su patria.

Ya estábamos en la puerta con ánimo de no parar hasta la Fuente de la
Teja, cuando llegaron don Hilario y el Conde de Rius, que bajaban de las
habitaciones de Su Majestad la Reina doña María Victoria. Hablando
pasaron junto a nosotros, como si no nos vieran. Creímos entender que
había sido mala inteligencia del Conde la designación del señor Peña
para la Capellanía de Lavanderas.

«Le han llamado---me dijo \emph{Mariclío}---porque a esta buena señora
le ha dado ahora por hacer obispos. Cree con esto desarmar a las damas
católicas que le han declarado la guerra. Equivocada está de medio a
medio, porque aunque propusiera una hornada episcopal de sacerdotes
virtuosos y entendidos, el Papa no los aceptaría\ldots{} Así lo dije
ayer a doña María Victoria, y ella me aseguró que secretamente, y sin
que lo supieran don Amadeo ni Víctor Manuel, había tendido un hilo de
inteligencia con el Vaticano, y por este hilo le habían dicho que sí,
que propusiera\ldots{} ¡Ay, no sabe esta buena señora con quién trata!
Yo le dije: `No te fíes. Suponiendo que Pío IX entre por el aro, no te
preconizará más que obispos carlistones, afectos a él más que a ti y a
tu marido\ldots{} Hija mía, no te metas con Roma, ni creas que amansarás
a las apostólicas damas, poniéndote todos los moños del catolicismo y
del papismo\ldots{}'. Y este bienaventurado Hilario Peña no se calará
nunca la mitra. Es hombre bueno, sabio y caritativo. No tiene
ambición\ldots; no quiere \emph{obispar}. Ya sabes que pertenece a la
militar orden de \emph{Santiago el Verde}, quiero decir que \emph{es de
Caballería}.»

\hypertarget{xiv}{%
\chapter{XIV}\label{xiv}}

No sé cómo escapamos de aquel antro, que tal me parecía\ldots{} Salimos
oyendo la voz lejana de don Hilario que decía: «No, no; nunca.» En la
calle nos encontramos \emph{Mariclío} y yo, y apenas tomamos la
dirección que ella indicara, noté que su persona se iba despojando de la
dignidad señoril, y su vestimenta desluciéndose hasta tomar las
apariencias humildísimas con que la vi en la gruta de Graziella. Pero a
medida que envejecía y se vulgarizaba, era mayor su agilidad, y su paso
tan vivo que no podía yo seguirla sin sofocarme. Yo me preguntaba:
«¿Cómo ha podido cambiar tan pronto de traje y facha?\ldots{} ¿Y dónde
demonios lleva escondidos los zapatos de medio lujo?\ldots{} ¿Y cómo
salimos de Palacio sin pasar por ninguna de sus puertas?\ldots{} ¿Y qué
se le habrá perdido a esta buena señora en la Fuente de la Teja?»

Por Caballerizas, Cuesta y Puerta de San Vicente, Puente del Manzanares
llegamos al popular sitio de recreo. Hormigueaba en él la descuidada
plebe; sonaban en estridente algarabía los organillos, los pregones y el
gozoso runrún de los merenderos. Por entre la turbamulta paseamos;
\emph{Mariclío} habló con dos aguadoras, yo con un mendigo lisiado a
quien llevaban en un carrito\ldots{} Llegamos a donde militares y
muchachas habían armado el incansable bailoteo. Daba gusto ver el
entusiasmo con que ellas zarandeaban sus cuerpos en aquel ejercicio,
agarrándose al hombre o brincando frente a frente y haciendo graciosas
figuras. «El baile---me dijo mi compañera de paseo---es la primitiva
manifestación del arte y del amor. En su ritmo verás el aleteo con que
la especie humana dice: `No quiero morir, sino vivir y reproducirme'.»

Contemplando los enardecidos grupos danzantes, y luego las parejas que
entre los espesos olmos se alejaban buscando la soledad,
\emph{Mariclío}, con lenguaje que sólo entendíamos el viento y yo, les
decía: «Divertíos en la edad gozosa\ldots{} Soldaditos y criadas, chicos
y chicas que comenzáis la vida en la sana esclavitud de las
obligaciones, no os detengáis, y de estos devaneos inocentes pasad a
mayores devaneos\ldots{} Casados o sin casar, cread españoles, traednos
ciudadanos, que es menester venga nueva generación a enmendar a esta,
desvaída y decadente. Traed acá nuevos hombres de quienes yo pueda
referir acciones altas y nobles.»

Seguimos andando\ldots{} Yo era un autómata\ldots{} En la Virgen del
Puerto y la Puente Segoviana, nos cruzábamos con parejas a quienes
\emph{Mariclío} hacía la misma recomendación de aumentar a toda prisa el
censo de España\ldots{} «Nueva gente\ldots{} y pronto, pronto\ldots{}
Hombres que traigan cerebros machos, corazones grandes y ternillas a la
medida de los corazones\ldots» Pasamos luego por la \emph{Tela}, donde
vimos enorme caterva de chiquillos jugando a la tropa con palos,
banderitas y morriones de papel. Los más audaces se disputaban el mando:
\emph{Yo soy Plim}, chillaba uno, y otro gritaba: \emph{Pues yo
Napolión}. \emph{Límpiate}\ldots{} Un tercero venía dando zancajos y
vociferando así: \emph{Quitaos, gallinas, que yo soy mi abuelo, y mi
abuelo se llamaba el Tío Pecinado}\ldots{} Formaban batallones; batían
marcha imitando con la boca el \emph{rataplán} de los tambores;
disparaban tiros, se acometían al arma blanca, tomaban la fortaleza de
un montón de piedras\ldots{} \emph{Mariclío} se metió entre ellos y
fogosa les decía: «No desmayéis, valientes chicos. Creced y dadme tela
para que yo corte a vuestra patria un vestido espléndido, y dadme
materia para que ese vestido salga recamado con estrellas de oro\ldots{}
Mandaos los unos a los otros, recompensaos, castigaos, para que
aprendáis la justicia. Sed guerreros chiquitos para que de grandes seáis
buenos ciudadanos.»

Otras estupendas cosas les dijo, y ellos, exaltados por tan sonoras
palabras, no vieron mejor modo de expresarnos su conformidad que
apedreándonos. Las peladillas silbaban en nuestros oídos\ldots{} Era un
disparar impetuoso y graneado que no nos hizo daño alguno.
\emph{Mariclío} se descuajaba de risa, y sin miedo a la pedrea les
enardecía de este modo: «Bien, hijos: no importa que me ofendáis ahora
si mañana os portáis como dignos y valientes. Seguid, seguid
jugando\ldots» Embocábamos la calle de Segovia, cuando mi brava
compañera me habló así: «Tito mío, estas diabluras de los rapaces y el
embeleso de las parejas de enamorados, me consuelan de la mísera vida
que arrastro en esta tu decaída tierra. Veo que abres tus ojazos,
admirándome sin conocerme, deseando que te diga quién soy y te explique
por qué vine al mundo, y cuáles son mi abolengo y familia. Sentémonos en
este sillar que aquí está como preparado para nuestro descanso.» Nos
sentamos, y he aquí lo que me contó:

«Somos nueve hermanas\ldots{} No te diré cuál es más joven o más vieja,
pues nacimos juntas de un mismo vientre\ldots{} Nuestro padre nos dedicó
a diferentes artes. Cada cual escogió la más de su gusto. Una de mis
hermanas se dedicó a bailarina, y ha venido muy a menos; es más
desgraciada que yo, y hoy nadie le hace caso. Dos fueron cómicas: la una
se dedicó a la tragedia, la otra a la comedia. Andan hoy regular;
consideradas sí, pero muy discutidas\ldots; que si eres, que si no
eres\ldots{} La que estudió para oradora brilla y aparenta, mas con poca
substancia. La que se aplicó a la tarea de componer versos heroicos,
está por los suelos, más que yo quizás; la que hace versos alegres va
viviendo\ldots, da qué hablar, y los desocupados la festejan. La que
actúa de observadora del cielo y del curso armónico de los astros, goza
de gran predicamento. Pero la que ha subido más en el aprecio de las
gentes y más éxitos alcanza es la que eligió el arte de la música, del
dulce canto y tañer de concertados instrumentos. A mí ya me ves. No
valgo para nada, por falta de materia con que pueda dar al mundo muestra
y señales de mi grandeza\ldots{} Las nueve hermanas nos vemos y nos
visitamos a menudo para comunicarnos nuestras glorias y desdichas\ldots»

Cuando esto decía, ya no estábamos en la calle de Segovia, sino
internados en las calles más bulliciosas de Madrid. Mi interesante
compañera se detuvo en un punto, donde oíamos dulcísimos acentos de
violines y de humanas voces melodiosas, y despidiéndose me dijo así:
«Aquí me quedo, que siento la voz de mi hermana, la que rige y gobierna
los reinos de la Música, y subiré a pasar un ratito en su
compañía\ldots{} Vete a descansar, que bien lo necesitas\ldots{} Haz por
dormirte; olvida lo que conmigo has hablado y visto, que todo es
figuración y embuste de tu cerebro enardecido y no muy sano\ldots»

Dejé de verla a mi lado\ldots{} Mi camino seguí claudicante y haciendo
eses\ldots{} Esto de las eses que yo hacía me puso en gran cuidado, pues
no recordaba yo haber bebido ni una gota de licor espirituoso. Alguna
cuchufleta oí referente a mis eses\ldots; la cabeza me pesaba como si en
ella se me hubiera metido todo el azogue de las minas de Almadén\ldots{}
No puedo asegurar cómo y en qué postura llegué a mi casa; pero es
indudable que en ella y en mi cama me encontré por la mañana, como quien
despierta, o más bien resucita\ldots{} Apenas puse mis huesos de punta,
me lié con Ido del Sagrario en agria disputa. Empezamos por sostener, yo
que las Musas eran diez, y él me contradijo con burlas diciendo que no
eran más que nueve, quizás ocho no más, pues una de ellas, la de la
Historia, se había dado de baja por no tener ya cosa bella o grande que
contar\ldots{} Estallé yo en cólera; quise pegarle, y habríamos tenido
en casa una tragedia si no entrara Nicanora con zorros y una estaca para
restablecer la paz. ¡Cómo estaría yo en aquellos días, que no hablaba
con ningún amigo sin que acabáramos poniéndonos de vuelta y media! Con
Mateo Nuevo reñí tan ásperamente que faltó poco para enredarnos a
pescozones. Por una palabra, por una sonrisa, desafié a Luis Blanc y a
Roberto Robert. A Ramón Cala, por haberme recomendado moderación en la
bebida (yo no lo cataba), le mandé los padrinos, que fueron Ido del
Sagrario y Roque Barcia.

Divagando solo, examinaba lo que bien puedo llamar mi conciencia mental,
y sentía que alguna pieza del aparato pensante no se hallaba en perfecto
engranaje con las demás. Yo quería pensar una cosa y me salía otra.
¿Cómo restablecer la ordenada función de mi cerebro? Consulté el caso
con Ido, muy práctico en tales achaques, y me dijo que tomase mucha tila
y no leyera más libro que \emph{Las Tardes de la Granja}, obra muy
distraída, o la \emph{Vida de Santa María Egipciaca}, que a él le había
probado muy bien. En esta situación de espíritu, llegaban a mí ecos
zumbantes del estruendo político en las Cortes y en la Prensa. A Sagasta
y Romero Robledo, \emph{el gallo de Cameros} y \emph{el pollo de
Antequera}, les traían locos por la transferencia de dos millones, que
la gente maleante dio en llamar \emph{Los dos apóstoles}. Traviesos eran
Sagasta y Romerito, y no reparaban en pelillos para engrasar la máquina
electoral. Y aun así no pudieron impedir que trajeran acta treinta y
cinco carlistas. Estos se preparaban en el Norte para obsequiarnos con
otra guerra civil\ldots{} ¡Bueno se iba poniendo esto\ldots!

Mis amigos políticos y particulares huían de mí, o me trataban como un
caso patológico. La vagorosa Delfina se presentó en mi casa un día,
luctuosa y con un negro velo por la cara, y en tono dulzaino y lúgubre,
revelándome su doble procedencia confitera y funeraria, me dijo: «Tito
de mi alma, tus amigos no hacen más que compadecerte; yo te compadezco y
trato de curarte. Ya escribí a tu familia. Tendrás pronto remedio. La
vida campestre te probará muy bien. Yo, cuando me quedé viuda, estuve
también algo tocada, y con dos meses de andar al zancajo en una dehesa,
pastoreando vacas y subiéndome a los alcornoques, sin cuidarme de que
los zagales me veían las piernas, me puse buena, y tan fuerte que al
volver habría podido levantarte en vilo para darte azotes, como lo hice
después\ldots{} bien lo sabes.»

A los tres días de esta visita, hallábame yo recién salido del lecho,
sentadito en incómodo sillón de gastados muelles y desiguales pelotes.
Trájome Ido mi desayuno, y apenas lo tomé con menos que mediano apetito,
me sumergí en hondísimas reflexiones. ¿Adónde iría con mi cuerpo aquel
día? Estando en los senos cavernosos de esta meditación, la mirada en el
suelo, el dedo en la frente, oí ruido de voces que venían del
recibimiento\ldots{} Alcé los ojos, y en la puerta de mi cuarto vi un
bulto, una persona que allí apareció como clavada. Era tan semejante a
mí, que creí ver la reproducción de mi figura en un espejo\ldots{} El
sujeto que suspenso me miraba era chiquitín como yo, con mi propia cara
más curtida, cabello gris y\ldots{} Lo diré de una vez. Aquel señor era
mi padre.

La inesperada presencia del autor de mis días sacudió todo mi ser,
privándome del habla por un mediano rato\ldots{} Y el pobre señor, más
envejecido que viejo, se conmovió intensamente al verme tan alicaído, si
bien su pena no tardó en dulcificarse, pues por la carta angustiosa de
Delfina, temía encontrarme en un manicomio. Pasada la efusión primera, y
dada cuenta de toda la familia, mi padre planteó la cuestión secamente.
Había venido por mí. Yo no dije nada; me sentía máquina rota. ¿Y cuándo
nos iríamos al pueblo\ldots? Aquella misma tarde. «Bueno\ldots{} pues
vámonos.» Así dije, y mi padre dio las órdenes a Ido para que aprontara
mi ropa y todo mi bagaje, con excepción de libros, pues no consentía que
llevase conmigo las causas de mi desarreglo mental, que eran la vida
loca de Madrid, el hervidero de las ideas disolventes, y las lecturas de
obras perversas que inducían a la inmoralidad y al crimen\ldots{} Él no
tenía nada que hacer en la Corte, que odiaba y maldecía\ldots{} «Yo no
me separo de ti---me dijo.---Tomaremos un bocado al mediodía\ldots; yo
con un caldo me arreglo. Hoy es vigilia de precepto.»

Fuimos a visitar a Delfina, y largo rato platicó mi padre con ella,
recordándole sus bondades con mi familia. Y entre otras remembranzas de
gratitud, sacó de la obscuridad del pasado la siguiente: «Usted,
Delfina, ha sido muy buena para nosotros. Cuando vino a Madrid el año 67
mi hermana Bonifacia con su marido, a consultar a los médicos su
enfermedad del pecho, estaba usted recién casada. Acompañó a mi hermana
en el visiteo de doctores; le regaló una magnífica torta de dulce, y
cuando el pobrecito Manuel murió, no quiso usted cobrarle nada por el
ataúd y hachones\ldots{} Esto no lo olvida mi hermana, que ahora vive en
Burgos.» Con estas y otras finezas nos despedimos, y Delfina me dio un
escapulario y una cajita de bombones de chocolate para que me
entretuviera por el camino\ldots{} Dos horas después, estábamos ya en la
estación del Norte, con una hora de anticipación a la de la salida del
tren, pues mi padre temía que este se le escapara, dejándole un día más
en este Madrid, objeto de todo su asco y aversión.

En marcha el tren, llevando en nuestro departamento de segunda tres
compañeros y dos compañeras de camino, mi buen padre, libre ya de la
inquietud del regreso, y gozoso de llevarme consigo, me franqueó sus
cariñosas intenciones. «Hijo mío, creo que sólo con sacarte del
laberinto de ese Madrid arrastrado y disoluto, te curarás de tus murrias
y del desvarío de tu cabeza. Te inficionaron los miasmas del vicio y de
la corruptela, ¿no entiendes lo que te digo?\ldots; pues corruptela
quiere decir el burlarse de las leyes de Dios, el no amarle ni temerle,
el andar en el tole tole de libertades, que yo llamo licencias, y el
querer meternos a los españoles en un fregado de ideas pestíferas y,
como quien dice, republicanas. Te lo diré más claro\ldots{} En los aires
limpios del pueblo soltarás toda esa podredumbre, y serás otro
hombre\ldots{} Echarás de tu cabeza todo el maleficio, dejando que entre
poquito a poco, como ave que busca su nido, la paloma del Espíritu
Santo.»

De esta figura que de su boca salió envuelta en seráfica sonrisa, debió
de quedar muy satisfecho el buen señor, pues con ella puso punto final,
y apoyando su venerable cabecita en la palma de la mano, se durmió como
un ángel. Era mi padre, don Matías Liviano y Pipaón, un hombre bueno y
simplísimo, incapaz de hacer daño a una mosca, de ideas petrificadas,
patriarcales, resultado del vivir estrecho en pueblos de corto
vecindario, sustrayéndose sistemáticamente a todo contacto con el vivir
que irradia de las grandes ciudades del reino. Alavés de nacimiento, se
estableció desde muy joven en Oña, patria de mi difunta madre, doña
Pascuala Zurbano y Calomarde. En Oña, el Cubo y Medina de Pomar poseían
mis padres algunas tierrucas y dos o tres casas de mala muerte con que
disfrutaban de un pasar modesto, insuficiente para los hijos que
aspirábamos a mejor vida. Mis dos hermanas casaron, la una con un
bigardo vizcaíno, bien cubierto del riñón, vamos al decir, rico; la otra
con un viudo joven de Miranda de Ebro, que traficaba en vinos de Rioja.
Yo, el más chico de la familia en edad y estatura, pues a mis hermanas
les tocó la talla que a mí me faltaba, anhelé desde niño horizontes más
amplios, y cuando pude valerme solo, me fui a Vitoria en busca de
alimento con que saciar mi apetito mental. No hallándolo en la capital
de Álava, planteme en Madrid, desde donde anudé relaciones con mi padre,
ofreciéndole villas y castillos, y pronosticándole mi próxima,
indubitable celebridad.

El sueño no quiso apagar mis arrebatados pensamientos. Mi desvelo fue
parte a que me fijase en una señora que a mi vera estaba, la cual durmió
hasta más allá de Ávila, y poco después, volviéndose a mí, me preguntó
que cuánto faltaba para llegar a Bribiesca. Al contestarle que allá iba
yo también, vi que era de agradable rostro, lozana y risueña. Al
instante reapareció en mi ser el caballero galanteador, sentí mi cabeza
despejada, y mi corazón henchido de amor a toda la humanidad femenina.
Empecé por acometerla con discretas finuras, sondeando hábilmente su
receptividad galante. Mantúvose firme un buen rato, ni admitiendo ni
rechazando las varas que yo quería clavarle; mas yo saqué las armas
retóricas de mi arsenal persuasivo, y a poco de medirlas con la recatada
concisión de la dama, supe que era viuda sin hijos, y que tenía fincas
en la Bureba\ldots{} Poco a poco fue entrando en el nimbo de simpatía
que sé formar entre mi persona y una blanda hembra. Desde Medina a
Valladolid la dama recompensaba mi rendimiento con sonrisas, y un juego
de ojos que fue como si las estrellitas del cielo se colaran en la
penumbra del coche. Más animado yo en cada estación, pues por estas
contaba yo las etapas de mi aventura, rompí a cantar, cerca de Burgos,
la cavatina de mi declaración, con la mala pata de que en los primeros
compases despertó mi padre, y estirándose y bostezando exclamó:
\emph{Alabado sea el Santísimo Sacramento del Altar}. Al terminar la
frase, hizo la señal de la cruz sobre su boca, y sacando el rosario se
puso a rezar\ldots{} Me había cortado el resuello\ldots{} ¡Ay, si no
fuera mi padre\ldots! Entre dos avemarías, pronunciadas a media voz, me
dijo: «Tito, ¿te encuentras bien? ¿Has podido dormir?»

---Sí, padre; he dormido. Estoy tan bien, tan bien, que ya se me han
quitado todos los males, y me siento tal y como fui en mis días de
fuerte salud, enteramente \emph{conmutativo y bilateral}.» El pobre
señor no me entendía, y siguió despachando su tercio de rosario.

\hypertarget{xv}{%
\chapter{XV}\label{xv}}

A poco de pasar de Burgos, envainó mi padre su rosario suspirando ya por
la llegada, y aunque sobraba tiempo, diome prisa para que recogiera
nuestros bultos y paquetes. «Por Dios vivo, Tito, no se nos quede algo.»
La señora guapa se arregló la cabeza y toquilla dirigiéndonos una mirada
que me pareció precursora de inteligencia. Sin duda le supo mal el
quedarse a media miel cuando el despertar de mi padre cortó bruscamente
la volcánica declaración que yo empecé a espetarle. «Hasta que pase
Santa Olalla no hay prisa» nos dijo; y en su acento creí notar cierta
dulzura que a mí solo dedicaba. Llegamos, y al ponerse en pie la señora
para salir vi con espanto que era coja, pero de una cojera de
solemnidad, pues tenía una pierna de palo, y se ayudaba de un
bastón\ldots{} En ninguna de mis conquistas, tuve tan \emph{mala
pata}\ldots{} Hice como que no me enteraba, y extremando mi finura y
prodigando las expresiones más corteses, la ayudé a bajar del coche. Los
demás viajeros seguían durmiendo profundamente. El frío era
intensísimo\ldots{} De mi brazo pasó la dama coja a los brazos de
personas que la esperaban\ldots{} Mi padre saludó a un cura, y luego al
dueño de los coches que llevaban diariamente el correo desde Bribiesca a
Medina de Pomar, pasando por Oña, nuestro pueblo\ldots{} Descansamos;
amaneció, y ¡al coche\ldots! Antes de las diez estábamos en la risueña y
monacal villa de Oña, donde me crié, y con las primeras travesuras
realicé mis primeras infantiles conquistas.

Declaro que me rejuvenecí y me fortifiqué con sólo pisar el suelo de
aquella villa guardadora de mis dulces recuerdos. El convento de
benedictinos con su iglesia y claustros y frondosas huertas, que
conservaban aún a mi parecer la huella de mis zapatitos agujerados a
poco de estrenarlos, renovaron en mi espíritu las alegrías de la niñez.
Con placer indecible me recreaba en las verdes orillas del río y en los
embalses de cristalinas aguas que los frailes tenían para sus recreos de
natación y pesca\ldots{} La menguada población me divertía menos. En el
tiempo que yo faltaba de allí, aumentado había el rebaño de curas; la
beatería del vecindario era ya un estado epidémico\ldots{} Para mí,
pasar de Madrid a Oña era como saltar de un planeta a otro. Mi padre,
que con tanto desprecio y horror hablaba de los \emph{miasmas} de
Madrid, no se daba cuenta del aire espeso de fanatismo que allí
respirábamos. Felizmente, corta sería nuestra estancia en Oña, y
cobrados unos cuartejos de la renta de dos casuchas y tierras pobres,
seguiríamos hasta Durango, donde mi padre, desde su viudez, vivía con mi
hermana Trigidia (nombre de una santa oñense), bien casada y
establecida.

Con mal tiempo y buen humor, metidos mi padre y yo en vehículos que
variaban de lo malo a lo pésimo, emprendimos la peregrinación hacia
Frías; de allí por el valle de Tobalina seguimos a Miranda de Ebro,
donde nos detuvimos para pasar un día con mi hermana Pascuala. De
Miranda seguimos en tren hasta Vitoria, y otra paradita, pues mi padre
no pasaba por allí sin visitar a sus parientes los Pipaones y Suredas,
todos redomados carcundas. La última etapa fue de Vitoria a Durango, por
Ochandiano, paso de la Peña de Amboto\ldots{} Y heme aquí, lectores que
bondadosos me seguís de mazo en calabazo, heme incrustado en una
sociedad de sentimientos y pensares tan opuestos a los míos, que me tuve
por transportado, no digamos que a otro planeta, sino al más lejano de
los mundos siderales. Vivía mi hermana en casa holgona, del tipo más
patriarcal. Su marido, Ignacio Zubiri, estaba ausente. Guardábase en la
familia cierto misterio, que al fin descifré suponiéndole en la facción.
Fruto lozano de este matrimonio eran tres chicos sonrosados y
mofletudos. Trigidia se alegró mucho de verme; como mi padre, celebraba
que me hubieran traído del infecto ambiente de Madrid a la sanidad de
los valles risueños entre montañas. Halagado de la buena vida material,
yo simulaba un apego mansurrón a la verde Vasconia.

La verdad, yo comprendía y admiraba las sólidas virtudes de la raza, su
contumaz apego a la tradición, cualidad meritoria cuando sirve de punto
de partida para el progreso, como acontece en Inglaterra; me agradaba la
lealtad de los hombres, la lozanía de las mujeres; los alimentos eran
muy de mi gusto: la rica ternera, el pescado que los más de los días
traían de Mundaca o Elanchove, las gallinas, patos y abundancia de
verduras que mi hermana recibía diariamente de sus caseríos. Las
borrajas, las habas, nabitos, y cuanto constituye la nutrición castiza
en el país, satisfacía mi paladar y me restauraba el estómago, tan
necesitado de vida nueva. Lo que no me entraba ni con escoplo era el
habla. Toda mi atención no era bastante para entenderla, y ni el oído ni
la mente podían habituarse a tan archiengorrosa cháchara. Mayormente me
afligía ver en el vascuence un valladar, un tremendo aislador para todo
amoroso intento. Siempre que inicié la conquista de alguna garrida
hembra campestre o frescachona criada, el maldito lenguaje me
descomponía y me desarmaba, pues ni yo les entendía una palabra, ni
ellas a mí más que si les hablara en lengua chinesca.

En aquel pueblo y en ambiente tan apropiado a un espíritu enteco, vivía
mi buen padre como si estuviera en las antesalas del Paraíso. Desocupado
y con sus cortas necesidades satisfechas, vegetaba y dormitaba como un
bendito a la sombra del dogma, que en aquel país es como una bóveda
solemne que protege y abriga las almas. En su credulidad candorosa, el
pobre don Matías Liviano y Pipaón no veía nada más allá de su vivir
cómodo, en lo material, y de su pensar estrecho dentro de la elemental
esfera religiosa. «Así lo encontramos y así lo hemos de dejar, hijo
mío,» era su única réplica cuando yo me permitía deslizar en su oído
alguna observación conforme a mis ideas. Viéndole tan tranquilo, tan
feliz dentro de su redoma, me parecía crueldad impertinente
contrariarle. Si le hubiera dicho que no creo en el Infierno, le habría
ocasionado tal vez un catarro gástrico, tal vez un ataque a la cabeza;
que su flaca salud pendía de cualquier disgusto. Si yo le hubiera dicho
que el Purgatorio no es más que un establecimiento industrial y
mercantil, de cuyos pingües rendimientos se nutre el cuerpo de la
Iglesia, el choque de mis ideas con su inefable quietud le habría quizás
provocado un torozón que le llevara al otro mundo. Y aunque él creía
tener asegurada la gloria eterna, por el pronto le iba bien aquí con las
borrajas, las habas, la merluza en salsa verde, los pichones y las
sabrosas sardinas de Elanchove.

Por esta causa, yo no me metía en discusiones con él ni con mi hermana,
ni con ninguna de las personas que a casa concurrían. Y aun le guardaba
la fina consideración de acompañarle en sus frecuentes visitas a Santa
María, seguidas de inmersiones larguísimas en la casa del cura, vicario
o arcipreste que en aquella santa iglesia gobernaba, con otros, las
almas duranguesas. Para sobrellevar tan fastidiosos plantones no tenía
yo paciencia, y esperaba al santo varón paseándome en el espacioso atrio
de la iglesia, donde me entretenía viendo salir y entrar chicas guapas,
no por beatitas menos interesantes.

Buena parte del tiempo que allí me sobraba, invertía yo en pasearme por
las anteiglesias o pueblecitos que rodean la villa. A todas las mujeres
que encontraba les pedía plática, con idea de ejercitarme en el
vascuence, lengua preciosa, les decía yo, que deseaba poseer\ldots; como
que mi estancia en Durango no tenía más objeto que aprender el idioma
vasco. Ya poseía veinticuatro lenguas, entre ellas todas las orientales,
y además el catalán y el chino. Con estas y otras sutilezas iba entrando
en la confianza de ellas, y como ya sabía no pocas frasecillas éuskaras,
me divertía, bromeaba, y con alguna logré asomos de intimidad, que
andando días llegaron a mayores, proporcionándome sabrosos ratos a la
sombra de espesos laureles o nogales.

Fuera de estos experimentos harto arriesgados y de compromiso, vivía yo
confinado en la desabrida normalidad de la casa y sociedad de mi
hermana, rezando el rosario con mi padre, oyendo la cancamurria de los
ojalateros que le hacían la tertulia, o el relato de lo que ocurría en
la facción lejana. Mi único recreo, las más de las tardes, era jugar a
la pelota con mi sobrino mayor y otros chicarrones del pueblo, en el
trinquete próximo a \emph{Barrencalle}, donde vivíamos.

Por las noches, arrimados a la lumbre si hacía frío, o reunidos en la
sala baja, había de aguantar el chaparrón de la ojalatería carlista, que
ni poco ni mucho me importaba. Ello era como vivir en un Limbo todo
tristeza nebulosa, y ya me cansaba ¡por Júpiter!, tan miserable vida.
Los asistentes a la casa eran vecinos de mi hermana y amigos de su
marido, algunos curas que olían a pólvora, y hombrachos aguerridos que
apestaban a incienso.

Una noche vi a mis buenos ojalateros tan movidos al optimismo, que hube
de prestar más atención a sus ardorosos comentarios. Según noticias
mandadas con un propio por mi cuñado Zubiri desde Lecumberri, donde a la
sazón estaba, el grito se daría muy pronto en la frontera de Navarra,
proclamando la Monarquía cristiana y su cabeza don Carlos, \emph{alias}
Duque de Madrid, nieto del glorioso don Carlos María Isidro. Habían
concluido, pues, las vacilaciones entre los consejeros del Rey; ya los
Elíos, los Radas del orden militar, los Morales y Manterolas del civil y
eclesiástico, habían superpuesto su opinión guerrera a la de los
Nocedales y Canga-Argüelles que en los ocios de Madrid predicaban la
paz. Ya el hijo de cien reyes, por la recta línea masculina,
desenvainaba el acero, y seguido de sus leales, pasaba la raya de
Francia, y con bravura y ardor repetía la frase guerrera del comunero
episcopal Acuña: \emph{¡Adelante mis clérigos!}

La buena sombra, que a todas partes me acompaña, deparome un amigo, cuya
compañía y grata conversación suavizaban la rigidez monótona de mi vida
en aquellos días de Mayo. Era el tal un donoso cura, don José Miguel
Choribiqueta, rector de la iglesia de San Pedro de Tavira, viejo ya el
hombre y cascado, algo enfermo de los ojos, que recataba con vidrios
verdes, carácter jovial, ameno y comunicativo. Asistente por rancia
costumbre a la tertulia de mi hermana, se aburría como yo de las
ojalaterías enojosas, y me hacía el favor de sacarme de paseo por las
alegres campiñas. En cuanto le traté, vi en él a uno de esos hombres
que, habiendo realizado en la plenitud de la vida lo que le imponía su
conciencia, llevando a la esfera de los hechos su fe, su valor y su buen
criterio, miraba con desdén a los que imitar querían en peores tiempos
los mismos actos y las mismas virtudes, o lo que fuesen. Don José
Miguel, héroe de la \emph{otra guerra}, no podía desechar la idea de que
lo pasado fue mejor, ni admitía que hubiera dos epopeyas en un mismo
siglo.

«A solas con usted, señor don Tito---me decía en castellano corriente,
aunque un poco turbio,---me reiré de estos majaderos, que quieren
repetir\ldots{} ya, ya\ldots{} para repeticiones estamos. Aquellos eran
otros tiempos, aquellos eran otros hombres\ldots{} Dígame usted, señor
don Tito, qué guerra pueden hacer, ni qué lauros conquistar Fulgencio
Carasa y Jerónimo García\ldots»

---No les conozco, amigo mío, y esos nombres escucho ahora por primera
vez.

---Pues no pierde usted nada con no conocerles\ldots{} Como si el mandar
tropas fuera cosa de juego\ldots{} Oiga usted. Yo mandé tropas desde el
33 hasta el convenio de Vergara, que Dios confunda; yo tengo mi cuerpo
lleno de agujeros, cicatrices y costurones. Yo\ldots; no es que yo lo
diga\ldots{} Ahí están los partes de la campaña, desde el gran
Zumalacárregui hasta el bribón de Maroto\ldots; en algún archivo
estarán\ldots; véanlos\ldots{} Pero no hablemos de mi humilde persona.
Yo le pregunto a usted si puede esperarse algo bueno de Jiménez de Rada,
que fue liberal y conspiró con Prim para traer la Revolución llamada de
Septiembre\ldots{} ¿Se concibe, pregunto yo, que Valdespina pueda hacer
algo? ¿Y de Calderón qué me dice? ¿En Elío tiene usted confianza?

---Yo, ninguna. No les conozco siquiera\ldots{}

---Y puesto a comparar, mi señor don Tito, diré a usted en confianza que
entre este reyezuelo y aquel otro respetable y sentado cristianísimo
monarca don Carlos María Isidro hay alguna diferencia\ldots{} me parece
a mí\ldots{} Y dígame ahora, hágame el favor, dígame: ¿Dónde tenemos un
Zumalacárregui, un Villarreal, un Gómez, un Zariátegui, un
Cabrera?\ldots{} En cambio, veamos los que han salido a la
palestra\ldots{} ¿Pero no se ríe usted? Yo me descuajo de risa. Han
salido armados de punta en blanco, Canaelechevarría y Solís, dos
clerigachos guerniqueses, que no pueden ni con el hisopo\ldots{} Le digo
a usted que esto es un paso de comedia\ldots{} También ha ido el
danzante de Urraza, síndico del Ayuntamiento\ldots{} Y ahora, mi buen
don Tito, no se enfade si le digo que su cuñado de usted, el marido de
Trigidia, Ignacio Zubiri, que anda no sé por dónde haciendo el papelón,
es un calzonazos que se asusta de ver pasar un conejo\ldots{} ¡Bonita
guerra nos traerán, bonita! Yo barrunto que estos van a su
negocio\ldots{} Guerra y guerra de figurón, para luego venderse al
Gobierno de Madrid, y pescar grados y galones. Otra vez el infiel Maroto
que vendió como carneros a los hombres de fe, a los guerreros cristianos
de España\ldots{} ¡Oh, España! ¿Quién te sacará de esta miseria?\ldots{}
Los leones que pelearon en aquella soberbia campaña, o se han muerto, o
están como yo con una garra en la sepultura. Nuestro galardón no está
aquí sino allá---añadió con solemnidad señalando al cielo con su
cayada.---Dios nos acoge en su santo seno, y dice a estos malos
imitadores: «Mequetrefes, no intentéis lo que es superior a vuestra
flaqueza. Dejad las armas hasta que me plazca resucitar a mis hombres, y
les mande a defender mi causa.»

En otro paseo, oyéndole los mismos o parecidos razonamientos, le dije:
«Según veo, esto será nube de verano, y todo acabará en corto tiempo,
por la poca lacha de la gente nueva y el abandono del Gobierno\ldots»

«No se duermen, no, los fantasmones de Madrid. Ya tiene usted a Serrano
en campaña. Ayer estaba en Tafalla\ldots{} ¡Por mi patrón San Miguel,
que no me dieran a mí más trabajo que hacer polvo a esos Serranos y a
esos Moriones, generales de teta, que aún no han llegado a la dentición
militar! Oiga usted, amigo: En uno de los encuentros que tuvimos con los
cristinos al retirarnos de Peñacerrada, no copamos a Espartero porque el
General Guergué, que entonces nos mandaba, no hizo caso de mí, que a
cada momento le advertía sus errores tácticos. Y a pesar de ello, supe
arrollar al entonces coronel don Juan Zabala, matándole mucha gente, y
al maldito Zurbano le tuve cogido\ldots{} Fue cuestión de minutos, señor
don Tito\ldots; debió la vida suya y la de su tropa al socorro que le
dio de improviso el General Rivero\ldots{} Pues verá usted otra: Días
adelante, mandaba yo la Caballería del General don Julián Alzaa\ldots{}
No tiene usted idea de las palizas que le di a Zurbano en Arechano, en
Gamarra y en otros lugares de Álava\ldots{} Pues digo, también el
General León me conocía\ldots{} Menudas cargas nos dimos, y si los
falsos historiadores le dicen a usted que en Belascoaín quedó vencedor
el Leoncito, no lo crea usted. El vencedor fue este cura.» Dijo esto
puesta la mano en el pecho, parándose, con lo que dio a su figura un
aspecto estatuario.

---Ha sido usted un héroe, señor Choribiqueta---le dije poniendo en ojos
y boca todas las formas de admiración.---He oído que también estuvo
usted en Ramales y Guardamino.

---Allí estuve\ldots{} ¿Cómo no? Bien armada se la teníamos a Espartero.
Pero la cobardía de Maroto nos birló la victoria\ldots{} El tal Maroto,
desde los fusilamientos de Estella\ldots{} y yo fui de los que escaparon
de milagro\ldots{} venía tramando su infame traición al Rey legítimo.
Bien nos la jugó a todos. Yo he servido a la causa de Dios desde sus
comienzos hasta que Maroto nos vendió miserablemente en el llano de
Vergara. En el Infierno está pagando su culpa\ldots{} Yo he servido a
las órdenes de Zumalacárregui, de Villarreal, de Cástor Andéchaga, del
Conde de Negri, de Guergué y de otros guerreros abnegados y valientes;
serví y luché sin ambición, despreciando ascensos, despreciando pagas,
comiendo un pedazo de pan y unas habas mal cocidas después de veinte
horas a caballo, o de medio día de combate; yo no miré jamás a ninguna
ventaja temporal; no miraba más que a Dios y a su santa doctrina\ldots{}
Cuando salí de mi casa para entrar en la facción, llevaba en mi cinto
sesenta y cinco duros, y cuando a mi casa volví después de la traición
de Vergara traía dos pesetas en plata, y otra, o poco más, en
calderilla\ldots{}

---¡Bien por los hombres valientes y honrados---exclamé---que sacrifican
a una finalidad altísima la conveniencia personal y la propia
vida!\ldots{} Y ahora, don José Miguel, me va usted a permitir que le
haga una pregunta: Cuando, terminada la campaña, dejó usted la
existencia militar para restituirse a la eclesiástica, ¿no sintió en su
alma los efectos de transición tan violenta?\ldots{} Yo me figuro que
usted no sabría ya ser cura\ldots; vamos\ldots{} que se le habría
olvidado hasta la misa, el modo de decirla\ldots{} y el rosario y las
preces más usuales.

---Le diré a usted. Cuando a mi pueblo y hogar volvía, con la pena del
convenio, deshecha y arrojada en el polvo la causa de Dios, venía yo
pensando eso mismo que usted dice, que se me había olvidado todo el
ritual\ldots{} Pues verá usted, señor don Tito: yo fui siempre especial
devoto de la Purísima Concepción. La Dulcísima Señora, San Miguel
Arcángel y el Señor San Pedro fueron y son mis abogados así en la guerra
como en la paz. A la Reina de los Ángeles me encomendaba yo en todos mis
aprietos, y con su amparo y el de los santos que nombro, salí felizmente
de todos los peligros\ldots{} Como digo, venía yo mustio y desconsolado
en un jamelgo que me proporcionó el cura de Placencia, y al divisar la
torre de mi pueblo querido, se me ensanchó el corazón\ldots, me entró en
el alma una luz celestial, y volviendo toda mi voluntad hacia la
Purísima Señora, le pedí que a la memoria de su siervo humilde volviera
todo lo que pudo olvidar en los trajines de la guerra\ldots{} Fue para
mí aquel momento el más solemne de mi vida, puede usted creerlo, momento
en que me sentí comunicado con la Virgen Santísima y con mis celestiales
patronos\ldots{} Esto no lo comprenderá usted, esto no está al alcance
de las personas de fe poco ardorosa. Pues bien, llego, me desmonto del
rocín, me quito las espuelas, y entro en la iglesia. Lo mismo fue verme
bajo la bóveda obscura, que recordar de golpe lo que había olvidado. Mi
memoria se vació de todo lo de la guerra, y se llenó de todo lo
eclesiástico. ¡Virgen Inmaculada, qué cosas! Lo que usted oye\ldots{} A
la media hora de mi llegada, me revestí y salí a decir mi misa.

\hypertarget{xvi}{%
\chapter{XVI}\label{xvi}}

Me entretenían lo indecible las conversaciones con el amable cura, tipo
singular del más violento hibridismo que puede ofrecernos la naturaleza
humana. Sólo España, fecunda en ingenios, en héroes, en santos y en
monstruos, nos da estos engendros de la razón y la sinrazón, de la fe
mística y el orgullo marcial fundidos dentro de un alma\ldots{} Y debo
añadir que el bravo veterano Choribiqueta era en su vejez un venerable
padre de almas, que cumplía sus deberes escrupulosamente y ejercía la
caridad con verdadera efusión cristiana.

Tanto como me agradaba la épica historia del clérigo y su franco
carácter, picante mixtura de lo divino y lo humano, me entristecía la
sociedad de mi casa, donde se oía tan sólo el áspero zumbido de los
ojalateros, y el comentar de verídicos o fantásticos incidentes de una
guerra lejana. Iban y venían emisarios, llevando masas de juventud y
trayendo noticias de las gestas de Navarra. También se hablaba de
política o sucesos de Madrid, afeándolos con groseras burlas. Había
caído el Gobierno de Sagasta, por la porquería de dos millones que el
Sagasta y un tal Romero habían sustraído de la caja del Tesoro público
para llevárselos a sus propias cajas. Decíase que si los gastaron en
elecciones; que en Madrid, el dinero es el mejor cebo para pescar votos;
que si los gastaron en comilonas y regalos a señoras guapas, cosa en
Madrid corriente por ser pueblo de continuos festejos y
cuchipandas\ldots{} En las Cortes se armó tal rifirrafe por este alivio
de dos millones que hicieron al Tesoro los indignos administradores del
procomún, que el Gobierno se tuvo que retirar, lavándose las manos con
el agua del río Manzanares, que es agua muy sucia\ldots{} Naturalmente,
vino otro Gobierno, con el indispensable Serrano al frente, llevando de
compañeros a Topete, a un señor Ulloa, a otro que llamaban Candau, a un
tal Elduayen y a otro que respondía por Balaguer. Estos señores, salvo
Serrano y Topete, que con Prim componían la trinidad revolucionaria,
eran para la gente duranguesa muy conocidos en sus casas.

Corrió Serrano a Madrid a tomar posesión del mando político, y
encargando al Topete que le hiciera la vez, como cabeza del Consejo de
Ministros, se volvió al campo de la guerra\ldots{} En tanto, mi padre,
mi hermana y otras personas que por su metimiento en la casa eran como
de la familia, apartaban a ratos su atención del grave negocio bélico
para ocuparse de mí. Queriendo resolver de golpe y porrazo el problema
de mi vida y asegurarme la felicidad, decidieron casarme\ldots{} ¿Con
quién? Con una zagalona, más alta que yo en media vara, llamada Facunda,
hija de un pariente de mi cuñado Zubiri, y heredera de cuatro caseríos
de valor, según dijeron, situados en la risueña vega que fertiliza el
río Durango. La que me destinaban para compañera de mi existencia en
todo lo que esta me durase, era\ldots{} Dejadme tomar resuello, que esto
es muy grave.

Era una muchachona desgarbada, más sosa que las calabazas que a mi
parecer crecen a la puerta del Limbo; tan cerrada en el habla vascuence,
que apenas podía decir en castellano frases premiosas, trabucando los
casos, descoyuntando la sintaxis como lo harían los mismos demonios.
Desde que la vi, me fue atrozmente antipática, por su ceño displicente,
la sequedad de su trato, y algo que en ella noté, como sombra o trasluz
de un brutal fanatismo. Casándome con ella, según me manifestó mi padre
en una sesuda conferencia, sería yo poseedor de cuatro caseríos, dos de
ellos en Santa Polonia, lo más hermoso de la vega de Durango; otro en
Malespera, y el cuarto en Leguineche. El cuidado de mis tierras y
ganados acabaría de limpiar mi cabeza de los \emph{miasmas} cerebrales,
que me habían puesto al borde de la locura en la mil veces endemoniada
Villa y Corte. Aunque estos proyectos y augurios me desconcertaban,
fingí conformidad con la idea paterna, esperando que algún inopinado
quiebro de mi destino me sacara de aquel compromiso sin oponerme
derechamente a los planes del pobre viejo.

Los padres de mi novia eran admirable pareja para presentar como
maniquíes vestidos al tipo éuskaro en un museo etnográfico. Con ambos
hablaba yo mediante intérprete, pues sólo jirones desgarrados del idioma
castellano les habían entrado en la mollera. El padre pareció mirarme
con simpatía y alegrarse de tenerme por yerno: dijo que, siendo yo
persona de mucha lectura y escritura, podía enseñar algo a la chica que
se conservaba cerril. No le habían enseñado más que a rezar y a escribir
y leer torpemente. Era un ángel, eso sí, muy buena y obediente; sabedora
de todas las artes caseras, y tan excelente labradora del campo que
valía por dos hombres de los más fornidos. La madre no fue, a mi
parecer, tan propicia, y puso el reparo de mi corta estatura, por lo
cual no haría buen ayuntamiento con la yegua que el Cielo le había
deparado por hija\ldots{} También la chica, mi novia o prometida,
Facunda Iturrigalde (allá van nombre y apellido), me motejaba por
chiquitín; la risa no iluminaba su rostro inexpresivo y mofletudo sino
cuando se hablaba de mi corta talla, y algo decía en vascuence que hacía
reír\ldots{} Era sin duda un concepto semejante al de \emph{La Niña
boba}, de Lope, cuando le presentan el retrato de medio cuerpo del novio
que le destinaba su familia: \emph{Eso es no tener marido---siquiera
para empezar}.

Esto me ofendía. Pues una tarde\ldots{} Dejadme tomar otro aliento, que
esto es gravísimo. Una tarde, digo, iba yo acompañando a mi novia desde
Durango a Santa Polonia. Una fatalidad benigna nos dejó solos, pues los
padres iban delante con el carro cargado de aprestos de fábrica,
herrajes, maderas, para una obra que habían emprendido en la mejor de
sus casas. Charloteaba yo con Facunda, dándole lección de lengua
castellana, y obligándola, con insistencia de dómine, a repetir temas y
conceptos de uso constante en la conversación. A propósito estiraba yo
mi acción escolar para retrasarnos en el camino y ponernos a mayor
distancia de los padres. Dos criados que nos seguían con un borrico,
cargado también de material, pasaron delante de nosotros, y en esto,
atardeciendo, atravesamos un grupo de nogales que con su sombra
anticipaban la noche y convidaban al descanso. Díjome Facunda que
aquellos nogales y otros que más allá se veían eran suyos. Entrome con
esto un vivo afán de posesión de la tierra y de lo que no era tierra. Y
pues esta y los ganados, el fruto vegetal y la carne animal habían de
ser míos, bien podía tomar posesión de todo en aquel instante.

Apenas pensado el propósito mío de hacer efectivos mis derechos, acudí a
la práctica, declarando a Facunda la pasión violentísima que el lugar
sombrío y apacible, el sosiego del campo y la hermosura de ella
levantaron al modo de tempestad en mi alma. Observé que mis palabras
ardientes en castellano declamatorio, parodia de las famosas endechas de
don Juan Tenorio en el sofá, la impresionaron hondamente y la movieron a
estupor y curiosidad seguida de infantiles risotadas. Estimando la
actitud de Facunda como un principio de consentimiento, me lancé de las
palabras fogosas a los actos atrevidos\ldots{} Echele los brazos, y ello
fue como si el algodón quisiera ceñir y sujetar el acero. Facunda, sin
dejar de reír como una chicuela, se defendió de mí con rápida
zancadilla. Caí al suelo en postura poco airosa\ldots{} Quise
levantarme\ldots{} Facunda, con vivo juego de infancia campesina, me
volvió a dejar tendido y sin gobierno de mis piernas, y cuando yo,
vencido y maltrecho, pedía misericordia, me increpó y vilipendió con
horroroso traqueteo de frases de burla en vascuence. Comprendí que
jugaba conmigo, y que celebraba con algazara jocosa el triunfo de su
fortaleza sobre mi debilidad miserable\ldots{} Terminó el juego
desliándose de la cintura un cordel y atándomelo al tobillo sin que yo
pudiera evitarlo\ldots{} Me ayudó a levantarme, y arreándome con su
varita, me llevó por delante. No me quedaba otro recurso que aceptar el
juego y seguir la broma. De la boca de Facunda salió una frase que me
dolía más que la caída y los varetazos. Haciéndome el tonto, y fingiendo
alegría, traduje a mi modo la frase. Creo que no era infiel esta
versión: «Vean, vean el cochinito que he comprado en la feria\ldots{} A
mi casa lo llevo\ldots{} Tres duros me costó\ldots{} Engordarelo para
San Martín. Cochinito, arre\ldots; arre, \emph{charrichu}.»

Llegué al caserío renegando de las bromas de la zángana Facunda, aspeado
de la prisa con que me llevó haciendo el \emph{charrichu}. Quisieron los
padres que me quedase a cenar con ellos; mas yo, pretextando quehaceres
en casa y órdenes de mi hermana, me volví a Durango por el mismo
caminito llano, a trechos sombreado por nogales corpulentos. Si aquellos
hermosos árboles no me fueron propicios, otros más arrimados al monte
habían sido mis sagrados bosques citereos, y váyase lo uno por lo otro.
Yo podía vanagloriarme de más de tres y más de cuatro conquistas en la
soledad nemorosa.

Añado ahora, como dato interesante, que después de mi frustrado ataque a
la virtud de Facunda, esta empezó a mostrarme afición, y a gustar de mi
compañía y lecciones; ya no se burlaba de mi estatura mezquina, ni me
daba a entender que era poco hombre para su corpulencia. Esto me
envanecía; mas no cambiaba mi invencible repugnancia de hacerla mi
esposa, por incompatibilidad o desproporción muscular y sanguínea.
Bestias había yo conocido que no me desagradaban. Bien vengas,
bestiezuela, para el amor, mas no para el matrimonio.

A los tres días de hacer yo el cochinito, supimos que en un lugar de
Navarra llamado Oroquieta, había dado el General Moriones un tremendo
palizón a los carlistas, echándolos a la frontera con su iluso rey,
desvanecido por la adulación de sus prosélitos montaraces, y por el
estímulo de las plumas y voces que en Madrid movía la turba de
neocatólicos y tradicionalistas hidrófobos, explotadores de la religión
como resorte de absolutismo. El desconsuelo y turbación que tal noticia
produjo en la villa de Durango, y marcadamente para mí en nuestra
tertulia o cabildo de ojalateros, ignorantes de cuanto concierne a
gobierno de pueblos y al fuero de ciudadanía, no es para referido. Unos
clamaban, otros gruñían\ldots{} Llegó mi cuñado Zubiri, desarmado,
rabioso, sin que la vista de su hogar y de su familia le consolase del
porrazo recibido en lo más delicado de su amor propio y en lo más duro
de su barbarie.

Por no desentonar en el coro, yo me mostré afligidísimo, como si la
derrota de Carlitos VII me quitase la breva de ser su Ministro
Universal; mi padre era la imagen de la consternación paralítica y
estupefacta, cual si oyera el son terrorífico de las trompetas del
Juicio final. Todos se hallaban igualmente cariacontecidos, incluso el
cura Choribiqueta, aunque este lo hacía por comedia, pues cuando
salimos, y a discreta distancia de mi casa nos hallamos, rompimos los
dos en la misma exclamación: «¡Tenía que suceder!» Sin disimular su
alegría, el valiente clérigo me dijo: «¿Estaba yo en lo cierto, querido
Tito? ¿Se puede esperar algo de un Carasa, de un García, de un Urraza?
¿Cabe en lo humano que nos traigan la Monarquía de Dios las cabezas más
vacías que tenemos en nuestra tierra?\ldots{} Amigo, cada día me
encontrará usted más aferrado a mi tema. Dios no quiere que haya dos
epopeyas dentro de un siglo.»

---En el otro será, don José Miguel.

---En el siglo XX resucitaremos\ldots, lo creo como si lo estuviera
viendo\ldots; resucitaremos los soldados de la fe para traer a España el
Reino de Dios.

Por la tarde fui con mi padre a visitar al amigo Choribiqueta, que a la
hora de ritual nos dio chocolate con exquisitos bizcochos. Y tomando los
tres el Guayaquil, repitió don José Miguel los solemnes conceptos
sibilíticos que había expresado ante mí\ldots{} Entusiasmado quedó mi
buen viejo, y no sentía sino que él no fuera también resucitado para ver
la maravilla del siglo XX. Al volver a casa, le vi engolfado en
soliloquios que eran destellos de la misma idea consoladora\ldots{}
Llevándome a su cuarto a la hora de acostarse, tomó el tonillo más
patético y dulce para decirme: «Tito, hijo mío, ya que trayéndote a esta
tierra de la virtud y de la fe, te hemos curado de tus desvaríos, yo te
ruego que apliques tu ingenio y dotes oratorias a ilustrar a estas
buenas gentes sobre aquel punto de la venida del Reino de Dios. Tus
ideas han cambiado de una punta a otra del pensamiento. Eras hereje, y
herejías y locuras y pestilencias predicaste. Hoy eres creyente y acatas
la ley divina. ¿Qué trabajo te cuesta regalarnos con un buen discurso
que instruya y consuele? Yo me he cansado de decir a todos los amigos de
acá que eres un verdadero pico de oro, que en Madrid entusiasmas, y que
alguna vez te sacaron en hombres tus oyentes. Pues si tales triunfos
obtenías cuando predicabas la mentira, ¿qué tendrás ahora, reformado y
arrepentido, proclamando la verdad? Yo, sin esperar tu consentimiento,
he dicho que mañana por la noche nos darás una conferencia en la sala de
esta casa, que es bastante capaz\ldots{} No, no me vengas con repulgos,
ni arrumacos de falsa modestia. No, Tito\ldots; yo he anunciado la
plática tuya, y no has de dejar mal a tu padre. Di que sí. Tienes la
noche y todo el día de mañana para prepararte. A más de los amigos, que
ya están en el ajo y esperan la función como pan bendito, convidaré a
las personas principales del pueblo, sacerdotes, señoras\ldots,
señoritas\ldots»

No dijo más. Lo pensé un instante, y accedí, representándome la sala, mi
sermón, mi triunfo\ldots{}

\hypertarget{xvii}{%
\chapter{XVII}\label{xvii}}

A continuación verás, oh lector amable y socarrón, mi formidable
discurso, precedido de un ligero introito descriptivo\ldots{} Mi hermana
y mi padre se encargaron de colocar a los caballeros y señoras en
ringleras de sillas puestas en tres lados de la sala, dejando la
cabecera de esta para las personas de más viso, y para desahogo del
orador. Yo improvisé una tribuna con tres sillas cuyos respaldos me
separaban del público, ofreciéndome apoyo y resguardo. Con cuquería
teatral me abstuve de aparecer ante mi auditorio hasta el momento de
comenzar mi oración. Desde la puertecilla por donde había de entrar miré
y examiné a mi público, conforme se iba instalando. Vi señores
acartonados, predominando los narigudos sobre los chatos, serios todos
como si estuvieran en misa; vi a la derecha, en el término más lejano,
señoras gordas, señoras flacas, algunas de buena presencia y aire
aristocrático dentro del tipo lugareño. En la primera fila lucía un
grupo de tres damas, una de ellas muy aventajada de pechos, la cara
bonita. Vestían todas de negro, con excesiva honestidad, pues apenas
dejaban ver el cuello carnoso. Sobre la obscura vestimenta se destacaban
escapularios y medallitas. Gente aldeana de ambos sexos ocupaba las
filas menos visibles, pues los sitios delanteros eran para el señorío y
los curas\ldots{}

Tal era mi público, arcano cuyo seno guardaba la rechifla o el aplauso.
Aunque nunca me ha faltado el valor en casos semejantes, sentía ligero
escalofrío, y mis ideas se acobardaron, refugiándose en lo más hondo del
cerebro\ldots{} Pero llegaba el instante en que el pundonor y el
sentimiento del deber habían de arrollar al miedo y a los falsos
escrúpulos con que el alma desconfía de sí misma\ldots{} Tendí mis ojos
sobre el apretado concurso. El aleteo de los abanicos me infundió
ánimos, no sé por qué. Tras de cada abanico, adiviné un corazón de
mujer\ldots{} ¡Ah!, mujeres. ¡A ellas!, me dije, y salí. Acogido fui por
un murmullo; que allí no se estilaban los aplausos. Puse las manos sobre
el respaldo de las sillas, que eran mi tribuna, y con firme aliento, y
plena conciencia de mi triunfo ante las damas y mujeres, solté las
primeras palabras: «Señoras\ldots, señores\ldots»

Una pausita, y seguí: «Soy un pobre peregrino que ha venido de la región
del pecado a esta comarca de la inocencia y las virtudes. Salí de aquel
infierno agobiado por el peso de mis culpas; pero la voz de Dios me
alentó en los primeros pasos; la voz de Dios me iluminó el alma; en el
áspero camino lloré mis errores, y una vez llorados y aborrecidos, el
arrepentimiento me dio nueva vida\ldots{} El ambiente puro de esta
tierra completó mi regeneración, y el ejemplo de vuestras virtudes me da
valor y alientos para dirigiros la palabra. Soy un alma que ha conocido
el mal, y ahora se espacía en el bien, sintiéndose hermana de las almas
buenas, y aspirando a perfeccionarse viviendo entre vosotros con
familiares lazos de amor. Os entrego mi corazón, os entrego mi alma
toda, para que la fundáis con las almas vuestras. Vuestro pensar es el
mío. No me falta más que poseer vuestra lengua, la más antigua y la más
hermosa del mundo, para poder con ella cantar en voz baja las bellezas
de vuestra tierra y en voz muy alta, pero muy alta, el nombre de Dios y
las glorias de su santa causa.»

Circuló murmullo de aprobación. Adelante. «Dios me ha dado el singular
galardón de traerme a su campo, a su solar amado y predilecto, donde
prepara la redención de la mísera España, que sería, como sabéis, su
nación preferida, si ella se organizase a la usanza vuestra, y
desechando sus vicios y desnudándose de la costra leprosa de sus
herejías, se vistiera del esplendor de vuestra fe y de la gala de
vuestras resplandecientes virtudes\ldots{} Pero ¡ah!, la redención de
España está lejos, queridas hermanas, queridos hermanos; y está lejos,
porque la vuestra, que ha de preceder al salvamento de todo el pueblo
ibero, no está cerca, no. Mucho tendréis que hacer aún, ¡oh gloriosos
vascos!, para poner el problema en su verdadero estado de intenso
desarrollo. Permitidme que os exponga las ideas, fruto de mis largas
meditaciones en esta tierra bendita. Oíd el parecer de un férvido
creyente que en largas noches, invocando el auxilio de la Divinidad, ha
estudiado el presente, adquiriendo la clara visión del porvenir\ldots{}
La causa de Dios triunfará en Vasconia, y en Vasconia tendrá su
principal asiento, cabeza de todos los reinos católicos de nuestra
España\ldots{} Habréis visto, amadas hermanas y hermanos, que la guerra
encendida para restablecer el imperio de la fe se ha visto frustrada.
Abrid los ojos y ved bien claro, como lo veo yo, que Dios no quiere
traeros la verdad por mano y designios de reyes grandes ni chicos, ramas
una y otra de un árbol podrido. No; no esperéis nada de los reyes, que
conquistan el suelo para hacerlo suyo y llenarlo de formas de tiranía.
Yo no distingo de reyes, ni disputo por legitimidades que sólo son juego
de palabras. Todos los reyes son ilegítimos, todos llevan en la cimera
de sus cascos estos o los otros signos; en la redondez de sus cabezas,
estos o los otros sombreros.»

Estupor en el público\ldots{} En él se oiría el vuelo de una
mosca\ldots{} Sin perder el hilo de mi razonamiento, observaba yo las
caras de mi auditorio, y en ellas vi asombro, terror\ldots{} Pero no me
importaba. Tomé aliento, bebí un poco de agua, que me habían puesto en
una mesilla cercana, y seguí muy sereno preparando la bomba cuyo
estallido debía ganarme la voluntad de todo el concurso. Seguí: «Mi
declaración os causa sorpresa, y alarma por un instante vuestras
conciencias honradas. Pues si a todos los reyes, decís, debemos
declararlos ilegítimos; si ninguno de ellos ha de traernos la luz
celestial; si no debemos luchar por reyes ni príncipes ni fantasmones
más o menos coronados y galonados, el orador quiere que instituyamos una
república, y esa república será el ara santa donde se consagre la unión
de todos los católicos pueblos, la paz, el bienestar, la dicha\ldots{}
Sí, hermanos queridos, la república es nuestra salvación.»

Inmensa ansiedad expectante en el público. Hice una pausa. Paseé mis
miradas arrogantes por las caras de señoras y caballeros, y como había
tomado ejemplo de \emph{Fray Gerundio} para producir los grandes efectos
oratorios, les dejé en el tormento de sus dudas, y cuando me pareció
bien, tomado otro traguito de agua, proseguí: «¡Sí, la república\ldots!
Pero no es aquella bacante semi-desnuda y escandalosa, hija de Satán,
que trastorna con su bello nombre y su infernal doctrina a los pueblos y
ciudades de Castilla; no es la bestia roja, sanguinaria, ebria de vino y
de mentirosas filosofías; no es esa, no. Esa república será barrida como
los despojos de Carnaval que ensucian las calles el Miércoles de Ceniza;
esa república tendrá sus altares en los manicomios, donde expirarán
todos los que la profesan, y donde se extinguirán sus alientos con
rugido de fieras moribundas. La república que yo preconizo y anuncio es
otra, es la que lleva en sus sienes, por corona, la luz del Espíritu
Santo, la que en los bordes de su clámide lleva bordadas las
inscripciones \emph{Fe, Esperanza y Caridad}, la que en su seno purísimo
agasaja la paz, la que con sus labios imprime el beso del ardiente amor
de Dios\ldots; esa república, hermanos queridísimos, es\ldots{} la
Iglesia.»

El enorme efecto se produjo, y aguzando mi voz para dominar los
murmullos de entusiasmo, remaché la frase: «\emph{La Iglesia católica,
apostólica, romana}\ldots» «Ya veis, cómo al arrancar de vuestras
opiniones la figura borrosa y descolorida de estos reyes de faramalla,
os presento la imagen de Cristo, Rey de los pueblos católicos, Cristo,
Rey de España\ldots{} Y siendo Vicario de Cristo, y su cabeza visible el
Romano Pontífice, os digo: Durangueses, pueblo todo vasco-navarro,
derribad los ídolos dinásticos, usurpadores de la autoridad, y poned en
el trono vacío la excelsa soberanía del Papa\ldots{} Oídme ahora este
argumento decisivo: ¿No nos gobierna el Papa en lo espiritual; no es él
quien nos impone el dogma y vigila su cumplimiento? Pues si gobierna en
lo espiritual, que es lo más, ¿por qué no ha de gobernar en lo temporal,
que es lo menos? ¿No se os había ocurrido este razonamiento? ¿No
pensabais que el gobernador espiritual debe gobernar también en el
terreno de las menudencias de la vida? ¿Qué es lo espiritual?: la vida
infinita. Pues englobad lo finito en lo infinito, mirad lo finito como
cosa baladí al lado de lo infinito.»

Entusiasmo loco. La convicción ganó todos los ánimos. Me aplaudieron. La
señora gorda y guapa más visible entre las damas, me miraba no ya con
admiración, sino con arrobamiento. Mi padre, sentadito en forma de
ovillo no lejos de mí, tenía ya el pañuelo tan mojado de sus lágrimas
que se las bebía por no poder secárselas. Adelante con mi bravo
discurso: «Ya sabéis, ¿qué católico no lo sabe?, que el Santo Padre
tenía en el centro de Italia sus Estados, de los cuales era Rey.
Donación del Altísimo eran aquellos Estados, los más felices de la
tierra mientras vivieron bajo el mando, bajo el dulcísimo gobierno de Su
Santidad. Pero el Infierno alborota la Italia; el Infierno coloca en el
trono de un pequeño reino de Italia llamado Cerdeña a un cerdo que lleva
el nombre de Víctor Manuel, y este cerdo arrebata al Pontífice sus
Estados, dejándole preso en su palacio. ¿Por qué ha permitido Dios tal
iniquidad? Porque previene a Pío IX mejor casa y estados mejores y reino
más grande\ldots{} Ya lo adivináis. Vuestros corazones se anticipan al
pensamiento\ldots{} El nuevo Estado Pontificio es España, y contra
España pontificia nada podrá el Infierno, ni los Víctor Manueles de los
cubiles de acá y de allá prevalecerán contra la voluntad de Dios\ldots{}
Pero Dios espera, Dios quiere que los pueblos que le son queridos se
penetren de su voluntad, y den muestra de querer realizarla. Claro es
que Dios puede hacerlo cuando guste; pero le agrada en extremo que su
pueblo más querido se anticipe, y reclame el honor de declararse
propiedad del Vicario de Cristo\ldots{} Ya lo sabéis, hijos de Vasconia.
Si el Espíritu Santo, como creo, ha sugerido a vuestras almas la idea de
la Pontificia República, no vaciléis, no durmáis, no esperéis a mañana.
Id a Roma, damas y varones escogidos de Dios, y decid al Supremo
Jerarca: Padre Santísimo, si Estados os quitó la iniquidad de un Rey,
tomadlos mejores y más ricos en la Península católica, donde la Reina de
los Ángeles tiene su más extendido y ferviente culto, en la tierra
bendita, madre de los santos y fundadores más gloriosos. Por de pronto,
podréis tener por vuestros los vastos dominios que se extienden desde el
Roncal a Carranza, salida y puesta del sol; por Septentrión, el
Cantábrico mar y cordillera pirenaica, y al Sur montañas de Burgos,
curso del Ebro\ldots»

Sonidos guturales, ayes de admiración, palmadas\ldots{} Estaban locos.
La señora gorda me comía con sus ojos. En ellos y en su lozano rostro
encendido por el calor y el entusiasmo fijé yo los míos, y para ella
dije: «Pedid al Santo Padre su bendición y os la dará gozoso, y vosotros
le diréis: «Santísimo Padre, mandad al punto a vuestro nuevo territorio
todos los frailes y monjas que tengáis disponibles, y que sean de
diferentes órdenes, sin que ninguna falte, y con la sola invasión de esa
católica hueste, dad por conquistado vuestro reino, y bien asegurado
contra heresiarcas y contra la peste de nefandos políticos. Los varones
religiosos, astros de virtud y profesores de fe, se difundirán por toda
la Península, y ya no hace falta más. Pero mandad muchos, Santísimo
Padre, los más robustos, los más enérgicos, los más sabios, y con ellos
mandad cuantas vírgenes o esposas del Señor tengáis en vuestros sacros
monasterios. No os arredre el número, que allí hay sustento y holgadas
casas para todos, y dinero de largo para cuanto hubieren menester.»

El regocijo de mi público iba en aumento, y yo, creciéndome y
agigantándome sin necesidad de tacones, llenaba el mundo, a mi parecer,
con mi exaltada oratoria, y al cielo tocaba con mi gesto no menos
elocuente que mi palabra. Para ofrecer a mi auditorio, en forma práctica
y fácilmente accesible a los más obtusos, la idea de mi República
Hispano-Pontificia, tracé el siguiente cuadro estadístico: «No
terminaré, señoras y caballeros, sin daros una síntesis clarísima de los
nuevos Estados de Dios, gobernados por su Vicario en la Tierra. Admitid
que las órdenes religiosas difundidas por toda España y adueñadas de las
conciencias, declaran constituida la divina República. Admitid esto y
dadlo por hecho, y veréis el grandioso espectáculo de una nación
organizada por el Espíritu Santo. Rey ni Roque no necesitamos, porque
nuestro soberano es el Papa, residente en Roma, o residente en España,
en la ciudad que más le conviniere. Los ministros podrán ser siete,
ocho, según lo demande el interés público, y serán escogidos entre los
arzobispos y los priores o abades de las Congregaciones. Desaparecerá,
pues, de un soplo la nube de politicastros que cual langosta devora toda
la riqueza del país. Congreso y Senado pasarán también al estercolero, y
en su lugar tendremos un Concilio permanente, que se formará con
individuos del Episcopado, Padres de la Compañía de Jesús, reverendos
párrocos y sabios religiosos de distintas órdenes. Los funcionarios
subalternos de los Ministros, los embajadores o nuncios serán también
obispos, deanes, arciprestes, según la categoría del cargo; los
gobernadores y alcaldes se reclutarán entre los párrocos de más
autoridad y circunstancias, y en cuanto a lo que hoy se llama
\emph{Tribunales de Justicia}, os diré que a la Iglesia le sobra
personal para constituir, con los sabios agustinos, dominicos y
jerónimos, cabildos jurídicos que vean y sentencien con recto juicio
todas las causas civiles, criminales y eclesiásticas\ldots{} Ya veo en
vuestros rostros que mentalmente formuláis una pregunta: ¿Y Ejército? Os
diré con la rudeza que pongo en mis opiniones, que el actual
\emph{elemento armado} será reconstituido después de una escrupulosa
purificación, para lo cual se formará un elevado Consejo presidido por
un obispo. Serán vocales de ese magno Consejo personas de acreditado
conocimiento y experiencia en lo militar y en lo religioso, que de sobra
tenemos, bien lo sabéis, varones doctos, guerreros y píos, que sepan
desempeñar función tan delicada.»

Vehemente aprobación, y voces afirmativas\ldots{} \emph{que sí},
\emph{que sí}\ldots{} Y yo me encaminé sereno y majestático 3 al
coronamiento de mi aparato lógico: «Sólo me falta deciros que para la
realización de este divino ideal, de lo que llamaríamos \emph{Política
de Dios} y \emph{Gobierno de Cristo}, hemos de establecer la estricta
unidad de sentimientos religiosos, hemos de conseguir que en toda la
Nación no exista una sola alma que discrepe del sacrosanto dogma. ¿Qué
necesitamos para este fin indispensable? Pues necesitamos un órgano, un
instrumento de limpieza, un salutífero purificador de las conciencias.
¿Y cuál es este órgano, este instrumento en que se combinan lo divino y
lo humano? En la mente de todos los que me escuchan, en sus labios, diré
con más propiedad, está la respuesta. El órgano purificante y unificante
es la dulce Inquisición\ldots{} Sí, la llamo dulce porque sus efectos
nos llevarán a un dulcísimo estado de beatitud, porque los rigores que a
veces empleara contra la herejía son cosa blanda en parangón de la paz y
dulcedumbre que ha de dar a la Nación, porque si emplea el fuego para
ahuyentar a los demonios, nos trae frescura y aire delicioso con el
batir de alas del sinnúmero de ángeles que el cielo nos enviará para
consuelo y alegría de las almas españolas.»

Delirio, palmoteo frenético, berridos de aldeanos, lloriqueo de
señoras\ldots{} Y yo \emph{tenza que tenza} como el célebre mentiroso
Manolito Gázquez, bailando en el aire, quiero decir que alentado por los
aplausos, disponíame a terminar del modo más airoso. Con rápida visión
retórica, comprendí que el final debía ser en extremo patético y dulzón.
Allá va: «Ya he cansado bastante a este noble auditorio; ya debe este
humilde orador católico volver a la obscuridad de que nunca debió salir,
de que salió por vuestra benevolencia y caridad, digo caridad porque tal
me parece el hecho de que os hayáis dignado oírme. Os debo gratitud
eterna por vuestra benevolencia, y a ella correspondo diciéndoos que ese
galardón de vuestras almas generosas es el más preciado que pude soñar.
No veáis en mi pobre discurso primores de inteligencia, ni recursos de
erudición, ni ornato de filigranas retóricas; no veáis más que ardor de
fe, y sinceridad de creyente postrado ante los altares de Dios. Lo que
habéis oído es fruto, más que del estudio, de la oración, de embebecer
el alma en la contemplación de la Divinidad y dormir en el éxtasis como
el niño inocente en el blando regazo maternal. En mí no veáis ciencia;
en mí no veáis la vana sabiduría que adquiere en los libros, obra
comúnmente de la superchería o del orgullo; en mí no veáis más que amor,
que es la fuente de todo bien, manantial que nace en la grada más alta
del Trono del Altísimo y viene murmurando suaves promesas hasta nuestras
almas sedientas. El amor de Dios que me abrasa con llama inextinguible,
me ha enseñado el amor de las criaturas. Mi enseñanza es amor, y
entiendo que el sublime plan de República Hispano-Pontificia sólo por el
amor puede traerse a la realidad. Y en verdad os digo que sin amor no
saldréis de la esclavitud en que vivís\ldots{} Amaos los unos a los
otros. Amad a vuestros enemigos, amad a vuestros amigos. Os lo dice y os
lo encomienda con efusión ardiente el que, subiendo desde el error a las
cimas de la verdad, aprendió esta suprema ley; el que vivió en el pecado
y se regeneró en la virtud; el que fue ciego y hoy iluminado por el
fuego de la fe, es todo sentimiento, todo piedad, todo amor\ldots{} He
dicho.»

El esfuerzo para terminar con brío, el espasmo oratorio me dejaron sin
aliento. Me vi en brazos de mi padre que al estrujarme en ellos me privó
de la respiración: el raudal de sus lágrimas me anegaba el rostro. Mi
hermana lloraba también, abrazándome y dándome besos, mientras el
bárbaro de su marido gritaba: «\emph{Lo que saber chico, pico de oro
tener hablando}.» El público en masa avanzó impetuoso hacia mí para
felicitarme con palabras cariñosas y exclamaciones de entusiasmo. La
señora gorda y bonita fue de las primeras que a mí llegaron, trayendo
consigo dos damas flacas un tanto narigudas, de cuyos labios oí plácemes
afectuosos en una jerga mixta de castellano y vascuence. En la señora
simpática pude advertir una subida coloración del rostro, del exceso de
entusiasmo, y un trémolo de la voz que me indicaba su modestia, como si
se creyera indigna de hablar conmigo. Apretándome las manos entre las
suyas calentitas, me dijo: «¡Qué sublime orador! ¡Qué gloria
oírle!\ldots{} Aquí no se ha conocido quien a usted iguale ni quien como
usted posea el arte de conmover.» A su correcto castellano contesté con
vehementes gratitudes, y ella, hecha un merengue, hablome de este modo:
«Sería cosa de pedir a usted que le oyéramos todos los días. Yo he
comprendido todo, tan bien lo decía usted y con tanta claridad lo
exponía. Todo lo he comprendido. Sólo me han quedado dudas en un punto.
¿En la nueva República, los militares vestirán el uniforme que hoy usan,
o un traje como los caballeros de Calatrava y Santiago, con birrete y
manto blanco?»

---Sobre ese punto y otros que no he podido explanar en esta oración
sintética---le respondí muy fino,---daré a usted explicaciones latas
cuando tenga yo el honor de visitar a usted para ofrecerle mis respetos.

---¡Oh!, cuando usted quiera.

---Molestaré quizás\ldots{}

---¿Molestia? Ninguna. Vivo sola con dos muchachas. Mi esposo está en
Cuba, empleado en la Aduana\ldots{} Salgo poco de casa. De ocho a nueve
todos los días voy a misa a Santa María, y por la tarde al
rosario\ldots{} Tendré mucho gusto\ldots{}

Despidiéndola cortésmente para dar paso a otras y otros que acudían a
mí, dije para mi sayo: «Conquista tenemos.» Largo rato duró el sofocante
jubileo de plácemes y apretujones. Las pobres aldeanas expresaban con
sencillez candorosa el deleite de haberme oído, y salían clamando:
«¡Viva Dios y viva el Santo Papa nuestro Rey!» Harto expresivos fueron
los padres de mi novia y mi novia misma. En los ojos de esta conocí que
había llorado. Apretome el brazo hasta el dolor, muda y bestial
expresión de sentimientos que parecían instintos. El padre me dijo que
sabía yo por doce obispos, y la madre me soltó este requiebro:
«\emph{Tanto como chico, grande ser tú, hijo mío, de saber y sermón
bonito}.»

La felicitación y el abrazo de mi amigo el cura Choribiqueta fueron
también muy expresivos, si bien con un poquito de reserva, que no pudo
disimular. Le invité a no escatimar conmigo su confianza, y a mis
razones contestó estas, que oí con el respeto debido a su grande
autoridad: «Muy bien, querido Tito, soberbio. Ha estado usted
imponderable en la dicción, sublime en la idea y plan del Gobierno de
Cristo, por su Vicario el Papa. Es usted un orador que se deja en
mantillas a los Manterolas de aquí y Castelares de allá. Conforme en
todo, menos en una cosa; y pues usted me pide franqueza, allá va mi
parecer sincero. Todo me ha parecido bien, menos la idea de meternos
aquí todos los frailes de la Cristiandad. ¿Para qué queremos aquí tal
aluvión y acarreo de regulares? Nosotros los seculares nos bastamos y
nos sobramos para todo lo que haya que hacer. Sobre que son en su
mayoría un hatajo de gandules que vienen aquí con hambre atrasada, y en
poco tiempo consumirían todas las subsistencias de la Nación, querrían
mangonear ellos solos y nos reducirían a una servidumbre vergonzosa. En
la clerecía de aquí hay bastante personal para desempeñar cuantas
funciones civiles, judiciales y aun militares se nos encomienden. De mí
sé decir, sin jactancia, que me creo tan apto como el primero para ser,
al par que un párroco modelo, un ejemplar alcalde. Sí, Tito, sí; yo
gobernaría esta villa mejor que nadie. Bien apañado estaría el pueblo, y
bien derechos andarían todos mis administrados, que al propio tiempo
serían mis feligreses. ¡Ayuntamiento y parroquia en una pieza! ¡Qué
gusto! Pues aún me sobraría tiempo para otro cargo, por ejemplo: maestro
de escuela\ldots{} A los chicos los despacharía yo en dos
palotadas\ldots{} Conque ya sabe usted lo que piensa un hombre que
siempre dice la verdad. Recorte usted eso de la traída de frailes y
monjas, y en lo demás conformes, y grandemente entusiasmado de su
talento, de su oratoria, de su arranque\ldots{} ¡Viva Dios Uno y Trino,
y la Purísima Concepción, Madre del Verbo, inspiradora de toda
elocuencia!»

De los demás curas recibí enhorabuenas, no todas ardorosas, algunas
bastante frías como de quien no ve con buenos ojos al que descuella
demasiado pronto, y gana con un solo acto la voluntad colectiva\ldots{}
Avancé en la sala para saludar a los que humildemente iban saliendo sin
atreverse a dirigir la palabra al gran orador. A muchos di mis
gratitudes, y en uno de los grupos rezagados que requerían con apreturas
la puerta de salida distinguí una cara de mujer que me dejó paralizado
de estupor. O yo veía visiones, o la que vi era \emph{Mariclío} en
apariencia equívoca, medio señora, medio aldeana. Con trabajo y
abriéndome paso como pude llegué hasta ella. Me miraba y reía. Cuando a
su lado estuve, acercó su boca a mi oído para decirme con susurro: «Eres
el granuja de más chispa que he visto en el mundo. He pasado un rato
delicioso oyéndote desatinar con tanta gracia y picardía. ¡Y esta pobre
gente tan consentida!\ldots{} Te han tomado por el Espíritu Santo.»

Interrogada por la razón de su presencia en la villa de Durango, me dijo
así: «Aquí he venido creyendo encontrar algo de provecho. Me parece que
nada bueno podré llevarme, como no sea tu discurso, que quizás, bajo la
forma de jácara o entremés de burlas, entraña no pocas verdades para el
día de mañana\ldots{} Pero no hablemos más aquí. Vivo en una posada o
parador, a la entrada del pueblo viniendo de Bilbao. Vete a verme cuando
puedas. Estaré algunos días hasta ver en qué para esta nueva humorada
facciosa\ldots{} En la posada, pregunta por \emph{doña Mariana}, o la
\emph{Madre Mariana}, que con tales nombres vengo, y por ellos soy
conocida. Adiós, Tito salado.»

\hypertarget{xviii}{%
\chapter{XVIII}\label{xviii}}

Maravillado me dejó la presencia de \emph{Mariclío}, pues aunque bien
conocía yo sus naturales tendencias a la ubicuidad, no esperaba verla en
aquel lugar de Vasconia, donde nada ocurría digno de los borceguíes ni
aun de las sandalias de mi ilustre amiga. Hice propósito de visitarla en
su posada, en cuanto tuviera un rato disponible. Viéndola escurrirse
entre el gentío saliente, acompañada de otra mujer que acaso sería su
posadera, pensé que mi discurso debió de causarle gran regocijo, y de
ello me alabé, pues yo también de dientes adentro me reía de mí mismo, y
celebraba el gracejo y socarronería con que supe tomar el pelo a los
inocentes y fanáticos durangueses. Ni en aquella tarde ni en todo el día
siguiente pude ver a \emph{Mariclío}, porque en mi casa menudeaban las
visitas. Tras de las visitas venían las invitaciones a comer, y hasta de
las monjas de Santa Susana y Santa Clara llegaron recaditos tiernos, con
la coletilla de que me verían con gusto en el locutorio.

Heme aquí de visiteo todo el santo día, sin olvidar a las monjitas, y
menos a mi predilecta, la que di en llamar \emph{señora gorda}, y ahora
designo por su verdadero nombre, doña Josefa Izco de Larrea. Ya
comprenderá el ladino lector que, encontrándola sola en mi primera
visita, juzgué oportuno aprovechar la buena coyuntura para colocar,
entre los tópicos vacíos de un vago parloteo, una pérfida declaración de
amor. Díjele que por las singulares circunstancias de mi vida y por la
exaltación a que había llegado, mi espíritu necesitaba un amor puro, un
amor místico, y que en ella veía el único ser capaz, por su exquisita
idealidad, de acoger aquel amor\ldots{} enteramente angélico, sin el
menor atisbo ni vislumbre de melindre sensual. Poniéndose colorada y
haciendo con su boca linda unos repliegues muy monos, contestó que
siendo el amor rematadamente puro, en \emph{toda la extensión de la
palabra}, afecto espiritual, sutilísimo y sonrosado, no tendría
inconveniente en\ldots{} Al siguiente día, después de acompañarla a
misa, le conté, como yo sabía hacerlo, la vida de Santa Cecilia y San
Valeriano, que fueron novios y tuvieron el gusto de ser martirizados
antes de casarse. Oíame Josefa Izco con arrobamiento, y encomiaba la
castidad como la virtud preeminente para ganar el cielo. Yo decía para
mi sayo: «Déjate estar. Ya hablaremos de eso dentro de ocho o diez
días.»

La primera vez que pude hacer un hueco en mis preocupaciones para
visitar a \emph{Mariclío}, tuve la desdicha de no encontrarla en su
casa. Díjome la posadera que había ido a Elorrio, y que ignoraba cuándo
volvería. ¿Qué pasa en Elorrio? A mi pregunta me contesta la buena
mujer: «No sé, señor. Sólo sé que allí está el General Serrano, alojado
en la casa de los señores de Urquizu\ldots{} Dos hermanos muy
principales. El uno fue a la facción, el otro está con Serrano. Andan
sobre esto muchos decires\ldots{} Parece que allá van los señores de la
Diputación de Vizcaya, o que Serrano y Urquizu irán a ponerse \emph{so
el árbol de Guernica} para tratar paces duraderas con don Carlos. No sé
si \emph{doña Mariana} es amiga del Serrano; pero allí está, viendo lo
que guisan. Es señora muy leída, que todo lo quiere saber, y no hay olla
en que no meta sus narices\ldots»

En tanto que esto ocurría, el éxito y fama de mi discurso,
\emph{Proclamación de la República Hispano-Pontificia}, repercutían
lejos o cerca de mí con diferentes efectos. Por una parte, mi padre
recibía de Madrid la noticia de que la conferencia, reproducida por la
prensa neocatólica, había levantado polvareda de alegría y entusiasmo.
Gabino Tejado, Carulla, Carbonero y Sol y otras encumbradas figuras del
ultramontanismo, me ponían sobre su cabeza. Se decía en Madrid que en la
Curia Romana era ya conocido el discurso, y que el propio Pontífice,
oído el dictamen de la \emph{Propaganda Fidæ}, lo consideró como
documento digno de ser comunicado a todo el mundo católico. Esto me
aseguró mi buen padre, babeándose de emoción; mas como no me mostrara
las epístolas en que tan lisonjeras cosas se le comunicaban, pensé que
algún ángel se lo había contado en sueños.

Por otra parte, llegaron a mí referencias totalmente desfavorables a mi
persona y discurso. Mi amiga mística Josefa Izco, cuando ya sus tiernas
afecciones iban derivando por suave pendiente hacia la impureza, me
informó con íntimo secreteo, de que dos curánganos aviesos, el uno
coadjutor en Santa María, capellán el otro de las Claras, tramaban atroz
conjura contra mí. Andaban diciendo que informados de mi persona y
antecedentes por sujetos llegados de Madrid, sabían que yo era un pícaro
redomado, un zascandil de la literatura y el periodismo, federal de
abolengo, masón y revolucionario callejero, y que mi famosa perorata fue
una burla infame de la honrada inocencia de los durangueses. Creía
Pepita Izco que los tales clérigos procedían así movidos de la envidia y
del reconcomio de su barbarie, y que yo sufría la injusta persecución
que siempre recae sobre el verdadero mérito. Pero me prevenía contra la
maldad de mis enemigos, que ya se preparaban para vilipendiarme
públicamente. El uno se proponía desenmascararme desde el púlpito,
contando mi vida de disipación y escándalo, y mis propagandas
demagógicas y ateas. El otro andaba ya en tratos con una pandilla de
mozos de brío, que me obsequiarían con una somanta, toreándome por las
calles y arrojándome del pueblo.

Ambas versiones archivé en mi mente para resolver, a su debido tiempo,
el partido que debía tomar. Pepa Izco no me engañaba; los optimismos de
mi padre me inspiraban confianza poca, y no era santo de mi devoción el
ángel que le traía los cuentos de Roma. Prevenido para lo que pudiera
ocurrir, volví a la morada de \emph{Mariclío}, que por dicha mía llegó
de Elorrio horas antes de pasar por Durango el Duque de la Torre, con su
séquito militar y civil en dirección a Zornoza. Di cuenta a la
\emph{Madre Mariana} de mis inquietudes, y me dijo que según sus
noticias no tendría yo más remedio que salir por pies, antes que se
descubriera la superchería picaresca del sermón con que embobé a los
durangueses. Había sido yo un diablo metido a predicador y profeta, y
aunque lo hice con donaire sutilísimo, tendría que pagar con el pellejo
mi descocado atrevimiento\ldots{} A estas severas razones añadió después
otras más blandas que me infundieron cierta tranquilidad: «Hazte el
desentendido de esos rumores contra ti, y esta tarde y mañana irás con
tu padre a Santa María, y con Choribiqueta darás tu acostumbrado paseo.
Yo me encargo de sacarte de esta rinconada en que te has metido. ¿Cómo?
Por de pronto antes de media noche recibirá tu padre un telegrama del
encargado de la Nunciatura en Madrid, diciéndole que el Papa desea y
pide que vayas sin pérdida de tiempo a Roma\ldots{}

---¡Yo\ldots; a Roma yo!

---No te alborotes, hijo. Tú has hecho la historia jocosa, la profecía
burlesca. ¿Qué otra cosa es tu \emph{República Hispano-Pontificia} más
que un divertido sainete? Pues yo, en estos días de horroroso tedio,
endulzo mis amarguras dándome un paseíto por el campo de la
\emph{Historia burlesca}, de la \emph{Historia chismográfica}, de la
\emph{Historia juguete}\ldots{} De varios modos nombro estos vagos
esparcimientos de mi triste vida. ¿No lo entiendes, tontín? Pues vete a
tu casa, y \emph{espera los acontecimientos}. Aunque estos sean
acontecimientos de puro recreo infantil para pasar el rato, no quedarás
mal servido, querido Tito, predilecto de las Musas bufonescas\ldots{} Yo
me iré esta noche en persecución de mi Duque de la Torre. Deseo saber si
hace algo que me obligue a cambiar estas rústicas alpargatas por el alto
y dorado coturno. Luego volveré aquí, donde espero verte, y me contarás
si te han dado la solfa y carrera en pelo que te corresponde por haberte
metido a intérprete del Espíritu Santo.

Obediente a su mandato, me retiré \emph{pian pianino} a mi casa y esperé
tranquilo los pícaros acontecimientos. A la hora de la siesta, llegó el
telegrama en que el secretario de Estado de Pío IX\ldots, no
reírse\ldots, comunicaba\ldots, no sé cómo decirlo para que mis lectores
no me tengan por loco\ldots{} En fin, que piensen lo que quieran\ldots{}
Los visajes que hacía mi padre al fijar sus ojos en el telegrama, la
cara que puso leyéndomelo, después de haberse enterado él detenidamente,
no caen dentro del dominio de la literatura descriptiva\ldots{} Yo, al
menos, no encuentro palabras para expresar el trémulo acento, la\ldots,
la\ldots{} transfiguración, el éxtasis final de mi buen viejo en tan
sublime instante. Y para complemento de la función, llegó una hora más
tarde el rector de Santa María con otro telegrama notificándole que la
\emph{Propaganda Fidæ} quería que yo explanase mi tesis ante ella\ldots;
vamos, que Roma me llamaba, Roma me reclamaba, no sé si para ponerme en
un altar, o para quemarme vivo.

Corrí a llevar la noticia a Pepita Izco, que no se resolvió a creerlo, y
aun indicó la idea de que en ello andaban los demonios. De vuelta a mi
casa, recibí el tercer telegrama. Era del encargado de los negocios
puramente eclesiásticos de la Nunciatura, diciéndome que a mi
disposición tenía los fondos necesarios para mi viaje\ldots{} ¿Creéis
que era broma?\ldots; y añadía que no perdiese el tiempo, pues el 25
salía vapor de Marsella para Civitta-Vecchia, y si me descuidaba no
tendría vapor hasta el 31\ldots{} Aquella noche nadie durmió en casa.
Todos parecían locos. Zubiri, mi padre, mi hermana, se reunían en
consejo de familia, y se separaban sin decidir cosa alguna. Trigidia, un
tanto recelosa de la procedencia de los telegramas, inclinábase a
suponerlos, como Pepita Izco, invención del mismo Infierno.

Lo primero que me dijo mi buen padre a la mañana siguiente, cuando
tomaba su chocolate, fue que antes de partir para la \emph{capital del
Orbe Católico}, debía dejar concertadas solemnemente mis nupcias con
Facunda, dando cuenta de ello al Sumo Pontífice en la primera entrevista
que con él celebrara, para que nos concediese su santa bendición, regalo
de boda el más preciado que la chica de Iturrigalde podía ambicionar.
Con todo me mostré conforme. Trató luego de la necesaria provisión de
dinero, y haciendo un gran esfuerzo y torciendo la boca como si algo le
doliera, sacó un envoltorio de papel con cuatro monedas de cinco duros,
que me enseñó diciéndome: «Esto para el viaje a Madrid, que harás en
primera, para que en primera te vea el Nuncio, Pro-nuncio, o lo que sea,
si baja a la estación a recibirte\ldots{} Ya sabes que tienes viaje
pagado desde Madrid a la \emph{capital del Orbe Católico}. Te
recomiendo, hijo del alma, que no te detengas en la Villa y Corte más
que el tiempo preciso para visitar al señor Pro-nuncio. Huye de los
amigos malos y de toda la pestilencia de aquel pueblo corrupto.»

Por la noche me dio las monedas de oro con tanta solemnidad como si
pusiera en mis manos hostias consagradas. Y al siguiente día me
asaltaron los padres de Facunda con arrumacos y zalamerías, amenazándome
con su enojo si volvía de Roma sin traer para su hija el espléndido
regalo de la bendición papal. En tanto la mozarrona corpulenta me
perseguía, como camella desmandada, por las calles y callejas del
pueblo, llamándome a su lado, pidiéndome conversación de amores cual si
me necesitara para inmediatas expansiones afectivas. También me acosaba
mi padre, dándome prisa para emprender mi viaje; no se me escapara el
vapor de Civitta-Vecchia.

Estaba yo en ascuas, pues Pepita Izco me dio noticias alarmantes de los
dos clerizontes que trataban de lanzar contra mí la brutal plebe, armada
de estacas. Indicios de esta ignominia observé al pasar por algunas
calles. Frente a la botica de Anabitarte vi un grupo que a mi paso
profirió voces chanceras acompañadas de siseos y carcajadas, y de la
lonja de Basterrechea salieron chiquillos desvergonzados que me
arrojaron hojas de berza y algunas peladillas\ldots{} En previsión de un
escandaloso conflicto, mi primer cuidado fue correr en busca de mi
protectora la \emph{Madre Mariana}, y tuve la suerte de verla entrar en
su posada a poco de estar yo allí. Sabedora ya de mis afanes, y
tomándolos a broma, me dijo sonriente: «¿Qué le pasa al ingenioso
Tito?\ldots{} ¿Quieres quedarte en esta feliz Arcadia?»

---No, Madre. Por todo el oro del mundo no estaría un día más en la
metrópoli de mi República Pontificia. Se la entrego al Papa y a sus
negros lugartenientes\ldots{} El problema es salir de aquí sin la cabeza
rota. Ampáreme usted, y si como parece abandona estos lugares
beatíficos, lléveme en su compañía y séquito, en calidad de secretario,
maletero, paje o como le plazca.

Sin otra forma de expresión que una sonrisa tranquilizadora, cogiome de
la mano y me llevó a su habitación, que era baja, obscura. Al entrar en
ella, encandilado por la luz solar, no pude distinguir si los informes
bultos que allí se parecían eran muebles, baúles o personas. \emph{Doña
Mariana} me arrojó, con empujón leve, en un asiento que no supe si era
sillón o sofá. Inciertas blanduras de muelles rotos y de pelotes
gastados me lastimaban las carnes. La señora me habló de viajar en coche
y en trenes, y cuando de mí se alejaba la reconocía tan sólo por la voz,
pues su figura se perdía en las tinieblas de aquel antro. Me consolaba
la idea de que \emph{doña Mariana} me llevaría consigo, y mi única
contrariedad era el tener que partir sin ropa, pues ni a tiros volvería
yo a casa de mi hermana para recoger mi equipaje\ldots{}

Pensando en esto, mis oídos, más que mis ojos, se sintieron como
sumergidos en una atmósfera de somnolencia, jugando con la ilusión y la
realidad. En el charloteo murmurante de \emph{doña Mariana} con personas
no vistas, se destacó un acento que me sonaba como la propia voz de
Graziella, mi hechicera y amiga en las noches febriles de la gruta de
marras. El dejo italiano de la invisible parlante y su gracia voluble
delataban a la ninfa; mas yo nada veía; la luz era escasa, temblorosa.
Creyérase que la producían llamas moribundas de candiles colocados en el
suelo de la estancia. Esta era de tal configuración, que desde mi
asiento yo no distinguía su término.

De improviso, vi a la \emph{Madre Mariana} junto a mí, no puedo decir si
sentada o en pie. Su voz sonaba quejumbrosa, diciéndome lo que, por ser
de ella, intento copiar \emph{ad pedem litteræ}\ldots{}

«Me vuelvo a los Madriles, porque ya he visto lo que dan de sí los
últimos acontecimientos de Navarra, y el fracasado intento de guerra
civil. Bien poca cosa es lo que puedo aprovechar de esta ráfaga
histórica, que pudo ser incendio, y no es más que fogata o llamarada
efímera. En un palacio de Amorevieta \emph{(Dos Amores)}, he dejado a
Serrano, que ayer trataba de paces con los diputados de este Señorío.
Con él hablé, y sus pensamientos y los míos han coincidido en la
necesidad nacional de poner cerrojos, candados y barrotes al templo de
Jano\ldots{} En los medios para lograr tal ventura no estamos acordes.
Serrano, ya lo sabes, es un león en los campos de batalla; pero en los
descansos de la guerra, toda la hiel se le endulza, y en su inocente
optimismo cree que con tratos y avenencias amistosas puede desarmar a
sus encolerizados enemigos. Yo le dije que sólo con la guerra cruda y
eficaz se puede obtener el beneficio de paces duraderas. No le convencí,
y allí estuvo parlamentando con los primates vizcaínos, y entre unos y
otros dejaron escritas unas que llaman \emph{bases}, y que son
montoncitos de arena movediza sobre los cuales nunca podremos asentar un
sólido edificio.»

Yo quise decir algo; pero las ideas que de mi cerebro bajaron a mis
labios helados, murieron en ellos sin producir el más leve sonido. Doña
Mariana prosiguió así:

«Estaba el Duque en lo cierto diciendo a los carlistas, por conducto de
Urquizu, que en guerra formal jamás vencerían. ¿A qué sostener una
campaña, que no tendría más consecuencias que convertir el risueño País
Vasco en campo de ruinas y desolación? Algunos cabecillas, como Iriarte
y Valdespina, no se daban a partido; otros firmaron en Mondragón un acta
en que autorizaban a Urquizu para tratar de paces con Serrano.» De la
boca de la \emph{Madre Mariana} salieron con limpia dicción nombres de
esos que se resisten a permanecer en la memoria del oyente:
\emph{Garibi}, \emph{Cengotita}, \emph{Arguinzonis}\ldots{} Entendí que
los dos primeros eran apellidos de cabecillas, el otro de un diputado
del Señorío de Vizcaya\ldots{} Luego pronunció otros nombres, que yo con
atención muy afilada intenté clavar en mi memoria. Pero entraban en ella
y al instante salían a perderse en el ambiente ahumado y tenebroso de
aquella estancia de aplastado techo y largura de túnel.

Turbado yo y soñoliento, pude formular en mi magín este razonable
juicio: «El suceso que la puntual \emph{Mariclío} trata de referirme es
de aquellos que se desvanecen en la Historia, y a los treinta o más años
de acaecidos, no hay memoria que los retenga, ni curiosidad que en ellos
quiera cebarse. El humo y la penumbra borran todo hecho que no tuvo
eficacia, y de él sólo queda un epígrafe, la etiqueta de un frasco
vacío.» Yo vi el letrero: \emph{Convenio de Amorevieta}, y ante él la
\emph{Madre Mariana} y su humilde interlocutor bostezábamos.

Pronunció luego la señora nombres vascos, que al salir de la clásica
boca cruzaban el aire con ruidillo comparable al del diamante que raya
el cristal\ldots{} \emph{Arguinzonis}, \emph{Urquizu}, \emph{Urúe}. Eran
estos los individuos con quienes Serrano hizo tratos para dar la paz a
la noble Vizcaya. ¿Qué convinieron? Indulto general a todos los
insurrectos carlistas que se presentaran con armas, dándoles todo género
de garantías para su seguridad\ldots{} Los que vinieran de Francia
podían quedarse en sus hogares sin ser molestados\ldots{} Los generales,
jefes y oficiales procedentes del Ejército, que se hubiesen alzado en
armas por la causa carlista, podrían ingresar en el Ejército con los
mismos empleos que tuvieron antes de su deserción. La Diputación de
Vizcaya se reuniría con arreglo a fuero, a la sombra venerable del
\emph{Guernicaco arbola}, para determinar la forma y manera del pago de
los gastos de la guerra\ldots{} La cuestión foral se trató vagamente en
una carta del Duque, ofreciendo que todo se arreglaría de común acuerdo,
mirando a la paz duradera\ldots{}

¿Qué resultó de esto? Nada. Vinieron días de una paz artificiosa. Fue
remisión de la fiebre carlista, cuyo germen permanecía latente en la
sangre vasco-navarra, prolongando el descanso para resurgir con más
fuerza. El tiempo no quiso hacer nido entre los papeles del \emph{Trato
de Amorevieta}, y la guerra dormida, o tan sólo amodorrada, despertó y
se puso en pie en los comienzos del año que venía\ldots{} De esto nada
puedo decir, y sigo mi cuento refiriendo sensaciones personales que no
carecen de miga histórica.

Cuando menos lo pensaba, sirviéronnos comida frugal. Yo vi a la
\emph{Madre Mariana} sentada frente a mí, con la separación de una
mesilla en que aparecían diferentes platos y viandas del género pobre y
barato. Servían mujeres, de las que yo no veía más que las manos y
antebrazos. Eran dos, pues yo distinguía tres manos, a veces cuatro;
pero de esta cifra no pasaban. Sus voces sonaron como un murmullo, vago
silabeo mezclado de inflexiones de jácara. «Que me maten, pensaba yo, si
esta voz y estas manos no son las de la ninfa hechicera.» Confirmaron
tal sospecha el olor y el gusto del vinillo blanco, en quien reconocí la
poción somnífera que me dieron en la gruta señalada en mis recuerdos con
la sencilla marca del \emph{número 16}\ldots{} Me dormí, mas no tan
profundamente que dejara de advertir la partida, el arrastrar de baúles,
la cháchara de las viajeras, que en vilo me llevaron a un coche, y en él
me acomodaron como un bulto más. Rodó el vehículo con estruendo\ldots;
rodó con él el tiempo descuidado, sin señalar las horas; rodó la noche
vaga, en cuyo seno las horas se dormían también olvidadas de sus
minutos\ldots{} y uno de estos despertó de súbito y me dijo: «Excelso
Tito, estás en Vitoria.»

\hypertarget{xix}{%
\chapter{XIX}\label{xix}}

Y yo dije al minuto: «Tu hora ¿cuál es?» Y no el minuto sino \emph{doña
Mariana} me contestó: «Déjate llevar, bobito. Del coche pasamos al
tren.» Me miré, me consideré, me vi como un niño chiquitín, que no podía
valerse. Sentí hambre. Pensé que me alimentarían con biberón. Manos
blandas me cogieron arropándome. Mis manecitas tocaron un abultado seno,
y balbuciendo dije: «¿Verdad que eres Graziella?\ldots» Y una mano menos
blanda me azotó en los cuartos traseros, y oí dulces palabras: «A
callar, a dormir\ldots{} ro\ldots» Por el traqueteo rítmico que venía de
abajo, conocí que no íbamos en coche, sino en el tren. Yo dormitaba, y
mi vago soñar, reproduciendo cosas pretéritas, era cortado a trechos por
el canticio melancólico que marcaba las estaciones y los puntos de
parada. Los sueños que elaboraba mi cerebro eran pasajes de intensa
zozobra, con opresión cardiaca y temor de inminente peligro. Mi primera
zozobra fue si alcanzaría o no el vapor para Civitta Vecchia\ldots{} Que
no lo alcanzaba; que salía momentos antes de llegar yo\ldots{} Allá va
el vapor sin mí; allá va\ldots{} Y en esto sonaba el triste canto:
\emph{¡Pancorbo, un minuto!}

Pensé yo que un minuto no me daba tiempo para embarcarme en otro
vapor\ldots{} El \emph{traca traca} del tren siguió arrullándome, y en
mi cerebro aparecía nueva inquietud opresora. En mi discurso de Durango,
se me había olvidado una parte importantísima. A muchos de mis oyentes
repugnaba la palabra República, aun retocada y ennoblecida con los
perifollos de \emph{Católica} y \emph{Pontificia}. «No, queridas
hermanas; no, hermanos del alma, no os alborotéis por la fealdad de una
palabra, similar de todo escándalo y del delirio de la sanguinaria
plebe\ldots{} Callad, escuchadme: os sobra razón, y en armonía con
vuestros sentimientos doy a los gloriosos Estados el nombre de
\emph{Imperio de Cristo}, \emph{Imperio Hispano-Pontificio}\ldots{} ¿Os
satisface? ¡Viva nuestro Emperador y Rey Pío I, quiero decir Nono, que
el número no hace al caso!» En esto la divina voz melancólica clamaba en
el silencio frío de la noche: \emph{¡Quintanapalla, un minuto!}

El espantoso ruido del tren pataleando sobre las placas giratorias al
entrar en una estación grande, me hizo saltar en el regazo de la
incógnita hembra que me agasajaba. Pregunté dónde estábamos, y oí que
habíamos llegado a Burgos. No me tranquilicé con la idea y el honor de
estar en la ilustre \emph{Caput Castellæ}, y seguí con mis ansias y
zozobras al compás del fogoso vehículo que me llevaba traqueteando a lo
largo de las Españas. Vi que contra mí venían los bárbaros jayanes
hostigados por dos curas impíos y soeces, deshonra de su clase. La
bestial plebe me apaleó; arrastrado fui por el suelo y lanzado a un
campo de ortigas 4\ldots{} Recogíame con dulce piedad Pepita Izco; me
lavaba las heridas, me bizmaba con delicadas manos; con el bálsamo de
sus caricias me restauraba el cuerpo y el alma, y llevándome a su casa
en brazos de las fornidas doncellas que la servían, en su propio lecho
blando y anchuroso me acostaba, ¡ay!, a punto que el cantor triste del
tiempo y de la noche decía, estirando la voz: \emph{«¡Torquemada, un
minuto!»} Oyéndolo, pensaba yo que Torquemada, con sus hórridas hogueras
y sus crueles suplicios, era más humano que la bestial plebe
duranguesa\ldots{}

Pasado este angustioso trance, volví a la primera zozobra: ¿Alcanzaría
el vapor para Civitta Vecchia? No lo alcanzaría, por no llevar el tren
la vertiginosa marcha necesaria para llegar a Marsella en corto tiempo.
Cuando creí que el cantor nocturno clamaría \emph{Marsella, parada y
fonda}, gritó: \emph{Venta de Baños, cambio de tren para
Santander}\ldots{} Pensé que siendo Santander puerto de mar, allí
encontraríamos vapor para Italia\ldots{} Pero no iba nuestro tren en
aquella dirección que me sacaría de mis apuros. Oí cantar \emph{Dueñas},
luego \emph{Valladolid}; después \emph{Arévalo},
\emph{Sanchidrián}\ldots{} Cuando pasamos de la patria de Santa Teresa,
la \emph{Madre Mariana} me tomó en sus brazos y me zarandeó gozosa
diciéndome: «Titín, chiquitín, arroja de tu mente todas las ideas, todas
las impresiones, recuerdos de aquella \emph{Carquilandia} que ha sido
para ti un destierro, en algún modo tedioso y mortificante. Pero no
creas que allí has perdido el tiempo, no; en aquella tierra de hombres
inocentes y bravos has aprendido más de lo que pensabas. Mucho vale,
hijo mío, el aprendizaje de cosas y personas que allá tuviste; mucho
vale el dato de Vasconia, documento vivido por ti, para que lo agregues
a los estudios que han de darte el total conocimiento de la vida
hispana.»

Con filial mirada y breves voces accedí a cuanto la cariñosa,
\emph{Mariana} me decía. En aquel punto me sentí tan extremadamente
chiquitín, que al colocarme ella al amparo de su brazo derecho, pude
medirme fácilmente, pude ver y comprobar que yo no era más largo que su
brazo, desde el sobaco a la punta de sus dedos. Yo menguaba, yo había
disminuido considerablemente de talla, y así debía creerlo mientras no
se me demostrara que ella crecido había hasta un tamaño doble o triple
del que tenemos por natural.

Al otro lado del vagón, dos mujeres arrebujadas y encogidas dormían
profundamente. Con el tapujo de sus pañolones no se les veía el rostro.
En los dos montones de arropadas carnes, inmovilizadas por el descanso,
descollaban las ancas poderosas. Esto vi a la incierta luz de la lámpara
cenital cubierta de un trapo verde. \emph{Doña Mariana} no dormía.
Sentada estaba en el rincón junto a la portezuela, teniéndome
agasajadito en el espacio, grandísimo a mis ojos, entre su brazo derecho
y el costado correspondiente. Blanduras tibias rodeaban mi mezquino
cuerpo en aquel nicho sagrado.

De él me sacó la Diosa cuando habíamos traspasado el caballete del
Puerto, y poniéndome sentadito sobre su muslo izquierdo, me dijo:
«Pronto veremos la claridad del alba. El día nos saluda siempre en este
paso de la Vieja a la Nueva Castilla. Y pues estamos, como quien dice, a
las puertas de esa Villa, cueva o nidal de todas las alimañas que
intervienen en la vida pública, aquí recobro la plenitud de mis
funciones, y uno de mis primeros actos será tomarte a mi servicio,
utilizando tu agudo ingenio y la sutileza con que te cuelas allí donde
algo se guisa que pueda interesarme. Tu vista y oído son excelentes
órganos de observación. Pequeño eres; más pequeño, casi imperceptible
serás cuando me sirvas en calidad de corchete, confidente y mensajero.»

Respondile que desempeñaría con orgullo cuantas encomiendas quisiera
encargarme, y cada palabra que salía de mis labios achicaba, a mi
parecer, mi ya corta estatura. O yo padecía una horrenda perturbación de
mis sentidos, o era del tamaño de un gatito en la edad juguetona. Mordía
yo suavemente un dedo de la \emph{Madre Mariana} para demostrarle mi
cariño, y con sus dedos me abrazaba ella y jugaba con mi cuerpecillo
blando y dúctil.

El tren descendía rápidamente. Amaneció\ldots{} Oí el clamor ferroviario
que nos dijo: \emph{Escorial}, \emph{cinco minutos}. Vino luego
\emph{Villalba}; siguió \emph{Torrelodones}\ldots{} Ya día claro,
\emph{doña Mariana} llamó a las mujeres durmientes, incitándolas a
prepararse para la llegada. Pero ellas continuaban como piedras en el
apretado envoltorio de sus mantas y mantones. La señora, puesta en pie,
se cubrió de un luengo balandrán; cogiome con viva manotada, y
doblándome sobre mí mismo me guardó en un hondo bolsillo de aquella
prenda lujosa.

Desde mi cárcel holgona y forrada de seda olorosa, oí la voz de la que
bien puedo llamar mi ama, despertando a las mujeres. Estas gruñían
desperezándose\ldots{} Con el canto de \emph{Pozuelo}, \emph{dos
minutos}, se confundía el ajetreo de las tres féminas requiriendo sus
maletas y cinchando con correas sus envoltorios de viaje. En tanto, yo
me desperezaba y sacudía en mi cárcel sedosa. Nada veía; pero al tacto
pude apreciar que no estaba solo y que otros seres blandos y menudos
iban conmigo en la prisión\ldots{} Total, que llegamos a Madrid.
Claramente percibí la salida del tren, el paso por la estación, la
entrada en un coche y\ldots{} ya no más, ya no más. Mis sensaciones se
perdieron en un sopor delicioso y rosado, tirando a violeta\ldots{} No
sé cómo expresarlo.

Al llegar a este punto, el más delicado, el más desaprensivo de esta
historia, me detengo a implorar la indulgencia de mis lectores,
rogándoles que no separen lo verídico de lo increíble, y antes bien lo
junten y amalgamen; que al fin, con el arte de tal mixtura, llegarán a
ver claramente la estricta verdad. A riesgo de que no me crean, les digo
que me encontraba en la plena conciencia de mi yo espiritual y físico;
yo era yo mismo en mi ser inmanente; gozaba la serena vida fisiológica,
la vida pensante y erudita, pues todo lo que supe sabía, y mi memoria se
armonizaba con mi entendimiento; yo estaba bien comido y perfectamente
apañado de todas mis necesidades y estímulos; yo bebía y fumaba; yo iba
por las calles saboreando la inefable dicha de que nadie me viera ni en
mi diminuta persona reparara; yo disfrutaba el placer de verlo todo sin
ser visto, y de ejercitar el don de la crítica, el don de la burla, más
precioso aún, sin que nadie por ello me molestase; yo podía reírme a
mansalva de todo ser viviente, del Rey para abajo, y no encontraba freno
ni obstáculo a mi observación fisgona; ante mí no había puerta cerrada
ni pared que me cortaran el paso; me congraciaba de mi suerte
diciéndome: «Por San Tito mi patrón y por Santa Clío mi madre, que es
linda cosa el oficio de duende.»

En calidad y funciones de tal, avanzaba yo una tarde por la Plaza de
Oriente, y después de rodearla toda contemplando el caballo de bronce,
me metí en Palacio por la puerta del Príncipe. En el largo zaguán, desde
la puerta al patio, me encontré de manos a boca con mi amigo Quintín
González, imponente y colosal portero, vestido de casacón colorado, con
los aditamentos solemnísimos de tricornio y cachiporra. Ante él me
planté puesto en jarras y le felicité por su hermosura monumental. Con
gran sorpresa mía, Quintín permaneció impasible y tieso, sin contestarme
ni fijar en mí sus miradas. En aquel momento me hice cargo por primera
vez de que yo era invisible o poco menos, y sin solicitar de nuevo la
comunicación amistosa con el amigo, acordeme de su mujer y de mi amoroso
enredo con ella en días lejanos, allá por los fines del 70 y principios
del 71.

Entráronme vivas ganas de ver a Nieves, y con resuelto paso me lancé a
las alturas por la escalera de Cáceres. Recorrí alegremente todo el piso
segundo, todo el tercero, rememorando alegres días. No encontré a la
esposa de Quintín en la habitación donde antes moraba; tampoco encontré
a mi pariente don José Folgueras, empleado en la Intendencia\ldots{}
Metime en diferentes casas cuyos inquilinos desconocía, y en una de
ellas se me apareció la frescachona Nieves, así llamada irónicamente,
pues era su persona el trasunto de los ardores caniculares. Había
mejorado considerablemente de posición y jerarquía, que bien lo
declaraban su compostura y traje, así como el adorno de la sala. En esta
la vi sentadita frente a un alabardero, el cual, inclinado con abandono,
le acariciaba las manos pronunciando las palabras galantes que inician
una campaña de amor\ldots{}

Yo me reía y observaba. Brincando pasé entre las piernas de uno y otro
sin que ellos se percataran de mi presencia. Salté a una silla; de esta
me encaramé en la cómoda; me entretuve mirando retratos colocados en
esterillas, y entre ellos vi el mío, que a Nieves regalé dos años antes.
La estancia revelaba un progreso enorme en el bienestar del matrimonio
Quintín-Nieves. Esta no era ya planchadora de la \emph{Real Casa}; debía
de ser azafata, moza de retrete o no sé qué\ldots{} De un brinco volví
al suelo. El alabardero, echando hacia atrás los vuelos de su capa
blanca, se aproximaba tanto a Nieves que su larga perilla rozó los
labios de ella. En uno y otro, la alegría del alma mostrábase con el
reír gozoso y voluble. De pronto Nieves cogió del sofá el tricornio de
su adorador y se lo puso. Con rápido andar corrió a mirarse en el
espejo. Tras ella fue el galán, y abrazándola por la cintura, ambos
contemplaron sus rostros risueños en el espacio reproducido por el
cristal. Yo me dije: «Vaya, vaya; ni aun en mi condición de invisible me
resigno a presenciar la felicidad ajena, con mi gorro bien calado y mi
velita en la mano. Abur, avecillas en celo; divertíos todo lo que
podáis.»

Salí de estampía y conmigo salió el gato de la casa, que por efecto de
la picante escena iba en busca de lo suyo. El ligero paso del morrongo
guió los pasos míos y tras él seguí escaleras abajo, no sé si por la de
Cáceres o por otra de las muchas que allí hay. Ya era de noche y el gas
alumbraba todos los pasajes, conductos y rincones del inmenso caserón
real. No puedo dar idea del sinnúmero de peldaños que descendí. En un
rellano encontré a mi gato, con otros individuos de su especie,
maullando y haciendo la carretilla. Su lenguaje no era para mí
totalmente ignorado. También ellos y ellas jugaban, se perseguían y se
enzarzaban en enredos amorosos\ldots{} Descendiendo más, el olfato y el
ruido de voces hondas me anunció las cocinas.

En ellas penetré, y vi la caterva de cocineros y marmitones que
aderezaban la real comida. Era también la hora de servirla, y en el
ancho recinto abovedado vi movimiento y barullo que me dejaron suspenso.
Daba el jefe voces de mando, como general en el momento crítico de una
batalla. Los hombres de blanco gorro hacinaban en las fuentes con ágiles
dedos las piezas de carne, legumbres y pescado, con el adorno de mil
porquerías comestibles. Otros armaban los castilletes de repostería y
postres de cocina. Todo el comistraje iba pasando al pie del ascensor,
por donde las copiosas bandejas subían al piso principal, como en los
buques de guerra suben los proyectiles desde la bodega hasta la batería
donde están emplazados los cañones.

Recorrí todo el antro, y movido de mi curiosidad intensa me metí en un
grupo de marmitones, que arreglaban las fuentes catando de todo por arte
o glotonería. Algunos de ellos comentaban con burletas el extraño gusto
de don Amadeo. No comía más que carne guisada simplemente, que los
italianos llaman \emph{lesso}, y patatas cocidas. Uno que parecía
italiano aseguró que lo mismo comía Víctor Manuel. El postre de nuestro
Soberano eran guindas en aguardiente que le mandaban de Turín,
aderezadas con pimienta en grado tan fuerte que cuantos lo probaban aquí
escupían los hígados.

La vista del monta-cargas me atraía. Reconocida ya la oficina culinaria,
me lancé a él escabulléndome entre rimeros de chuletas y montañas de
hojaldre. Subí\ldots{} Encontreme en una habitación donde estaba la
estufa en que se colocan las fuentes para conservar el calor. Allí, los
mozos, a la voz de un maestresala llevaban los manjares al comedor
llamado \emph{de diario}. Con rápido paso en el comedor me colé. Vi al
Rey y a la Reina en las respectivas cabeceras. Vi damas, gentiles
hombres, militares de la guardia, ayudantes del Rey, y oí la festiva
charla trilingüe, pues sobre el castellano, a lo largo de la mesa,
flotaban frases y conceptos italianos y franceses. Exploré con alegría
juguetona la hermosa estancia; contemplé las pinturas del techo, los
espejos, cuadros y tapicerías que ornaban las paredes, las suntuosas
mesas, relojes y candelabros\ldots{} Ni encogido ni perezoso, creyendo
que vistas las alturas y los medios debía investigar también lo
rastrero, me metí debajo de la mesa, y la recorrí holgadamente de punta
a punta por la calle que dejaban libre los pies de las dos filas de
comensales.

Allí me entretuve observando los bien calzados piececitos de las
señoras, las caladas medias y los bajos finísimos guarnecidos de
encajes. Por otro lado vi botas con espuelas, conteras de sables,
pantalones galonados\ldots{} Hasta mí llegaba, repercutido por la madera
que allí era mi techo, el sonido de la conversación ceremoniosa. La mesa
era para mí una caja armónica que me transmitía las inflexiones más
leves de la voz humana. La Reina hablaba un castellano gramatical,
premioso, aprendido por principios. Los entorpecimientos de su palabra
revelaban el temor a equivocarse. Don Amadeo hablaba torpemente, como
quien todo lo aprendía de oídas y sin estudio. Al fin de la comida me
regocijó la escena en que el Rey, con galantería maleante, quería
obsequiar a señoras y caballeros con las famosas guindas de Turín. Todos
declinaban riendo el honor de probarlas. Una dama, cuyo nombre ignoro,
dijo que una vez que cató las guindas se le abrasó la boca y estuvo
enferma de estomatitis. Un caballero, ayudante del Rey, alabó a este por
tener su boca indemne contra el fuego. La risa terminó con libaciones
discretas de jerez y \emph{champagne}. Todos bebieron menos el Rey que
no cataba el vino.

Terminada la comida, desfilaron. Yo salí de los últimos, y pude ver a
los camareros bebiéndose lo que quedaba en algunas copas. Como esto no
me interesaba, corrí tras de las reales personas, y de estancia en
estancia llegamos a una que llamaban (después lo supe) \emph{Despacho
del Rey}. La Reina con las Condesas de Almina y de Constantina formó
corrillo en el testero principal, junto a la chimenea entonces apagada.
Sobre ésta lucía un retrato de María Luisa, por Goya, maravilla de la
pintura. Embelesado estuve un rato mirando la figura genuinamente
borbónica de aquella Reina frescachona, de boca hundida y ojos de fuego.
El pintor, atento a destacar lo más hermoso del modelo, se había
esmerado en reproducir su brazo incomparable.

Retozando sobre la blanda alfombra de Santa Bárbara, me enteraba yo de
cosas y personas. La tertulia de Sus Majestades después de comer no era
muy lucida. Ningún personaje de importancia, ningún prócer de primera
fila, vi entre los asistentes a la real sobremesa. Toda la concurrencia
era puramente palatina y del Cuarto Militar. Habló la Reina del Convenio
de Amorevieta, que estimaba beneficioso\ldots{} por el momento\ldots{}
Díaz Moreu le dio detallada explicación de las bases de aquel arreglo;
elogió con ardor al Duque de la Torre, hombre de altas miras. Según
dijo, el Convenio sería discutido en las Cortes y tendría la aprobación
de todos los elementos dinásticos. Esperaba que de esta discusión
saldría el Gobierno con mayor fuerza. Hablaron después de Ruiz Zorrilla,
lamentando su alejamiento de la vida pública, en su retiro de Tablada.
Doña María Victoria expresó tímidamente sus dudas de la eficacia del
Convenio de Amorevieta. ¿Quién podía responder de que los carlistas,
rehechos más allá de la frontera, no volverían con mayor furia a
encender la guerra civil? Contra su terquedad nada valdría la razón,
nada el interés de la Patria. Extremando su galantería, Díaz Moreu no se
atrevió a disipar en absoluto las dudas de la Reina y casi las confirmó
diciendo: «Tal vez, Señora. Vuestra Majestad discurre siempre con
admirable previsión. El carlismo es de calidad muy dura,
irreductible\ldots{} Con esa gente no hay día seguro.»

Por lo que después oí de labios de doña María Victoria, comprendí que
esta señora se cuidaba de los asuntos públicos y en ellos ponía toda su
atención. En su grande ánimo prevalecían la idea y propósito de
consolidar en España la dinastía de Saboya. Manteniendo su propia
persona en cierta obscuridad modesta, enderezaba su voluntad firmísima
hacia el porvenir de sus hijos en tierra hispana\ldots{} Hecha esta
observación pasé a fisgonear en el grupo que al otro lado de la estancia
formaba el Rey con los amigos de su mayor intimidad. Allá me fui ligero,
resbaladizo, invisible. Lo que oí agazapadito debajo de la silla en que
don Amadeo se sentaba, merece capítulo aparte.

\hypertarget{xx}{%
\chapter{XX}\label{xx}}

Lo primero que le cuento al lector amable y antojadizo es que nuestro
buen Rey saboyano desdeñaba los riquísimos tabacos habanos de regalía,
de que había grande acopio en la Casa Real. El mismo desaire que sus
amigos hicieron a las abrasadoras guindas de Turín hizo él al tabaco
generoso y suave de la \emph{Vuelta Abajo}. Por hábito y gusto fumaba el
hombre los apestosos cigarros que en Italia llaman \emph{virginia},
consistentes en un luengo y nefando cachirulo que lleva en su ánima una
paja, sin la cual no hay quijadas que los hagan arder. Amable y guasón,
a sus amigos ofrecía las cajas de habanos diciéndoles: \emph{Fumen eso;
yo virginia}. Para evitar el continuo encender de fósforos, que sin
fuego constante no hacía tiro la pajilla, Su Majestad tenía en una
mesita cercana una vela encendida, y a la llama de esta aplicaba el
chicote.

Junto al Rey estaba el Barón de Benifayó, Montero Mayor de Palacio,
alto, moreno, expresivo, de arqueadas cejas, lentes de oro. Como hablaba
de corrido y limpiamente el italiano, con él descansaba don Amadeo del
suplicio del idioma español, que en dos años no había podido dominar. A
la vera de don Amadeo vi otros señores, que no pude identificar por mi
desconocimiento del personal palatino. Vestían de paisano. ¿Era uno el
General Gándara o el General Rosell? ¿Era el otro don Cipriano Segundo
Montesinos? No puedo asegurarlo. Reconocí a Dragonetti, a Díaz Moreu y
al General Burgos, de uniforme, que dejaron a la Reina conversando con
las damas, el Conde de Rius y otros dos palaciegos gordinflones que yo
no conocía.

En el corrillo del Rey, la conversación era frívola, de temas fugaces
que pasaban rápidamente de boca en boca. En un momento que a mí me
pareció solemne vi a la Reina levantarse. Hizo una reverencia de Corte,
y seguida de las damas se retiró a sus habitaciones. Empezó el desfile
de los caballeros en dirección de la Saleta, hasta que solos quedaron
don Amadeo y Benifayó. Encendió Su Majestad otro \emph{virginia}. El Rey
y su Montero hablaron breve rato en italiano bajando la voz, pues aunque
nadie quedaba en la estancia, temían el misterioso escuchar de las
paredes. Servidores galonados pasaban por el \emph{Despacho del Rey}.
Les sentí cerrando puertas y apagando luces en las habitaciones
próximas. Pensé que mis funciones inquisitivas me ordenaban no apartarme
del Rey y su Montero hasta saber qué harían. A mi parecer, dejaban
correr el tiempo esperando la ocasión oportuna para escabullirse de
Palacio. No me engañaba.

Llegó un instante en que el silencio y la tranquilidad, tardíos
cortesanos, se posesionaron de la Casa de los Reyes. Don Amadeo y su
Montero se filtraron, vamos al decir, por la puerta de servicio. Pisando
quedo y sin decir palabra, atravesaron un pasillo alumbrado con mecheros
de gas. Torcieron a la derecha, luego a la izquierda. Ningún servidor
les salió al paso, ni tuvieron otro testigo de su escapatoria que mi
traviesa personalidad invisible. Llegaron, llegamos debo decir, a la
escalera de caoba que llaman \emph{de la Intendencia}. Descendimos
suavemente. Gemían los peldaños alfombrados bajo las pisadas de ellos,
no de las mías; que yo era poco más que un espíritu\ldots{} Fuimos a
parar a un pasillo: en él vi dos servidores que estaban en el ajo, y
saludaron con leve reverencia. De allí salimos a la Plaza de la Armería,
donde esperaba un coche de un solo caballo y cochero sin librea. Entró
el Rey en el coche; tras él Benifayó. Algo noté entre el Montero y el
oficial de guardia, que me indicó la connivencia de este. No necesito
decir que me colé de un brinco dentro de la berlina, achantándome
bonitamente en la bigotera\ldots{}

El coche partió hacia la Plaza de Oriente y calle del Arenal. Era la
noche plácida, de mejor temple que el día, como suele acontecer en las
primaveras matritenses. Por la Puerta del Sol y calle de Alcalá
discurrían los vecinos noctámbulos que salen de los teatros para meterse
en los cafés. En Recoletos vimos poca gente; no faltaban los ciudadanos
de la última capa social que tienen por alcoba y cama las sillas de
hierro, o la escalinata de la Casa de la Moneda. Desierta estaba la
Castellana. El coche la recorrió en casi todo su largo y fue a parar en
un hotel próximo a la calle de la \emph{Ese}. Alguien abrió desde dentro
la verja, y la berlina penetró en un jardinillo de incipiente
frondosidad. Momentos después, las tres ilustres personalidades
franqueábamos corta gradería y entrábamos en una linda sala bien
iluminada, donde fuimos recibidos por una dama\ldots{} Espérate un poco,
picaresco lector, que esto es muy delicado.

Era la tal de mediana talla, bien formada y no mal constituida de carnes
y anchuras. Mi primer cuidado fue examinarle bien el rostro, que vi
entonces por primera vez. Mi crítica lo declaró tan agraciado como
hermoso; la tez morena, ojos expresivos, grande la boca, tan abundante
el pelo que no se contenía dentro de sus límites naturales,
extendiéndose por delante de la oreja como un rudimento suave de
varoniles patillas. El conjunto de tal rostro tenía el encanto de la
originalidad, que en arte como en belleza es poderoso atractivo.
Sentáronse los tres arrimados a una mesa, la dama y el Rey juntitos,
mano con mano; frente a ellos Benifayó\ldots{} Yo me subí de un brinco a
la consola próxima para ver bien y pescar todo lo que hablaran. La
señora, que vestía luenga bata de seda blanca libremente descotada,
dejando ver los linderos de un lozano busto, revelaba en sus ojos
chispos y en su franca sonrisa el gozo de ver terminada felizmente una
larga y ansiosa espera. Anhelaba, sin duda, comunicar a su regio amigo
impresiones guardadas durante lentas horas y aun días. La ocasión de la
dichosa confianza llegaba al fin. No podía contenerse, y prorrumpió en
estas calurosas manifestaciones: «Ya supongo, \emph{mon lion brave et
généreux}, que no te habrás tragado el pastel que llaman \emph{Convenio
de Amorevieta}. No te fíes del Duque. Su intención no es mala; pero en
la diplomacia militar no da pie con bola. Los carlistas tratarán ahora
de rehacerse, y volverán pronto más insolentes y feroces a disputarte el
Trono\ldots{} Si las Cortes aprueban el \emph{Convenio}, el Duque, ¡oh
Rey mío!, te pedirá la suspensión de garantías, pues sin hacer mangas y
capirotes de la Constitución no podrá gobernar.»

\emph{---Yo contrario}. He jurado \emph{(giurato)} la Constitución.
Gobernar sin ella no puede ser. \emph{Yo contrario}.

---No debiste consentir que don Manuel, desalentado y aburrido, se
retirase a Tablada. Ten presente, Rey de España por los 191, que no has
venido aquí a continuar la política de los malditos Moderados, de los
Unionistas rutinarios y pasteleros. Por ese camino no vas a ninguna
parte.

---Es cierto, Adela. \emph{Yo conforme}.

---Ni la guerra puede ser sofocada para siempre sino con la guerra
misma---dijo ella disfrazando la pedantería con mohínes graciosos,---ni
la política debe estancarse o petrificarse\ldots, no sé cómo
decirlo\ldots{} No has venido a España para gobernar como la pobre doña
Isabel\ldots{} Para ese viaje no necesitabas alforjas\ldots{} Fíjate en
este refrán castizo; repítelo para que se te grabe en la memoria\ldots{}
Alforjas\ldots; a ver, a ver cómo nos pronuncias esa jota\ldots{}

Intentó el Soberano un aprendizaje de pronunciación castellana; mas lo
hizo tan desgraciadamente, que él mismo se reía de su torpeza antes que
los demás riéramos. En esto entró un criado, vestido de frac, con dudosa
corrección, y colocó en la mesa servicio de té, con galletitas y
emparedados. A una orden de la señora, desapareció el sirviente,
volviendo al punto con un mazo de los infernales cigarros
\emph{virginia}, predilectos de Su Majestad. Cayeron los dos caballeros
sobre los \emph{sandwichs}, mientras la señora servía el té, y a mí, lo
confieso, me asaltó la idea de plantarme en la mesa y comer con ellos,
satisfaciendo mi hambre nocturna. Mas recordando mi calidad de sabandija
perteneciente al mundo suprasensible, me abstuve de tomar parte en el
refrigerio. Temía que un rasgo de animalidad me descubriese, deshaciendo
el artilugio que me había transformado de persona grave en duende
corredor. Si una indiscreción o exceso de travesura me restituyese de
súbito a mi ser propio, ¡no te arrendara yo la ganancia, pobre Tito!

Entre mordiscos a los emparedados y sorbitos de té, la dama de las
patillas anudaba la serie de sanos consejos al amigo y Rey. Intervino
Benifayó realzando con tímidas palabras la persona del General Serrano.
Entre la dama y el Barón se trabó una donosa controversia, en que
salieron a relucir duques y duquesas con otras bien conocidas personas
de la \emph{crema} social. En todo lo que allí se dijo puse yo mi
atención; pero mis funciones en cierto modo históricas me obligan a
seleccionar los conceptos que oí, reservándome tan sólo los que
entrañaban algún interés público.

«Si vale el consejo de una mujer---dijo la dama poniendo su blanca mano
sobre el hombro de Amadeo,---yo diría que debías mandar a Tablada un
mensajero\ldots; persona discreta y aguda tenía que ser\ldots; un
mensajero que pudiera cazar con lazo de buenas razones a Ruiz Zorrilla
y\ldots{} Debes tener muy presente, león de Saboya, que para remover del
fondo a la superficie la vida política, las costumbres políticas, y
\emph{toda la pesca}, determinó Prim traer a España un Rey nuevo, un Rey
\emph{de fuera} que nos diese lo que no teníamos, y acabara con el
tejemaneje moderado y unionista. Hacer una revolución, poner todo patas
arriba, cambiar de dinastía para volver a las viejas mañas, al
polaquismo, al hoy tú, mañana yo, me parece que es como si quisiéramos
aplicar a la vida de la Patria el juego de las cuatro esquinas\ldots»

En un tris estuvo, podéis creérmelo, que saltara yo desde la consola al
regazo de la patilluda señora para felicitarla por su atinado consejo.
¡Qué discreción, qué talento, qué golpe de vista! Yo me decía: «De casta
le viene al galgo. Ya sé que te engendró el primer escritor del siglo.»
Abstraído un momento en estas consideraciones, vi que el Rey y la dama
blanca se escabullían por una puerta próxima al mueble donde tenía yo mi
observatorio. Advertí disminución de la luz\ldots{} El bueno de Benifayó
¿dónde estaba?\ldots{} Creí verle arrimado a la mesa hojeando una
revista ilustrada\ldots{} Creí que salía por la puerta que nos había
dado ingreso. Por primera vez desde que era duende dudaba de la justeza
de mi perfección visual. Pero es mi deber no interrumpir mi cuento; que
para seguir con vista y oído el curso de la humana vida en estas
historias me llevaron al recatado lugar donde me encontraba. Adelante,
pues.

La fatalidad me obliga, ¡oh lector agudísimo y picaruelo!, a continuar
en forma que sin duda no ha de agradarte. Tengo que emplear en mi
escritura los signos simbólicos más discretos. Meto la mano en una
escarcela bien provista que me colgó de la cintura mi \emph{doña
Mariana}, y saco un puñado de puntos suspensivos y los derramo sobre el
papel para que te entretengas leyéndolos o descifrándolos. Ahí
van.\dotfill

\noindent\dotfill

Aturdido recorrí brincando toda la habitación; salí al jardín; no vi
alma viviente. El coche no estaba. ¿Había partido en él Benifayó para
volver más tarde? No lo sabía ni me importaba averiguarlo. Cerrada la
puerta de hierro, trepé por las enredaderas que cubrían la verja y de un
brinco me puse en la calle. Al pisar el suelo de la Castellana me
reconocí en mi normal estado físico. Yo era quien era, Proteo Liviano,
conocido por Tito en el vago mundo del periodismo y de las letras. Mi
primer cuidado fue desandar a buen paso la Castellana, Recoletos\ldots{}
En la Cibeles el reloj de Buenavista me dijo que eran las dos de la
mañana\ldots{} Tomé el camino de mi casa, calle del Amor de Dios,
hospedaje de doña Nicanora, esposa del evaporado filósofo don José Ido
del Sagrario.

Agasajado en mi cama me adormecí jugueteando con estos acertijos: ¿Era
verdad que mi buen padre me había llevado a Durango, que hice allí vida
patriarcal y soñolienta entre carlistas fieros y curas de armas tomar?
¿Eran reales las figuras de Choribiqueta, Fabiana Iturrigalde y Pepita
Izco? ¿Había yo en efecto espetado a los cándidos durangueses un
discurso chancero sobre la \emph{República Hispano-Pontificia}? ¿Era
verdad que la \emph{Madre Mariana} me había sacado de aquel atolladero,
tomándome a su servicio, para lo cual hube de transformarme en duende
minúsculo y gracioso, sutil espía de la historia privada?\ldots{} Si
todo esto fue mentiroso aparato forjado por mi exaltada imaginación y de
ello puede resultar que lo verosímil sustituya a lo verdadero, bien
venido sea mi engaño, y allá van, con diploma de verdad, los bien
hilados embustes.

En aquellos días anduve de bureo político con mis amigos Mateo Nuevo,
Roberto Robert y don Santos La Hoz, que me felicitaban por haber
recobrado mi equilibrio cerebral. Fui a la tribuna de las Cortes; oí un
gran discurso de Cristino Martos de fiera oposición al Gobierno;
presencié los ardientes debates sobre el \emph{Convenio de Amorevieta},
terminados con votación que dio al Gobierno formidable mayoría. A pesar
de esto corrían voces desfavorables para la situación Serrano-Topete.
Decíase que el Duque, abrumado por las dificultades que se le venían
encima, había pedido al Rey la suspensión de garantías y que don Amadeo
respondió secamente con su acostumbrada fórmula: \emph{Yo contrario}.
Despiertos y animosos, los radicales corrieron en Comisión a Tablada
logrando atrapar a don Manuel Ruiz Zorrilla y traerlo a Madrid. Total,
lector mío cachazudo, que sobrevino la quinta o sexta de las crisis que
amenizaron aquel reinado. Cayó el Duque de la Torre, dejando el puesto a
Ruiz Zorrilla, que formó Ministerio con Martos, Montero Ríos, General
Córdoba, Ruiz Gómez, Beránger, y Gasset y Artime. Íbamos viviendo.

Engalláronse más los alfonsinos. Hablaban de la Restauración como si la
tuvieran en la mano. Los federales del grupo intransigente y levantisco
echaban bombas. Los Clubs y Casinos ardían en protestas, en arengas
fogosas, en amenazas furibundas a todo lo existente. Me pidieron que
hablara y hablé, soltando todo el surtidor de mi nativa facundia
oratoria. Nadie me atajó; a nadie parecieron extremadas mis
lucubraciones. La misma boca que predicó en Durango la \emph{República},
mejor dicho, el \emph{Imperio Hispano-Pontificio}, vociferaba en Madrid
anunciando el próximo advenimiento del \emph{Federalismo Sinalagmático y
Cantonal}. ¡Abajo la Unidad centralista y corruptora, arriba el Cantón
autónomo que por medio del Pacto reconstruiría la patria libre,
devolviendo al ciudadano su dignidad y soberanía! Aplausos frenéticos y
plácemes cariñosos recompensaban mi palabrería furiosa.

La corriente social me devolvió, entrado ya el mes de Julio, al
afectuoso trato de Mateo Nuevo, que generosamente me ayudaba en mis
penurias. Volví a frecuentar su casa, Montera, 11, donde acudían casi
todos los amigotes mencionados en los comienzos de este libro. El
jacobino \emph{Tribunal del Pueblo} ya no se publicaba; pero existía,
con el nombre de Redacción, el punto de cita de los que regían las
muchedumbres populares, titulándose \emph{presidentes de los Comités de
distrito}, \emph{presidentes de Juntas revolucionarias}, con otras
denominaciones que sólo han servido para distracción y entretenimiento
de los partidos avanzados. A poco de frecuentar la sala cuyos balcones
caían a la obscura calle de los Negros, me dio en la nariz olor de
conspiración aguda.

Al comunicar mis sospechas a un amigo candoroso, este me dijo: «Sólo se
trata de producir en Madrid la conveniente alarma con objeto de que el
Gobierno no saque tropas de aquí para mandarlas a las plazas de
provincias. Se prepara\ldots, en confianza te lo digo\ldots, un
movimiento general en toda España. Ahora va de veras. Se alzarán Ferrol,
Santoña, Cartagena, Sevilla, Badajoz, \emph{etcétera}. Ello está tan
bien dispuesto que el triunfo es seguro, tan seguro como tenerlo en la
mano. No falta más que una cosa, Tito, y es producir en Madrid agitación
tan grande que el Gobierno no pueda sacar tropas. ¿Lo entiendes? Ello es
clarísimo. Te digo esto con la mayor reserva. No hables a nadie\ldots»

No daba yo gran crédito a tales monsergas. Mil veces había llegado a mis
oídos el susurro de alzamientos generales o locales sin que los hechos
correspondieran a las risueñas esperanzas. El optimismo de los
revolucionarios sencillotes y pillines, que creen lo que sueñan, es un
fenómeno habitual en tiempos turbados. Manteníame yo escéptico,
convencido de que no había más revolución que la formulada en ardientes
discursos, revolución puramente teórica y verbal. Por eso yo, sempiterno
hablador, era el primer revolucionario de la época y el primer oráculo
de un resurgimiento que no quería venir. La Patria no podía contar aún
con la acción de sus hijos, y debía contentarse con la resonante
canturía de sus oradores. Desconfiado de la eficacia de la acción,
continuaba yo atento al trajín de los conspiradores, y a su chismorreo
sigiloso en la vacía redacción de \emph{El Tribunal del Pueblo}. De ello
me distrajo, al promedio de Julio, el hallazgo feliz de una
mujer\ldots{}

Tomo aliento, amados lectores, con lo cual, al contarlo, expreso mi
sorpresa y turbación ante la súbita emergencia de un pasado lisonjero.
La mujer que se me apareció en la calle de la Sal, junto al arco de la
Plaza Mayor, era la poética, la romántica Obdulia con quien compartí las
venturas del amor en los comienzos del reinado de Amadeo I\ldots{}
Obdulia, ¡oh!\ldots{} Tito, ¡ah!\ldots{} Al tiempo de lanzar estas
exclamaciones se juntaron en febril apretón nuestras manos, y con frase
entrecortada nos dimos informes recíprocos de la salud y vida de uno y
otro. La linda criatura estaba flaca, ojerosa, manchado el rostro de
pecas rojizas; y el desarreglo y suciedad de su ropa indicaban pobreza,
malestar, infortunio\ldots{} Díjome que se había casado, por imposición
de su familia, con el desagradable mastín negro Aquilino de la Hinojosa.
Ya lo sabía yo. \emph{Oí contar de un náufrago la historia}. La
\emph{náufraga} era mi pobre y desdichada Obdulia.

\hypertarget{xxi}{%
\chapter{XXI}\label{xxi}}

Ávida de referir sus cuitas, la infeliz mozuela me contó que, a poco de
casarse, vio en su marido el más perverso animal de la Creación. Lo que
llamamos luna de miel fue para Obdulia completa desilusión del
matrimonio. Ella era delicada, sensible y de finísimo trato; él grosero,
brutal, insaciable en la comida y otros apetitos. Al mes de casada pensó
en divorciarse; habló con un abogado amigo suyo, y como este le dijera
que en las leyes españolas no tenemos divorcio, dio en la idea de
suicidarse, saltando de un brinco hacia \emph{las palmeras de Sión}. Le
faltó valor para el salto mortal: ni con fósforos, ni con braserillo,
supo determinarse\ldots{} Pensó acudir a mí; me buscó; dijéronle que yo
vivía en magnífico arreglo con una tendera de la Concepción Jerónima.
Acercose allá y le salió al encuentro una señora llamada Cabeza que
quiso descabezarla\ldots{} En tanto, Aquilino iba de mal en peor,
agravando sus defectos. No le bastaba el oficio de afinador para
sostener su casa y sus vicios. Dedicose a la compra, venta y alquiler de
pianos, y tales desatinos hizo y en tales enredos se metió, que fue a
caer en las mallas del Código penal.

«En mi casa---decía suspirando---no entraban más que procuradores y
alguaciles. Yo no vivía; el apetito y el sueño me abandonaron; consuelo
de mi angustia era el llanto, consuelo también un librito de poesías de
Selgas que por las noches me calmaba los nervios, y aquellos versos de
Espronceda: \emph{¿Por qué volvéis a la memoria mía\ldots?} Hace unos
meses vino a verme y a consolarme Celestina Tirado, que se metió a
beata\ldots, no sé si lo sabes\ldots, y anda en trajines de religión.
Díjome que en la iglesia hallaría mi remedio; que fuese a misa y a
confesar, y que rezara mis tercios de rosario con devoción. Mi antigua
señora la Marquesa de Navalcarazo me llamó para recomendarme el mismo
medicamento de Celestina: Religión, misas, novenas, y pronunciar a toda
hora el nombre de Jesús, \emph{que endulza el alma y la
boca}---\emph{más que con la miel y azúcar---con sólo sus cinco
letras}\ldots»

Cogidos de la mano íbamos paseando despacito bajo los soportales de la
Plaza Mayor. La doliente historia de mi amiga quedaba cortada en un
suceso que nos abría camino para reanudar nuestra vieja novela
interrumpida. Aquilino de la Hinojosa no estaba en Madrid. Dos semanas
antes de lo que se refiere, había ido a Villaviciosa de Odón a recoger
la menguada herencia de una tía suya que murió en aquel pueblo. Para
ciertas diligencias judiciales tuvo que trasladarse a Navalcarnero; al
regreso volcó la galera en sitio de peligro; rodando cayó el afinador en
una barranquera, donde le recogieron descalabrado y con una clavícula
rota. Personas caritativas le llevaron a Villaviciosa, y en casa de unos
parientes estaba en cura, que habría de ser larga. «Ayer---me dijo
ella---recibí su primera carta después del siniestro. Está dado a los
demonios. Me escribe poniendo en cada renglón una blasfemia. Le tienen
bizmado y entablillado, sin poder moverse. ¡Dichosa herencia, que no es
más que un melonar, cuatro almendros y una casuca sin techo! Me dice que
tiene cama para dos meses; manda tres duros por el ordinario y cuatro
recibos de treinta reales para cobrar alquileres de pianos. Me
recomienda la economía y que no vaya a verle, pues está bien cuidado por
su prima doña Melchora.»

Fáltame referirte, lector de mi alma, la última declaración de Obdulia,
que es del tenor siguiente: «Vivo en el 23 de esta Plaza, allí, en un
entresuelo, encima de la taberna que hace esquina a la calle del \emph{7
de Julio}. Con las pesetejas que me ha mandado ese, y diez duretes que
me dio mi señora la Navalcarazo, vivo pobre, y solita porque he
despedido a la muchacha que me servía\ldots» No necesito decir más para
que se comprenda que en aquel mismo día senté mis reales en el
modestísimo y lóbrego albergue de mi antigua y moderna conquista, la
señora de la Hinojosa. Los que no han vivido en un entresuelo de la
Plaza Mayor, con ventanas mezquinas, bajo la visera de los soportales,
no saben lo que es obscuridad en pleno día. Nunca pensé yo cobijar mi
persona en tal ratonera; pero la exaltada pasión y el donaire de mi
socia me convertían la tristeza en gozo y las tinieblas en luz.
Aderezaba Obdulia nuestras comiditas. Más de una vez, por evitarnos ir a
la compra y la molestia de encender lumbre, bajábamos a comer a la
taberna, donde nos servían platos de judías de \emph{batallón}, tajadas
de bacalao y otros condimentos de pobres. El tabernero era muy amable y
nos ponía la mesa en un aposento interno, donde rara vez veíamos
comensales.

Por cierto que una noche me encontré de manos a boca con Serafín de San
José, el esposo de mi antigua barragana, la eximia señora doña Cabeza.
Aquel soez vagabundo, muy mal vestido y con cara de hambre atrasada,
hablaba sigilosamente con un bigardo de mala catadura, entreverando las
tajadas de bacalao con tragos de tinto. De la mesa donde estaba vino a
saludarme, y me dijo que su mujer se había arreglado otra vez con el
zascandil de Alberique. ¡En qué distinguida sociedad estábamos! El
despacho grande de la taberna hervía de parroquianos lenguaraces.
Siempre que por allí pasábamos de refilón oíamos conceptos groseros,
iracundos, entre los cuales saltaba, como nota picaresca, una idea
política.

Ultimados mis quehaceres volví a casa, un poco tarde, en la noche del 18
de Julio, y marco esta fecha porque sobrevino de improviso un suceso
histórico. Hallé a Obdulia nerviosa y asustada: «¡Gracias a Dios que
llegas!---me dijo, saliendo a la escalera.---Entremos; vas a saber una
cosa tremenda. No te asustes; no va con nosotros. Siéntate\ldots{}
Recordarás que pedimos al tabernero para esta noche un pote gallego, que
a ti tanto te gusta. Queriendo yo aprender cómo hacen este guiso, bajé a
la cocina y estuve un rato con la \emph{señá} Sebastiana. Luego me fui
al mostrador, con el señor Tomás. De allí a la trastienda. Oí palabras
sueltas de los \emph{puntos} que bebían y charlaban\ldots{} Até mis
cabos\ldots{} Volví al mostrador; el señor Tomás y un hombre de mala
facha, que llaman \emph{el tío Martín}, secreteaban\ldots{} Pesqué
alguna frase que me abrió las entendederas\ldots{} En fin, chico, te
diré lo que he podido traslucir: Esta noche matarán a don Amadeo. ¿A qué
hora? Cuando los Reyes vuelvan de los Jardines del Retiro a Palacio.
¿Sitio? La calle del Arenal. No te rías. Verás cómo resulta cierto. Otra
cosa: el pote gallego se ha pegado, y en su lugar nos mandarán unas
chuletas de vaca y patatas fritas. Andan abajo esta noche muy
desconcertados. ¡Qué caras he visto en la trastienda! Para mí, son los
mismos que mataron a Prim.»

No di gran importancia al cuento de Obdulia; pero tampoco lo eché en
saco roto. Mientras cenábamos, comentando la conjura tabernaria, hice
propósito de dar un soplo al Gobierno civil para que este tomase las
precauciones propias del caso. Pero a nadie conocía yo en las
Delegaciones ni en las antesalas del Gobernador. En estas dudas acordeme
de mi pariente \emph{Sebo}, cuyas relaciones familiares con la primera
autoridad de la provincia, don Pedro Mata, me constaban de manera
positiva. Tranquilamente despachamos nuestras chuletas, por cierto medio
chamuscadas, medio crudas, y salimos a buscar en calles y jardines el
aire y la expansión nocturna con que templábamos el ardor de los días
caniculares. Después de hacer escala en la casa de Telesforo del
Portillo (Olivar, 4), bajamos al Prado; dimos unas vueltas por
Recoletos; descansamos en un aguaducho, y ya cerca de media noche
cogimos la calle de Alcalá, y en la Puerta del Sol dudamos si tomaríamos
la calle Mayor, que era nuestro derrotero, o la del Arenal. Éramos como
trasnochadores que no se retiran a su casa sin ver una piececita de
teatro. «Por sí o por no---dije a mi señora postiza---sigamos la
dirección que han de llevar los Reyes y veremos si sale sainete o
tragedia.»

Recorriendo despacio la calle del Arenal vimos en la esquina del
callejón de San Ginés a Serafín de San José con blusa larga. Advirtiendo
que se recataba de nosotros creí sorprender en él cierto aire de
filósofo pensativo. Al pasar por Bordadores dos hombres cruzaron a la
acera de enfrente. Obdulia me hizo notar que bajo las blusas de aquellos
tipos se marcaba el bulto de trabucos o retacos. Hacia la calle de las
Fuentes creí ver al señor Tomás, con chaqueta parda y boina. Ya nos
acercábamos a la calle de la Escalinata, cuando sentimos venir coches
que nos parecieron de Palacio. Retrocedimos. Era, en efecto, la
carretela descubierta en que volvían de los Jardines el Rey y la Reina,
con el General Burgos. Detrás venía otro carruaje\ldots{}

No tuvimos tiempo para mayores observaciones porque de súbito sonaron
disparos. Los fogonazos brillaban en un lado y otro de la calle.
Encabritados los caballos (luego supimos que eran yeguas), se paró el
coche. Púsose en pie don Amadeo. El General Burgos atendió a escudar a
la Reina con sus corpulentas anchuras\ldots{} Confusión, espanto\ldots{}
Los transeúntes se agolpaban curiosos o corrían atemorizados. Obdulia y
otras mujeres lanzaban al aire sus chillidos. Del coche que venía detrás
descendió el Gobernador don Pedro Mata enarbolando su bastón. Surgieron
polizontes como por magia. Nuevos disparos. La carretela de los Reyes
partió a escape hacia Palacio: una de las yeguas cojeaba. Entablose
rápida lucha entre policías y paisanos. Estos huyeron, en veloz corrida,
hacia las Descalzas y Santo Domingo\ldots{} Busqué a Obdulia, que en el
tumulto se apartó de mí. La encontré en la esquina de la calle de las
Fuentes. Volvimos al lugar trágico y vimos entre varios heridos a uno
yacente, rígido; parecía muerto. Obdulia reconoció al \emph{tío Martín}.
Allí estuvimos, atentos al ardoroso comentario del suceso, hasta que
trajeron la camilla para llevarse al que todos creían cadáver. Y
agregándonos a la comitiva de curiosos desocupados y chicuelos, fuimos
tras de la camilla hasta la Casa de Socorro de la Plaza Mayor. De allí
pasamos a nuestra casa, advirtiendo al entrar en ella que había en la
taberna estrecha custodia de policías.

A la mañana siguiente, atraído del febricitante interés que despierta un
lugar trágico, me fui a la calle del Arenal. Gran golpe de gente había
frente a una tienda de cristales situada entre la Costanilla de los
Ángeles y la Travesía de los Donados. Los curiosos impertinentes no se
hartaban de mirar y señalar las huellas de los proyectiles en el zócalo
y en el rótulo de la tienda. De improviso, los que formábamos \emph{el
respetable público} de la tragedia fracasada vimos llegar al propio don
Amadeo, acompañado de su amigo Dragonetti y de su ayudante Díaz Moreu.
Rodeado de la plebe novelera miró y remiró las señales de los balazos.
Muchos de los que allí fisgoneaban tenían a gala el señalar al Rey algún
desperfecto que Su Majestad no había visto.

De la tienda salió una señora joven que parecía la dueña, y
graciosamente invitó al Rey a que pasara, si quería descansar. Daba las
gracias don Amadeo, permaneciendo en la calle, cuando se destacó del
personal de la tienda una señora mayor, que ofreció al Rey un proyectil
que había penetrado en el local, incrustándose en la anaquelería.
Agradeció don Amadeo el obsequio y quiso gratificar a la señora, mas
esta no admitió el dinero. Despidiose el monarca sombrero en mano, con
su habitual cortesía, y a pie se volvió a Palacio, escoltado por un
pelotón de vagos y precedido de un destacamento de chiquillos.

Acerqueme yo a la señora mayor, que en la puerta de la tienda quedaba,
contemplando al pueblo soberano, y de manos a boca le dije: «He tardado
un rato en reconocerla, insigne \emph{Mariclío}, porque está usted hoy
un poco desfigurada, con mayor peso de ancianidad que el que tenía la
última vez que la vi. A su disposición me tiene para cuanto guste
mandarme.»

---A este ensayo de tragedia---me dijo, enseñándome un pie---he venido
con mis zapatos de orillo, como ves. No había motivo ni asunto para
mejor calzado. Los badulaques de anoche, movidos a un acto que no tenía
más objeto que producir miedo para que el Gobierno no saque tropas a
provincias, han procedido neciamente. El provecho de este regicidio sin
regicidio será para los partidarios del niño Alfonso. ¿Por ventura son
estos los que os aconsejan y dirigen?

Nada le respondí, pues mis observaciones no habían de llegar a la altura
de su autoridad. Ofrecime de nuevo a prestarle cuantos servicios me
encomendara, y con gusto la vi bien dispuesta en favor mío. Díjome que a
la sazón moraba en la portería de la Academia de la Historia, porque sus
cortos haberes no le permitían mejor acomodo. La \emph{capitis
diminutio} a que había llegado, en la desabrida etapa histórica del Rey
saboyano, deslucía su ancianidad gloriosa. «Lo que mayormente me
aflige---añadió, rompiendo conmigo la multitud para seguir juntos por la
calle del Arenal---es la flaqueza femenil de los partidos monárquicos y
la inconsistencia de los que vociferan en las filas avanzadas, indicio
seguro de la poca virilidad del pueblo hispano. Todo lo que aquí pasa es
cosa de ópera cómica, tirando a bufa. He pensado en darme de baja, como
dice tu amigo Ido del Sagrario, y transferir mis nobles funciones a mi
hermana Talía, que las desempeñará muy bien, encargando algunos
numeritos de polka y tango a mi hermana Euterpe\ldots{} El quita y pon
de Ministerios que sólo difieren en la medida y rumbo de sus tonterías;
la conspiración de las damas católicas, con su armamento de peinetas y
florecillas de lis, pertenecen al orden literario del entremés con
tonadilla y ovillejos. Habrás oído, entre tus amigos, planes de
levantamientos en plazas fuertes y ciudades populosas. No hagas caso,
hijo. ¡Batallones que se echan a la calle, guarniciones que se
pronuncian! ¡Sueños locos de paisanos ociosos, que gobiernan el mundo en
las mesas de un café o la redacción de periódicos bullangueros! Todos
esos que se levantan, lo que hacen es acostarse, y entre sábanas se ríen
de los conspiradores de alfeñique\ldots{} Hace pocos días, he visto a
los niños de las Peñuelas jugando al pronunciamiento. La demagogia misma
procede hoy con más simplicidad que barbarie. Los ideales exaltados son
ahora instintos movidos por la imbecilidad.»

Acompañé a la señora hasta la calle del León, y me volví a casa. A mi
consorte accidental referí mi encuentro con \emph{doña Mariana}, y traté
de explicarle la condición de esta y su doble calidad real y quimérica.
Pensé yo que Obdulia no me entendería, pero como en la naturaleza
cerebral de la bella joven prevalecían la ensoñación poética y el bello
mentir, admitió como verídico el cuento de \emph{Mariclío} y de sus
inauditas transformaciones. «¡Ay Tito de mi vida---me dijo
consternada---que felices seríamos si esa divina dama nos llevara por
esos mundos como duendes o muñequitos que pueden esconderse, si a mano
viene, dentro de una cajita de caramelos! Sabrás que en esta renegada
casa estamos sobre un volcán. Apenas saliste tú para la calle del
Arenal, entraron dos policías y me marearon con preguntas; que si yo,
que si tú\ldots{} Respondiles que no teníamos nada que ver con el
atentado; que nosotros somos vecinos, pero no cómplices del señor Tomás
y sus compinches. Antes te dije, querido Tito, que estábamos sobre un
volcán\ldots{} Son dos volcanes, dos. Porque si vuelve Aquilino mal
curado de sus mataduras no pararé hasta el suicidio\ldots, y que me
entierren en un cementerio bonito, con cipreses y adelfas. En caso de
que mi maridillo se quede por allá, será posible que nos prendan por el
aquel de regicidas, y nos separen quizás para siempre. Eso no, Tito mío:
vámonos, salvémonos.»

Fácilmente me comunicó Obdulia sus recelos, y por tranquilidad suya y
mía resolví una pronta mudanza. Recogida nuestra ropa, un colchón y
otras cosillas, y dejando en la casa los trastos menos necesarios, nos
fuimos a mi hospedaje de la calle del Amor de Dios. De sus graves
inquietudes descansó Obdulia con la grata compañía de Nicanora y del
dulce filósofo don José Ido. Este mostraba paternal solicitud por la
espiritual joven que llevé a su casa. Hablaron de literatura y teatros,
y Obdulia le recitó con lírica declamación, versos que embelesaron al
esmirriado señor\ldots{} Mi compañera no pisaba la calle por temor a un
encuentro desdichado. Echándoselas de médico, Ido la declaró anémica y
diagnosticó los baños de mar como infalible tratamiento. ¡Buenos
estábamos para viajecitos y expansiones estivales!

Pasaba yo los mejores ratos del día persiguiendo a \emph{doña Mariana},
o en su grata compañía cuando me deparaba Dios el encontrarla. Una
tarde, platicando en la portería de la Academia, me sorprendió, mejor
diré, me asombró gratamente con estas inesperadas razones: «Ocioso está
el gran Tito, y la ociosidad es el achaque peor que puede caerle a un
hombre de ingenio. De tu listeza y de tu travesura necesito yo estos
días, sin que me sea forzoso darte la condición, modo y sutileza física
que te di al traerte de Durango a Madrid. Tal como eres y en compañía de
esa moza chiquita y romanticuela, que es ahora tu mujer adventicia, irás
a donde yo te mande. Ya sabes que el Rey Amadeo sale hoy para una
excursión a diferentes ciudades del Norte. Tú irás también por allá. Mas
te destino a una sola plaza, Santander. Me consta que van también para
allá gentes peligrosas de uno y otro sexo. En fin, tú lo has de
ver\ldots{} Observa lo estrictamente verdadero; no me traigas acá
mentiras adornadas.» Sacó de entre sus ropas un taleguito, y me lo
mostró con estas dulces palabras: «Apurando mis recursos te doy billete
de ida y vuelta para ti y para tu chiquilla, y una suma prudente para el
gasto de tres semanas. Toma. No tardéis más de dos días en poneros en
camino. Buen ojo, actividad y criterio. Adiós.»

\hypertarget{xxii}{%
\chapter{XXII}\label{xxii}}

Ya me tenéis otra vez, lectores picarescos, oficiando de guindilla
histórico, sin conmutación de mi ser físico en entidad
peri-espiritual\ldots{} Lo que se alegró mi Obdulia cuando en casa le
mostré el saquito milagroso, no hay para qué decirlo. Veraneo, baños de
mar, costa cantábrica, ¡qué porvenir tan poético y delicioso! En dos
días arregló la romántica sus trapitos por el figurín más económico, y
nos largamos con viento cálido en busca del viento fresco. ¡Por qué modo
tan peregrino se habían realizado los deseos emigratorios de Obdulia y
su anhelo de ambiente marino, conforme a la docta indicación del
filósofo-médico Ido del Sagrario! En el estado de nuestro ánimo se nos
representó como un paraíso la ciudad Cantábrica, que en aquel tiempo
bien podría llamarse la \emph{ciudad harinera}, porque su hermoso puerto
se veía poblado de buques de vela cargando harina, o descargando los
ricos frutos coloniales. Obdulia, que nunca había visto el mar, se
embelesaba contemplando el grandioso muelle, el trajín comercial, los
barcos de arboladura gallarda; y cuando en nuestro primer paseo vagoroso
traspusimos el cerro de Miranda, la vista del Océano impetuoso colmó el
estupor de la pobre muchacha. ¡Aquello sí era poesía!\ldots{} ¡Aquello
era el camino de América, el camino para todo el \emph{más allá}
terrestre y acuático!

A los dos días de vagar por la ciudad y sus alrededores, probando
distintos alojamientos, nos instalamos definitivamente en una casita del
alto de Miranda, donde pagábamos dos pesetas por la habitación, y
comíamos por nuestra cuenta. Éramos dichosos en aquella vida libre y
modesta. Los dos íbamos a la compra, y Obdulia guisaba. Lo restante del
día lo empleábamos en largos y deleitosos paseos: ya nos extendíamos
hasta Cabo Mayor, y desde lo alto del faro contemplábamos el mar en toda
su majestad y bravura, o bien, después de recrearnos en las hermosuras
del Sardinero, íbamos a coger azucenas y clavellinas silvestres a la
península de la Cerda. También dirigíamos nuestros pasos tierra adentro,
revolviéndonos por toda la ciudad, entretenidos con la faena de las
harinas en el puerto, o viendo el arribo de las lanchas pescadoras.

A los seis días de esta descansada vida llegó el Rey, con séquito
militar y civil no muy lucido. Recibiéronle las autoridades y le
alojaron en la Aduana, edificio viejo donde estaban las oficinas del
Gobierno Civil y de la Administración de Hacienda. Antes o después de
don Amadeo (no puedo precisarlo), llegó de Santoña el batallón de línea
que debía custodiar a Su Majestad y hacerle los debidos honores. Como en
la ciudad no había cuartel, por ser plaza desguarnecida y en extremo
pacífica, la autoridad militar ordenó al alcalde que expidiera boletas
de alojamiento para albergar a la tropa. El Alcalde, señor Sañudo, era
convencido republicano, y sin faltar al respeto que al Jefe del Estado
debía, replicó que no estaba dispuesto a molestar al vecindario y que
acomodasen a los soldados en la forma militar más adecuada.

En esto ocurrió un suceso digno de la historia. Como la visita del Rey
fue tan precipitada, no hubo manera de prepararle decoroso alojamiento.
Elegido para este fin el local alto de la Aduana, habitación del
Gobernador civil, lo pintaron deprisa y corriendo para disimular su
fealdad y porquería, y esto se hizo la víspera de la llegada del Rey.
Pasó este una noche de perros en su incómodo albergue, apestado del
insufrible olor de la pintura, y al amanecer abrió los balcones,
buscando aire respirable. Ante este imprevisto contratiempo acudieron
los ediles a don Juan Pombo, el ricacho del pueblo, que ofreció para
morada real un lindo palacete del Sardinero, conocido por \emph{La casa
de Pepe Pombo.} Y allá se instaló el Rey, encantado de la belleza del
sitio y del relativo esplendor de su nueva residencia.

Al propio tiempo fue resuelto, del modo más simple, el conflicto del
alojamiento militar. En las suaves colinas verdes que rodean el
Sardinero y entre los espesos grupos de pinos, se emplazó un lindo
campamento con tiendas de lona. El vivir de los soldados día y noche en
aquel alegre vivaque, dio al hermoso paisaje un cierto encanto de
popular romería. Para embellecer más el cuadro fondeó en el abra del
Sardinero la fragata \emph{Vitoria}. Desde el ventanucho de nuestra
casita, Obdulia y yo, contemplando las tiendas, los pinares, la tropa,
los bañistas y la grandiosa nave, creíamos ver el más lindo nacimiento
que se pudiera imaginar.

En aquel amenísimo rincón de la Montaña hacía don Amadeo vida campestre,
desplegando libremente sus aficiones democráticas. A distintas horas se
le veía divagando en dirección de Cabo Menor o de La Magdalena,
acompañado de Díaz Moreu y Dragonetti. Por las tardes, cuando la música
tocaba en \emph{El Pañuelo} (plazoleta triangular entre la Casa de
Baños, las fondas y el palacete de Pombo), le veíamos en la turbamulta
de paseantes, ojeando a las señoritas guapas y charlando jovialmente con
sus amigos\ldots{} De la llaneza democrática del Rey oímos contar
innumerables casos. Alguien le había visto llegar de noche, solo, a su
vivienda y llamar a la puerta tirando de aldabón, como cualquier vecino
trasnochador\ldots{} Otros le sorprendieron en el interior de su palacio
inspeccionando las obras de decorado. Viendo a un obrero que clavaba una
guarda-malleta, subido en débil escalera, puso en esta el Rey su mano y
dijo: «Cuidado con caerse, amigo. Siga usted clavando; yo mantengo.»

Una mañana, paseando Obdulia y yo por la Segunda Playa, vimos una dama
guapa y melancólica, con traje veraniego enteramente blanco: «Ya tenemos
aquí a la de las patillas» dije a Obdulia, que cebó en ella sus miradas.
Un rato fuimos tras ella, acechándola con discreto espionaje. La vimos
llegar pausadamente hasta Los Molinucos; volvió luego por la playa en
baja marea, fijando sus ojos en la arena húmeda como si buscara en ella
alguna inscripción borrada por las aguas. Subió después hacia Las
Llamas; se sentó en un ribazo. Sin duda esperaba. ¡Qué triste es
esperar, esperar al que no llega, al que no acude puntual a la cita! La
espiábamos con tanta discreción que no podía sospechar nuestra
vigilancia\ldots{} Llegó el momento en que la belleza patilluda daba por
terminado su desesperante plantón. En su rostro pálido creíamos advertir
el despecho y la ira. Subió paso a paso hacia el pinar llamado de
Aparicio. De tiempo en tiempo volvía sus ojos hacia el paso de la
Primera Playa. Aquel mirar era el último residuo de esperanza. En la
carretera subió a un coche de los que llaman cestas, y partió cuesta
arriba en dirección de la ciudad\ldots{}

De once a doce, me cuidaba singularmente del baño de Obdulia. Ayudábala
yo a desnudarse y vestir el traje marino; con ella descendía por la
playa hasta dejarla en poder de Germán, el fornido bañero; y en el
límite del agua, mojándome los pies, la miraba entre las blandas olas,
remojándose con toda la fe de una bañista que busca la salud. A la
salida le ponía la capa, y a la caseta volvía con ella, donde quedaba
sola con su felpuda sábana y su ropa. Yo me paseaba viendo el ir y venir
de mujeres en remojo, y singularmente me fijaba, como los demás
curiosos, en una señora inglesa, esbelta, rubia y guapísima, que nadaba
como un pez. Al salir de las aguas, la recibía su marido capa en mano y,
como yo a Obdulia, la llevaba derechamente al secadero de la caseta.

Un amigo que en el entretenido vagar de la playa me salió, un
conocimiento de estos que se traban y se destraban en la sociedad
balnearia, entabló conmigo coloquio chismográfico, del cual refiero lo
estrictamente substancial: «¡Brava mujer es esta inglesa! ¡Vaya unas
hechuras, vaya una tez de rosa y nácar\ldots! ¿Ha visto usted qué
piernas? Para escultura no hay como las inglesas. Su marido es
corresponsal del \emph{Times}, el primer periódico de Londres. Celebra
conferencias políticas con el Rey, y el Rey las celebra de otro género
con la \emph{corresponsala}. ¿No lo sabía usted? Viven en una de estas
fondas, no sé si en \emph{Zaldívar} o en \emph{Barbotán}\ldots{} Dicen
que Amadeo y su nuevo amor se ven en una casa del Paseo del Alta.»

Camino de nuestra casa, dije a Obdulia: «Me parece que tendremos lío. En
el mar proceloso se baña una bellísima nadadora, de nacionalidad inglesa
y \emph{corresponsala} del \emph{Times}. A esta señora le hace cucamonas
nuestro amado Soberano, y digo tan sólo cucamonas por no dar mayor
gravedad a un caso que conozco por simple chismorreo público.» Debo
añadir ahora que, sin darnos cuenta de ello, Obdulia y yo nos sentíamos
posesores de no sé qué poder metafísico, con el cual penetrábamos en la
intimidad de los hechos y en la conciencia de las personas que en
Santander y su famoso balneario vivían. Hallábame yo dotado de una
facultad intuitiva, al modo de reflejo de la vida externa en mi retina
cerebral, facultad que a Obdulia se comunicaba, resultando que los dos
teníamos un vago conocimiento de cuanto sucedía.

Por esta pasmosa virtud anímica supimos, sin que nadie nos lo dijera,
que la dama patilluda moraba en el \emph{Hotel del Comercio}, el más
decentito de la ciudad (Muelle, número 1), antigua casa próxima al
edificio de la Aduana, donde el Rey habitó una noche y estuvo a punto de
perecer envenenado por la reciente pintura del local. Tuvimos asimismo
la visión de que Adela pasaba parte del día en su aposento solitario,
atormentada de hondas inquietudes. Escribía cartas con febril mano, y
las rasgaba en pedazos antes de concluirlas. Por no bajar al comedor se
hacía servir en su habitación. Como Calipso en su gruta, \emph{ne
pouvait se consoler} de la partida de Ulises.

Una mañana, en el Sardinero, presenció todo el público de bañistas y
curiosos una estupenda regata. La nadadora inglesa se alejó, con
gallardo deporte, como unas treinta brazas. Al volver hizo la plancha,
meciéndose graciosamente sobre las movibles ondas. Cuantos estábamos en
la playa admirábamos su belleza y arrojo. Apenas la hermosa oceánide
hizo pie para volver a tierra firme, se nos ofreció un espectáculo
emocionante. De la Caseta Real, colocada del lado de Piquío, salió Su
Majestad con dos amigos al recreo de su baño, que más bien era un alarde
de resistencia deportiva, pues si como Rey había quien le aventajara,
como nadador difícilmente se le encontrara rival. Le vimos alejarse
braceando, hizo la plancha, continuó aguas adentro; a una distancia
doble de la que había recorrido la bella nadadora inglesa, se volvieron
los amigos del Rey y este siguió, impávido, convoyado por una lanchita
que tripulaban los bañeros.

En la playa, la nutrida fila de espectadores aumentaba por momentos.
Corrían de boca en boca voces de admiración y entusiasmo: «Es un
pez\ldots{} Hace rumbo a la \emph{Vitoria}\ldots{} Que llega\ldots{} Que
no llega.» A veces le perdíamos de vista por interponerse la curva de
una onda; después reaparecía. Hubo momentos en que sólo pudieron verle
los espectadores que miraban con gemelos; por fin, estalló en el público
la exclamación: «¡Que llega! ¡Que llega!» A bordo de la fragata sonaron
las cornetas, anunciando la presencia del Soberano. Los que tenían
gemelos vieron a los oficiales que descendieron la escala para
recibirle. La nadadora inglesa, que había tenido tiempo de vestirse, era
la más regocijada entre el público, la que con más énfasis aplaudía y
encomiaba el arriesgado ejercicio del regio tritón, glorioso deudo de
Neptuno. Don Amadeo se quedó a bordo; para llevarle su ropa y
servidumbre vino la falúa de vapor de la fragata.

La misma tarde de este suceso vimos en el Sardinero a la dama blanca y
melancólica. Después de voltijear en las inmediaciones de la residencia
real, vino al Pañuelo, donde alguien la enteró de que don Amadeo
continuaba en la fragata. Supo también que a bordo había un poquito de
fiesta, merienda o refresco. La lancha de vapor iba y venía, llevando
convidados. Por sus propios ojos vio Adela que entraban en la falúa el
corresponsal del \emph{Times} y su bella señora. Momentos después de
este grave incidente la vimos en la playa, excitadísima, hablando con
Díaz Moreu. Su palabra era tan vehemente, su actitud tan resuelta y su
gesto tan vivo, que creímos que le arrancaba los cordones al ayudante
del Rey. Este empleaba toda su habilidad cortés en aplacar el enojo de
la dama, y sus razones discretas terminaban con una negativa rotunda:
Imposible llevarla a bordo. Volvió la embarcación a recoger más gente, y
se llevó a Díaz Moreu, al alcalde Sañudo y a dos o tres militares de la
guarnición. La hermosa Dido, abandonada contra el fuero de amistad y
amor, mostraba claramente su despecho y celosa furia cuando embarcó en
la jardinera de dos caballos para retirarse a su gruta del Hotel del
Comercio.

No sé decir si yo veía o si adivinaba; mas la certidumbre penetraba en
mi espíritu, y continúo mi cuento seguro de llevar delante de mi pluma
la luz de la verdad. Obdulia y yo veíamos lo distante; oíamos las voces
lejanas\ldots{} A consecuencia de lo anteriormente referido, la hermosa
y desdichada señora, que por su talento y su belleza mereció los favores
del Rey, se vio lanzada a extremos de pasión y venganza, si reprobables
en la estricta moral, dignos de indulgencia como desahogo casi legítimo
de un alma burlada. Iracunda y ciega pensó que su papel en aquel drama,
medio personal, medio histórico, era responder al secreto agravio con
agravio público y resonante. Pues se la despreciaba indignamente, pues
se la sustituía por una inglesa extravagante y zancuda, no se retiraría
de la escena sin escándalo. ¿Qué menos hacer podía que dar publicidad a
trece cartas escritas de puño y letra por el Rey de España, don Amadeo
I?

Aferrada locamente a esta resolución, castillo formidable de la flaqueza
femenina, hizo saber al Rey lo que proyectaba. Alarma y susto en la
pequeña Corte del Sardinero; mensajes, recaditos\ldots{} Pronto se vio
que la deidad irritada no cedía. Sonaron los primeros fragores del
escándalo: la tempestad estaba cerca\ldots{} Transcurrieron dos días; al
tercero presentose en el Hotel del Comercio y en la estancia de la dama
un caballero amigo del Rey, pidiéndole conferencia reservada. Sentose
Dido abandonada junto a la mesilla donde pasaba las horas escribiendo y
rasgando cartas, e invitando al caballero a sentarse frente a ella, le
preguntó el motivo de su visita.

«Comprenderá usted, Adela---dijo el caballero,---que el objeto de esta
entrevista no puede ser grato para mí. Confío en la discreción de usted,
en su talento, en su bondad. Es usted buena. Por tal la he tenido
siempre. Bien sabe el respeto y la consideración con que la tratamos
todos sus amigos. Vengo decidido\ldots, ¿no lo presume usted?\ldots, a
recoger las cartas de Su Majestad.» Desplegando toda la táctica femenil,
Adela contestó que las cartas podían ser documentos históricos y que en
este caso pertenecían a la Nación. No creyó el caballero que el asunto
era de los que pueden tratarse con sutilezas del ingenio, y sacando de
su cartera un sobre repleto de billetes de Banco, lo puso sobre la mesa
y dijo así:

«Las relaciones de Su Majestad con usted han terminado, Adela. Mi
opinión es que usted las ha roto, no él. Sea como fuere, Su Majestad no
consiente que usted quede desamparada. Tome usted esto\ldots{} Son cien
mil pesetas.»

Airada respondió la señora que quizás vendería los \emph{documentos
históricos} por una palinodia del Soberano, reconociendo su veleidad y
poniéndole remedio\ldots{} Por dinero no los daría nunca. Entablose una
breve y agria disputa. Dido enamorada se defendía fieramente contra el
abandono. El mensajero del Rey, hombre que iba derecho al bulto y no
gustaba de inútiles parloteos, sacó del bolsillo un revólver, y
poniéndolo de golpe sobre la mesa, soltó este ultimátum: «O me da usted
las cartas, o la mato a usted ahora mismo.»

Por distintos estados emotivos pasó rápidamente la dama. En el espacio
de unos segundos se mostró colérica, medrosa, soberbia, humilde\ldots{}
Con incierto paso llegose a un maletín donde guardaba sus alhajas. Sacó
las cartas, y con furioso ademán las arrojó sobre la mesa. El mensajero
tuvo serenidad para contarlas. Vaciando el sobre de los billetes y
metiéndolas en él, para guardarlas cuidadosamente en su bolsillo, se
retiró con fría reverencia. No hay noticias del tiempo que tardó Adela
en recoger la indemnización de guerra, última página de su historia de
amor.

\hypertarget{xxiii}{%
\chapter{XXIII}\label{xxiii}}

En medio de la placidez de la vida campestre y balnearia, no se
extinguían absolutamente las inquietudes de Obdulia. Una noche despertó
sobresaltada y pegando gritos. Había soñado que hallándose en la
frescura y recreo de su baño, vio venir de mar afuera un horrendo
tiburón, abierta la espantosa boca con triple fila de dientes. El
cetáceo no era otro que Aquilino de la Hinojosa, metido en aquel disfraz
para devorar a su cónyuge infiel. Entre las mandíbulas del monstruo
marino estaba ya cuando despertó de la pesadilla. El terror le duró
largo rato después de despierta, y sólo a fuerza de cariños pude
tranquilizarla, prometiéndole además el exterminio del afinador, en
cuanto le cogiese a tiro.

La partida del Rey, que embarcó para visitar otros puertos de la costa;
las enojosas lluvias, que anunciaban la declinación de la temporada, y
la merma fatal de los dineros de \emph{Mariclío}, nos dieron el toque de
marcha\ldots{} En el tren, camino de Madrid, la casualidad nos deparó la
compañía de aquel joven, no diré amigo, sino conocido, que en la playa
del Sardinero me dio noticias de la \emph{corresponsala} del
\emph{Times} y de sus amores con el Rey. Era el chismoso profesional, el
hombre de las anécdotas galantes, de las historias que son el fermento
de la ociosidad en casinos y cafés. Tomando pie de una frase nuestra
pegó la hebra de su croniquería escandalosa, y después de referir con
picantes pormenores el enredillo del Rey con la dama inglesa, nos colocó
el relato de otras aventuras amadeístas, acaecidas en el último
invierno.

La primera ocurrió en una casa (proximidades del Teatro Real) donde
vivía un personaje que, por su elevado cargo, despachaba muy a menudo
con el Rey. Esposa del tal personaje, cuyo nombre no quiso revelarnos el
cuentista, era una mujer bella y arrogante a quien Amadeo conoció en un
baile de Palacio. Ligerilla debía de ser la dama, pues sin gran
resistencia \emph{tomó varas} del Rey y concertaron una entrevista.
¿Dónde? En la propia casa de ella, aprovechando las largas ausencias que
al marido imponían sus obligaciones burocráticas\ldots{} Acudió el
Soberano a la cita, no sin prevenirse contra posibles contingencias
desagradables. Por si el marido se presentaba inopinadamente en su
domicilio, se dispuso que un confidente del Rey se situase en la puerta
de la calle, con hábiles instrucciones para cortarle el paso. La
maquinación era del género más picaresco\ldots{} Pues señor; llegó como
se temía el confiado o desconfiado caballero, y el emisario,
acometiéndole al bajar del coche, le dijo con bien fingida premura:
«Estoy aquí esperándole a usted para comunicarle, de parte de Su
Majestad, que en Palacio le espera para tratar con usted de un asunto
urgentísimo.» Refunfuñó el marido\ldots{} Antes de ir a Palacio subiría
un momento a su casa. Pero el confidente le atajó con frase apremiante,
angustiosa: «No, no; no hay que perder momento. Su Majestad espera
impaciente. El asunto es muy grave.» El engañado personaje se dirigió
velozmente a Palacio, donde preparado le tenían otro gracioso ardid. Al
encuentro le salía Dragonetti, obligándole a una larga antesala. Su
Majestad estaba conferenciando con Cialdini, embajador de Italia, en las
habitaciones de Su Majestad la Reina\ldots{} A la hora larga de este
bromazo recibía don Amadeo al personaje y con él trataba de un asunto
administrativo, que el cuentista no dijo, y en verdad no hacía falta
para redondear el cuento.

Oída y celebrada esta picante aventura, de cuya veracidad no respondo,
el chismoso, sin tomar respiro, continuó la serie: En otro lance
amoroso, los satélites del Rey tuvieron que simular un robo para
proteger la difícil salida del \emph{galantuomo}; en otra sacaban a la
señora por una puerta secreta, o bien descolgaban al caballero desde el
balcón al jardín, y en todas resultaba que el marido era tonto. Nos dijo
seguidamente que don Amadeo había traído de Italia una cuadrilla de
rufianes para organizar aventuras tan diabólicas. No daba yo gran
crédito a esta importación rufianesca. Añado por mi cuenta que los
referidos lances de seducción eran de corte italiano más que español, y
en ellos se advertía el cinismo malicioso de Bocaccio antes que las
artimañas sutiles de la picaresca de acá\ldots{} Lo que he recogido de
boca del chismoso tiene un hueco en estas páginas como documento vivo de
cierta opinión insana que se proponía desprestigiar al Rey Amadeo,
poniendo en circulación estas liviandades indecorosas y a veces
ridículas.

Llegamos a Madrid en perfecta salud. En la casa de huéspedes no había
otras novedades que un aumento molestísimo de estudiantes de Medicina, y
que el gran don José, en un ataque agudo de su depresión cerebral,
pasaba largas horas sumergido en hondas meditaciones sobre el misterio
de la Inmaculada Concepción. Sabedora de mi llegada, fue a verme Delfina
Gil, suponiendo que yo venía de Roma. Por carta de mi hermana Trigidia
tuvo noticia del revuelo que armó mi discurso, y de los telegramas del
Papa llamándome a \emph{la capital del Orbe Católico}. Seguí yo la
broma, y a sus preguntas acerca de la salud del Padre Santo, le dije que
estaba bueno, sin otro achaquillo que un corrimiento de muelas que le
obligaba a tomar continuamente buches de malvavisco. Le describí con
frase hiperbólica la Basílica de San Pedro, y la Capilla Sixtina, donde
oía yo misa todos los días frente a la pintura del Juicio Final. Añadí
que el Sumo Pontífice me había colmado de bendiciones y finezas, dándome
de añadidura una misión secreta para la Reina doña María Victoria, la
cual me recibiría en audiencia un día próximo. De esto no podía decir
una palabra más. \emph{Ítem}. Yo comía todos los días con mi amigo del
alma el Cardenal Fieramosca, de la \emph{Propaganda Fidæ}\ldots{} Los
buenos católicos estábamos de enhorabuena porque la prisión del Santo
Padre tocaba a su fin. El bárbaro Víctor Manuel, movido de
arrepentimiento y del acerbo dolor de su culpa, estaba dispuesto a
postrarse de hinojos ante el solio pontificio, cubierta de ceniza la
cabeza, besando sucesivamente los escalones, hasta poner sus labios en
la sandalia de Pío.

Por el efecto que en Delfina causaron estas gordísimas trolas, comprendí
que le faltaría tiempo para comunicarlas a los beaterios y sacristías
que frecuentaba\ldots{} Como mi Obdulia no se aliviara de su terror, le
ordené que no saliera de casa. Yo andaba en busca de la \emph{Madre
Mariana}, sin poder dar con ella. Los porteros de la Academia de la
Historia me dijeron que después de pasarse tres días y tres noches en la
biblioteca, la vieron salir una noche con don Marcelino. Presumieron que
habían ido a la Academia de la Lengua, calle de Valverde. Don Marcelino
había vuelto; \emph{doña Mariana} no. Ansioso de hablar con ella, la
busqué en ambas Academias y en la de Ciencias Morales y Políticas, en la
imprenta de la \emph{Gaceta} y en la Armería Real. Todo inútil.

Al volver a mi casa encontré en ella a Ramón Cala y a Felipe Ducazcal,
que me esperaban para que les acompañase a la guarida de don Francisco
Torquemada, prestamista y anticuario, con objeto de proponerle la venta
o alquiler de algunas prendas de uso mujeril, consideradas ya como
arqueológicas. De buen grado les acompañé a la calle de San Blas, y,
enterado yo del asunto, entablamos negociaciones con el adusto usurero,
gastando los tres enorme dosis de paciencia y saliva para persuadirle de
que le proponíamos un buen negocio. Tratábase de adquirir o alquilar
cierto número de peinetas de carey, altas y labradas, en forma de teja.
Usadas por nuestras abuelas, ya pertenecían al coleccionismo. Torquemada
las tenía preciosas y pedía por ellas un sentido. Se convino al fin en
que las cediera por dos días, depositando una cantidad como garantía de
puntual devolución.

Colaborando en la travesura que se traían mis amigos, nos procuramos
mantillas blancas y negras en diferentes casas de préstamos, y en lo
restante del día y mañana siguiente organizamos la graciosa mascarada
que había de desvirtuar y corromper la manifestación de las católicas
damas alfonsinas. No fue empresa difícil reunir y contratar dos docenas
de \emph{mozas del partido}, bonitas las unas, atarascadas las otras,
útiles todas para el efecto que nos proponíamos obtener. El pícaro
Ducazcal sacó, no sé cómo ni de dónde, ocho carretelas de lujo, algunas
blasonadas, con lucidos troncos de caballos.

La función resultó brillante, abigarrada, jocosa. Salieron aquella tarde
las alfonsinas aderezadas con sus mantillas y peinetas, creyendo
realizar de este modo una protesta muda contra la nacionalidad exótica
de nuestros Reyes. Ridículo, afectado y artero resultaba el españolismo
de nuestras clases altas. Las que desde el segundo tercio del siglo
habían renegado de todo lo castizo, arrojando al montón de las
prenderías las modas españolas, y vistiéndose, comiendo y hablando a la
francesa, salían ahora con la tecla de adoptar preseas sacadas del
Rastro indumentario. Bien hicieron los pícaros de la política en poner
frente a ellas el manchado espejo de un Rastro moral.

La pantomima de aquella tarde fue lucida, y digámoslo claro, vergonzosa.
A lo largo de la Castellana, la ilustre señora y reina doña María
Victoria pasó ante la muchedumbre carnavalesca arrostrando, con severo
continente, el desaire público con visos de injuria. Nunca la vi tan
revestida de alta nobleza y majestad. En su rostro y actitudes no se
conoció si había sabido distinguir las verdaderas de las apócrifas
damas. Mis amigos y yo nos entretuvimos en actuar como puntuales
cronistas de salones, digamos de sociedad, y fuimos enumerando el
mujerío manifestante, en sus dos estamentos constitutivos. La fatalidad
política había confundido lo más aristocrático con lo más villanesco. Y
sobre la bullanga femenil oíamos una estruendosa carcajada de la Moral
Pública.

Oficiemos de revisteros imparciales: allí estaban la Navalcarazo y la
Yébenes, señaladas por su furibundo catolicismo; la Campo Fresco, de
agudísimo ingenio; la Belvís de la Jara, la Ruy Díaz, ilustres
importadoras de toda elegancia francesa; la Villares de Tajo y la
Gamonal, flor y nata de la aristocracia burguesa; la Trastamara, la
Monteorgaz, la Villaverdeja y la Tordesillas, de remoto abolengo
histórico. Entre los caballeros vimos a Paquito Uclés, a Pepe Armada,
Jacinto del Pulgar, Guillermo de Arancis, Manolo Montiel y otros
\emph{que sería prolijo enumerar}\ldots{} De la otra banda deben ser
citadas con preferencia \emph{Paca la Alicantina, Marquesa del Cieno};
la Eloísa, muy conocida en todos los círculos\ldots{} viciosos; \emph{la
Clotildona}, la Rosa Huertas, \emph{Pepa la Sastra}, espléndida de fofas
carnes; la \emph{Napoleona}, la \emph{Condesa del Real Cuño}, \emph{la
Sílfide}, \emph{la Moño Triste} y otras tales, cuyos linajudos nombres
se escapan avergonzados de la pluma cuando queremos escribirlos.

Al volver yo de la Castellana con Roberto Robert y Mateo Nuevo,
encontramos a Pepe Ferreras, el periodista más discreto y agudo de
aquellos tiempos, hombre que sabía cual ninguno poner el dedo en la
parte doliente de todo suceso político y mostrar el grave daño que
padecíamos. «He visto la indigna comedia de esta tarde---nos dijo.---No
se concibe mayor oprobio de un país, ni mayor torpeza de las clases
altas, que nos han traído la intervención del fango social en la vida
política. En el estúpido atentado contra el Rey y en esta farándula
repugnante veo yo el principio del fin. La responsabilidad es de todos,
sin excluir las instituciones. Queríamos un Gobierno constitucional,
sensato, estable, y en dos años llevamos ya seis crisis si no recuerdo
mal. En política todo puede admitirse, menos el barullo, el caos y la
falta de orientación. ¿A dónde nos lleva este don Manuel? ¿Continuará la
marcha emprendida en su primer Ministerio, o nos precipitará de tumbo en
tumbo hacia lo desconocido? Digamos con don Salustiano: \emph{Dios salve
al Rey}. A la Reina no hay que salvarla, que bien alta está en el
concepto público. Si ella gobernara, tendríamos Saboyas para rato. Pero
no nos caerá esa breva. Lo peor del caso es que todo esto, y
principalmente lo que esta tarde hemos visto, resulta en provecho de los
Borbones\ldots{} Y yo pregunto a ustedes, señores republicanos tibios y
calientes, señores demagogos y socialistas de la Internacional, ¿harán
ustedes algo duro y hondo, algo que no sea esta labor de tontería y
aturdimiento? Si no cambian de tocata, la Restauración viene; vendrá
traída por todos, y principalmente por ustedes; la tendremos aquí
después que armemos el gran barullo\ldots, el gran barullo\ldots{} Y si
no, al tiempo, al tiempo\ldots{} el gran barullo.»

Repitiendo la frase última, rutinaria muletilla en él, se despidió de
nosotros, y yo seguí sopesando en mi mente las palabras proféticas del
sutil periodista y augur Pepe Ferreras.

\hypertarget{xxiv}{%
\chapter{XXIV}\label{xxiv}}

En lo restante de aquel Otoño, esta Nación sin ventura, como cuerpo en
que circula sangre viciada, se llenó de granos, manchas eruptivas y
forúnculos, síntomas de la enfermedad o \emph{gran barullo} pronosticado
por Ferreras. En todo el territorio del Norte, alta Cataluña,
Maestrazgo, provincias de Levante, apareció la sarna de las partidas
carlistas, y tras ellas vino el picor y desazón de las partidas
republicanas. No sabía el Gobierno a dónde acudir primero: aquí salía
del paso rascándose; allá se aplicaba emolientes; nos contentábamos con
ir viviendo, con ir tirando, mientras el mal estuviera limitado a la fea
y desapacible afección dermatológica\ldots{} Continuaban
infructuosamente mis diligencias para encontrar a la \emph{Madre
Mariana}. Si por una parte me dolía mi orfandad, por otra tuve algunas
satisfacciones de carácter doméstico. La intranquilidad en que Obdulia y
yo vivíamos se calmó con las noticias que de Villaviciosa trajeron el
ordinario y otras ordinarias personas. Lejos de mejorar, Aquilino iba de
mal en peor, por la falsa soldadura de la clavícula, y aún tenía
camastro para otros dos meses o más. Eso íbamos ganando.

Con los dinerillos que dio a mi mujercita la Marquesa de Navalcarazo,
por ciertas labores de aguja, y algo que yo ganaba escribiendo en
\emph{El Diario del Pueblo}, fundado por mi amigo Valero de Tornos,
pagábamos nuestro pupilaje, y aún nos restaba para menudencias y
honestos placeres. Debo decir, entre paréntesis, que en mi Obdulia se
armonizaba el romanticismo con las cualidades del perfecto economista.
Gracias a ella podíamos regalarnos diariamente en \emph{La Perla}, yo
con mi café, ella con su vasito de leche merengada.

Los billetes del periódico nos permitían el goce del teatro: en el Circo
de \emph{Paúl} nos entreteníamos oyendo a la Williams, actriz bonita y
salada que con el gracioso Rosell representaba el \emph{Mambrú}, pieza
de circunstancias llena de picardía. En el \emph{Teatro Circo} vimos dos
o tres veces el famoso zarzuelón \emph{Barba Azul}; en \emph{Capellanes}
nos descuajábamos de risa con la desvergonzada revista \emph{Los
prófugos de Ultramar}, sátira del escándalo de los Dos Millones que,
según la gente maliciosa, afanaron Sagasta y el pollo antequerano.

Pero lo que más nos encantaba y divertía era el arte maravilloso de la
célebre prestidigitadora Benita Anguinet, en \emph{Variedades}.
Titulábase la función \emph{Los milagros de la brujería}, y como yo
había sido un poco brujo hallaba singular deleite en aquel espectáculo
de escamoteos, sorpresas, juego de luz y tinieblas, que confundían la
mentira con la realidad. Era la Anguinet una señora simpática, gorda sin
menoscabo de su agilidad: encontraba yo en ella un parecido notable con
Pepita Izco, heroína de mi breve idilio místico y sensual de Durango.
Por esta razón eran más calurosos mis aplausos a la mágica de opulentas
carnes y sortilegios diabólicos. Una noche, estando Obdulia y yo en
segunda fila, vi en la primera a mi pasado amor María de la Cabeza
Ventosa de San José. Estaba con Alberique. A la salida nos miraron con
desdén olímpico, como diciendo \emph{adiós pobreza}. Les pagamos en peor
moneda, riéndonos descaradamente de su inflado empaque burgués.

Entrado ya Diciembre, el buen pueblo republicano de Madrid agregó al
interés de los teatros un motincillo callejero, nuevo síntoma de la
grave dolencia hispana. Hallábase una noche deliberando la \emph{Junta
Suprema del Consejo de la Federación Española}, cuando sonaron tiros en
la Puerta del Sol. ¿Qué ocurría? Que los Comités de los distritos habían
acordado, por sí y ante sí, lanzarse a la calle. Corriose la trifulca a
la Plaza de Antón Martín, tradicional baluarte republicano, y allí fue
sofocada por las tropas que llevó el General Pavía. Entre los
revolucionarios figuraban el famoso Espiga, el comandante Decref y
Carlos Caro, Cerrudo y otros paisanos. Hubo bastantes heridos y un solo
muerto, el lacayo del coche de Ruiz Zorrilla, víctima inocente del celo
de un diputado, señor Boceta, que se empeñó en recorrer el \emph{campo
de batalla} en el propio carruaje oficial del Presidente del Consejo.

Los treinta y cinco prisioneros de aquella descabellada intentona fueron
puestos en libertad a la mañana siguiente\ldots{} A mi parecer,
produjeron aquel fugaz movimiento \emph{Las Hojas Revolucionarias} que,
a falta del periódico \emph{Tribunal del Pueblo}, publicaban mis amigos
de la calle de la Montera. Entre aquellas \emph{Hojas} obtuvo enorme
circulación la titulada \emph{El Rey se va}, escrita por la
propagandista republicana Modesta Periú. No era ella la única hembra que
valerosamente luchaba por la Causa, pues otra, llamada Guillermina
Rojas, anduvo a tiros con las tropas de Pavía en la plaza de Antón
Martín.

A los pocos días de esta zaragata, los buenos y sencillos
revolucionarios se las prometían muy felices. Hallándome yo una noche en
la redacción de \emph{El Diario del Pueblo} escribiendo mi \emph{Crónica
del día}, vino a darnos plática un amigo, jovenzuelo y candoroso, el más
activo satélite de don Juan Contreras y del \emph{Consejo Federal}, que
forjaba los rayos de la revolución. «Ya la tenemos armada, querido
Tito---me dijo con sigiloso misterio.---Ahora va de veras. Será cuestión
de días el triunfo de la República Federal. Sevilla, Barcelona, Cádiz,
Cartagena, están a punto de pronunciarse. La \emph{Junta Suprema} y los
prohombres han discutido largos días, triunfando al cabo la idea del
levantamiento general. Esto que te digo lo sé por el propio García
López\ldots{}

«Puedes estar seguro, como si lo hubieras visto, de que anoche salió
para Andalucía Nicolás Estévanez. ¿Crees que va de paseo o a echar
discursos? No, chico. Lleva la sagrada misión de cortar todos los
puentes de Despeñaperros, de levantar partidas, sublevar las poblaciones
de Linares, Andújar, Bailén, La Carolina, cerrando al Gobierno toda
comunicación con las plazas de Andalucía. Tú conoces a Estévanez;
comprenderás lo que puede esperarse de su capacidad y audacia. Nicolás
es el águila de las guerrillas. No te digo más\ldots{} Dentro de algunos
días podremos decir, no El Rey se va, como nuestra brava heroína la
Modesta Periú, sino \emph{El Rey se ha ido}. Día de júbilo tendremos.
¡Con qué gusto veré partir a don Amadeo, al Dragonetti y a los rufianes
que ha traído de Italia para sus trapicheos amorosos! Lo sentiré tan
sólo por la Reina, francamente lo digo. Esta doña María Victoria es tan
buena y simpática que no parece Reina, sino una señora cualquiera. Yo me
quito el sombrero al verla pasar, y le perdono el ser italiana. Ya sabes
que cría a sus hijos. Me consta que este verano, paseando por las
inmediaciones del Escorial, encontró un niño abandonado que chillaba
pidiendo teta. Pues lo recogió y le dio de mamar, no con biberón, Tito,
sino a sus propios pechos. Tú que sabes tanto de Historia, me dirás si
has leído algún pasaje de reinas o emperatrices que hayan hecho
esto\ldots»

Tomé nota mental de los cuentos que me trajo aquel majadero inocente, y
seguí observando los acontecimientos que marcaban la fiebre y el
creciente malestar de la Madre España. Entre domésticos goces y fáciles
trabajos transcurrieron los días de Diciembre, hasta la placentera
semana de Navidad y Año Nuevo, que fue para nosotros alegre y descansada
por lo que voy a referir. Se hospedó en nuestra casa por pocos días un
rico labrador toledano, residente en Bargas, que nos invitó a pasar las
fiestas en su campestre vivienda, holgona y bien abastada de cuanto ha
menester la vida. Aceptamos con gratitud, y allá nos fuimos con él en un
galerín que salía de la Cava Baja. En el viaje y en el pueblo todo nos
pareció delicioso: el campo totalmente desnudo de árboles, nos
encantaba; la morada de nuestro amigo y anfitrión se nos antojó palacio
principesco; cuanto veíamos era reflejo del gozo de nuestras almas.

En don Casiano vimos el más cumplido, el más gallardo y obsequioso
hidalgo campesino; en su mujer, doña Dulce, la más bella, la más airosa
y afable dama labradora de estos reinos; en sus cinco niños, cinco
ángeles que reproducían la hermosura y simpatía de sus padres. La casa,
enorme y toda de planta baja, era el ideal de la humana vivienda:
anchurosas estancias, patios y corrales poblados de alimaña volátil y de
toda cuatropea cerdosa, ovejuna y caballar. Completo la figura del gran
don Casiano diciendo que militaba en el republicanismo federal, y que
tanto en él como en su linda consorte reconocimos las ideas más amplias
y generosas. Estábamos, pues, Obdulia y yo en el Paraíso terrenal, y
nuestra única pena era que antes de Reyes tendríamos que salir de él.

No hay que hablar de la opulencia de las comidas, del diario consumo de
pollos, palomos, conejos y cabritos. Lo que digo: aquello era más que el
Paraíso, era Jauja. Tenían los niños, en una de las principales
habitaciones, un magnífico Nacimiento con la mar de figuras, montañas de
corcho, nubes de algodón, sin fin de pastores, Reyes Magos, y un
escuadrón de Húsares. Obdulia, que era maestra en artes infantiles, les
completó la decoración con ramaje de carrascas, un lago cristalino, en
que patinaban elefantes y camellos, y un ferrocarril que comunicaba el
Cielo con la Tierra. La Nochebuena, iluminado el altarejo con
innumerables candelas, brillaba como ascua de oro. Niños de la vecindad
agregados a los de casa, nos regalaron con el concierto angélico de
panderetas, zambombas, rabeles, cánticos y alilíes de entusiasmo.

A la mañana siguiente, los ciegos, que recorrían el pueblo cantando
villancicos, vinieron a la casa, donde se les aseguraba copiosa limosna.
Eran mendigos astutos y oportunistas que variaban el sentido de sus
coplas, acomodándolas a las ideas de las personas cuyo aguinaldo
requerían. Y como el buen Casiano gozaba en toda la comarca fama de
republicano ardiente, los ciegos cantaban de este modo el natalicio del
Hijo de Dios: \emph{Camina la Virgen pura}---\emph{con San José
liberal}---\emph{para el Santo Nacimiento}.---\emph{República Federal}.
Venía luego el estribillo, que era el \emph{Me gustan todas}, con música
de \emph{El joven Telémaco}.

Otras coplas copio que nos hicieron mucha gracia: \emph{En la mitad del
camino}---\emph{iba San José cansado}.---\emph{Fue a llamar a una
posada}---\emph{y le salió un moderado}.---\emph{A otra posada
llamó},---\emph{ya fatigado de andar},---\emph{y le dijo el
posadero}:---\emph{entra, Pepe federal}. Por aguinaldo recibieron, con
la calderilla, un pan y un chorizo por barba. En la calle les encontré
luego, cantando también en forma libre para halagar al pueblo cuyas
ideas liberales conocían: \emph{Vinieron los pastorcitos}---\emph{a
besarle pies y manos};---\emph{Jesucristo muy contento}---\emph{porque
eran republicanos}. Me contaron que en la casa del párroco, tachado de
carcunda, cantaban así: \emph{Viva Jesús Nazareno},---\emph{juez de
nuestra Religión}.---\emph{Viva Jesús Nazareno}---\emph{y don Carlos de
Borbón}. Frente al cura, como en todas partes, terminaban con el
estribillo: \emph{Me gustan todas,---me gustan todas,---me gustan
todas---en general}\ldots{}

Con la llegada de los Reyes Magos, día triste para los escolares, nos
despedimos de nuestros espléndidos anfitriones. Trance amarguísimo era
dejar las ricas ollas, y el trato exquisito de doña Dulce, su digno
esposo y agraciada prole. Pero no había más remedio. Proponiéndome yo no
volver a Madrid sin pasar unos días en Toledo, para que Obdulia pudiese
dar un vistazo a la Catedral y demás monumentos, el propio don Casiano
nos llevó en un cochecillo a la Imperial Ciudad, instalándonos en la
Posada de la Sangre, donde nos pagó una semana de hospedaje. Hombre tan
bueno y dadivoso despertaba en mí tal admiración y gratitud, que hube de
considerarlo como un enviado de Dios.

El tiempo húmedo y ventoso no nos estorbó para recorrer y registrar las
maravillas toledanas, desde la inmensa Catedral, relicario de todas las
artes, hasta los últimos rincones arqueológicos, como el Cristo de la
Luz y el Cristo de la Vega. Rendidos de nuestras caminatas por las
empinadas y torcidas calles, nos acogíamos a nuestra Posada, al amparo
de la sombra del amigo Cervantes. Una noche, cenando en anchurosa cuadra
junto a la cocina, vi a la \emph{Madre Mariana} que hacía por la vida en
una larga mesa, poblada de arrieros y caminantes. Dos mujeres estaban a
su lado, y todos los comensales departían alegremente. Con respeto
supersticioso me acerqué a la Señora y le besé la mano. Ordenó ella que
Obdulia y yo nos agregáramos a su compañía, y así lo hicimos gozosos.
«Celebro encontrarte, querido Tito---me dijo.---Aquí me tienes
descansando en esta ciudad que es uno de mis solares predilectos. Me
distraigo remembrando cosas de tiempos muy lejanos. Es dulce y
confortante hacer revivir los Concilios de Toledo, las cuitas del Rey
Sabio, el Rito Mozárabe y charlar con los cardenales Mendoza, Cisneros,
Cilíceo, Carranza, y con mis buenos amigos Juan Guas y el Greco.»

Oyendo a la Señora creí encontrarme en los senos vaporosos de un mundo
quimérico. Las dos mujeres que acompañaban a la divina Clío atrajeron
poderosamente mi atención. La una, bella y altiva en su madurez, era la
mismísima Viuda de Padilla; la otra, joven y bonita, Santa
Leocadia\ldots{} Entre los hombres, todos de vigorosa complexión goda o
castellana, de rostros enjutos y tallas procerosas, vi al Rey Wamba, a
San Ildefonso, a Jiménez de Rada y Jiménez de Cisneros, a Illán de
Vargas, al Pastor de las Navas, y a otros, extranjeros españolizados,
que eran sin duda Copín de Holanda, los Borgoñas y Theotocópuli. También
creí reconocer al poeta Garcilaso y al comunero Padilla.

\hypertarget{xxv}{%
\chapter{XXV}\label{xxv}}

Cenamos diferentes manjares castizos; se obscureció la estancia, y al
volver en tropel a nuestros dormitorios, \emph{Mariana} me estrechó la
mano diciéndome: «Descansa un poco, que en el primer tren de mañana nos
iremos a Madrid. No sé si sabrás que está a punto de estallar un huracán
político por susceptibilidades y resquemores de los caballeros de
Artillería. No te digo más por esta noche\ldots»

En efecto, reunidos en el tren, a temprana hora, \emph{Mariclío}
prosiguió de esta manera sus graves informes: «El ventarrón nos ha
venido por el nombramiento de don Baltasar Hidalgo para el mando de una
división en el Ejército del Norte o de Cataluña\ldots{} no estoy bien
segura: lo mismo da. Recordarás la parte que se atribuye a Hidalgo en
los trágicos acaecimientos del cuartel de San Gil (1866). Fuera o no
culpable el entonces capitán de Artillería, sus compañeros le tomaron
entre ojos. Apartado del Cuerpo, Hidalgo ha prestado servicios en Cuba;
ha merecido y obtenido ascensos: hoy es Mariscal de Campo, sin que sus
compañeros de Arma hayan protestado de verle en tan alta jerarquía. El
disgusto de ahora se funda en que los artilleros no quieren ser mandados
por don Baltasar. Distante de Madrid he formado el juicio de que esto es
un aparato político para derribar al Gobierno y poner en graves
apreturas al pobre Amadeo. Sé que los llamados Constitucionales andan en
este enredo y que los oficiales de Artillería se reúnen nocturnamente en
casa de Ulloa. Pronto se sabrá la verdad. Hoy se abren las Cortes, allí
parirán estos montes y veremos sí sale ratoncillo inocente o dragón
infernal.»

Mientras hablaba la Señora examiné a las dos mujeres que iban en su
compañía. Ya no vi en ellas las poéticas facciones de la viuda de
Padilla y Santa Leocadia, sino, antes bien, vulgares rostros de dos
criadas, que al propio tiempo eran marisabidillas capaces de escribir al
dictado sendos tomos de Historia. Con una de ellas charlaba Obdulia,
refiriéndole sus impresiones de Toledo, y la otra me dio noticias del
nuevo incendio de guerra civil en el Norte y Cataluña. Las facciones de
Guipúzcoa, mandadas por Lizárraga, pisoteaban el \emph{Convenio de
Amorevieta}; Durango ardía en pasiones belicosas; Pepita Izco, olvidada
de mí, bordaba banderas para los batallones de la Fe, y mi amigo
Choribiqueta, dando de mano a su atavismo, presentía ya que podían caber
dos epopeyas dentro del espacio de un solo siglo. Horizontes teñidos de
sangre cerraban la vista por el Norte y parte de Levante. La pobre
España, arrullada en los brazos de la Fatalidad, aguardaba su sentencia
de muerte o vida con expectación pavorosa.

Al llegar a Madrid, \emph{doña Mariana} concertó conmigo lugar y ocasión
para comunicarnos; podía yo prestarle ayuda en la grave crisis que el
Destino elaboraba en su profundo taller histórico. Conforme a estas
advertencias, una mañana, entrado ya Febrero, me llamó a la casa del
reverendo sacerdote don Hilario Peña, a quien hallé trabajando en su
biblioteca, algo aliviado de la gota, metido en el laborioso afán de
terminar su magna obra del \emph{Clero Mozárabe}. Frente a él, en la
misma mesa atestada de librotes y papeles, escribía rápidamente la
\emph{Madre Mariana} en largas hojas de papel pergaminoso. Apenas me
acerqué a ellos para saludarles, vi entrar a Graziella, trayendo
servicio de café con leche y tostadas para los dos, mejor dicho, para
los tres, pues me invitaron a participar de su desayuno. Entraba y salía
la ninfa, diligente y cuidadosa, como ama de llaves sobre quien pesa el
gobierno de una casa. No hablaba más que lo preciso. Pasado un rato,
cuando el cura, la \emph{Madre} y yo hablábamos de los asuntos públicos,
reapareció con bayetas calientes para defender del frío las piernas y
pies de su amado señor.

«Hemos sabido---me dijo la \emph{Madre},---que el Rey de Italia ha
escrito a don Amadeo ordenándole que a todo trance se sostenga en el
trono, para lo cual es indispensable que se ponga al lado de la
oficialidad de Artillería, y que no consienta la disolución de un Cuerpo
tan noble y fuerte. Tenemos, pues, que Amadeo se coloca frente a su
Gobierno. Si prevalece el criterio del Rey, veremos a Ruiz Zorrilla y a
sus radicales hechos polvo. Volverá el Duque, volverán los unionistas
con los resellados del progreso. ¿Qué ocurrirá después?\ldots{} Ven acá,
Graziella: tú, que eres el numen de la nueva Italia, traído a nuestra
tierra como un soplo vivificador, dinos lo que te inspiran tus hermanas
las ninfas del Arno y Tíber.»

La vivaracha Graziella, que en aquel momento acababa de poner bajo los
pies de don Hilario una estufilla con brasas de carbón de encina, apoyó
sus codos en la mesa, y en el tono jovial y picaresco que tan bien se
armonizaba con su liviandad, nos dijo: «Víctor Manuel teme a los
Carbonarios, teme a los sectarios de Mazzini y a los venecianos que han
heredado las doctrinas de Manín. No quiere que se pase más allá de la
Monarquía democrática. Le asusta la República; cree que si su hijo
flaquea en España y se deja arrollar por el radicalismo, tengamos aquí
un ensayo de Gobierno popular con gorro frigio. La dichosa monterita es
para él como para mí la mala sombra, la getattura. Le dice a su hijito
que se arrime a los cañones. Sin cañones no se puede vivir. Lo mismo
pienso yo, que también soy de artillería. Como venga el gorro colorado,
el Rey \emph{galantuomo} ve perdido el trono de Portugal, donde tiene a
su hija María Pía, perdido el trono de España, en peligro también el
suyo, aunque asentado en la popularidad.»

---Si es verdad lo que nos cuenta esta loca---dijo don
Hilario,---tenemos resuelta la cuestión. El Rey se va con los caballeros
de Artillería; Zorrilla y Córdoba se meten en sus casas; vuelve el
Duque\ldots{} Resulta que aquí siempre estamos lo mismo. Entran y salen
los eternos perros sin tomarse el trabajo de cambiar sus collares.

---Lo que yo veo, mi buen don Hilario---dijo \emph{Mariana},---es que
aquí andan sueltas todas las pasiones menos la del patriotismo, única
pasión que da salud y vida a los pueblos enfermos. Ya sabemos quién es
el Ginés de Pasamonte que mueve los hilos de este retablo. Al pobre
Amadeo le ponen en un dilema de mil demonios: de una parte su juramento
de Rey constitucional; de otra la conservación de un trono que unos y
otros han convertido en mueble de guardarropía. Aquí despuntan
acontecimientos dignos de mí. Graziella, sácame del arca grande mis
borceguíes de tacones de plata\ldots{}

En la segunda visita que días después les hice, me recibió Graziella
sola, luctuosa y suspirante. Don Hilario estaba en cama, con ataque
agudísimo. \emph{Doña Mariana}, que había salido a sus menesteres y a
visitar a sus hermanas, no tardaría en volver. Decidime a esperarla para
comentar con ella el suceso corriente. Las Cortes habían discutido la
disolución del Cuerpo de Artillería, aprobando la conducta del Gobierno
por ciento noventa y un votos.

\emph{«Gettatura, gettatura}---exclamó la ninfa, llevándose las manos a
la cabeza.---¡Los ciento noventa y uno que le trajeron, ahora le
despiden!» Desapareció la hechicera voluble y yo me quedé solo en la
biblioteca, sin otra distracción que leer los tejuelos de los libros y
curiosear en los rimeros de papeles. Llegó \emph{Mariclío}; hablamos un
rato; volvió a salir presurosa. No sabré dar medida del lapso de tiempo
que permanecí solito en la silenciosa estancia. Anocheció; me adormecí
en la holgada blandura de un sillón. Conservo la vaga idea de haber
visto a Graziella entrar con una triste lamparilla de catacumbas. La
tenue claridad nocturna se fue trocando en luz de claro día, y cuando mi
cerebro se despejó de las nieblas del sueño, advertí con espanto que no
estaba en la biblioteca del docto don Hilario, sino en la quimérica
gruta de aquella casa del número 16, tragada por la tierra en Maravillas
o Monteleón. Entró la diablesa itálica desgreñada y en paños menores a
traerme café con leche; y poco después llegó \emph{doña Mariana}, de
cuyos labios, para mí divinos, oí la grave relación que a la letra
copio:

«El nudo se ha roto ya, y a estas horas el arduo conflicto artillero ha
pasado al montón de los hechos consumados. Las consecuencias serán por
algunos bien vistas, por otros lloradas\ldots{} Los jefes y oficiales,
doloridos por el agravio que a tan noble Cuerpo se infería, presentaron,
como sabes, solicitudes de cuartel, retiro o licencia absoluta según la
situación de cada uno. Como era natural, el Gobierno las admitió.
Paralelamente a esta moral de los ofendidos, los Generales palatinos
Gándara, Rosell y Burgos, en connivencia y contacto secreto con Serrano
Bedoya, el Duque de la Torre y todo el patriciado constitucional,
preparaban un acto de audacia política que bien podría llamarse
\emph{golpe de Estado}. Del Rey te diré que patrocinaba el movimiento
conforme a las ideas, planes y temores de su señor padre. La Casa de
Saboya se asusta del radicalismo y pretende afianzar en las dos
penínsulas la Monarquía democrática.»

---Ya lo sabemos, \emph{Madre}---dije yo.---El numen italiano no quiere
cuentas con la República. Víctor Manuel cree que está lejos aún la
emancipación de los pueblos latinos.

---Así es, hijo mío---prosiguió \emph{Mariana}.---La conjura para sacar
triunfante al Cuerpo de Artillería no vacilaba en rebasar los linderos
de la prudencia. No bastaría derribar al Gobierno radical; era forzoso
barrer el Parlamento, en cuyo seno convulso \emph{ciento noventa y un}
votos aprobaron la reconstitución del Arma de Artillería, elevando a los
sargentos a la categoría de oficiales y substituyendo los jefes con
individuos técnicos de otros Cuerpos. Para dar eficacia positiva al
pensamiento de los conjurados se acordó el siguiente plan: Enganchadas
las baterías en el cuartel de San Gil y en el del Retiro, con su
oficialidad y jefes naturales a la cabeza, saldrían a la calle con la
marcialidad que es de rigor así en las paradas como en los
pronunciamientos. Los de San Gil debían detenerse en la puerta del
Príncipe, donde se les incorporaría el Rey con el escuadrón de su
Escolta. Dado este paso, ¿qué faltaba ya? Seguir adelante, disolver las
Cortes y crear la dictadura interina, de donde saldría un nuevo
artificio constitucional, impuesto por las circunstancias\ldots{}
Preparado estaba ya todo, cuando llegó de Palacio la contraorden. No
había nada de lo dicho. A desenganchar. Quedaron los soldados en su
ordinaria vida de cuartel y los jefes y oficiales se acogieron al
descanso de sus casas.

---Ya me figuro el reverso de la escena, señora \emph{Madre}; mejor será
decir que lo adivino. Con el fuerte apoyo que le daba la confianza de
las Cortes, Ruiz Zorrilla llevó a la sanción del Rey el Decreto
reorganizando el Cuerpo de Artillería, y don Amadeo\ldots{} fue
débil\ldots{}

---Débil no, querido Tito. Fue consecuente con los compromisos que le
impuso su dignidad al venir a España. Reflexionó; hizo exploración de su
conciencia; puso fin con solemne arranque a sus veleidades y ligerezas.
Recordó su juramento ante las Cortes. Sus ojos vieron en letras de fuego
las palabras memorables con que expresó su propósito de \emph{no
imponerse a la soberanía de la Nación}, y firmó.

---Y ya tenemos a los sargentos en los puestos de los oficiales. Me da
en la nariz que algunos de los agraviados ofrecerán sus servicios a
Carlos VII.

---Así será, hijo mío. La Nación está en presencia de graves turbaciones
y luchas sangrientas. Para salir viva de ellas necesita sacar de su ser
el poder anímico que hoy parece adormecido. Fracasada la conjura de los
constitucionales, la rabia del pataleo les inspira resoluciones
sumamente cómicas. Entérate de esto: la Duquesa de la Torre ha dimitido
su cargo de Camarera Mayor de la Reina, y el Duque renuncia a todos sus
empleos, títulos y condecoraciones. La figura de Amadeo se ha crecido a
mis ojos. Presumo que en su mente germina y florece la idea de la
abdicación. ¿Estamos frente a un acontecimiento digno de mí?

Sorprendido quedé viendo el arrogante ademán con que \emph{Mariana} se
levantó de su asiento. La sorpresa fue pasmo y admiración cuando la vi
transfigurada de vieja caduca en matrona gallarda, de rostro helénico y
figura escultórica. Temblé de emoción al oír el vibrante sonido de su
voz, pronunciando este imperativo llamamiento: «Graziella, ven; ha
llegado la hora. Saca del arcón mi clámide más hermosa. Tráeme la
diadema y el coturno\ldots{} ¿No entiendes, tonta?\ldots{} Mis
borceguíes de tacones de oro.»

\hypertarget{xxvi}{%
\chapter{XXVI}\label{xxvi}}

Con potente acción de mi voluntad sobre mis sentidos logré
desembarazarme de aquel mundo quimérico, y me restituí a la vida normal,
volviendo a mi casa y a la comunicación afectuosa con mis amigos. Valero
de Tornos, alfonsino, y Ramón Cala, republicano, me llevaron al
Congreso, y en pasillos, tribunas y Salón de Conferencias noté agitación
y vocerío que me recordaban \emph{el gran barullo}, pronóstico de
Ferreras. Por aquel cálido y tempestuoso ambiente corría como centella
esta frase lumínica: \emph{El Rey abdica}. Pepe Ferreras, que por su
autoridad y claro sentido de las cosas formaba corrillo en cuanto
hablaba, puso el paño al púlpito y nos dijo: «Don Amadeo se va; don
Amadeo vuelve la espalda a este pueblo de orates y nos deja entregados a
nuestras propias locuras. No creáis, como algunos dicen, que a la Reina
le cuesta trabajo desprenderse del Trono español. Es todo lo contrario.»
Como sobre este punto se moviera ligera discusión en el corrillo, el
buen zamorano, mascando un puro rebelde al fósforo y a las quijadas,
prosiguió así:

«Por una dama discretísima, la más afecta a Su Majestad la Reina, he
sabido que esta planteó a su marido la cuestión en forma concluyente. No
tenía ya paciencia para soportar los desprecios del patriciado de
señoronas, que habían manifestado con descortesía su fanatismo y su
inferioridad mental. ¿Querían Borbones? Pues dárselos. La santa Señora,
que siente nostalgia honda de su tierra y de su casa ducal, saldrá de
aquí dejando memoria eterna de sus virtudes. A cambio de esto no se
llevará ni una hilacha. Huye de nosotros para librarse de los dos
fantasmas que llenan su alma de terror: el carlismo y la Internacional.
Anhela sacar a su esposo y a sus hijos de un país donde no hay hombres
que sepan domar las pasiones, y establecer un Gobierno que sea garantía
de la libertad y de la paz\ldots{} Estos sentimientos y razones han
ganado el ánimo del Rey, que, como ustedes saben, no tiene ambición. La
Corona no le deslumbra; por conservarla y traer a la razón a los
elementos que componen esta olla de grillos no quiere emplear la fuerza,
ni derramar sangre española. Por tanto, es irrevocable su resolución de
abdicar la Corona, y así lo ha manifestado a don Manuel Ruiz
Zorrilla\ldots{} Así lo ha manifestado\ldots{} así lo ha dicho.»

Más tarde, recorriendo distintas cavidades de aquel horno de pasiones y
disputas, me encontré a otro corrillo donde Llano y Persi y don Santos
La Hoz vaciaban en los oídos las noticias más recientes: el Rey había
encargado a don José de Olózaga el mensaje de abdicación; mas no
habiéndole gustado la forma y algunos conceptos del documento, encargó
nueva redacción de él a don Eugenio Montero Ríos. Llegó en esto la
noche, y el zumbar de colmena aumentaba en el Congreso. Metiéndome en
todos los corrillos vi al propio Rivero esculpiendo, con su voz dura y
su gesto autoritario, la Historia de España en aquella memorable noche
del 10 al 11 de Febrero de 1873. Por la voz, el ceño y el ademán, don
Nicolás María Rivero era un cíclope ceceoso que hablaba dando
martillazos sobre un yunque. Oponíase airadamente a la pretensión de
Zorrilla que, acariciando aún la esperanza de disuadir al Rey de su
propósito, intentó suspender las sesiones de Cortes. Rivero, firme y
tozudo en la idea contraria, quería reunir Senado y Congreso,
constituyendo así la Asamblea Nacional (llamada por algunos Convención),
que al recibir la renuncia del Rey asumiría todos los poderes.

Como teníamos jarana para toda la noche, me fui a cenar con Ramón Cala y
don Santos la Hoz a la taberna de la calle del Turco, donde es fama que
se dieron cita los matadores de Prim. Volvimos al instante al Congreso,
que estaba en sesión permanente. En las inmediaciones del edificio, por
Floridablanca y Carrera de San Jerónimo, había gentío expectante.
Relajada la disciplina de ujieres y porteros, entraban, salían y andaban
por aquella casa los ciudadanos, en revuelta familiaridad con diputados
y senadores. Corrían de grupo en grupo noticias estupendas. En uno se
aseguraba que \emph{ya no había nada de lo dicho,} que el Rey se quedaba
entre nosotros, ganoso de nuestra felicidad; en otro decían que los
constitucionales procuraban entenderse con el Gobierno para buscar
\emph{la consabida y tan acreditada} fórmula de concordia, que
permitiera seguir turnando mansamente en los pesebres del presupuesto;
más allá oímos que Serrano enviaba un recadito al General Moriones para
que acudiese a Madrid con algunas fuerzas.

En estas contradicciones y resoplidos \emph{del gran barullo} de
Ferreras se pasó la noche. Me fui a dormir a mi casa, y en la mañana del
11 traté de volver a mi puesto o atalaya de la Historia. Pero a la
familiar licencia de la tarde y noche anteriores para franquear el
edificio, había sustituido un rigor extremado. Los ujieres no dejaban
pasar ni una mosca, y hube de mantenerme en la calle observando los
grupos que circundaban el templo de las leyes. Allí me encontré con las
furibundas mesnadas de Mateo Nuevo, de García López y con muchos
individuos de la \emph{Junta Suprema del Consejo de la Federación
Española}. Vi cuadrillas de hombres armados, inquietos y vociferantes.
Busqué ávidamente entre la multitud a Nicolás Estévanez, y no le hallé
ni nadie me dio razón de él. Ya perdía yo la esperanza de colarme en el
Congreso, cuando mi buena suerte me deparó a Moreno Rodríguez, a cuyos
faldones me agarré para romper la terrible consigna porteril. En las
tribunas no se cabía. Cuando pude meter el hocico en la de la Prensa,
con terribles ahogos y apreturas, ya se había leído el mensaje de
abdicación de Amadeo I. Poco después conocí el documento y pude apreciar
su entonación viril y el amargo lamentar de un Rey que no logró la paz y
ventura de sus pueblos. Quejándose de la crudeza implacable con que
luchaban los partidos, decía: «Si fuesen extranjeros, al frente de estos
soldados tan valientes como sufridos, sería yo el primero en
combatirlos; pero todos los que con la espada, con la pluma, con la
palabra agravan y perpetúan los males de la Nación son españoles, todos
invocan el dulce nombre de la Patria, todos pelean y se agitan por su
bien, y entre el fragor del combate, entre el confuso, atronador y
contradictorio clamor de los partidos, entre tantas y tan opuestas
manifestaciones de la opinión pública, es imposible atinar cuál es la
verdadera, y más imposible todavía hallar el remedio para tamaños
males.»

En otro lugar se expresaba de este modo: «Nadie achacará a flaqueza de
ánimo mi resolución. No habría peligro que me moviera a desceñirme la
Corona si creyera que la llevaba en mis sienes para bien de los
españoles\ldots» A renglón seguido pedía, en su nombre y en el de su
esposa, que se indultase a los autores del atentado de la calle del
Arenal. Y terminaba, con frase patética, haciendo renuncia de la Corona
por sí, por sus hijos y sucesores, y despidiéndose \emph{de la noble y
desgraciada España} con toda la efusión de su alma generosa.
Suspendieron la sesión para redactar la respuesta que las Cortes debían
dar al Rey dimisionario. Crecía la efervescencia en el interior del
Congreso, y fuera la inquietud popular era ya imponente. Para calmar los
ánimos, salió Figueras a una ventana, por la calle de Floridablanca, y
pronunció una breve arenga, cuya síntesis era esta: «De aquí saldremos
muertos o con la República votada.»

Encargado Castelar de contestar al Rey, redactó en breve tiempo un
elocuente mensaje. Se reanudó la sesión. Presidía Rivero; a su lado se
sentaba Figuerola, presidente del Senado. Los escaños rebosaban de
legisladores de diferentes capacidades y cataduras. Todo lo que allí
pasaba era irregular y contrario a la Constitución, según la cual no
podían deliberar juntas en ningún caso las dos Cámaras. Sobre el fondo
de un silencio majestático fue leído el mensaje de Castelar, un adiós
ceremonioso al Rey caballero, que prefería la paz de su hogar al tumulto
de una Patria hirviente y postiza. El estilo grandilocuente y ampuloso
del orador poeta lucía en todo el documento. Flores y más flores
arrojaban las Cortes sobre la persona del Soberano dimitente y de su
augusta y amada esposa. Se les despedía con galas retóricas, lindísimas
y bien olientes, ofreciéndoles, como poético galardón, la ciudadanía de
un pueblo independiente y libre. \emph{Ite, missa est}.

\hypertarget{xxvii}{%
\chapter{XXVII}\label{xxvii}}

Sin discusión fueron aprobadas la renuncia del Rey y la respuesta o
responso que le dieron las Cortes al asumir todos los poderes. A Palacio
acudió una Comisión presidida por Rivero, la cual debía poner a manos de
Su Majestad dimisionaria los tiernos adioses de la \emph{tan noble como
desgraciada España}. En el acto palatino, que según me dijeron fue
solemne y triste, Rivero, con la trémula voz de un cíclope conmovido,
pidió al Rey y a la Reina el honor de estrecharles la mano, y no hay que
decir que tal honra le fue cordialmente otorgada. Los Reyes dijeron para
sí: \emph{Adiós, mundo amargo}.

Primer trámite del Parlamento después de lo relatado fue la renuncia del
Gobierno, que ya estaba como el alma de Garibay. Inmediatamente se
presentó la proposición pidiendo que se proclamase la República. El
debate fue ordenado y serio, sin más acritud que el corto pero grave
altercado entre Martos y Rivero. Este, movido de su temperamento
irascible y despótico, exigió duramente a los que fueron ministros de
don Amadeo que ocuparan interinamente el banco azul. Saltó Martos de su
asiento, como enconada fierecilla, y con aplauso del Congreso dijo entre
otras cosas: «No está bien que empiecen las formas de la tiranía el día
en que se despide el poder monárquico.» Estas palabritas hirieron a don
Nicolás en lo más vivo, obligándole a descender, con runflante protesta,
del augusto sitial\ldots{} ¡A votar, a votar! \emph{Doscientos cincuenta
y ocho votos contra treinta y dos} decidieron que España no era ya
Monarquía, sino República. \emph{Laus Deo}.

Procediose a elegir Poder Ejecutivo. He aquí el primer Ministerio de la
República: Presidencia, Figueras.---Estado, Castelar.---Gobernación, Pi
y Margall.---Gracia y Justicia, Salmerón (don Nicolás).---Hacienda,
Echegaray.---Guerra, Córdoba.---Marina, Beránger.---Fomento,
Becerra.---Ultramar, Salmerón (don Francisco). Cuatro de estos señores
pasaron de ministros de don Amadeo a ministros de la República con la
corta pausa de un trámite parlamentario. Martos vitoreó calurosamente a
la República, a la integridad de la Patria y a Cuba española, y Figueras
anunció días de ventura bajo un régimen de concordia, paz y
libertad\ldots{} El cambio de instituciones, que parecía mutación
teatral con subir y bajar de telones pintados, fue acogido por el pueblo
con alegría más expansiva que escandalosa. Las multitudes que invadían
las calles próximas al Congreso se difundieron fraccionándose. El más
nutrido destacamento fue a parar a la Puerta del Sol, irradiando su
ardor patriótico con vítores, cánticos, músicas y desahogos inocentes,
sin molestar a nadie ni llegar a las tonalidades demagógicas. En Antón
Martín el tumulto fue más vivo, y aparecieron banderas aparejadas
precipitadamente por ciudadanas en quien se juntaban el republicanismo y
la majeza. En la Plaza de la Cebada, en Maravillas, San Gil y demás
puntos estratégicos de las expansiones madrileñas, el entusiasmo no
traspasó los límites de la moderación. Ello fue como un plácido regocijo
lugareño, festejando \emph{la traída de aguas} o la elección de un
alcalde muy querido en la localidad.

Con puntualidad absolutamente espontánea, pues no mediaron órdenes ni
avisos, aparecieron iluminados casi todos los balcones de Madrid en la
noche del 11 al 12 de Febrero. Obdulia y yo recorrimos algunas calles, y
en las de Alcalá y Arenal contemplamos las lucecitas balconarias,
haciendo de todas ellas recuento y análisis. Eran como letras, palabras
y conceptos de una página histórica, escrita con hachones y farolillos.
Sin más auxilio que nuestro criterio y el conocimiento en cierto modo
adivinatorio que teníamos del vecindario matritense, leímos aquella
página y la diputamos por vergonzosa y repugnante. Las casas de los
republicanos, que eran los legítimos triunfadores en la jornada del 11
de Febrero, estaban a obscuras, y en cambio los palacios aristocráticos,
las moradas de las damas católicas y de los señorones alfonsinos y
carlistas brillaban con espléndido alumbrado, signo de lisonjeras
esperanzas. Mayormente nos escandalizó la cínica refulgencia de las
casas donde se albergaban los corifeos del viejo progresismo, que hasta
el día 10 fueron cortesanos y servidores de don Amadeo.

Pasando junto al Teatro Real en dirección de la plaza de Oriente, me
tocó en la espalda, llamándome por mi nombre, una mujer enlutada,
cubierto el rostro de negro velo. Por la voz conocí a Graziella, y
rogándole que abandonara el tapujo, le dije: «Numen de Italia, ¿también
tú nos dejas?»

---Bien quisiera volver a mi Patria---contestó la ninfa con voz
tremante.---Esta patria postiza me rechaza. ¡Oh, España!\ldots{}
\emph{Vedo l'armi, vedo le mure, ma la gloria non vedo}.

---Hechicera del Arno y Tíber, hija del Cardenal Fieramosca, ¿quién te
trajo a España?

---Me trajeron, diez años ha, unos pobres coristas de ópera. Era yo
mocita cuando mis padres rebuznaban, en este teatrón, los corales del
\emph{Moisés} y de \emph{La Gazza Ladra}. Ya sabes lo que fui cuando
abandonada de mis padres me metí en la vida \emph{traviattesca}. Mucho
he visto, mucho aprendí en esta tierra de la donosa picardía\ldots{}
Dragonetti me conoce bien. Voy a Palacio a despedir a unos parientes
míos que moran en las alturas, los rufianes del Rey. Quiero dar a todos
mis tiernos adioses.

---Sigue mi consejo, Graziella, y vete con los de tu raza.

---No puedo, queridos amigos Tito y Tita; que en Madrid he de quedarme
al cuidado de mi anciano protector y amigo del alma don Hilario. A
proceder así me mueve con mi cariño la ambición intensa que me llena
toda el alma. ¿Sabes lo que ambiciono?\ldots{} No te rías\ldots{} Aspiro
a que vosotros, los locos de la Federal, hagáis obispo al sacerdote más
ilustrado y virtuoso que existe en las Españas míseras. Con el oro y la
plata de mis ahorros le he comprado ya la mitra y báculo\ldots{} Dentro
de pocos días adquiriré un magnífico pectoral que he visto en el Monte y
un soberbio anillo, que espero besaréis con devoción tú y todos tus
compinches\ldots{} En fin, apresurad el paso, que yo tengo prisa. Si
queréis entrar en Palacio, venid conmigo.

En esto nos hallábamos frente a la inmensa mole de la casa de los Reyes,
huraña y obscura, contrastando lúgubremente con las luminarias de la
Burguesía infatuada y de la Aristocracia enloquecida.

\hypertarget{xxviii}{%
\chapter{XXVIII}\label{xxviii}}

Momentos después, mi \emph{Tita} y yo, por virtud del poder milagroso
que llevábamos en nuestras almas, nos convertíamos en gatitos diminutos
y recorríamos, con jugueteo y brincos invisibles, la Saleta, la
Antecámara y Cámara, y otras regias estancias. Un hado benéfico,
protector de nuestro sagaz espionaje, nos permitió ver el solemne
desfile que era fin y principio, engarce o eslabón entre dos
interesantes etapas históricas. Delante iban damas y palaciegos rodeando
a las servidoras que conducían a los dos niños mayores, Manuel
Filiberto, ex-Príncipe de Asturias, de cuatro años de edad\footnote{Hoy
  Duque de Aosta.} y Víctor Manuel, de tres años y dos meses\footnote{Hoy
  Conde de Turin.}. Seguía el ama que llevaba en brazos al ex infante
Luis Amadeo Fernando, nacido en Madrid el 29 de Enero: su edad, catorce
días\footnote{Hoy Duque de los Abruzzos, explorador del Polo Norte.}. En
torno a esta criatura se agrupaban los Marqueses de Dragonetti y otras
personas de alta jerarquía, italianas y españolas. Detrás iba don Amadeo
grave y sereno, sin expresar pena ni alegría, vestido de viaje. La
corona y atributos monárquicos se habían quedado en el suelo del
Despacho del Rey, al pie del retrato de María Luisa.

Daba el brazo el Monarca dimisionario a su digna y santa esposa, doña
María Victoria, envuelta en pieles. No se le veía más que el rostro
pálido, con marcadas huellas de dolencia reciente. No parecía pesarosa
de abandonar la colosal vivienda que fue para ella lugar de ansiedad y
martirio. A los que fueron sus servidores despedía con sonrisa graciosa
y afable. Creímos que les decía: «No me llevo más que lo mío, marido y
mis hijos. Os dejo todo lo vuestro, una corona que no ambicioné y un
título de Reina que no fue para mí más que una palabra vana.»

Rodeaban a los Reyes personas finchadas de estas que llaman hombres
públicos. No transcribo nombres porque no estoy bien seguro de acertar
en mis designaciones. Había entre ellos algunos militares que en ocasión
distinta enumeré en estas páginas. Confundido entre la turbamulta, y
como si quisiera ocultar con su persona su desconsuelo, iba Ruiz
Zorrilla, con luto y resignación en el rostro macilento. En la cola de
la procesión vi a mi adorada señora \emph{Mariclío}, tan grande que no
había techo de suficiente alteza para su figura majestuosa. Vestía la
clámide griega, calzaba el coturno y ceñía su frente la diadema cuyos
reflejos iluminaban el Espacio y el Tiempo. Su rostro clásico, sus
labios mudos y sus ojos divinos decían: «Al fin encontré la página
hermosa. Ahora soy quien soy.»

El momento más triste y grandioso de aquel éxodo fue el descender de la
comitiva por la Escalera de Honor, entre alabarderos rígidos, sin música
ni voces que turbaran el fúnebre silencio. Sólo el rumor de las pisadas
marcaba el lento caminar de una época, declinando hacia los senos del
Tiempo que traen la sanción de los actos y el juicio de la Historia.

Y nada más\ldots{} Se obscureció la escalera, se obscureció el Palacio,
apagose el ruido de las pisadas. Nos vimos envueltos en tinieblas de
panteón\ldots{}

fin de amadeo i

Santander-Madrid, Agosto-Octubre de 1910.

\flushright{Santander-Madrid, Agosto-Octubre de 1910.}

~

\bigskip
\bigskip
\begin{center}
\textsc{fin de amadeo i}
\end{center}

\end{document}
