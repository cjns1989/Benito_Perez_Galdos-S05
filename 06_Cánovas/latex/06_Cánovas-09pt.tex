\PassOptionsToPackage{unicode=true}{hyperref} % options for packages loaded elsewhere
\PassOptionsToPackage{hyphens}{url}
%
\documentclass[oneside,9pt,spanish,]{extbook} % cjns1989 - 27112019 - added the oneside option: so that the text jumps left & right when reading on a tablet/ereader
\usepackage{lmodern}
\usepackage{amssymb,amsmath}
\usepackage{ifxetex,ifluatex}
\usepackage{fixltx2e} % provides \textsubscript
\ifnum 0\ifxetex 1\fi\ifluatex 1\fi=0 % if pdftex
  \usepackage[T1]{fontenc}
  \usepackage[utf8]{inputenc}
  \usepackage{textcomp} % provides euro and other symbols
\else % if luatex or xelatex
  \usepackage{unicode-math}
  \defaultfontfeatures{Ligatures=TeX,Scale=MatchLowercase}
%   \setmainfont[]{EBGaramond-Regular}
    \setmainfont[Numbers={OldStyle,Proportional}]{EBGaramond-Regular}      % cjns1989 - 20191129 - old style numbers 
\fi
% use upquote if available, for straight quotes in verbatim environments
\IfFileExists{upquote.sty}{\usepackage{upquote}}{}
% use microtype if available
\IfFileExists{microtype.sty}{%
\usepackage[]{microtype}
\UseMicrotypeSet[protrusion]{basicmath} % disable protrusion for tt fonts
}{}
\usepackage{hyperref}
\hypersetup{
            pdftitle={Cánovas},
            pdfauthor={Benito Pérez Galdós},
            pdfborder={0 0 0},
            breaklinks=true}
\urlstyle{same}  % don't use monospace font for urls
\usepackage[papersize={4.80 in, 6.40  in},left=.5 in,right=.5 in]{geometry}
\setlength{\emergencystretch}{3em}  % prevent overfull lines
\providecommand{\tightlist}{%
  \setlength{\itemsep}{0pt}\setlength{\parskip}{0pt}}
\setcounter{secnumdepth}{0}

% set default figure placement to htbp
\makeatletter
\def\fps@figure{htbp}
\makeatother

\usepackage{ragged2e}
\usepackage{epigraph}
\renewcommand{\textflush}{flushepinormal}

\usepackage{indentfirst}

\usepackage{fancyhdr}
\pagestyle{fancy}
\fancyhf{}
\fancyhead[R]{\thepage}
\renewcommand{\headrulewidth}{0pt}
\usepackage{quoting}
\usepackage{ragged2e}

\newlength\mylen
\settowidth\mylen{……………….}

\usepackage{stackengine}
\usepackage{graphicx}
\def\asterism{\par\vspace{1em}{\centering\scalebox{.9}{%
  \stackon[-0.6pt]{\bfseries*~*}{\bfseries*}}\par}\vspace{.8em}\par}

 \usepackage{titlesec}
 \titleformat{\chapter}[display]
  {\normalfont\bfseries\filcenter}{}{0pt}{\Large}
 \titleformat{\section}[display]
  {\normalfont\bfseries\filcenter}{}{0pt}{\Large}
 \titleformat{\subsection}[display]
  {\normalfont\bfseries\filcenter}{}{0pt}{\Large}

\setcounter{secnumdepth}{1}
\ifnum 0\ifxetex 1\fi\ifluatex 1\fi=0 % if pdftex
  \usepackage[shorthands=off,main=spanish]{babel}
\else
  % load polyglossia as late as possible as it *could* call bidi if RTL lang (e.g. Hebrew or Arabic)
%   \usepackage{polyglossia}
%   \setmainlanguage[]{spanish}
%   \usepackage[french]{babel} % cjns1989 - 1.43 version of polyglossia on this system does not allow disabling the autospacing feature
\fi

\title{Cánovas}
\author{Benito Pérez Galdós}
\date{}

\begin{document}
\maketitle

\hypertarget{i}{%
\chapter{I}\label{i}}

Los ociosos caballeros y damas aburridas que me han leído o me leyeren,
para pasar el rato y aligerar sus horas, verán con gusto que en esta
página todavía blanca pego la hebra de mi cuento diciéndoles que al
escapar de Cuenca, la ciudad mística y trágica, fuimos a parar a
Villalgordo de Júcar, y allí, mi compañero de fatigas Ido del Sagrario y
yo, dando descanso a nuestros pobres huesos y algún lastre a nuestros
vacíos estómagos, deliberamos sobre la dirección que habíamos de tomar.
El desmayo cerebral, por efecto del terror, del hambre y de las
constantes sacudidas de nervios en aquellos días pavorosos, dilató
nuestro acuerdo. Inclinábame yo a correrme hacia Valencia, impelido por
corazonadas o misteriosos barruntos. Di en creer que hallaría en tierras
de Levante a mi maestra \emph{Mariclío} y que por ella tendría
conocimiento de la preparación de graves sucesos. Pero a Ido le tiraba
hacía Madrid una fuerte querencia: su mujer, sus amigos, su casa de
huéspedes. La ley de adherencia en las comunes andanzas aventureras nos
apegaba con vínculo estrecho. Desconsolados ambos ante la idea de la
separación, cogimos el tren en La Roda y nos plantamos en la Villa y
Corte.

Largos días permanecí recluido en mi aposento pupilar de la calle del
Amor de Dios. La casa estaba desierta por ausencia de los estudiantes de
San Carlos que gozaban ya de la dilatada vagancia veraniega. Prisionero
me constituí en mi celda, sin osar poner los pies en la calle, no sólo
por aburrimiento, sino por tener mis bolsillos tristemente limpios y
mondos de toda clase de numerario. Olvidado me tenía mi excelsa Madre,
sin que mi conciencia ni mi razón explicarme supieran la causa de tal
abandono, pues nada hice ni pensé que pudiera desagradarla. Cuantas
veces acudí a la portería de la Academia de la Historia en busca de los
emolumentos que allí, solícita y puntual, me consignaba \emph{Doña
Mariana}, hube de volverme desconsolado y con las manos vacías a mi
pobre hospedaje. Por fin, avanzado ya el mes de Agosto, ¡oh inefable
dicha!, la portera de la docta casa me entregó con graciosa solemnidad
un paquete que contenía suma moderada de los sucios \emph{papiros} que
llamamos billetes de Banco, y una cartita cuyo interesante contenido
devoré con mis ojos en el corto trayecto de la calle del León a la del
Amor de Dios.

«Perdona, mi buen muñeco---decía la carta,---si tan largo tiempo estuve
sin acudir a tus necesidades. Con la presente recibirás ración no muy
cumplida del pan de la vida social. Gástalo con tiento, mantente en la
justa ponderación de la economía y la prodigalidad\ldots{} Estoy donde
estoy. No me verás tan pronto. Vivo en obscuro escondite, acechando un
hecho histórico que tú no has previsto y yo sí. No pocos caballeros
españoles y algunas damas alcurniadas quieren engendrar un ser político,
que representará la transformación capital de la familia hispana. Es lo
que el bueno de Víctor Hugo llamaba \emph{un gozne de la
Historia}\ldots{} Yo me entretengo mirando a los que ponen sus manos
pecadoras en esta labor mecánica. Unos se esfuerzan en engrasar la
espiga y el anillo del gozne para que el doblez se efectúe sin aspereza
y con silencio decoroso; otros, en su afán de terminar prontito, salga
lo que saliere, doblarán la Historia con maniobra violenta, y el
chirrido del metal giratorio se oirá hasta en la China\ldots{} ¿No
entiendes esto, historiador travieso y chiquitín?\ldots{} Vístete bien,
ahora que tienes dinerito fresco, y no busques tu sastre entre los de
medio pelo. Reanuda y cultiva tus antiguas amistades, y disponte a
estrecharlas nuevas relaciones que te salgan al paso. No desdeñes a los
hombres de pro\ldots{} El pro se acerca taconeando recio\ldots{} La
pobretería se aleja pisando con el contrafuerte\ldots{} Adiós, hijo. En
cuanto lleguen las brisas de otoño, que avivan la natural frescura y
alegría de los madrileños, diviértete lo que puedas. Si sientes apetito
de lecturas, pon a un lado al amigo Saavedra Fajardo, y entretente con
el \emph{Manual del perfecto caballero en sociedad}, consagrando algunos
ratos a la \emph{Moda elegante}.»

Confuso me dejó la epístola, que leí cuatro veces, y aunque algo pude
descifrar de su sentido recóndito, no llegué al pleno dominio de las
ideas expresadas por la Madre en aquellas líneas, escritas con genuino
trazo de Iturzaeta\ldots{} Septiembre se me pasó en renovar mis
amistades de Madrid, y en ponerme al habla con sastres y zapateros.
Amenguaba ya el calor; pero aún se veían en el Prado grupos de paseantes
y tertulias de gente distinguida: formábanlas familias que no habían
podido ir a baños y otras que se volvieron antes de tiempo, repatriadas
por la escasez de pecunia. En diferentes corros y tertulias mariposeaba
yo en las tardes y noches de variado temple. También gustaba de
arrimarme a los puestos de agua, frecuentados por parroquianos de
distinta marca social, bastos, finos y entrefinos.

Ved ahora la cáfila de amigos que me salieron al encuentro en el Prado y
sus aguaduchos: Luis Blanc, Moreno Rodríguez, Serafín de San José,
Telesforo del Portillo \emph{(Sebo)}, Patricio Calleja, Mateo Nuevo,
Fructuoso Manrique, David Montero, \emph{Dorita}, Niembro, Emigdio
Santamaría, Díaz Quintero, María de la Cabeza, Delfina Gay, y el
imponderable \emph{don Florestán de Calabria}, que se presentó ante mí
con flamantes apariencias de limpieza y elegancia. Apartados del grueso
de la concurrencia, que paladeaba el agua fresca con azucarillos y
aguardiente, echamos un parrafito. Díjome que a femeniles influencias
debía un empleíto escribientil en el Círculo Popular Alfonsino, y que
desde que se puso en contacto con las \emph{personas decentes} había
empezado a echar buen pelo, como lo demostraba su ropa.

A mis anhelos de conocer el paradero de \emph{Leona la Brava}, contestó
que estaba en París. ¿Fue quizás con el hinchado figurón de los
monumentales sombreros? No; el tal no gozaba ya la privanza de la dama
de Mula; con su fatuidad \emph{chisteriforme} habíase retirado, dejando
el puesto a un protector nuevo, caballero separado de su mujer,
regordete, calvoroto, afeitado el rostro y muy pulido de vestimenta,
íntimo amigo de don Francisco Cárdenas, de don Manuel Orovio, y
asistente pegajoso a la tertulia del Conde de Cheste. Noté en \emph{don
Florestán} cierto pudor para revelarme el nombre de aquel sujeto; sin
duda quería guardar el incógnito de uno de los hombres de pro que le
habían protegido. No insistí, seguro de descifrar el acertijo en cuanto
\emph{Leona} volviera de su excursión parisiense. ¡Y que no vendría poco
ilustrada en todo género de novelerías y elegancias! Terminó el
pendolista sus referencias diciéndome con cierta vanagloria: «Fíjese
usted, don Tito; el amigo de doña Leonarda es de los que tienen más
metimiento en el palacio Basilewski, donde reside la que fue nuestra
Soberana, quien como usted sabe abdicó ya en su hijo don Alfonsito.»

Quedé con don Genaro en que me avisaría puntualmente la fecha de la
\emph{rentrée} de \emph{La Brava}, y ya no volví a verle hasta mediados
de Octubre. En tanto, los amigos cuyo trato frecuentaba yo por aquellos
días, me confirmaron en la idea de que la sociedad española quería
cambiar de postura, como los enfermos largo tiempo encarnados sin
encontrar alivio. Notaba yo la \emph{lenta pero continua} inclinación de
las voluntades hacia un ideal que a primera vista deslumbraba,
desviándose de los ideales pálidos ya y marchitos. Dábame en la nariz el
olor del aceite con que los más sagaces querían engrasar la bisagra
histórica, y a mi oído llegaba el crujir de los impacientes y el
retemblido del aparato con que se hacen los dobleces de la vida de un
pueblo.

En la última decena de Octubre tuve conocimiento del regreso de Leonarda
y de su domicilio, calle del Saúco, a espaldas del Ministerio de la
Guerra. Juzgando indiscreto visitarla sin previa petición de venia, eché
por delante un recadito con el de \emph{Calabria}, y por el mismo
conducto recibí un pase para penetrar en la gruta de la ninfa. Era la
casa linda, coquetona, mejor apañada y dispuesta que la de la calle de
Lope para un vivir descuidado y placentero. En el carácter de
\emph{Leona} no advertí mudanza: era la misma mujer afable, cariñosa y
sugestiva que descubrí en el tempestuoso ambiente del \emph{Cantón
Cartaginés}. En su habla encontré notorio progreso, pues no se daba
reposo en la tarea de perfeccionar su léxico. Apenas abrió la boca, me
saltó al oído el decir exquisito, que revelaba un trato frecuente con
personas de cepa \emph{moderada}. Con estos refinamientos se confundía
un gracioso empleo de galicismos de buen tono, y el desaprensivo
chapurrar de términos franceses, entreverados con lo más corriente de
nuestro lenguaje.

Apenas cambiamos las primeras cláusulas de afecto y remembranzas,
\emph{Leona} me soltó en nervioso estilo el relato de sus impresiones de
París, juzgando con criterio justo todo lo que había visto, sin dejarse
llevar del prurito de la admiración ni columpiarse en los espasmos de la
hipérbole, como es uso y costumbre de los que llevan a la gran Lutecia
todo el bagaje de sus almas provincianas. El buen gusto apuntaba ya en
mi dulce amiga, anunciando la deliciosa ecuanimidad de la \emph{mujer de
mundo}. «Vivíamos en la \emph{Rue Richepanse}, muy cerquita de la
Magdalena y a poca distancia de la Plaza de la Concordia---me
dijo.---Nos retirábamos tarde, porque casi todas las noches íbamos al
teatro. A media mañana nos levantábamos, y yo empleaba largo rato en mi
\emph{toilette}, que allí, Tito mío, hay que mirar bien cómo sale una a
la calle. Almorzábamos, unas veces en el \emph{Café Anglais}, que es lo
mejor de París; otras veces en \emph{Vefour}, en las arcadas de una
plaza que llaman \emph{Palais Royal}. Por probar de todo, y para que yo
me enterara bien de lo que es aquel gran pueblo en lo tocante a
comistrajes, íbamos algunos días a unos \emph{restauranes} baratitos,
pero \emph{la mar} de buenos, que llaman \emph{Bullones o Duvales}.»

A su caballero daba \emph{Leona} el nombre de Alejandro, que a mi
parecer era denominación familiar convenida entre ellos, pues según mis
barruntos, el tal personaje figuró después en la Historia no muy
lucidamente con nombre bien distinto. «Después de almorzar---continuó
diciendo \emph{La Brava},---mi Alejandrito me dejaba en el Hotel y se
iba a sus negocios, que no eran otros que la conspiración alfonsina.
Largas horas pasaba en el Palacio de la Reina; visitaba al marqués de
Molins, a Salaverría, al Duque de Sexto, a don Martín Belda y a otros
que yo no recuerdo, todos ellos metidos en esa contradanza del
alfonsismo. Cansábame yo de estar encerrada en el Hotel, y algunas
tardes cogía mi sombrero y mi sombrilla y me marchaba a pasear por los
bulevares, llegándome hasta la Puerta de \emph{San Denis} o un poquito
más allá. Yo podía decir lo que dicen que dijo Cúchares cuando le
preguntaron si se había divertido en \emph{París de Francia: aqueyo es
mu aburrío. To er zanto día está uno olivarej arriba, olivarej
abajo}\ldots{} Y no te creas, Tito, que era \emph{Leona} costal de paja
para los franchutes. \emph{Olivares} arriba y abajo me seguían dos, tres
y a veces hasta cinco moscones haciéndome el amor, y diciéndome cosas
que yo entendía muy bien sin saber una palabrita de aquel habla. Pero,
dándome la mar de pisto y con muchísima dignidad, seguía mi camino sin
hacerles caso y me metía en la fonda.»

No volví a ver a \emph{Leona} hasta una noche de Noviembre, en el teatro
Real, a donde la llevaba con frecuencia su afición a la ópera, nueva
señal de adelanto en su carrera de cultura. Después de buscar a Leonarda
por las regiones \emph{paradisíacas} la encontré en delantera de palco
por asientos, localidad que abonada tenía con dos amigas guapas,
elegantonas y de la propia marca social. En los entreactos picoteaban
las tres pasando revista con picante estilo a la concurrencia de damas,
y señalando indiscretamente a sus \emph{editores responsables},
confundidos en la turbamulta de gente distinguida, conservadora y
alfonsina. Sobre la negrura de los fraques se destacaban las calvas,
relucientes algunas como bolas de billar. La ópera de aquella noche era
\emph{Roberto el Diablo}, cantada por Rosina Penco, el tenor Nicolini y
el bajo David. Poco pude hablar con mi amiga en aquella ocasión porque
de improviso llegaron al palco unos pollastres esmirriados, en traje de
etiqueta, que entablaron voluble conversación con las tres damas,
acosándolas con bromas de mal gusto y cuchufletas impertinentes. Me
retiré a mi localidad del Paraíso un tanto mohíno y desconsolado.

Más dichoso fuí la noche del estreno de \emph{Aida}, hacia el 10 o el 12
de Diciembre, porque tuve la precaución de tomar anticipadamente la
delantera de palco por asientos inmediata a las que ocupaban las tres
ninfas. Sentado junto a mi amiga pude charlar con ella cuanto me dio la
gana. «Esta noche---me dijo \emph{Leona}---tenemos el teatro \emph{au
grand complet}. Sabrás, Titín salado, que hace tres semanas me da
lecciones un profesor de francés, a quien conocerás el día que vuelvas
por casa. Como los temas se me salen de la boca sin pensarlo, te
pregunto: \emph{¿Tienes el cordón azul de la sobrina del hermano de mi
jardinero?} Mi respuesta fue: \emph{No tengo el cordón de la bella
hermana del sacristán; pero tengo la inmensa satisfacción de contemplar
de cerca tus negros ojos y de admirar los blancos dientes que asoman
entre esos labios de coral cuando iluminas el teatro con tu sonrisa}.

---Cállate la boca, Tito, que no estamos solos---me contestó \emph{La
Brava}.---Mejor será que eches tus miradas por esta sala espléndida. En
aquel palco tienes a la Campo Alange con su hija Luisa, que esta noche
se lleva en el Real la palma de la hermosura. En la platea del
proscenio, debajo del palco de los ministros, verás a la Medinaceli.
Buena mujer, verdad. ¿Te gusta? ¡Ah, pillo!\ldots{} Más arriba, en los
entresuelos, está la Fernán Núñez y su hija Rosarito, \emph{très
gentille}, con otras chicas muy guapas. Sigue mirando. ¿No ves a la
baronesa de Hortega con su palco lleno de señorones?

---Sí. Y en el palco de al lado la de Navalcarazo.

---\emph{Pardon, moncher Tit}. No es la de Navalcarazo, sino la de
Híjar\ldots{} Allí tienes a Robles, el empresario del teatro, un
caballero alto, moreno\ldots{} En la platea de abajo la Montúfar, guapa,
carnosa. Tras ella el Marqués de Bedmar, Heredia Spínola y otro
alfonsino vejancón que no recuerdo cómo se llama. En aquella platea,
mira, Sardoal, Ricardo Álava y unas señoras que no conozco. En el palco
de al lado la Perijaa con la Acapulco.

---Y luego sigue la de Ahumada\ldots{}

---\emph{Pardon, mon ami}. Me sé de memoria a todo el señorío de Madrid,
lo que llamamos \emph{gens du monde}. Esa que dices tú es la Folleville,
con la Belvís de la Jara, la Campoalegre y Pepito Montiel\ldots{} Vuelve
tus ojos al entresuelo y verás a la Villavieja con el Marqués de
Yébenes, el neo más rabioso que hay en todo el universo mundo.»

Cambiando bruscamente de cháchara, sin dejar de prodigar los
\emph{pardones} a cada instante, me quitó \emph{Leona} los gemelos para
mirar a las butacas. «En el pasillo central, allí, al extremo, de
espaldas a la orquesta, tienes al caballero más pomposo y elegantón que
hay en el teatro---me dijo.---Es \emph{Monsieur le Marquís du Bacalaó}.
A él se acerca en este momento mi Alejandrito. Reconócelo por la calva,
que es de las que hacen época en la historia del poco pelo. Sé lo que le
está diciendo. Cosa muy interesante. En el segundo entreacto te lo
contaré, pues el primero pronto se acabará\ldots{} ¿No ves en otro grupo
a Ramón de Navarrete? ¡Oh, \emph{le grand critique de société}, por mal
nombre \emph{Asmodeo}! Dicen que es más viejo que la Cuesta de la Vega,
pero está muy espigadito todavía.

---Ya, ya. También andan por ahí don Ignacio Escobar, y Jove y Hevia.

---Ahora entra Ramón Correa con Cruzada Villamil\ldots{} A callar, a
callar, que empieza el segundo acto\ldots{} Esta ópera me va gustando
mucho. Hoy leí el libreto y sé que pasa en el Egipto, donde están las
Pirámides. ¿Saldrán aquí esas Pirámides? Me gustaría verlas.»

Terminado el acto segundo con el grandioso concertante que sigue a la
marcha de las trompetas, \emph{Leona} se dispuso a comunicarme las
interesantes novedades políticas que, según ella, conocía mejor que
nadie en Madrid. Recatando su rostro tras el abanico, me dijo con
afectada reserva: «Has de saber, querido Tito, que don Alfonso ha dado
un Manifiesto a la Nación, escrito en un Colegio no sé si de Inglaterra
o de Alemania. Hasta ahora no se ha hecho público ese documento, que
dice cosas muy bonitas.

---¿Lo has leído tú?

---\emph{Pardon}. No lo he leído. Pero mi Alejandro, que recibió un fajo
de ellos para repartirlos, me ha contado todo lo que trae. Cosa buena.
Como que está escrito por Cánovas, \emph{voilà}.

---Sí, sí. Dirá\ldots{} ya se sabe\ldots{} todo lo que es de rigor
cuando los Reyes destronados quieren que se les franqueen los caminos o
los atajos de la restauración.

---Dice\ldots{} que seamos buenos\ldots{} \emph{Pardon}\ldots{} no es
eso\ldots{} Dice que viene a reinar por haber abdicado su mamá, que a
todos abrirá de par en par las puertas de la legalidad, o como si
dijéramos, que todos entrarán al comedero para llenar el buche,
\emph{passez moi le mot}\ldots{} Y pone más, Tito; escucha: que si al
igual de sus antecesores será siempre buen católico, como hijo del siglo
ha de ser verdaderamente liberal.

---Dos ideas son esas, \emph{ma chérie}, que rabian de verse juntas.
¿Liberal y católico? ¡Pero si el Papa ha dicho que el liberalismo es
pecado! Como no sea que el Príncipe Alfonso haya descubierto el secreto
para introducir el alma de Pío IX en el cuerpo de Espartero\ldots»

\hypertarget{ii}{%
\chapter{II}\label{ii}}

En el tercer entreacto de \emph{Aida}, Leonarda, coincidiendo con mi
excelsa Madre, me aconsejó que me pusiese a tono con la situación que se
veía venir. Don Alfonso estaba en puerta, aunque otra cosa pensasen los
cándidos \emph{provisionales} y los que creyéndose listos andan a
tientas por las obscuridades de la vida. Al Gobierno de Sagasta no le
llegaba la camisa al cuerpo y se defendía deportando a Filipinas a todos
los que juzgaba sospechosos. Sospechoso era el país entero, que pedía
orden y paz, metiendo de una vez en cintura a los malditos carcas y a
los insurgentes de Cuba. A tan atinadas observaciones, que mi amiga
expresaba en lenguaje más llano del que yo uso, agregó luego estos
familiares consejos, inspirados en un claro sentido de la realidad:
«Cuídate ahora de la buena ropa, porque se ha concluido el reinado de
los cursis y de la pobretería. Arrímate a Cánovas, que es el hombre de
mañana, y si no tienes medios para hacerte su amigo yo te los
proporcionaré. Qué, ¿te asombras? Esta pobre \emph{Lionne}, que te
parecerá una \emph{doña Nadie}, tiene hoy un poder que ya lo quisieran
más de cuatro.»

Al final de la ópera, entre el tumulto de los aplausos que prodigó el
público a Tamberlick y a la Fossa, me dijo Leonarda que por \emph{don
Florestán} me avisaría para celebrar una entrevista y ponerme al tanto
de los acontecimientos. Despedime cariñosamente de ella y de sus dos
amigas, que tengo el gusto de presentar a mis lectores, presagiando que
tal vez las encontraremos más tarde en nuestro camino. La una era María
Ruiz, menudita y graciosa; la otra Carolina Pastrana, ojinegra, blanca y
gordezuela; ambas liadas con alfonsinos de riñón bien cubierto que no
debo nombrar porque ya entrábamos en la era de la hipocresía, del mírame
y no me toques, y del buen callar, que llamamos Sancho.

Con la mayor parte de los ministros del Gabinete Sagasta tenía yo pocas
relaciones. Al Presidente no le había visto desde el tiempo de don
Amadeo. A Ulloa y Romero Ortiz les trataba superficialmente. Por cierto
que este, en su despacho de Gracia y Justicia, adonde fui con una
comisión de postulantes gallegos, nos habló del \emph{Manifiesto de
Sandhurst} con marcado menosprecio. El único Ministro con quien tenía yo
franca amistad era el de Fomento, Carlos Navarro Rodrigo, el cual en
Noviembre me manifestó su proyecto de fundar un gran periódico que
defendiera \emph{la pura doctrina constitucional}, contando conmigo para
redactor político. ¡A buenas horas mangas verdes!

Una tarde, a fines de Diciembre (creo que fue por Inocentes, día más día
menos) fui a verle a su despacho de la Trinidad, y me le encontré
demudado y tan nervioso que su lengua gorda no articulaba las palabras
con la claridad debida. «¿Pero no sabe usted lo que pasa, Tito?---me
dijo, anonadándome con su gesto y el aire imponente de su procerosa
figura.---Esto es inaudito. Vivimos en un país de locos\ldots{} Por
telegrama de hoy se ha sabido que en Sagunto, el General Martínez Campos
ha proclamado Rey de España al Príncipe Alfonso. ¿Es esto racional, es
esto patriótico?\ldots{} ¿Qué personalidades del Ejército le han ayudado
en su loca empresa? Se habla de Jovellar, de Balmaseda, de los Dabanes,
de Borrero; no sé\ldots{} no sé\ldots»

Acto seguido entraron precipitadamente en el despacho los Directores
Generales y los Secretarios, con sin fin de papelotes que traían a la
firma. El Ministro, con presurosa mano, garabateaba su testamento. Al
despedirme, don Carlos me dijo: «Nuestro periódico se quedará para
mejores tiempos. Ahora mismo voy a ver a Serrano Bedoya y a Primo de
Rivera, para saber qué determinan el Ministro de la Guerra y el Capitán
General de Madrid\ldots{} Esto no puede quedarse así\ldots{} Algo muy
gordo pasará\ldots{} Quizás no pase nada\ldots{} Veremos\ldots»

Caviloso me volví a mi casa, y al subir la escalera sentí mi espíritu
lanzado a un torbellino de ideas contradictorias. La renovación social y
política que se anunciaba ¿era un paso hacia el bienestar nacional o un
peligroso brinco en las tinieblas?\ldots{} Apenas entré en mi aposento
me dio la ventolera de ponerme los trapitos de cristianar para salir al
visiteo de las personas de pro, obediente a las sabias indicaciones de
\emph{Mariclío} y de \emph{Leona la Brava}. Yo me había hecho a la
entrada de invierno elegante ropita para andar por el mundo: pantalones
de última moda, chalecos vistosos, levita inglesa y un gabán con forros
de seda y cuello y bocamangas de piel, que quitaba el sentido. Este rico
indumento completábase con espléndido surtido de corbatas, guantes,
botas de charol y sombrero de copa \emph{dernière façon}.

Disponiéndome para vestirme busqué mi ropa en la percha y en un armario
de luna que me habían puesto mis patrones para mayor decoro de la
estancia hospederil, y busca que te busca, no encontré ninguna de
aquellas ricas prendas que me costaron un dineral. Contrariado primero,
furioso después, empecé a pegar gritos:

«¿Qué es esto? ¡Don José, Nicanora! ¿Dónde está mi ropa?» No tardó en
acudir a mi desesperado llamamiento el filósofo Ido, que trémulo y
confuso, me dijo: «Ilustrísimo Señor: llega Vuecencia a su casa
trastornado, falto de memoria. Las tres y media serían cuando llamaron a
la puerta dos individuos con uniforme, que me parecieron ordenanzas de
la Presidencia o ujieres del Parlamento. Venían de parte de Vuecencia
por su ropa elegante para vestirse allá, no sé donde\ldots{}

---Yo no he pedido mi ropa, ¡canastos, mil porras!---exclamé fuera de
mí.---Es usted un simple, don José. Se ha dejado usted robar.

---Señor, yo me lo creí porque\ldots{} verá\ldots{} A eso de las dos y
cuarto me encontré en la calle a ese amigo de Vuecencia\ldots{} don
Serafín de San José\ldots{} el cual me dijo que para que don Alfonso
venga con más aquel, se quería formar hoy mismo un Ministerio de
conciliación y de ancha base, pero muy ancha\ldots{}

---¡Qué demonio de conciliación ni qué ocho cuartos!

---Conciliación del orden con el desorden, de la libertad con el palo,
de Cheste con don Salustiano de Olózaga. Ya ve usted si es ancha la
base\ldots{} Al saber esto y al ver que Vuecencia me pedía su
ropa\ldots{} \emph{francamente, naturalmente}\ldots{} pensé que era su
Ilustrísima uno de los llamados a componer ese Ministerio, y que tenía
que vestirse a escape por mor del juramento y de la toma de
posesión\ldots{}

---¡Qué juramento, que posesión, ni qué cuerno! ¡Señor don Ido del
seguro, señor don Ido de la cabeza, basta de enredos y venga pronto mi
levita, mi gabán, mi\ldots!

---Excelentísimo señor don Tito---exclamó Sagrario consternado y casi
lloroso.---Lo que he tenido el honor de decir a Vuecencia es el mismo
Evangelio.

---Déjeme usted de Evangelios, señor mío. Ya empiezo a creer que esto es
una broma de los estudiantones de San Carlos que tiene en su casa, los
más traviesos, los más alocados, los más pillos, hablando mal y pronto,
que hay en Madrid\ldots{} Esas diabluras de niños mal educados no las
tolero yo. Que los aguanten sus padres, que no supieron darles mejor
crianza\ldots{} Y usted, señor don Ido, señor don Dejado de la mano de
Dios, usted es responsable de este despojo. Ya verán todos quién es
Tito. Esta misma tarde daré parte a la policía y\ldots»

En esto presentose Nicanora, y con tan sinceras y persuasivas palabras
confirmó lo dicho por su esposo, que yo quedé perplejo, sin saber qué
pensar. El desgaste de energía me llevó a un estado de atontamiento que
pronto fue laxitud soporífera. Dije a mis patrones que me dejaran solo,
y me tumbé en el sofá, cuyos muelles cortantes habían sufrido aquel
verano esmerada reparación\ldots{} Rumor de misteriosas voces atormentó
mis oídos. Otra vez me sentí en poder de los entes invisibles que en
ciertas ocasiones de mi vida dirigían a su antojo mi conducta social. Y
eran precisamente los espíritus malos, bien distintos de aquellos
benéficos protectores que más de una vez endulzaron mi existencia.

De improviso, me hizo saltar en el sofá un anhelo irresistible de
echarme a la calle. Y como ya no podía, por falta de la ropa buena,
visitar a la aristocracia política, resolví vestirme con un trajecillo
raído, añadiendo la capa venerable, astrosa, digna de pasar de mi casa
al Rastro, y el hongo abollado que sufrió los rigores del asalto de
Cuenca, pues la chistera número dos habíala destinado a medir garbanzos.
Iba, pues, como uno de esos cesantes crónicos que todo lo esperan de las
algaradas demagógicas. En la calle me sentí populacho, y hube de
contenerme para no gritar \emph{¡Abajo Alfonso}! \emph{¡Viva la libertad
de cultos y el desestanco de la sal!} En mis oídos resonaba la cháchara
de los espíritus maléficos, aviesos y burlones. Tal era mi aturdimiento
que llegué a desconocer los sitios por donde iba. A menudo recibía
empujones de los transeúntes con quienes tropezaba, y en todos ellos
creí ver \emph{moderados} o alfonsinos orondos, insolentes, pavoneándose
en celebración de su triunfo.

Sin saber cómo ni por dónde, cual cuerpo inconsciente lanzado por el
acaso a los laberintos callejeros, llegué a la Travesía de la Parada y a
la taberna de Ginés Tirado. Entre los parroquianos que allí mataban el
tiempo encontré al maestro de obras Cerrudo, Perico \emph{el de los
Mostenses}, el corredor de vinos \emph{Botija}, el churrero \emph{Paja
Larga}, el tipógrafo Vicente Morata, Antonio Merino, profesor de
esgrima, y otros desaforados patriotas cuyos nombres no recuerdo.
Llevome Ginés a una mesa situada en lo más obscuro del establecimiento.
Formé ruedo con dos o tres de aquellos puntos, y un aprendiz de medidor
nos sirvió de lo añejo. Pedí al tabernero noticias de su hermana
Celestina, y me dijo que se hallaba en el piso alto y que le mandaría un
recadito para que bajase a verme.

Caía la tarde. Las luces de gas encandilaban mis ojos. Yo bebía sin
darme cuenta de las copas que a mis labios llevaba\ldots{} Sobre mi alma
iba cayendo un velo de tristeza desgarrada, por cuyos intersticios veía
las caras de los hombrachos que rodeaban la mesa, y oía jirones de una
charla política tocante a la \emph{venida de los higos chumbos}, o como
dijo \emph{Paja Larga}, del \emph{elemento alfonsino}\ldots{} En medio
de aquellas sensaciones caóticas vi aparecer a Celestina, que se sentó a
mi lado. En sus facciones angulosas, huesudas y secas, nariz de tajante
caballete, barba muy saliente con cuatro pelos en guerrilla, creí ver la
caricatura de un rostro aristocrático. Por la manera de liarse el
pañuelo a la cabeza, su parecido con el Dante resultaba perfecto.
Saludome con arrumacos y carantoñas, echándome su brazo por los hombros.

Pasado un lapso de tiempo que no sé precisar, Celestina me convidó a
comer; accedí; desaparecieron los bebedores; sentáronse a la mesa dos
muchachas graciosas y joviales, la una más linda que la otra; sirvieron
tortilla con jamón, tajadas de bacalao en el condimento que llaman
\emph{soldados de Pavía}, conejo en salsa y bartolillos; todo ello
remojado en abundancia con peleón, cariñena, moscatel y caña\ldots{}
Entre un tumulto de risotadas que repercutían dolorosamente en mi
cerebro, se nublaron mis ojos, me congestioné, perdí el conocimiento.

Mis sagaces lectores suplirán aquí la mutación de teatro que yo no puedo
describir porque no me hice cargo de ella. Cuando empecé a recobrar el
sentido me vi en la calle, ¡ay Dios mío!, llevado en vilo por cuatro
personas, dos de las cuales me parecieron mujeres. Mis conductores no
podían tenerse de risa y hacían chistes a costa mía, burlándose de mi
lastimoso estado. Quise hablar y no pude\ldots{} Caballero lector,
prepárate para otra mutación. Sumergido nuevamente en profundo sopor, no
me di cuenta de nada hasta que recobré súbitamente mi lucidez,
encontrándome en una pobre estancia, tumbado en mísero camastro\ldots{}
En pie, junto a mí, vi dos mujeres: la una era \emph{el Dante}, la otra,
la más linda muchacha de las que comieron conmigo en la taberna.

Transcurridos los primeros instantes de estupefacción hablé de esta
manera: «Pero Celestina, ¿qué es esto, qué me ha pasado?

---No es nada, señor de Liviano---me contestó la figura
dantesca.---Comió usted con gana y empinó más de la cuenta; de aquí que
se le fuera el santo al cielo\ldots{} Se nos quedó usted como difunto y
nos dio la gran desazón. Para ver de resucitarle y que recobrara su tino
le trajimos a esta casa, que no es la mía, sino la de esta joven, mi
amiguita, que aquí vive con su tía Simona. La vivienda no es de lujo,
como ve. Pero sí bastante apañada para su comodidad. Aquí puede usted
estar todo el tiempo que quiera, hasta que su caletre y sus nervios
entren en caja.»

Mostré en cortas palabras mi gratitud, dirigiéndome a la mocita gentil,
a quien di, no sé por qué desvarío dantesco, el nombre de
\emph{Beatrice}. «No me llamo Beatriz sino Casiana, para servir a usted
caballero don Tito---me dijo la graciosa muchacha.---En mi casa está
usted seguro y tranquilo. Nadie le molestará.» Como yo tratase de
indagar el lugar donde me encontraba, Celestina lo describió de esta
manera: «Estamos a la vuelta de la Escalerilla, frente a los Mostenses,
en el local donde radicó (vamos al decir) la redacción de \emph{El
Combate, aquel papel} donde escribió Paúl y Angulo, de quien se dijo que
tuvo que ver en la muerte de Prim. ¡Ay qué gracia, don Tito: está visto
que donde quiera que usted va, allí encuentra la Historia!» Con esta
frase y otras igualmente donosas se despidió la Tirado, diciendo que era
ya más de la una de la noche. Cuando la vi retirarse, después de
encarecer a Casiana que me cuidara con la mayor solicitud, creí que
salía para dar su acostumbrado paseo por el Infierno y Purgatorio de la
\emph{Divina Comedia}.

Solo ya con mi linda guardiana y aposentadora, esta se apresuró a
meterme en la cama. Hízome levantar; arregló el lecho con sábanas
limpias y buenas mantas; me quitó las botas; me ayudó a desnudarme con
todo recato y honestidad; me acostó, arropándome cuidadosamente; puso la
luz en lugar donde no me molestara, y sentose a mi lado. Tras de algunas
palabras mías de agradecimiento, contestadas por ella de una manera
discreta, caí en sueño profundísimo\ldots{} Desperté muy avanzado ya el
día, sintiendo en mi cabeza y en todo mi ser los efectos de la
reparación orgánica. Mi cerebro recobraba su lucidez. Yo era yo; me
reconocí como el Tito despabilado y clarividente de mis mejores días.
Llegose a mí Casianita, risueña y amable, trayéndome una taza de café
con leche. Bendiciendo su solicitud, me incorporé para tomar mi
desayuno. Apenas puse la taza vacía en las manos de la mozuela, esta se
sentó al borde de mi lecho, y con grácil llaneza y sinceridad, me
enjaretó este discursillo interesante:

«Ya está usted en mi poder, caballero don Tito, y lo primero que oirá de
mi boca es que ya no le suelto. Celestina me dijo anoche: Ahí te lo
dejo, Casiana; asegúralo bien, y haz cuenta de que con ese hombre
chiquito, te ha venido Dios a ver. El buen apaño que buscabas, ya lo
tienes. No es un cualquiera el señor que te ha caído del cielo, y aunque
le ves mal trajeado y alternando con gente de taberna, es como si
dijéramos un grande hombre, con \emph{muchisma} influencia y
\emph{muchismo} poderío. Yo no valgo nada; pero soy buena, aunque me
esté mal el decirlo, sé gobernar una casa y hacer la felicidad de un
caballero de circunstancias que no pique muy alto en sus pretensiones.
En mí tendrá usted una criada para todo y una mujer fiel que le
proporcione paz, alegría y cariño.»

Corté el discursejo pidiéndome antecedentes de su persona y familia.
¿Cuál era su estado, cuál su condición presente? Premiosa, suspirando a
ratos y haciendo lindos pucheritos, me dio a conocer los rasgos
culminantes de su breve historia. La señora con quien vivía era su tía.
De su madre, ausente, poco bueno tenía que decir ¡ay!, pues ella fue
quien la llevó a la desgracia. Con emoción y vergüenza me suplicó que no
la obligase a dar más pormenores de su deshonor y de la maldad de su
madre. «En fin, don Tito---añadió resumiendo en precipitadas razones la
confesión de sus desventuras;---ya sabe usted quién soy. La pobre
Casiana se acoge al buen corazón de usted. Ampáreme, señor, téngame
consigo para que mi vida sea menos aperreada y menos afrentosa.»

Confieso que la chica empezó a interesarme y que en mí sentía, con la
viva compasión, albores o remusguillos de un afecto incipiente. La
muchacha prosiguió: «Puede usted hacer mucho por mí, señor don Tito. Y
si quiere hacerlo con reserva, mejor. Con reserva debe ser, porque usted
es persona muy alta. Me lo ha dicho Celestina y todos los que estaban en
la taberna de Ginés Tirado. Usted vino anoche a la tasca\ldots{} ¡ya lo
sé, ya lo sé yo!\ldots{} disfrazado de pobre con una capa vieja, un
traje de papel secante y un sombrero que parece un acordeón. Esos
disfraces se los pone usted para vigilar a los que conspiran contra el
Gobierno y descubrirles todos sus trampantojos. Pero a mí no me la da,
que yo le he visto en la calle vestido muy majo, con botitas de charol,
gabán de pieles y un chisterómetro reluciente que da la hora\ldots{}

»Usted se sonríe y me mira con ojos cariñosos---continuó tras una breve
pausa.---Ya veo que me amparará. Ya no lo dudo\ldots{} Y lo primero que
le pido, don Tito de mi alma, no es que me dé de comer, no es que me
vista decentita; lo primero que le pido es que me enseñe a leer y
escribir o que me ponga un maestro que me dé lección\ldots{} porque soy
una burra\ldots{} no entiendo una letra\ldots{} no sé escribir una
palabra\ldots{} Y el ser una burra, créalo como Dios es mi padre, me
mortifica tanto, no, me mortifica más que el no ser mujer honrada.
¡Ay\ldots{} cuando yo le cuente cómo ha sido la infancia de esta
pobrecita Casiana, se espantará usted!\ldots{} De los cinco a los diez
años anduve por las calles, descalza, con un ciego que tocaba la
bandurria. Largo tiempo pasé durmiendo en un banco sin más abrigo que
unos trapajos indecentes. El abandono en que me tenía mi madre no se
cuenta en un año. Me alquilaba para pedir limosna con mendigos
asquerosos y borrachines.»

\hypertarget{iii}{%
\chapter{III}\label{iii}}

Las ingenuas declaraciones de Casianilla, infeliz pájara vagabunda y
analfabeta, me interesaban más a cada instante, y su afán de aprender a
leer y escribir despertó en mí los más puros sentimientos de tierna
simpatía. Cuatro días permanecí en aquella casa bien alimentado, bien
servido, \emph{como fuera Lanzarote---cuando de Bretaña vino}.
Suavemente, por naturales atracciones y accidentes circunstanciales,
fuimos entrando la mozuela y yo en franca intimidad. La tía de Casiana,
Simona, era una mujer tan avezada al trabajo casero que ni un momento
daba paz a sus manos bastas, así en la cocina como en el barrido y
fregoteo de las humildes habitaciones. Cuando ya me encontraba
restablecido y en disposición de salir a la calle, Casiana, infatigable
y hacendosa, me arregló la capa disimulando con hábil aguja los sietes
que la deslucían, y adecentando a fuerza de bencina y cepillo mi
desdichada ropa. En medio de estas faenas solía presentársenos de
improvisto \emph{El Dante}, para darnos buenos consejos y señalarme con
profética autoridad la conveniencia de recobrar mi alta posición.

Por fin, la vaciedad de mis bolsillos que en aquella ocasión pedía
inmediato remedio, me lanzó a las calles, llevando conmigo a la que ya
conceptuaba como inseparable compañera. Réstame decir que en el período
de mi corto encierro acabaron los agitados días del año 74 y empezaron
los de su sucesor. Estábamos, pues, en los infantiles comienzos del 75,
entre la Circuncisión y los Santos Reyes, cuando Casiana y este humilde
cronista atravesábamos medio Madrid alegremente y cogiditos del brazo,
para dirigirnos a la portería de la Academia de la Historia, donde
esperaba encontrar, con noticias frescas de la Madre, los dineritos que
tanta falta nos hacían\ldots{} No me engañó el corazón. Puso la portera
en mis manos el paquete, diciéndome: «Feliz año, don Tito,» y salimos mi
amiga y yo, no diré que brincando de alegría, pero poco menos. Propuse a
Casiana que bajáramos al Prado para descansar y leer detenidamente la
carta de mi Madre. Así lo hicimos, y sentaditos en el escaño de la verja
del Botánico, me consagré a leer, con el debido respeto y devoción, la
carta de \emph{Mariclío} que así decía.

«Para que te vayas enterando, mi buen Tito, te mando estos apuntes
producto de mi observación directa en los risueños lugares de Levante.
Días ha encontrábame yo en las ruinas del teatro romano de Murviedro,
rememorando la espantosa ocasión de la caída de la heroica Sagunto en
poder del furioso Aníbal, cuando mi fiel criada \emph{Efémera} me trajo
el aviso de que en el caserío llamado de \emph{les Alquerietes} ocurría
un suceso, que no por previsto era menos interesante para mí. Volando
fuimos allá \emph{Efémera} y yo, y vimos numerosas tropas del Ejército
del Centro formadas en cuadro. Frente a ellas, el General Martínez
Campos, rodeado de brillante Estado Mayor, pronunciaba con ronca
elocuencia un militar discurso, comenzado con negra pintura de los males
de la Patria y concluido con proponer la panacea de su invención, la
cual era proclamar Rey de las Españas al joven Príncipe Don Alfonso.

»Yo vi a Martínez Campos el 27 de Diciembre por la noche, cuando llegó a
Sagunto en una tartana, acompañado del Teniente Domínguez. Estábamos él
y yo en la misma posada. Ya sabes que aprecio mucho a este General,
reconociendo en él cualidades de bravo militar y honrado caballero. Me
ha dolido verle metido en este enredo. Si la Restauración era un hecho
inevitable, impuesto por fatalismo histórico, los españoles debían
traerla por los caminos políticos antes que por los atajos militares.
Cánovas opinaba como yo, y al fin ha tenido que doblar su orgullosa
cerviz ante la precipitada acción de las espadas impacientes.

»Al tanto estaba yo de lo que tramó don Arsenio en el Ejército del
Centro, antes de irse a Madrid; de la misión que llevó a la Corte el
Comandante Aznar, de las conferencias que tuvo con Martínez Campos, y de
la clave convenida para que este viniese a dirigir y encauzar el
movimiento. La clave telegráfica, que pasó por mis manos, decía:
\emph{naranjas en condiciones}. Las primeras tropas que se unieron al
General para dar \emph{el grito} fueron las que mandaba el Teniente
Coronel Aragón, Jefe de la reserva de Madrid. Las demás no tardaron en
agregarse.

»Con mis propios ojos vi al General Martínez Campos, la noche que llegó
a Sagunto, escribir tres cartas que mandó a su destino con el Comandante
Salcedo. El sobre de una de ellas decía simplemente: \emph{Brigada
Laguardia.---Villarreal}. La segunda carta iba dirigida a don Pablo
Corral, Teniente Coronel de la misma Brigada. Y la tercera al Coronel
Borrero, Jefe del Regimiento de la \emph{Constitución}, que se hallaba
en Castellón de la Plana. Tras el emisario mandé a \emph{Efémera}, hija
del Tiempo, educada por Eolo, y yo me fui a dar una vuelta por Valencia,
para ver lo que allí pasaba. Cuando me reuní con \emph{Efémera} dejé a
esta al cuidado de lo que ocurriera en Villarreal y volé a Castellón,
donde observados directamente los actos y palabras del General Jovellar
que mandaba uno de los Cuerpos de Ejército del Centro, comprendí que la
Restauración era ya un hecho, y que por la vulgaridad de aquellos
sucesos, la Historia no debía precisar pormenores que carecían de todo
interés.

»Apunta, hijo, apunta en media página el resumen de las directas
observaciones de tu Madre. Ayudaron a la fácil traída de don Alfonso los
hermanos don Luis y don Antonio Dabán, Borrero y don José Bonanza, el
Jefe de Estado Mayor Brigadier Azcárraga; el Teniente Coronel Aragón,
los Comandantes Aznar y Salcedo, y casi todos los jefes y oficiales de
la Brigada Laguardia y del Cuerpo de Ejército mandado por Jovellar.
\emph{Efémera} y yo nos reíamos de la llaneza ramplona con que en España
se desarrollan y se redondean estas revoluciones pacíficas que llaman
pronunciamientos. El de Sagunto fue una comedia, \emph{El juego de las
cuatro esquinas}, representada en un escenario de algarrobos.

»Y por último, no olvides que entramos en una época de buenas maneras,
distinción y elegancia. Ya \emph{se llevan} los chalecos de fantasía y
los botines blancos.

»Adiós, muñeco mío. Ten juicio. Si no te escribo ni me ves, sabrás de mí
por la veloz \emph{Efémera}.»

Afirmándome en la resolución que tomé apenas recibidos los dineros y la
cartita, cogí por un brazo a Casiana y nos fuimos a mi mansión
hospederil. Grande fue la sorpresa del matrimonio Ido al verme entrar
con la bonita res que había cazado en mi ausencia de cinco días.
Acostumbrados a mis extravagancias y a la presteza genial con que yo
emprendía y realizaba las amorosas conquistas, mis patrones suprimieron
toda indiscreta pregunta. Adelanteme yo a satisfacer su curiosidad,
diciéndoles en tono que excluía todo comentario: «Esta señorita que
traigo de la mano vivirá conmigo en esta misma habitación o en otra muy
próxima. Prepare usted, Nicanora, una buena cama y los muebles más
decorosos que haya en la casa.» Y tirando del paquete que acababa de
recibir saqué el fajo de billetitos y puse dos en manos de mi patrona,
diciéndole: «Ordeno y mando que esta señorita y yo comeremos en nuestras
habitaciones, apartados de la turbamulta de estudiantillos alborotadores
y zaragateros. Cobren mis atrasos si los hubiere. Abriremos la mano en
el dispendio, pues como ustedes saben, vienen tiempos en que las
personas han de ser estimadas según su prestancia y el tono que se den
al presentarse en el escenario social.»

Cuando esto decía, miré a la percha, abrí el armario de luna, y vi con
asombro y júbilo que toda mi ropa buena había vuelto a los colgaderos
donde estuvo antes de su inexplicable desaparición. Antes que yo pidiera
explicaciones de aquel prodigio, el filósofo don José pronunció estas
solemnes palabras: «Excelentísimo Señor: los mismos ordenanzas galonados
que se llevaron la ropa, la trajeron a los dos días, intacta y sin el
menor deterioro.

---Vamos, lo que yo pensé: un bromazo de los pícaros escolares.

---Dispénseme, Ilustrísimo Señor; no está en lo cierto. La broma, según
he podido yo entender por mis cálculos políticos, fue de don Antonio
Cánovas, que aquel día tenía gran interés en que Vuecencia no se pusiera
al habla con don Práxedes Mateo Sagasta, ni con el Capitán General de
Madrid, señor Primo de Rivera.

---Bien podrá ser---dije yo con fingida seriedad.---Me maravilla, señor
Ido, su descomunal \emph{pesquis} y la justeza de sus puntos de vista,
así en lo privado como en lo público. Y ahora, querido, ordene usted que
nos sirvan a la señora y a mí un suculento almuerzo.»

Mientras almorzábamos, por cierto con soberano apetito, solté el chorro
de mi locuacidad sobre el buen Ido del Sagrario, que ceremoniosamente
nos servía. «Don José de mi alma---le dije.---Voy a encomendar a usted
una misión, en cierto modo sagrada, que no dudo desempeñará
cumplidamente por ser usted tan cuidadoso patrón como ilustrado
pedagogo. Esta joven, cuyo nombre es \emph{Casiana de Vargas Machuca} y
procede de una de las más ilustres familias españolas, ha venido a ser
mi compañera por una serie de lamentables desdichas que no es oportuno
referir. En edad crítica para las niñas, entre los trece y catorce años,
padeció una terrible enfermedad de cerebro. ¡Ay don José! Casi
milagrosamente escapó con vida de aquella hondísima crisis. Pero perdió
en absoluto la memoria de cuanto aprendiera en la niñez. Aquí la tiene
usted modosa, dulce, cortita de genio, dotada de toda la perspicacia
compatible con su inocencia. Más le falta\ldots{} le falta\ldots{} En
fin, ilustre amigo: Casiana no sabe leer ni escribir.»

Asombrado quedó mi patrón, y brindose como viejo maestro de escuela a
reparar en corto tiempo la \emph{deficiencia educativa de la señorita de
Vargas Machuca}. «Esta misma tarde---le dije yo---proveeré a usted de
fondos para que compre una \emph{Cartilla}, el \emph{Catón}, el
\emph{Fleury}, el \emph{Juanito}, papel de escribir, pizarra, y todo lo
que sea menester para la primera enseñanza. La enfermedad quitó a la
niña la memoria, pero le dejó su talento natural, y con tan buen maestro
como usted recobrara en un periquete la sabiduría que perdiera.»

Muy orondo y como las propias mieles se puso el bueno de Ido. No veía ya
las santas horas de dar comienzo a su faena educativa. Cuando nos
quedamos solos, Casiana, soltando la risa, me dijo: «¡Ay, Tito, qué
graciosos embustes le has metido! ¡Vaya con decirle que me llamo
\emph{Vargas Machaca}, cuando mi apellido es Conejo!

---Y mañana le diré que por la línea materna eres Imón de la Mota, y que
te corresponde el título de Baronesa de Canillas de Aceituno, con sus
miajas de grandeza de España.»

---En el mismo tono de amable socarronería seguimos departiendo largo
rato, y a media tarde, adecentándome un poco sin llegar a ponerme los
atavíos señoriles, nos fuimos a la calle. Deseaba yo ponerme al habla
con algunos amigos para enterarme de todo lo actuado políticamente en
los días de mi eclipse. Estuvimos en el café de Venecia y en el de San
Sebastián, donde sólo encontré a dos amigos periodistas, Fabriciano
López y Mateo Carranza, que habían hecho campañas furibundas en la
prensa avanzada durante los pasados días, y a la sazón dejaban traslucir
su movible criterio con estas o parecidas manifestaciones: «Nosotros, a
la chita callando, hemos infiltrado el alfonsismo en toda España.»

Imitando la flexibilidad de sus conciencias, les presenté a Casiana como
una prima mía de grandes conocimientos pedagógicos, que había llegado de
Cuba con la noble aspiración de ocupar una plaza en la Escuela Normal de
Maestras. Subiéndose a la parra y poniéndose muy hueco, ofreció Carranza
su influencia para colmar los deseos de la ilustrada joven, pues era muy
amigo del nuevo Director de Instrucción Pública y esperaba tener un
puesto preeminente en las oficinas del Ramo.

Por Fabriciano y Mateo adquirí frescas noticias del raudo cambio de
situación que mi Madre llamaba \emph{gozne} o doblez histórico. Apenas
comprendieron Sagasta y sus Ministros que al pronunciamiento de Sagunto
se adhería con blanda unanimidad toda la fuerza militar del Centro y del
Norte, se apresuraron a retirarse por el foro cantando bajito. Se hizo
la pamema de detener en el Gobierno civil al imponderable don Antonio
Cánovas, el cual pasó algunas horas en el despacho del Gobernador señor
Moreno Benítez, obsequiado por este, y recibiendo plácemes, mimos y
reverencias de innumerables \emph{hombres públicos}, arrimados
temporalmente a un sol que alumbraba antes de nacer. Don Emilio, amigo
de Cánovas, le envió al Gobierno Civil una cama para que descansase
cómodamente en su breve cautiverio. Por tal fineza, el ilustre malagueño
favoreció después a su amigo con rápidos adelantos en la carrera de la
Magistratura.

Al día siguiente, si no estoy equivocado, después de un fugaz e ilusorio
poder omnímodo del Capitán General de Madrid, Primo de Rivera, se
constituyó la indispensable Junta con figuras culminantes del
alfonsismo. Poco después, \emph{maese} Cánovas, como quien cambia los
títeres de un retablo, compuso en esta forma el llamado Ministerio
Regencia: \emph{Presidencia}: Cánovas.---\emph{Estado}: don Alejandro
Castro.---\emph{Gracia y Justicia}: don Francisco
Cárdenas.---\emph{Hacienda}: Salaverría. ---\emph{Guerra}:
Jovellar.---\emph{Marina}: Molins.---\emph{Gobernación}: Romero
Robledo.---\emph{Fomento}: Orovio.---\emph{Ultramar}: Ayala.

Prosigo ahora mi cuento mezclando sabrosamente lo personal con lo
histórico. Sabed, lectores míos, que Casianita dio comienzo a sus
lecciones con ardiente entusiasmo, y que el docto profesor, contentísimo
de las aptitudes y aplicación de su discípula, aseguraba que pronto
leería de corrido y que sus adelantos habrían de ser prodigiosos. Como
la señorita de \emph{Vargas Machuca} deletreaba mañana y tarde, y
gustaba de emplear el resto del día ayudando a Nicanora en la cocina y
en los trajines de la casa, yo salía solo a recorrer el mundo.

Una tarde, Felipe Ducazcal me llevó al Círculo Popular Alfonsino,
hervidero de pretendientes al sin fin de plazas que brindaba la
Restauración a los españoles necesitados. Allí me encontré a Carranza,
que ya se había colado en la Dirección de Instrucción Pública; a Modesto
Alberique, que andaba tras una secretaría de Gobierno de provincias; a
don Francisco Bringas, que, bien asegurado en Fomento por la protección
de Orovio, brindaba sus influencias a la gentuza advenediza; a \emph{don
Florestán de Calabria}, que del empleo escribientil que tenía en el
Círculo, quería saltar a una plaza de la Calcografía Nacional.

Entre los que vendían protección me topé con Telesforo del Portillo
\emph{(Sebo)}, colocado ya en un buen puesto del Gobierno Civil, a las
órdenes del secretario don Federico Villalba. Serafín de San José había
sido llevado al Ayuntamiento por el nuevo Alcalde, Conde de Toreno. Mi
amigo Fabriciano López, a quien yo había conocido largos años en la
intimidad de Llano y Persi, Felipe Picatoste y el Marqués de Montemar,
progresistas de abolengo, tenía ya labrado un nido en la Secretaría de
la Presidencia, donde estaban colocados Carlos Frontaura, Lafuente,
Fernández Bremón y el joven Esteban Collantes. También encontré allí al
simpático Vicente Alconero, que no iba ciertamente al olor de los
destinos, sino por pasar el rato. De la conversación que con él sostuve,
saqué la sospecha de que tenía puestos los puntos al acta de diputado
por el distrito de La Guardia.

Se me olvidaba consignar\ldots{} y no extrañéis el desorden de mi
cabeza, pues ya sabe mi parroquia que yo endilgo mis cuentos brincando
locamente de idea en idea\ldots{} olvidé referir, digo, que el día 2 de
Enero del 75 salieron de Madrid los individuos designados para traer al
Rey Alfonso de las lejanas tierras donde se encontraba. Componían dicha
Comisión el Marqués de Molins, los Condes de Valmaseda y Heredia
Spínola, y don Ignacio Escobar, director de \emph{La Época}, todos
hombres muy serios y de encopetada representación para el caso. Una de
las primeras medidas del Ministerio Regencia fue suspender a rajatabla
los siguientes periódicos: \emph{El Imparcial, El Pueblo, El Correo de
Madrid, La Bandera Española, El Cencerro, La Prensa, El Gobierno, La
Iberia, La Igualdad, El Orden, La Civilización} y \emph{La Discusión}.

Habituado a la lectura matinal de mis periódicos favoritos, el vacío de
prensa me causaba tristeza. A Casiana le tenía sin cuidado que no
entraran \emph{papeles} en casa, porque \emph{le estorbaba lo negro}, y
además, le sabía mal que pasara yo largas horas agarrado al
\emph{Imparcial} o al \emph{Pueblo}. Cada día se metía más en las
honduras del \emph{Catón}, y sus ocios los consagraba, con no menor
celo, al trabajo físico. Una mañana me la encontré en la parte interior
de la casa, fregando los suelos, de rodillas, con los brazos al aire y
las manos moradas de tanto darle a la bayeta. Como rasgo característico
de su feliz adaptación a la nueva vida, contaré que los estudiantillos
de San Carlos solían acosar con bromas de mal gusto a mi hacendosa
compañera; pero esta les contestaba en breves y agrias razones, y si
ellos insistían, refrenaba sus audacias a bofetada limpia.

A menudo era visitada Casiana por su tía Simona, y cuando la encontraba
en el trajín de sus lecciones, permanecía la pobre mujer pasmada y muda
cual si presenciase un acto milagroso. Analfabeta era también Simona, de
las empedernidas e incapaces de enmienda, por causa de su edad. Se
consolaba mentalmente admirando el fervor de la muchacha, y la paciencia
del escuálido maestro que le iba metiendo en la cabeza tanta sabiduría.
Terminada la lección, tía y sobrina salían hablar de sus conocimientos y
relaciones.

Refiriéndose a Celestina Tirado, aseguró un día Simona haber descubierto
que la hermana del tabernero Ginés tenía trato con los demonios; vivía
en sociedad con una tal \emph{Grosella}, italiana o cosa así, y ganaban
la mar de dinero adivinando lo que no se ve y curando con bebedizos a
los desamorados. A lo mejor se iban por los aires en busca del
\emph{Gran Cabrío} para celebrar las misas demoniacas. Desde que
Celestina andaba en estos trotes se le había puesto la cara más huesuda
y le habían salido en la barbilla, en la nariz y en las orejas unos
pelos largos y feos.

Una tarde, solos Casiana y yo en nuestra habitación, platicábamos sobre
lo mismo. Mostrábase mi amiga incrédula de las cosas sobrenaturales que
su tía le contaba. Sostenía que eso de las almas del otro mundo que
vienen al nuestro no tiene realidad más que en los cuentos de viejas.
Díjele yo que existen verdades y fenómenos fuera de la acción de
nuestros sentidos; que no debemos rechazar en absoluto en contacto de
nuestro mundo con otros lejanos o próximos, aunque invisibles\ldots{} y
estando en estas amenas divagaciones vi que entraba en la estancia una
imagen, una persona, una mujer, sin que precediera el tintín de la
campanilla, ni anuncio ni aviso alguno. Di algunos pasos hacia la
extraña visitante, y antes que yo le preguntara si en mi busca venía, oí
su voz melodiosa que así me dijo: «¿No me conoce, señor don Tito? Soy
\emph{Efémera}, la mensajera de su divina Madre.»

\hypertarget{iv}{%
\chapter{IV}\label{iv}}

La recadista de mi Madre era una figura estatuaria, vestida con luengo
túnico negro algo transparente\ldots{} El estupor me cortó la palabra.
Pero con instintivo movimiento traté de reconocer si era real o
quimérico el bulto de aquella singular aparición. Al tocar con mi mano
su hombro sentí la dureza y el frío del mármol, y vino a mi memoria lo
que me aconteció en la fonda de Tafalla una mañana, cuando llamó a mi
puerta con dedos de piedra una figura, que si no era la misma que
delante tenía, se le asemejaba mucho. «Ya sé quién es usted---dije
balbuciente.---En Tafalla\ldots{} ¿se acuerda?

---Sí; me acuerdo---respondió ella con voz dulce y queda,
sonriendo.---Yo fuí la que llevó a usted un recado de mi santa Señora,
en Tafalla, sí\ldots{} cuando hicieron honras fúnebres al General Concha
antes de traer acá su cadáver\ldots{} y ahora vengo otra vez de parte de
\emph{Mariclío}.

---¿Me trae usted carta?

---No, don Tito. El mensaje de hoy es verbal y se lo comunicaré a usted
en pocas palabras. La que todo lo ve y lo sabe, ha dispuesto que su fiel
muñeco\ldots{} perdone si le doy este nombre cariñoso\ldots{} se prepare
para ir a visitar a don Antonio Cánovas.

---Pero yo no soy amigo de ese señor. No le he tratado nunca.

---¿Y qué importa? Yo tampoco le trataba, y hace días hablé con él como
hablo ahora con usted\ldots{} Ya sabe lo que dice don Antonio: que
\emph{ha venido a continuar la Historia de España}.

---Pues iré, iré. Pero no sé qué pretexto buscar para introducirme, para
pedir audiencia\ldots{}

---No se inquiete por eso. Es fácil, casi seguro, que el propio
Presidente le abra a usted camino llamándole a su despacho.»

Diciendo esto saludome con ligero movimiento de cabeza y dio media
vuelta para retirarse. Salí yo tras ella pasillo adelante. En el
recibimiento la despedí con expresiones inefables de gratitud y ternura:
«Adiós, \emph{Efémera}. Gracias, \emph{Efémera}\ldots{} ¡Bendita sea mi
Madre que te ha mandado a mí, bendita tú que me traes un destello de su
mente divina!\ldots» No conservo memoria de haber abierto la puerta. La
visión salió no sé cómo ni por dónde\ldots{} Tampoco sentí el sonido de
sus pies de mármol bajando la escalera\ldots{}

Al volver a mi estancia, vi que Casiana, reclinando su cabeza en el
respaldo del sofá, estaba como adormecida. Al llegar yo a su lado se
despabiló y me dijo: «Tito, tú hablabas aquí con alguien. ¿Quién era?

---No te asustes. Era una señora, una tal \emph{Efémera}, que vino a
traerme un recado.

---¿Cómo dices que se llama? ¿Efe\ldots?

---\emph{Efémera}, nombre que quiere decir la historia de cada día, el
suceso diario, algo así como el periódico que nos cuenta el hecho de
actualidad.

---¡Ah\ldots{} ya! ¿De modo que esa \emph{doña Femera} viene a ser un
periódico vivo que no dice las cosas escritas sino habladas?

---Justo, así es. ¡Oh, Casianilla, tú tienes mucho talento y todo lo
comprendes!»

Desde aquella tarde no se apartó de mi mente la idea de que don Antonio
me llamaría para echar un parrafito conmigo. ¿Era verdad el anuncio que
me trajo la vagarosa \emph{Efémera}, o era un artilugio de los espíritus
familiares que a ratos venían a divertirse con el pobre Tito?

Mientras llegaba la ocasión de salir de dudas, Casiana y yo matábamos el
tiempo acudiendo a presenciar todo suceso pintoresco que el flamante
reinado nos ofrecía. Un luminoso día de Enero se puso Casiana el más
decente de sus vestiditos, yo la pañosa con embozos de terciopelo
carmesí que adquirí con los dineros de la Madre, y nos fuimos al Prado a
presenciar la entrada del nuevo Monarca.

Había yo visto el solemne paso procesional de adalides revolucionarios
victoriosos, o de Reyes y Príncipes que venían a traernos la felicidad,
y calculaba que todas estas entradas aparatosas eran lo mismo
\emph{mutatis mutandis}: gran gentío, apreturas, aplausos, un punto más
o un punto menos en el diapasón de los vítores, la chiquillería subida a
los árboles, y los balcones atestados de señoras que sacudían sus
pañuelos como espantando moscas. En algunos casos hubo también soltadura
de palomitas que volaban despavoridas, huyendo del popular entusiasmo.

Una procesión de carácter bien distinto, tétrica y desesperante, y que
marchaba en sentido inverso, dejó en mi alma impresión hondísima: la
salida del cortejo fúnebre de Prim para el santuario de Atocha. Señaló
una coincidencia que me resultó irónica: en el mismo sitio donde vi la
entrada de don Alfonso de Borbón había visto pasar el entierro del
grande hombre de la Revolución de Septiembre, que dijo aquello de
\emph{jamás, jamás, jamás}.

Entró el Rey a caballo. Vestía traje militar de campaña, y ros en mano
saludaba a la multitud. Su semblante juvenil, su sonrisa graciosa y su
aire modesto le captaron la simpatía del público. En general, a los
hombres les pareció bien; a las mujeres agradó mucho. Al subir don
Alfonso por la calle de Alcalá, el palmoteo y los vivas arreciaron, y en
los balcones aleteaban los pañuelos de un modo formidable. Tras el Rey
marchaba un Estado Mayor brillantísimo. Lo que más gustó a Casiana,
según me dijo, fue el juego de colorines de las bandas con que se
adornaban los señores cabalgantes a la zaga del Soberano barbilampiño.
Igualmente me preguntó si aquellos caballeros tan majos y revejidos eran
Generales, y si el Rey jovencito les mandaba a todos. Después contempló
embelesada el paso de los coches en que iban los Ministros y el alto
personal palatino, cargados de plumachos, galones y cruces, y quiso
saber si aquellos pajarracos eran también marimandones; a lo que yo
contesté: «Todos los que ves vestidos de máscara mandan; pero más que
ellos mandan sus mujeres y otras tales, esas que están encaramadas en
los balcones, y algunas que andan por aquí.»

En esto sentí que una mano enguantada me tiraba de la oreja. Volvime y
me encontré frente a \emph{Leona la Brava}, que iba con una de sus
amigas del Teatro Real, Carolina Pastrana. Tras un rápido saludo,
Leonarda me dijo atropelladamente: «Que tienes que ir a ver a don
Antonio Cánovas; pero pronto, pronto. Hoy te mandé una cartita con el de
\emph{Calabria}. Si no la has recibido, en tu casa la encontrarás. En
ella te digo que si don Antonio no te llama, no faltará un amigo que te
lleve a su presencia.»

Antes que yo pudiera contestar, \emph{Leona} se fijó en Casiana,
requiriendo trato con expresiones francas, afectuosas: «¡Ah!, esta es la
muchachita que has pescado en el río revuelto de tu vida. Es linda de
veras. Parece buena chica y tú estás muy contento con ella\ldots{} Todo
lo sabemos Tito, y no tienes que guardar misterio con nosotras.»

Intervino entonces la Pastrana, diciendo con bondadoso acento: «¡Oh! Nos
han dicho que es una gran profesora, que es punto fuerte en el arte de
enseñar.

---¿Sabe francés?---interrogó \emph{La Brava} interesándose por mi
amiga.

Con monosílabos balbucientes intentó Casiana formular una contestación,
y yo acudí en su auxilio, respondiendo por ella: «Todo lo sabe. Pero es
tan tímida que no se explicará bien hasta que tome confianza.

---Quedamos en que visitarás al Jefe---saltó \emph{Leona}, presurosa por
seguir su camino.---Si el grande hombre te ofrece una posición, tú harás
un poquito de coqueteo y melindre, y acabarás por aceptar, quedando muy
satisfecho, \emph{ça va sans dire}.»

Con poco más de una parte y otra terminó el coloquio, siguiendo las dos
mujeres hacia la Cibeles. Ya los soldados que cubrían la carrera
formaban en columna de honor para el desfile. Las voces de mando, los
toques de clarín y corneta, daban al nuevo cuadro la brillante animación
ruidosa que tanto agradaba al pueblo de Madrid. Las masas de curiosos se
arremolinaban, buscando salida por una parte y otra. Nos corríamos hacia
la fuente de Neptuno queriendo ganar la Carrera de San Jerónimo, cuando
Casiana, atormentada por una idea, me habló de este modo: «Dime, Tito,
¿aquellas mujeres son damas o qué?

---Damas son, querida; pero de esas que llaman \emph{de las Camelias}.

---Pues, según me han dicho, la dama \emph{de las Camelias} era tísica,
y estas no están enfermas del pecho: chillaban como demonios.

---Los tísicos son ellos.

---Y dime otra cosa, Tito: los hombres de esas mujeres ¿son los que iban
antes en coche, con plumachos y requilorios dorados?

---Sí, hija mía. Uno de ellos llevaba casacón bordado con muchos ojos;
el otro, casaquín, llave de oro, calzón corto y media de seda.

---Y los que visten de esa manera ¿son Duques o Marqueses?

---En algunos casos, sí. En otros son Jefes Superiores de
Administración, Gentiles-hombres, o se les designa con diferentes motes
muy bonitos.

---Pues, según dice Ido, tú lucirás pronto si quieres todas esas
garambainas, y estarás muy guapo.

---No te digo que no. Cuando se pone el pie en el pórtico de este mundo
que hoy has visto, nadie sabe a dónde podrá llegar.

---Otra cosa, Tito---dijo Casiana rasgando su linda boca en franca
risa.---¿Llegará un día, no digo que mañana ni pasado, un día del tiempo
venidero, en que tú y yo seamos también Marqueses, Jefes de la Sagrada
Administración o personas gentiles de las llaves doradas?

---¡Ya lo creo que podrá ser! Muchos han pasado por aquí que subieron
del lodo a las cimas.

---Ahora vuelvo a mi tema: aquellas mujeres guapas que nos hablaron
antes ¿también mandan?

---¡Qué si mandan! Más que el Rey. Más que nadie. En muchas ocasiones
son ángeles tutelares que reparten la felicidad entre los ciudadanos.»

Mirome Casiana con espanto, abierta la boca, y yo me apresuré a
cerrársela con estas maduras reflexiones: «En la procesión que ha pasado
frente a nuestros ojos, multitud engalanada rebosando satisfacción y
alegría, has visto el mundo de los pudientes, de los administradores,
mayordomos y capataces de la cosa pública, mecanismo cuyas piezas mueven
las cosas privadas y todo el tejemaneje del vivir de cada uno. ¿No lo
has entendido, verdad? Pues te lo diré más a la pata la llana. Lo que
hemos visto es el familión político triunfante, en el cual todo es
nuevo, desde el Rey, cabeza del Estado, hasta las extremidades o
tentáculos en que figuran los últimos ministriles; es un hermoso y
lucido animal, que devora cuanto puede y da de comer a lo que llamamos
pueblo, nación o materia gobernable.

»Sabrás ahora, mujercita inexperta, que los españoles no se afanan por
crear riqueza, sino que se pasan la vida consumiendo la poca que tienen,
quitándosela unos a otros con trazas o ardides que no son siempre de
buena ley. Cuando sobreviene un terremoto político dando de sí una
situación nueva, totalmente nueva, arrancada de cuajo de las entrañas de
la patria, el pueblo mísero acude en tropel, con desaforado apetito, a
reclamar la nutrición a que tiene derecho. Y al oírme decir pueblo ¡oh
Casiana mía!, no entiendas que hablo de la muchedumbre jornalera de
chaqueta y alpargata, que esos, mal o bien, viven del trabajo de sus
manos. Me refiero a la clase que constituye el contingente más numeroso
y desdichado de la grey española; me refiero a los míseros de levita y
chistera, legión incontable que se extiende desde los bajos confines del
pueblo hasta los altos linderos de la aristocracia, caterva sin fin,
inquieta, menesterosa, que vive del meneo de plumas en oficinas y
covachuelas, o de modestas granjerías que apenas dan para un cocido.
Esta es la plaga, esta es la carcoma del país, necesitada y pedigüeña, a
la cual ¡oh ilustre compañera mía!, tenemos el honor de pertenecer.»

Cerró Casiana su linda boca en el curso de mi perorata y luego, con
grandes suspiros, expresó que iba entendiendo y lamentando la pintura
que yo le hacía de nuestra sociedad. Tomado un breve respiro, proseguí:
«En todo tiempo, y más aún cuando ocurren cambios de situación tan
radicales como el que estamos viendo, la caterva de menesterosos bien
vestidos, agobiada de necesidades por el decoro social de los señoritos
y los pujos de elegancia de las señoras y niñas, cae como voraz langosta
sobre el prepotente señorío engalanado con plumas, cintajos, espadines,
cruces y calvarios, porque esa casta privilegiada es la que tiene en sus
manos la grande olla donde todos han de comer. Aquí la industria es
raquítica, la agricultura pobre, y los negocios pingües sólo fructifican
en las alturas. La turba postulante se agarra a todas las aldabas, llama
a todas las puertas, tira de los faldones de los personajes
empingorotados, pide auxilio con discretos tirones a las mujeres
legítimas de los tales\ldots{} y a las que no son legítimas. Ya irás
comprendiendo, Casianilla, el manejo que se trae la inmensa tribu de
desheredados, y la misión benéfica que desempeñan, en algunos casos y a
hurtadillas, las dos mujeres guapas con quienes hemos hablado hace un
ratito.»

Terminé diciéndole, en forma que ella pudiera entenderlo, que España era
un país algo comunista. Por los canales contributivos venía todo el
caudal a la olla grande, de donde salía para repartirse en mezquinas
raciones entre el señorío paupérrimo de la flaca España. «He dado el
nombre de olla grande---añadí---a lo que en lenguaje político llamamos
Presupuesto.

---¡Virgen de la Paloma!---exclamó Casiana con risueña
espontaneidad.---Pues yo te digo ahora, Tito de mi alma, que seremos los
bobos de Coria si no metemos nuestra cuchara en ese bendito
\emph{porsupuesto}.»

Subíamos por Medinaceli y San Antonio del Prado, camino de nuestra casa,
cuando pasó ante mí la fantástica \emph{Efémera}, cual visión rápida que
fue a perderse entre los altos abetos que rodean la estatua de
Cervantes. Con ella iba otra mujer, vestida también de flotante y negro
túnico. ¿Era \emph{Graziella}? No puedo asegurarlo. Sólo diré, que en su
rauda fulguración de relámpago, las dos mágicas figuras lanzaron hacia
mí una mirada insinuante, cariñosa\ldots{} Y no hubo más.

El rigor cronológico, al cual inútilmente quiero acomodar la serie de
mis históricos relatos, me ordena referir que en la tercera semana de
Enero del 75 se me presentó Fabriciano López, quien como sabéis ya tenía
un puesto en las oficinas de la Presidencia. Según me indicó, estaba yo
en la lista de las personas que don Antonio Cánovas citaría para ser
recibidas en el despacho presidencial. Ignoraba la fecha en que me
tocaría la vez; y como al propio tiempo me dijera que en las covachuelas
de la calle de Alcalá tenían su abrigado albergue algunos funcionarios
de la clase de literatos y periodistas, todos amigos míos, allá me fui
con Fabriciano, movido del deseo de tantear el terreno en previsión de
lo que pudiera suceder.

En la hospedería burocrática de la Presidencia me encontré a don Carlos
Frontaura, ameno y regocijado escritor satírico, creador de \emph{El
Cascabel}, el periódico más divertido y chusco que hizo las delicias de
la burguesía matritense en aquellos lustros; a Campo Arana y Puente y
Brañas, autores de comedias y zarzuelas que tuvieron sus días de aura
popular; al excelente y hábil periodista Pepe Fernández Bremón, que
durante un cuarto de siglo mantuvo después su acreditada firma en
\emph{La Ilustración Española y Americana}.

Por mi primera visita entendí que en el asilo presidencial no eran
grandes los quehaceres de los buenos muchachos que allí tenían cómodo
acogimiento: unos leían periódicos, otros tertuliaban entre el humo de
los cigarrillos; iban y venían de una parte a otra, pasándose de mano en
mano papeles con trabajos vagamente iniciados. Todo indicaba la
plantación de un árbol burocrático que pronto daría flores y quizá algún
fruto.

Largo rato permanecí en aquella feliz Arcadia, oyendo el tañido de la
ociosa zampoña pastoril. Fabriciano y Fernández Bremón lleváronme al
despacho del Subsecretario, Saturnino Esteban Collantes, y a él me
presentaron. Era un joven discreto y afable, hijo del famoso político
del antiguo régimen don Agustín, nombrado a la sazón Ministro
plenipotenciario en Portugal. En la breve conversación que tuve con el
Subsecretario, adquirí la certidumbre de que mi nombre figuraba en la
lista de los presuntos visitantes de Cánovas. Pero el Presidente estaba
muy atareado en aquellos días\ldots{} Ya se me avisaría la fecha de la
entrevista.

Una larga semana tardó en llegar el aviso. En cuanto lo recibí me puse
la levita y las demás prendas \emph{de vestir}, me encasqueté la
\emph{bimba} y ¡hala!, a la Presidencia. Mediano rato me tuvo Esteban
Collantes en su despacho, esperando que salieran varios señores que
estaban dándole la jaqueca a don Antonio. Eran unos comisionados de
Málaga, un cacicón murciano, y el caballero de reluciente calva y
maneras elegantes a quien vi en las butacas del teatro Real la noche del
estreno de \emph{Aida}, hallándome en delantera de palco por asientos
junto a \emph{Leona la Brava}.

Despejado el terreno pasé yo, y atravesando el salón donde se reunía el
Consejo de Ministros, llegué al despacho del Presidente. A muchos
personajes de primera magnitud política había yo visitado en mi vida;
pero ninguno me causó tanta cortedad y sobresalto como don Antonio
Cánovas del Castillo, por la idea que yo tenía de la excelsitud de su
talento, por la leyenda de su desmedido orgullo y de las frases irónicas
y mortificantes que usar solía. Apenas cambiamos las primeras frases de
saludo, empezó a disiparse la leyenda del empaque altivo, pues me
encontraba frente a un señor muy atento y fino, y de una llaneza que al
punto ganó mi voluntad. Hízome sentar a su lado, en un sofá casi
frontero a la mesa de despacho, y hablamos\ldots{} quiero decir, él
habló y yo escuché, atento a su palabra enérgica, vibrante y un poquito
ceceosa.

«Deseaba verle, señor Liviano---me dijo,---porque he tenido ocasión de
leer páginas sueltas referentes al Cantón de Cartagena, escritas por
usted en el propio cráter de aquella revolución empezada sin tino y
concluida sin grandeza. Más que páginas, son notas trazadas al vuelo
frente a los acontecimientos, ya en los bastiones de Galeras o San
Julián, ya en la cubierta de los barcos sublevados. Esas notas
borrajeadas con el desgaire que imponen la premura del tiempo y la
nerviosidad del observador, me encantan a mí lo indecible, porque en
ellas veo como el primer aliento de la Historia, libre aún de artificios
y llevando en sí el aroma de la veracidad.»

Quedose el buen Tito de una pieza oyendo estos elogios, y por un momento
llegó a creer que el Presidente le tomaba el pelo. Mi estupor fue tal
que ni acerté a darle las gracias por tan increíbles piropos. Don
Antonio, ajustándose los lentes y alzando luego la cabeza, movimientos
en él muy comunes, prosiguió así: «Ya sé lo que va usted a decirme, y es
que esas páginas, esas notas, esos que mejor será llamar apuntes o
bosquejos, han sido escritos efectivamente por usted; pero no se han
publicado. Y usted pensará: \emph{¿cómo puede este señor haber leído mis
escritos si aún no han tenido la sanción de la letra de molde?} Pues si
no lo sabe le diré que tengo una loca afición a los estudios históricos.
A mí llegan diversos papeles interesantes, trozos de la Historia viva
que aún destilan sangre al ser arrancados del cuerpo de la Humanidad. Yo
los leo con avidez; los ordeno, los colecciono\ldots{} ¿Cómo llegaron a
mí los escritos de usted? No lo sé ni me importa saberlo\ldots»

Al oír esto sentí un tenue desvarío en mi cabeza, miré a un lado y a
otro\ldots{} ¡Jesús me valga!\ldots{} Creí que en la cabeza del sofá
erguíase grandiosa y colosal la figura de mi Madre, la divina
\emph{Clío}.

\hypertarget{v}{%
\chapter{V}\label{v}}

Segundos no más tardé en sustraerme al mundo quimérico para volver a la
esfera real. El sagaz estadista, adoptando el tono familiar apropiado al
asunto que quería tratar conmigo, me dijo así: «Sé que es usted amigo de
Cárceles y de otros que tuvieron parte muy visible en las locuras del
Cantón; seguramente lo es usted también de \emph{Tonete} Gálvez, que,
según mis noticias, fue la cabeza más firme y el brazo más fuerte en las
jornadas de Cartagena. Estará usted enterado de que los cantonales que
escaparon en la \emph{Numancia} permanecieron largo tiempo en Orán,
encerrados en un castillo. El Gobierno francés dispuso, a fines del año
anterior, internarlos en la provincia de Constantina. Contreras y su
ayudante Rivero accedieron a ser internados; Manuel Cárceles, Germes,
Gálvez y Gutiérrez obtuvieron un salvoconducto para fijar su residencia
en Suiza. Allá se fueron, creo que en Diciembre último. Y ahora pregunto
yo a don Proteo Liviano: ¿Están aún en Suiza? ¿Algunos de ellos ha
vuelto a España? Dígame lo que sepa. Habla con usted el amigo, no el
gobernante, y debo advertirle que estoy decidido a no perseguir a nadie,
ni aun a esos cuatro que, como usted sabe, están condenados a muerte.
Las realidades del Gobierno y la fuerza indudable de la Situación que
presido me imponen la clemencia. Oportunamente pienso dar una amnistía
general, que ha de comprender a esos ilusos, más románticos que
criminales. Espero que me diga usted, si lo sabe, el paradero de
Cárceles, Germes, Gutiérrez y Gálvez, y no vacilo en indicar que me
intereso singularmente por este último. Antonio Gálvez es un hombre de
bien; un político de ideas extraviadas, pero muy puro y muy sincero;
caudillo valiente hasta la temeridad. Sus sentimientos generosos le
impulsan hacia el bien, y si alguna vez hizo el mal fue por obedecer
ciegamente a la pasión revolucionaria.»

Asentí con fuertes cabezadas y algún monosílabo a lo que don Antonio me
decía en elogio a Gálvez. Como yo declarase con toda ingenuidad que
ignoraba el paradero de los emigrados del Cantón, el Presidente me
sorprendió con este rasgo de franqueza: «Tenemos una policía detestable.
No veo en ella más que la proyección más inútil y desmayada de nuestro
matalotaje burocrático. Si yo tuviera tiempo y no me agobiaran
atenciones de superior importancia, intentaría organizar un Cuerpo de
Seguridad muy a la moderna. Pero es más difícil crear aquí una buena
policía que poner en pie de guerra un gran Ejército. Por esa caterva de
vagos, mendigos y soplones, que no otra cosa son nuestros actuales
corchetes, ha sabido el Gobierno que andan por Madrid algunos
presidiarios de los escapados de Cartagena. Me han hablado de un armero,
muy hábil por cierto, que trabaja en la calle de los Reyes, y de un
vejete que se dice aristócrata napolitano y al parecer es gran
pendolista y pintor de ejecutorias. De seguro habrá en Madrid muchos más
y usted quizá los conozca. Ya comprenderá que no trato de perseguirlos.
Si esos infieles viven de su trabajo y no hacen daño a nadie, arréglense
como puedan. Lo que yo deseo de usted, señor Liviano, es que por esa
gente o por otra indague si está Gálvez en Madrid. En caso afirmativo,
trate de verle y dígale de mi parte que no se dé a conocer y se le
proporcionará buen recaudo para retirarse a Beniaján o Torre Agüera, sin
peligro alguno\ldots{} Y ahora, dispénseme, don Proteo, que yo dé a
usted esta comisión, puramente confidencial y amistosa. Esto queda entre
nosotros, y si dan resultado sus investigaciones y tiene la bondad de
venir a manifestármelo, ya sabe que con sólo presentarse a Esteban
Collantes será usted recibido por mí cuando guste.»

Prometí al caudillo alfonsino ocuparme desde aquel mismo día en dar los
pasos necesarios para satisfacer lo más pronto posible sus deseos, y me
despedí con todo el rendimiento y veneración que persona tan ilustre
merecía. Al atravesar el Salón de Consejos para retirarme, flaqueaban
mis piernas y mi cabeza no estaba muy firme. Cuando salí al vestíbulo me
alzó la cortina una mujer\ldots{} ¡Por Júpiter, era
\emph{Efémera}!\ldots{} Mi retirada fue más bien escapatoria. No vi a
don Saturnino Esteban Collantes ni a ninguno de los amigos de la
Secretaría\ldots{} Bajé a trompicones la escalera. En cada rellano, en
el zaguán y en la puerta se me apareció una, dos y veinte veces la
figura de \emph{Efémera}, con su túnico negro y su mirada dulce y un
poquito guasona\ldots{} En la calle tiré hacia el Prado, sin rumbo ni
dirección razonable. Me sentía sin aplomo, enloquecido. La mensajera de
\emph{Clío} no me abandonaba. Volví a verla en la esquina de la calle
del Turco; después junto al palacio de Alcañices. A lo largo del Prado
se repitió la visión, desvaneciéndose gradualmente.

Al llegar a mi casa iba totalmente persuadido de que la entrevista con
Cánovas era un nuevo fenómeno de la vida quimérica. Ni don Antonio me
había dicho nada, ni yo le vi, ni puse los pies en la Presidencia. Todo
había sido un bromazo impertinente de los espíritus picarescos que en
aquella temporada pasaban el rato divirtiéndose conmigo. El resto del
día permanecí en mi casa sumido en tristes cavilaciones, sin que los
halagos de Casiana pusieran término a mis melancolías. ¿Cómo era posible
que el Jefe del Gobierno, atento a los problemas políticos que debían
consolidar la Restauración, descendiese a la nimiedad de inquirir el
paradero de los desgraciados cantonales? La amistad protectora con que
distinguía Cánovas a \emph{Tonete} Gálvez ¿era un hecho real o un
desvarío de mi cerebro debilitado? Estas dudas me atormentaron hasta la
siguiente mañana en que mí espíritu empezó a serenarse, y di en pensar
que tal vez no era un sueño mi entrevista con el árbitro de los destinos
de España.

Fuese o no verdad el fenómeno, una fuerza misteriosa me impulsó a
inquirir y olfatear la pista de Gálvez. Vi a David Montero, y ni este ni
\emph{Dorita} me dieron luz alguna. Busqué a Fructuoso Manrique, que
vivía con \emph{Graziella}, no ya en la calle de San Leonardo sino en la
del Limón. En el taller de amenas hechicerías permanecí un rato
entretenido con las donosas diabluras de la italiana, y tuve el gusto de
acariciar al cuervo y al búho que gravemente colaboraban en las
operaciones de la casa. Ni Fructuoso, ni \emph{Graziella}, ni Celestina
Tirado, que entró de la compra con cesta repleta y un conejo de campo
para ponerlo con arroz en la comida de aquel día, sabían una palabra de
lo que afanosamente trataba yo de averiguar.

Cuando ya me despedía desalentado, saltó \emph{Graziella} con la idea de
apelar a la Cartomancia, arte muy eficaz para descubrir tesoros ocultos
y personas escondidas. Agarró la diablesa los naipes, y después de
barajarlos y hacer sobre ellos la mar de garatusas, pronunció sobre el
humo de un braserillo palabras hebraicas, llamó al cuervo que saltando a
su hombro le picó en el oído, y tras un nuevo sobar y manoseo de las
cartas trazando sobre una de ellas crucecitas con saliva, me dijo en
tono pausado y altísono: «Angélico Tito; encamina tus pasos vacilantes
hacia Perico Niembro, que te dará la luz que deseas.»

Ni corto ni perezoso corrí a ver a Niembro, el cual, después de un largo
palique en que se mantuvo escamón y misterioso, me mostró una carta de
Gálvez, fechada diez días antes en Lausanne. Ya me consideré satisfecho;
ya podía dar al gran estadista la precisa información que anhelaba. De
regreso a mi casa, revivió en mí la idea de que la famosa entrevista fue
soñación quimérica o mofa de los socarrones espíritus. A pesar de esto,
y temeroso de que no me dejaran llegar a la presencia de Cánovas,
endilgué mi levita y chistera, y me fui con maquinal impulso al caserón
de la calle de Alcalá. Contra lo que esperaba y temía, el Subsecretario
me recibió amablemente y me introdujo en el Salón donde vi como unas
veinte personas, entre las cuales reconocí al Marqués de Molins, a don
Fernando Cos Gayón, a Pepe Cárdenas, a Elduayen, a Valero de Tornos, y a
otros que por su empaque provinciano parecían embajadores del caciquismo
rural.

Iba Cánovas de grupo en grupo, repartiendo formulillas afectuosas y
equívocas, dulces ofertas que a nada comprometen. Yo me mantuve
apartado, esperando a que el Presidente me viese y me concediera el
honor de un breve coloquio. De improviso vino a mí el grande hombre, y
llevándome junto a una ventana, en una sola cláusula condensó el saludo
y la interrogación referente al encargo que me había hecho.
Comprendiendo que el laconismo se me imponía, saludé y contesté con
estas breves razones: «Señor don Antonio, he visto una carta, datada en
Lausanne con fecha 18 de este mes, en la cual dice Gálvez a su amigo
Perico Niembro que aún no sabe cuándo podrá volver a España.»

Pareciome que quedaba satisfecho el jefe de la Situación, y fuí
despedido con esta fórmula cortés: «Dispénseme, señor Liviano. Ya ve
usted cómo estoy de gente.»

Salí, y en la antesala me sorprendió la voz de Fernández Bremón, que
desde la puerta de la Subsecretaría me dijo: «No te vayas, Tito.
Precisamente estaba en acecho de ti para que no te me escaparas.»

Cogiome del brazo para llevarme a su oficina y allí, sentados \emph{vis
a vis} a un lado y otro de la mesa de trabajo, el sutil periodista me
dejó estupefacto con esta inesperada manifestación: «Por encargo de mi
Jefe te pregunto si aceptarías una posición decorosa, correspondiente a
tus méritos literarios y a tu conocimiento de la sociedad española. Por
el pronto tendrías una plaza en provincias, y más adelante vendrías a
Madrid.»

La sorpresa no me permitió formular una contestación inmediata y
terminante. Con medias palabras me mostré muy agradecido a la bondad del
Presidente\ldots{} Mas no podía, no debía dar\ldots{} ¿cómo
decirlo?\ldots{} dar a mis ideas de toda la vida un brutal
esquinazo\ldots{} Saltar tan de súbito al campo alfonsino, parecíame un
acto de cínica desvergüenza. Sólo el pensarlo me amargaba y me dolía
como un remordimiento.

Apuró Bremón los argumentos más ingeniosos para combatir una
susceptibilidad que a su juicio era producto de romanticismos mandados
recoger. Dignidad tan fieramente escrupulosa y arisca entraba ya en los
términos del mal gusto\ldots{} Disputamos, primero con serenidad,
después con cierto agridulce. Por fin, deseando yo cortar por el momento
la cuestión, le dije: «Pepe, lo pensaré. Déjame reflexionar y mañana
hablaremos.»

Abandoné la Presidencia con el recelo de encontrarme a \emph{Efémera},
cuya vaga presencia precedía siempre a las burlas de los ociosos
geniecillos maleantes. Al llegar a mi casa habíase afirmado en mi ánimo
la resolución de no admitir del alfonsismo una merced indecorosa.
Respetaba yo a Cánovas y le admiraba por su elevado entendimiento, por
su saber de Historia y de política, así como por su palabra enérgica y
sugestiva, esmaltada con los donaires de un ingenio sutil. Pero no
quería en modo alguno entregarme a la Restauración, induciéndome a ello
no sólo el vocerío de mi conciencia, sino el hecho de tener asegurado un
vivir modesto por el estipendio que de mi divina Madre recibía.

Decidido a rechazar con toda entereza el soborno, me personé al día
siguiente en las oficinas de la Presidencia, y reiteré a mi amigo
Fernández Bremón mi negativa exponiéndole exclusivamente las razones de
conciencia y dignidad, pues del subsidio materno que aliviaba mi pobreza
no tenía yo que dar conocimiento a ningún nacido. En esto llegaron al
despacho Frontaura y Campo Arana, y con ellos me dejó Bremón, llamado en
aquel instante a la Subsecretaría. Los ociosos funcionarios y yo
charloteamos más de media hora de cosas de teatros, comentando la
fulgurante aparición del genio de Echegaray en la escena española. Fue
como un huracán tonante y luminoso que trocó las emociones discretas en
violentos accesos de furia pasional; deshizo los gastados moldes,
infundió nueva fuerza y recursos nuevos al arte histriónico, electrizó
al público, y lanzó al campo de la crítica, en espantable remolino, los
ardientes entusiasmos revolcándose con las tibiezas rutinarias.

Cuando nuestras voces bajaban de tono hablando de Calatañazor, Arderíus,
Escríu y otros graciosos comediantes, volvió Fernández Bremón, y
llevándome aparte me dijo lo que a la letra copio para que el lector se
percate bien de la sorpresa que recibí al oírlo: «Se estima y se respeta
tu delicadeza al rechazar lo que se te propuso. Pero hay otra cosa,
Tito. Consta en la Subsecretaría que tienes a tu lado a una parienta
próxima recién venida de Cuba, una joven ilustradísima que posee todos
los conocimientos y títulos para ejercer el magisterio en condiciones
insuperables. Como supongo que en esa señorita no existirán los motivos
de delicadeza que a ti te obligan a renegar de la protección oficial,
dime el nombre de tu prima, sobrina o lo que sea, y se le dará una de
las plazas de Inspectoras de Escuelas que se crearán en estos días.»

Mediano rato estuve pensando la contestación que debía dar. Mi
conciencia me acusó de prestarme a una superchería si aceptaba, pues
Casiana no había pasado del \emph{be o ene, bon, be u ene, bun}. Luego,
mi voluntad un tanto picaresca quiso ahogar a la conciencia,
dictaminándome la conformidad con lo que se me proponía. Vacilé. Mi boca
trémula hizo una emisión de monosílabos que expresaban el pro y el
contra. Sentí en mi cabeza un leve desvanecimiento. Miré en derredor.
Frontaura y Campo Arana habían desaparecido.

En la mesa de despacho una mujer escribía silenciosa, haciendo con sus
lindos morros muecas infantiles\ldots{} ¿Era la vaporosa \emph{Efémera}?
No puedo asegurarlo. Sólo afirmo que en mi ánimo se extinguieron las
dudas, y sin miedo a la superchería dije a Bremón: «Si quieres, ahora
mismo te daré el nombre.» Acordeme entonces de que el apellido de
Casiana era Conejo, palabreja innoble y bajuna que a mi parecer
envilecía la persona de una Maestra Superior, y resolví traducirlo al
portugués, diciendo a mi amigo: «Apunta, Pepe, apunta el nombre:
\emph{Señorita doña Casiana Coelho}\ldots{} y por más señas \emph{Coelho
de Portugal}.»

Seguro estoy de que al leer esto, mis fieles parroquianos preguntarán:
«¿Y \emph{Efémera}?» Honradamente les contesto que no la vi al salir de
las covachuelas presidenciales, ni acierto a discernir si una figura de
flotante ropaje blanco, que iba delante de mí por las calles de Alcalá y
Cedaceros, reproducía la vagorosa estampa de la recadista de mi Madre.
Creo haber notado que se detuvo a comprar \emph{El Cencerro} en la
esquina de la calle de Gitanos, y que por esta vía húmeda y tabernaria
desapareció.

Me fui a mi casa, y entretuve la tarde repasándole las lecciones a
Casiana y oyendo el voluble disertar de mi buen patrón sobre materias
políticas y militares. «Sabrá usted, ilustre don Tito\ldots{} ¿y cómo no
ha de saberlo si un día sí y otro también hociquea usted con don Antonio
Cánovas?

---Párese un poco, don José---dije cortándole el discurso.---Yo no he
hablado con Cánovas. Por mis ideas y por mi insignificancia no sé, ni
puedo, ni quiero tratar a personas tan altas.

---Respeto, Excelentísimo Señor, las razones que Vuecencia tiene para
hacerse el chiquito---prosiguió Sagrario.---¡Sabe Dios lo que se traerá
Su Ilustrísima entre ceja y ceja! No me meto, no quiero meterme en
escudriñar su interior, las ideas, los propósitos, los planes que algún
día han de salir a la luz pública. Yo, que no veo más que lo que tengo
pegado a mis narices, pregunto: ¿Qué va a pasar aquí?\ldots{} No alterno
con sabios ni con gentes de grandes lecturas. Lo que sé lo aprendí
oyendo la voz del pueblo, \emph{vox caeli} que dijo el Latino. Todas las
mañanas voy a la compra, como Vuecencia sabe, y un ratito en la tienda,
otros en los cajones y puestos de los Tres Peces, me voy enterando de
los dichos que corren de boca en boca. Cuando vuelvo a mi casa y me
recojo en mi discernimiento natural, de lo que me entró por el oído y de
lo que yo discurro saco la verdadera enjundia y el meollo de eso que
llaman la Cosa Pública.

---Muy bien, don José. Los ruidos de la calle, traídos al crisol del
entendimiento, nos dan la verdadera clave de la opinión de un pueblo.

---Y \emph{francamente, naturalmente}, un hombre que ha vivido mucho,
que ha tratado innúmeras personas de arriba, de abajo y de en medio, que
ha sufrido adversidades personales y públicas viendo pasar ante sus ojos
tantas mudanzas, revoluciones y cataclismos, tiene derecho a decir: yo
veo lo que no se ve, yo presiento el suceso que aún está escondido en
los pechos de los que engendran la actualidad de hoy y la actualidad de
mañana. Y como pienso muy al derecho, al derecho le digo a Vuecencia,
señor don Tito, que su amigo don Antonio Cánovas\ldots{} amigo, ¿eh?,
aunque Su Ilustrísima lo niegue por razones de sigilo
diplomático\ldots{} está tragando mucha quina, una barbaridad de quina,
apretado entre dos muelas cordales, pues de una parte pesan sobre él los
malditos \emph{moderados}, los Chestes, Moyanos y Orovios que le piden
neísmo, intolerancia y tente tieso, y de otra parte le acosan los
alfonsinos que vienen de lo de Alcolea y quieren franquicias, unas
miajas de Soberanía Nacional y vista gorda para el libre pensamiento.

---Así es, amigo Sagrario. Lo que usted cuenta no es nuevo para mí.

---Pero hay algo más que usted no sabe, o si lo sabe no quiere decirlo,
y es que la Reina doña Isabel está dando las grandes tabarras a don
Antonio: solicita que la dejen venir acá, creo que para mangonear y
meterse en lo que ya no debe importarle. Con Pezuela y Roca de Togores
se entiende por cartitas dulces que menudean lo que usted no puede
figurarse\ldots{} Los \emph{moderados} escupen ya por el colmillo;
quieren ser los amos y que Cánovas gobierne a gusto de ellos. Por esto
yo digo a todo el que quiera oírme: aquí va a pasar algo\ldots{} Ya se
habrá usted enterado de que el rey don Alfonso, que se fue a Zaragoza y
Tudela a los cuatro días de llegar a Madrid, marchó después a Peralta,
donde acudieron los Generales Moriones, Laserna y Ruiz Dana, y con estos
y Jovellar, Primo de Rivera, Despujols, Terreros, Portilla, Morales de
los Ríos y otros, celebró Consejo para acordar el plan de operaciones.

---Sí, ya lo sé. Y el 22 de Enero largó sendas alocuciones a los
habitantes de las Provincias Vascongadas y Navarra y a los soldados del
Ejército del Norte.

---¡Consejo de Generales, alocuciones! Y yo pregunto: ¿Se trata de dar
el golpe definitivo a la negra facción, organizando descomunal batalla
con todos esos ilustres caudillos y el total contingente de nuestras
valientísimas tropas? ¿Estará próximo ese día de júbilo, ese día grande,
principio de la redención de España? Para mí, no hay duda, reunidos
todos esos elementos que han de constituir una hueste tan poderosa como
las de Alejando y César, la victoria es indudable. Venceremos, señor don
Tito, barreremos de nuestro suelo y de una vez para siempre esa escoria
del retroceso, esa inmundicia del absolutismo, esa paparrucha indecente
de la legitimidad. ¡Oh alegría, oh inmensa dicha de las almas
liberales!\ldots{} Un abrazo, don Tito. Y tú, Casiana, ven aquí\ldots{}
¡Un abrazo al amigo, al patrón, al maestro!»

\hypertarget{vi}{%
\chapter{VI}\label{vi}}

En los primeros días de Febrerillo loco, mi amigo Prieto y Villarreal me
llevó a una reunión de zorrillistas en casa de Cristino Martos.
Concurrieron a ella todos los que seguían a don Manuel y muchos
militares de los que quedaron defraudados y vencidos el 3 de Enero de
1874. Asistí yo al conciliábulo como simple testigo, y no despegué los
labios por no sentir mi ánimo dispuesto para ninguna clase de campañas
políticas. Había levantado don Manuel Ruiz Zorrilla la bandera de la
República frente a la Restauración, y tales fuerzas militares y civiles
agrupó a su lado, que el Gobierno alfonsino creyó preciso disponer el
extrañamiento de aquel gran ciudadano, rebelde y tenaz.

Decretado el ostracismo de don Manuel el 4 de Febrero, con la coletilla
de que no podría volver a España sin permiso previo del Gobierno,
aquella misma noche fue puesto en ejecución. Los zorrillistas y otras
personas unidas al temible revolucionario por vínculos de amistad,
hicieron acto de presencia en la estación del Norte.

Representando el ideal vencido que la Restauración quería lanzar del
suelo patrio, estaban en el andén Castelar, Salmerón, Carvajal, Rivero,
Echegaray, Martos, Pablo Nougués, Aguilera, Pedregal, García Ruiz y
otros muchos. Del estamento militar vi a los Generales Izquierdo y
Lagunero y al Brigadier Carmona, que salieron pitando para el destierro
al día siguiente.

Entre los amigos distinguíanse por su significación alfonsina don Pedro
Salaverría, Ministro de Hacienda, y el simpático Subsecretario de la
Presidencia, Esteban Collantes. De dónde provino la amistad de
Salaverría con don Manuel, no lo sé; la de Esteban Collantes y García
Ruiz tuvo su raigambre en la tierra palentina, donde Ruiz Zorrilla o su
señora poseían extensa propiedad rústica. La despedida fue triste y
afectuosa; los abrazos, efusivos; discreto el entusiasmo.

A este acto que considero público, y si queréis histórico, sigue en mis
crónicas otro que también me parece digno de perpetuarse en letras de
molde, y los escribo engarzados en una sola página para que resalte
mejor la desacorde calidad de ambos sucesos. Una tarde de aquel mismo
Febrerillo, que ahora llamo loco de atar o loco furioso, hallábame yo
solo en mi aposento, trasladando al papel con nerviosa escritura mis
impresiones de los pasados meses, cuando\ldots{} ¡ay Dios mío!\ldots{}
vi entrar a una mujer sin que la precediera rumor de pasos ni sonsonete
de campanilla. Llegose a mi mesa la fantasma, y yo, sin sorpresa ni
espanto, con la mayor naturalidad del mundo, le dije: «Hola,
\emph{Efémera}; bien venida seas. ¿Me traes carta de mi adorada Madre?»

Ella, dejando caer su izquierda mano marmórea sobre la mesa, alargó
hacia mí la derecha con un pliego, mientras sus labios helénicos
articulaban estas palabras que me sonaron cual si las transmitiera pos
ráfagas del aire una voz muy lejana: «No te traigo carta de tu Madre,
sino este pliego que me han dado para ti.»

Y yo, rasgando ávidamente el sobre y enterándome de su contenido,
exclamé: «¡Ah! La credencial nombrando a Casiana Inspectora de Escuelas.
Gracias. Mi buena Madre no se cansa de favorecerme\ldots{} Tú no
ignoras, \emph{Efémera}, que Casiana \emph{Coelho} es mujer meritísima,
muy versada en la teoría y práctica del arte pedagógico\ldots{} ¿Por qué
no descansas a mi lado?\ldots{} ¿Qué dices? ¿Qué no te sientas? ¡Oh!,
divina mensajera; tu destino es correr, volar, llevando por el mundo la
verdad del momento. Del conjunto de estos átomos, aglomerados por el
Tiempo, se forma la verdad histórica en lustros, en siglos\ldots{}
Espera un poquito, que quiero hacerte algunas preguntas. ¿Qué me dices
de mi Madre? Ya sé que por su condición inmortal está exenta de toda
enfermedad. Su salud es inalterable. Varían tan sólo su apariencia
personal y las vestiduras que cubren su noble cuerpo. Cuéntame: ¿qué
calzado gasta en estos benditos días para andar por el mundo? ¿Lleva por
ventura el alto y ceremonioso coturno, señal de la grandeza histórica?»

La recadista de \emph{Clío}, con solemnidad un tantico risueña,
contestó: «No lleva el coturno, sino unos holgados borceguíes de burdo
paño, decorados con papeles de rojo y gualda, talco y purpurina,
imitando el esplendor áureo del calzado de los Dioses, falsedad que sólo
engaña a ciertos académicos. Usa la Madre estos borceguíes blandos y de
figurón, porque se los impone la suciedad y dureza del suelo que
recorre, todo fango y guijarros puntiagudos.

---Muy bien, \emph{Efémera}. Y ahora dime otra cosa\ldots{} Esto se
refiere a mi persona\ldots{} Escucha. Con toda sinceridad y franqueza me
responderás a lo que voy a preguntarte. ¿Es verdad o es mentira que yo
he visitado a don Antonio Cánovas, hablando a solas con él de asuntos
políticos y particulares?

---La verdad y la mentira de los hechos no caen debajo de mi
jurisdicción. Lo que a mí me concierne es el contacto de las
inteligencias en las anchas regiones del espíritu. Del uno al otro
cerebro saltan las ideas como chispas de un fuego que es el generador de
la concomitancia y simpatía. Recojo yo estas chispas y las comunico
entre los seres, hállense próximos o distantes\ldots{} Es lo único que
puedo contestar al señor don Tito. Tengo prisa. Adiós.»

No me dio tiempo a formular nueva pregunta ni a darle mis tiernos
adioses. Desapareció en forma semejante a las magias de teatro. En vez
de volverse para tomar la puerta se desvaneció en la cavidad del
aposento, dejándome absorto, atontado y sin respiro. Apenas me repuse de
la emoción de tal escena, recorrí con rápida vista la credencial.
Nombraban a Casiana Inspectora de Escuelas con sueldo de diez mil
reales. En nota aparte me decía Bremón que si la señorita \emph{Coelho
de Portugal} ocupaba sus horas en dar lecciones particulares a
domicilio, quedaría relevada de todas las obligaciones de la Inspección,
salvo la de cobrar su sueldo a primeros de cada mes\ldots{}

Guardé el nombramiento, en el que vi un signo de los tiempos. Todo era
ficciones, favoritismos y un saqueo desvergonzado del
presupuesto\ldots{} Después de un largo titubeo, decidí no dar
conocimiento a Casiana de aquel momio inverosímil y esperar, esperar a
que se pusieran de acuerdo los ángeles que me favorecían y los demonios
que me burlaban.

Una noche, avanzado ya Febrero, cuando Casiana y yo volvíamos de ver una
funcioncita en el próximo teatro de Variedades, donde trabajaban actores
tan graciosos como Luján y Riquelme, nos encontramos a don José Ido en
estado de gran consternación y abatimiento. Creímos que Nicanora estaba
con el \emph{histérico} o que habían llegado noticias desagradables de
Rosita, de quien se dijo días antes que se hallaba ya fuera de cuenta.
No era nada de esto. Dejo al propio Sagrario la explicación del enigma,
reproduciendo el texto fiel de sus acongojadas manifestaciones:

«¡Ay don Tito de mi alma, qué pena, qué horrible desengaño! Ya sabe
Vuecencia que hace dos días venían corriendo unos rumores sumamente
halagüeños para la Patria y para la Libertad. Las voces públicas decían
en tiendas, porterías, plazuelas, cafés, estancos y boticas que en el
Norte estábamos dando una gran batalla, mejor dicho, que ganamos una y
luego dimos otra más reñida y sangrienta, ganándola también; que en la
tercera batalla, el suelo quedó totalmente cubierto de cadáveres
carlistas en una extensión de cuatro leguas a la redonda. Saturio, el
amolador de las Niñas de Loreto, me dijo ayer que de resultas de esa
terrible matazón de carcundas, los pocos que de estos quedaron salieron
por pies, desapareciendo al otro lado del Pirineo.

»Pero ¡ay!\ldots{} esta mañana, cuando más contento iba yo entre los
puestos de los Tres Peces, empezaron unos runrunes que dejaban
patidifusos aun a los que no les dábamos crédito. Hice mi compra, y
donde quiera que yo iba la voz pública seguía cantando el
\emph{miserere}. Al entrar en la Plaza de Matute, para comprar vino en
el almacén de Roque, me encontré al amolador y al sacristán de las Niñas
que discutían en medio de la calle. El sacristán, que es más neo que
Judas y más borracho que Noé, se dejó decir que a los liberales nos
habían dado un palizón horroroso\ldots{} Qué tal sería la somanta, que
los carlistas cogieron prisionero al Rey don Alfonso y se lo llevaron a
Estella.»

Siguió diciendo el manso filósofo que del sofoco que tomó al oír tales
desatinos le flaquearon las piernas, y tuvo que arrimarse a la pared
para no dar con su pobre cuerpo en el suelo. Luego se equivocó de tienda
y le armaron el gran escándalo por pedir tinto de mesa en una cerería.
Al referirnos esto, se acentuaba tanto la flaccidez del rostro del buen
hombre que los huesos se le transparentaban debajo de la piel, y la nuez
le crecía desaforadamente.

«Esta tarde---prosiguió mi atribulado patrón, sentándose para tomar
aliento,---me fui a Buenavista con la esperanza de que mi primo Macario,
sargento de la brigada obrera de Estado Mayor, me sacara de mis
horribles dudas y me dijese la verdad de lo acontecido en Navarra. ¡Ay
Dios mío, cuánto sufre un corazón patriota cuando el demonio enreda las
cosas de la guerra!\ldots{} Lo que ha sucedido es cosa desdichada y
lastimosa; pero no tanto como las asquerosas mentiras que contaba esta
mañana el rapavelas de las Niñas de Loreto. Parece, según reza el
telégrafo, que entre dos pueblos llamados si no recuerdo mal Lácar y
Lorca, hubo un momento en que por milagro de Dios Nuestro Señor no cayó
Alfonso XII en poder del faccioso.

---Estas cosas de la guerra---dije yo, dándole ejemplo de
serenidad,---son para miradas despacio. Esperemos los despachos
oficiales que nos darán relación detallada de los hechos. Tranquilícese,
don José; tomémoslo con calma, que ni por una victoria debemos perder el
sentido, ni por un descalabro hacer malas digestiones. La grandeza de un
pueblo no está en la guerra sino en la paz; la desdicha de los españoles
consiste hoy en que para llegar a la paz tenemos que pelearnos
fieramente unos con otros. A los labradores hemos convertido en
soldados, y ahora falta que los mansos obreros del terruño se cansen de
andar a tiros y vuelvan a coger el arado.»

A la noche siguiente no falté a la tertulia que algunos amigos teníamos
en el Café de Zaragoza. Casiana iba conmigo. Asiduo concurrente a
nuestras mesas era el Capitán Palazuelos, a quien yo conocí de Teniente
el año anterior: a la sazón prestaba servicio en la Subsecretaría de
Guerra. En cuanto llegué se puso a mi lado y me refirió lo que sabía del
suceso de Navarra, acaecido no lejos del siniestro lugar en que murió
trágicamente el General Concha. He aquí su relato sucinto:

«El 2 de Febrero, si no estoy equivocado, el jefe carlista Mendiri
atacaba con preferencia al segundo Cuerpo del Ejército, por suponer que
con el General en Jefe, Primo de Rivera, hallábase el Rey Alfonso. En la
tarde del 3, cuando menos lo esperaba la división Fajardo, compuesta de
dos brigadas (una de las cuales estaba en Lácar bajo el mando de Bargés
y la otra en Lorca), embistieron los de Mendiri el pueblo de Lácar con
extraordinaria bravura, llevando consigo a Cavero, Pérula y no sé quién
más, con aguerridos batallones y bastantes piezas de artillería. Ante lo
formidable del ataque flaquearon los nuestros; oyéronse gritos de:
\emph{¡Estamos vendidos! ¡Sálvese el que pueda!}, y el Regimiento de
\emph{Valencia} se dispersó, siguiéndole al poco rato los soldados de
\emph{Asturias}. Ni Fajardo ni Bargés cuidaron de poner centinelas en
los altos de Alloz y de Murillo, y a ello se debió principalmente el
descalabro.

»Cuando Fajardo, que estaba en Lorca, oyó los primeros disparos, se puso
al frente del Regimiento de \emph{Gerona} y se dirigió a la montaña que
separa aquel pueblo del de Lácar. Mas nada pudo hacer para dominar la
confusión en aquella hora fatídica. El desaliento era unánime, lo mismo
en los jefes que en los soldados. También se dispersó el Regimiento de
\emph{Gerona}, y el brigadier Viérgol se vio forzado a retirarse del
sitio de peligro. Primo de Rivera, ocupado entonces en el emplazamiento
de piezas de Artillería sobre Monte Esquinza y en hacer pruebas de
puntería sobre los pueblos enemigos, acudió en auxilio de los de Lácar y
Lorca, logrando remediar un tanto el desastre.

»En la madrugada del 4, el General Fajardo, al frente de la tropa, con
las cajas de caudales, botiquines y material de guerra, salió de Lorca,
retirándose hacia Esquinza. También los de Mendiri se desmandaron, y
viendo este que sus tropas se lanzaban al saqueo y al inútil
derramamiento de sangre, retirose a Estella. En el Ministerio aseguran
que el Rey no estuvo en peligro más que breves instantes. Alguien ha
dicho que se hallaba en la torre de una iglesia situada entre los
pueblos de Lácar y Lorca. Según las versiones oficiales, Su Majestad
permanecía en su alojamiento de Villatuerta, donde oyó muy de cerca los
disparos de fusilería. Cuentan que dijo a los que le rogaban que no se
aventurase a salir: \emph{Un Rey no debe ocultarse cuando silban las
balas a su alrededor}. Cómo y en qué forma salió de su alojamiento, no
he logrado saberlo. En Guerra me han dicho, sin precisar la hora, que el
Rey emprendió la marcha a galope tendido hacia Puente la Reina.»

A mis observaciones sobre la obscuridad del relato de Palazuelos,
contestó este: «Ha de pasar algún tiempo antes que sean conocidas en
todos sus pormenores las jornadas dudosas y equívocas que hoy designamos
con los nombres de Lácar y Lorca. Entiendo yo que la Historia, cuando se
ve precisada a referir con verdad acontecimientos de esta índole, pasa
grandes apuros y se ve ahogada en perplejidades enojosas. Los que
intervinieron en estas acciones, procediendo con negligencia o
aturdimiento, no ponen en sus despachos la debida fidelidad. Si es
sospechoso el testimonio de los nuestros, también lo es el de los
enemigos, que siempre exageran y sacan las cosas de quicio cuando han
tenido algún momento afortunado\ldots{} Los carlistas cantaron victoria
al recogerse a Estella. Ya veremos quién cantará el último.»

Cuando terminó el Capitán su bosquejo de Historia equívoca, nos
enredamos en otras pláticas más amenas y en bromas y diálogos picantes
que no nos corrompían las oraciones. Amenizaba las tertulias cafeteras
un pianista navarro llamado Cárcar, que solía venir a nuestra peña
brindándonos las piezas de su repertorio que más nos agradasen. Aquella
noche, para quitarnos el amargor de las desagradables peleas de Lácar y
Lorca, le pedimos que tocara jotas y rondallas, pues era consumado
maestro en la música popular de su tierra. Hízolo prodigiosamente y los
aplausos creo que se oyeron en Getafe. Hartos de conversación y de
música nos retiramos, no sin que Casiana hiciera la indispensable
requisa y acopio de terrones de azúcar para endulzar nuestro café
matutino. Con este típico detalle queda bien demostrado que en aquella
dichosa era de distinción y elegancia habíamos escogido lugar
preeminente en la esfera de la cursilería.

Pocas noches pasaron hasta una que en cierto modo debo llamar memorable,
porque en el diálogo familiar que tuve con Ido del Sagrario no faltaron
unas briznas de Historia. «Venga usted acá, excelso patrón---le dije,
viéndole entrar en casa \emph{cabiztivo} y \emph{pensibajo}.---Acérquese
y le contaré un suceso que disipará sus murrias, colmándole de
satisfacción y alegría\ldots{} Aquí tiene usted a Casiana, su ilustre
discípula, que pronto va a saber más que el maestro.

---Así lo creo y lo deseo, Excelentísimo Señor---dijo el filófoso,
tomando asiento a respetuosa distancia.

---Ya sabe Casiana el suceso de autos que voy a contarle a usted, y se
ha puesto muy contenta\ldots{} Ea, no quiero dilatar el plato de gusto
que le tengo a usted preparado. Oiga, don José, y vaya sacudiendo las
tristezas que le agobian desde que supimos la terrible trapatiesta de
aquellos malditos pueblos navarros. ¡Ánimo, valiente, que no hay mal que
cien años dure, ni desdichas que no terminen con algo lisonjero!\ldots{}
Pues, señor: don Alfonso XII celebró en Puente la Reina Consejo de
Generales, donde se acordó lo que no sabemos ni nos importa. De allí fue
a Pamplona y luego se dirigió a Logroño, con objeto de visitar al Duque
de la Victoria. ¿Qué tal? Su ídolo de usted, el invencible Espartero,
recibió al joven Monarca con las demostraciones de afecto más efusivas,
y pidiendo a sus ayudantes la cruz laureada de San Fernando, que él ganó
en las gloriosas campañas de la primera guerra civil, la puso en el
pecho del simpático reyecito. Debo añadir amigo don José, para que usted
se esponje, que al realizar don Baldomero este acto de acendrado
monarquismo, elogió calurosamente la conducta de Alfonso XII en la breve
campaña que a usted le tiene tan compungido.

---Algo es algo. ¡Viva el Duque!---exclamó Ido.---Me complace el suceso;
pero siempre me queda un dejo de aquellos amargores.

---\emph{Sursum corda}. Recobre usted su fe en la libertad; hínchese de
patriotismo; nos hincharemos todos\ldots{} Y ahora, don José, cuídese de
que nos sirvan la cena. ¿Verdad, Casiana, que el patriotismo nos
desarrolla furiosamente las ganas de comer?\ldots{} Oiga, señor
Sagrario: para celebrar el suceso con la debida solemnidad, dígale a
Nicanora que nos ponga una tortilla de seis huevos, para los dos, y esas
chuletas a la \emph{papillote} por las cuales merece su esposa de usted
el título de Cocinera de los Dioses.»

\hypertarget{vii}{%
\chapter{VII}\label{vii}}

Menudas jaquecas daban a don Antonio los señores del lastre
reaccionario, que pesaba brutalmente en la nave de la Situación. Por el
sistema \emph{efemerídeo} que me había revelado la Madre, introducía yo
mi pensamiento en el cerebro del grande hombre. Allí se me comunicaba su
iracundia por las enormidades que imponerle querían los bárbaros del
vetusto \emph{Moderantismo}. Ponían estos el grito en el cielo al ver
que los primeros puestos de la Política, de la Administración y del
Ejército eran arramblados por \emph{la taifa de Septiembre}, y se
aprestaron a las represalias metiendo a don Francisco Cárdenas, Ministro
de Gracia y Justicia, en el jaleo de derogar la Ley de Matrimonio Civil
de 18 de Junio de 1870. Con tal atropello resultaron concubinatos los
matrimonios legalmente contraídos, y naturales los hijos habidos en
ellos. Horrísona tempestad levantó en la Prensa y en la opinión este
atroz desafuero, y mientras el Papa se frotaba las manos de gusto, el
jefe de los alfonsinos rabiaba en silencio, viendo frustrado su sano
propósito de cimentar su política en el Manifiesto de Sandhurst.

Nadie me contaba el estado mental del Presidente del Consejo. Sentíalo
yo en mí mismo por el contacto misterioso del pensar \emph{canovístico}
con el pensar de este humilde vocero de la vida hispana. Por el mismo
artilugio milagroso pude apreciar que no hicieron maldita gracia al
insigne malagueño los airados decretos con que Orovio puso en la calle y
desterró a los Catedráticos de la Universidad Salmerón, Giner de los
Ríos, Azcárate y otros, lumbreras de la Filosofía y del Derecho, y
apóstoles de la libertad de conciencia. Por este acto de brutal
intolerancia y por sus pintorescos chalecos, transmitió su nombre hasta
los alrededores de la posteridad el Marqués de Orovio que, aparte su
ciego fanatismo, era una persona decente y honrada.

Con un bello desorden que a mi parecer da colorido y sabor picante a las
minucias históricas, os contaré que el Rey don Alfonso, muy contento con
la cruz laureada que Espartero puso en su pecho, partió de Logroño a
Burgos, y después de visitar Valladolid y Ávila regresó a Madrid, donde
\emph{las masas oficiales} le recibieron con palmas. En tanto, su madre
doña Isabel no cesaba de mover el ánimo irritable de los borbónicos
netos para que le abrieran brecha o caminito por donde colarse en el
suelo patrio. Suspiraba por la espesura florida de Aranjuez; necesitaba
una estación balnearia para la primavera, y en verano no podrían
pasarse, ni ella ni las Infantitas, sin los baños de mar.

Cánovas, que profesaba el principio filosófico-político de mantener a
las Reinas Madres alejadas del foco de la gobernación, indicó a doña
Isabel, con muchísimo respeto, la residencia de Mallorca para sus
esparcimientos y regocijos primaverales y veraniegos. En esto, sabedor
Carlos VII de los anhelos de su augusta prima, le escribió brindándole
para su descanso y recreo las Provincias Vascongadas donde él
reinaba\ldots{} Ridícula es la carta en que el Pretendiente ofrecía las
playas vizcaínas y guipuzcoanas a doña Isabel para su temporada estival.
Entre otras simplezas se dejó decir lo siguiente: «Si quieres ir a
Lequeitio o Zarauz, donde estuviste en otras épocas, puedes ocupar los
mismos palacios que entonces habitaste, pues no creo posible que en tal
caso los marinos de tu hijo continuaran bombardeando aquellos puertos, y
si lo intentasen, tengo cañones de suficiente alcance para que te dejen
tranquila.» Doña Isabel fue lo bastante discreta para no aceptar la
farandulesca protección de su primito. ¡Estaría bueno que las dos ramas
que habían desgarrado el cuerpo de la pobre España disputándose un trono
durante más de medio siglo, hicieran paces vergonzosas por los baños de
ola de Lequeitio!

Si buenas dosis de acíbar tragó Cánovas por las imposiciones \emph{del
elemento retrógrado y obscurantista}, como diría Ido, no fue mala
compensación la dulzura de ver entrar en la legalidad al truculento
guerrillero don Ramón Cabrera, culminante figura del carlismo. Conviene
consignar algunos antecedentes familiares de este gran suceso. Cuando el
llamado \emph{Tigre del Maestrazgo} pasó el Pirineo en 1840, perdida ya
la causa de don Carlos, fue a parar a Inglaterra, donde la fama de su
temerario arrojo rodeó su nombre de una aureola de trágica leyenda. En
Londres se destacó vigorosamente su atezado rostro, su mirada
fulgurante, el aspecto de fiereza medioeval, y se contaban las
cicatrices que hacían de su cuerpo un heroico jeroglífico. No
necesitaron los ingleses forzar su imaginación para ver en Cabrera una
figura genuinamente \emph{shakespiriana}.

Pasado algún tiempo, la leyenda del guerrillero y su presencia personal
interesaron el corazón de una dama inglesa, protestante, rica y noble.
La dama y el héroe contrajeron matrimonio con todas las de la ley.
Entró, pues, Cabrera en una vida pacífica y burguesa, a la cual se
atemperó fácilmente el adalid más terrible, sagaz, activo y sanguinario
que ha existido en nuestras discordias civiles. Determinó esta evolución
del carácter de Cabrera el genio de su esposa, que supo subyugar la
fiereza del cabecilla insigne.

El \emph{tigre} cedió a la blanda ferocidad de la \emph{tigresa},
convirtiéndose en apacible cordero. Un amigo de Cabrera, que le había
conocido en España, me contó que una noche fue a visitarle a su casa de
Londres, situada en el West, junto a un Square o plazoleta jardinada. Al
entrar en esta encontró a don Ramón, de frac, fumándose tranquilamente
un puro. Al abrazar a su amigo, \emph{el tigre} domesticado le dijo: «Me
encuentra usted aquí porque mi mujer no me deja fumar en casa.»

En rigor de verdad debe decirse que más que la señora contribuyó a la
domesticación de la fiera el plácido ambiente de un país liberal y
protestante, de un país en que imperaba la justicia y el orden, en que
los ciudadanos vivían dichosos ejercitando sus derechos y sometidos al
suave rigor de las leyes. A nadie pudo sorprender que un hombre tan
inteligente y agudo como Cabrera evolucionase radicalmente, acabando por
abominar de la salvaje guerra dinástica de su país, y se asqueara de las
vesanias y horrores en que él desplegó todo su coraje. Últimas palabras
de esta conversión fueron los intentos de transigir con don Amadeo y aun
con la República, y, por último el acto decisivo de reconocer a don
Alfonso como el único Rey posible en España. A este feliz resultado se
llegó mediante negociaciones en que intervinieron de una parte el Duque
de Santoña, Merry del Val y Pareja de Alarcón, y de la otra el señor
Homedes, sobrino del famoso guerrillero, y otros amigos de este.

En un Manifiesto publicado en París, dijo Cabrera a los carlistas con
buenas formas que el absolutismo teocrático era una estupidez en
nuestros tiempos, y que del lema de la bandera facciosa dejaba a los
fanáticos el \emph{Rey}, llevándose consigo el \emph{Dios} y
\emph{Patria}. Don Carlos espetó contra su antiguo General un enfático
documento, privándole de todos sus títulos, empleos y honores, castigo
que al flamante alfonsino le tenía sin cuidado. En cambio don Alfonso
incluyó el nombre de Cabrera en el escalafón de Capitanes Generales,
reconociéndole el título de Conde de Morella y todas las condecoraciones
que ganara en los campos de batalla, peleando contra la causa liberal.

Figurando ya en la Grandeza militar y social del nuevo reinado, el de
Morella se instaló en Biarritz para trabajar más de cerca en pro de
Alfonso XII. Muchos carlistas prestigiosos se fueron con él, y la
estrella del Pretendiente empezó a perder su brillo, anunciando un
próximo eclipse. Aquel amigo que había encontrado a Cabrera en la
plazoleta del West londinense fumándose un habano, me contó que en
Biarritz la transformación de la figura del \emph{tigre} superaba en
radicalismo a la mudanza de sus ideas y de su carácter. Se había dejado
la barba; su rostro no carecía de serenidad placentera; el empaque y la
ropa delataban la rigidez protestante y el característico tono
británico. Hablando, salpicaba de sus labios un ligerísimo acento
inglés. \emph{¡Oh tempora, oh mores!}

Mezclando sabiamente lo útil con lo dulce, conforme al precepto del
Latino, os contaré que Casiana \emph{Coelho} adelantaba maravillosamente
en sus estudios. Había pasado el \emph{Catón}, y ya leía sin grandes
tropiezos las primeras páginas de la infantil enciclopedia llamada
\emph{Juanito}. En la escritura, vencido el agobio de los palotes y el
duro aprendizaje de letras sueltas, escribía palabritas enteras con
limpieza caligráfica y puro estilo de letra española. Gozaba yo lo
indecible viéndola trabajar, y el paciente Sagrario me profetizó que el
año próximo la señorita de \emph{Vargas Machuca} sería un portento de
ilustración.

Continuaba yo manteniendo en reserva la famosa credencial de Casiana, y
como mi conciencia repugnaba la villanía burocrática de cobrar el sueldo
de la \emph{Señora Inspectora} sin que esta prestase al Estado servicio
alguno, inclinábame a permanecer a la expectativa, sospechando que el
tiempo o los espíritus amables me traerían una solución decorosa. En
tanto, deslizábase mi vida sosegada y sin quebraderos de cabeza, viendo
pasar los días grises y melancólicos: si alguno traía un suceso digno de
atención, el siguiente se lo llevaba para diluirlo en las penumbras del
olvido.

Redondeaba mi tranquilidad la paz amorosa de mi unión con Casianilla,
cuya modestia, docilidad y aptitudes caseras, encantábanme lo indecible.
La compenetración de nuestros caracteres y de nuestros gustos llegó a
ser tal, que mi pensamiento rechazaba con horror la idea de separarnos.
Ya he dicho, y ahora repito, que nos habíamos declarado muy a gusto
figuras culminantes en la flor y nata, o dígase \emph{crema}, de la
cursilería.

Para que mis simpáticos lectores se rían un rato, les contaré lo que
hacíamos mi compañera y yo, ganosos de afianzarnos y sobresalir
dignamente en aquella interesante clase social. Sigo creyendo que la
llamada \emph{gente cursi} es el verdadero estado llano de los tiempos
modernos, por la extensión que ocupa en el Censo y la mansedumbre
pecuaria con que contribuye a las cargas del Estado. Atención,
caballeros. Mi Casiana era su propia modista. Juntos íbamos los dos a
comprar las telas; luego, entregábase la pobre chica al corte y
confección en la mesa del comedor, guiándose con patrones hechos de
papel de periódico y figurines sebosos, que le traía no sé de dónde su
tía Simona. Largas horas de la tarde y la noche dedicadas a la costura,
sin sustraer tiempo al estudio, completaban la obra, y cuando llegaba la
ocasión de las probaturas, estas se hacían en mi presencia para requerir
mi opinión de hombre de mundo y corregir los defectos que yo advirtiera.

Sepan también las edades futuras que mi compañerita se arreglaba los
corsés, echando piezas nuevas allí donde hacían falta, renovando
ballenas, ojetes y cordelillos. En cuanto a los polisones ¡ay!, yo,
Prometeo Liviano, era el fabricante de aquellos absurdos aditamentos.
Tras cortos ensayos llegué a dominar el armadijo de alambres y
crinolina, que hubiera causado vergüenza y horror a la \emph{Venus
Calípige}. Agradecía Casiana esta colaboración convirtiendo en lindas
corbatas para mí los retazos sobrantes de sus vestidos. Sus hábiles
manos \emph{confeccionaron} igualmente un chaleco que resultó tan bien
cortado y \emph{fashionable} como los de Orovio.

Cuando teníamos aderezado nuestro equipo nos echábamos a la calle
pistonudos y fachendosos, y exhibíamos nuestras personas en Recoletos,
la Castellana y el Retiro, saboreando el efecto que causábamos en la
plebe ignara. A los teatros íbamos comúnmente con el noble carácter de
\emph{tifus}, acudiendo a la fina amistad de Ducazcal, Arderíus y otros
rumbosos empresarios. Rara era la noche en que faltábamos al café,
prefiriendo los que tenían piano y violín, complemento artístico de la
frescura de la leche merengada y del rico chocolate con picatostes.
Deliciosos ratos pasábamos en las \emph{soirées} cafeteriles, entre la
escogida sociedad de señoras equívocas y señoritas del pan pringado, sin
olvidar a última hora la rapiña picaresca de terrones de azúcar.

Procedía yo de esta manera extremando las formas de ordinariez
presumida, no por el corto gasto que tal vida supone, pues bien podía
dármela mejor, sino porque se me habían hecho odiosas las elegancias
faranduleras y la hinchada presunción traídas a la sociedad española por
el cambiazo de Sagunto.

Me cargaban los hombres jactanciosos y vacíos que se habían elevado de
la pobreza cesantil a las harturas del presupuesto, gentes por lo común
holgazanas, marimandonas, atentas no más que a encarnar en sí mismas la
pesadumbre del armatoste burocrático. Me reventaban los Condes y
Marqueses, mayormente los de nuevo cuño, sacados por don Amadeo y don
Alfonso del montón de indianos negreros, de mercachifles enriquecidos o
de agiotistas sin conciencia. Me encocoraban los señores pudientes, que
rebajando su jerarquía ancestral entregábanse al servilismo palaciego y
monárquico. Detestaba, en fin, todas las vanidades que se habían
mancomunado para contener los progresos de nuestra Patria, y encerrarla
dentro de unos moldes que no podría romper sin nuevas y más iracundas
revoluciones.

Como yo me tenía por superior a toda esta turbamulta, materializaba mi
desprecio adoptando la modalidad que a mi parecer era contrafigura del
señorío infatuado, rémora contumaz de la vida española. Y cuando ante él
ostentábamos Casiana y yo nuestros atavíos fachosos, mentalmente les
decíamos: «Miradnos bien. Somos cursis por patriotismo.»

Mis odios más vivos recaían sobre una casta de señoritos en su mayor
parte salidos de las Universidades, ricos por su casa, y algunos
participantes de las delicias de la nómina. Trastornadas estas criaturas
por las parambombas que introdujo la Restauración, elevaron a fórmulas
dogmáticas el arte y reglas de la elegancia. A todos los que no
tuviéramos exquisita hechura personal, en modales y ropa, nos miraban
como a raza inferior, no más digna de aprecio que las turbas gregarias
despectivamente llamadas \emph{masa obrera}. Entre ellos y los de abajo
ponían una barrera de lenguaje, neologismos extraños, chistes y camelos,
mezclados de una galiparda insubstancial.

Citaré el caso de uno de estos mancebos de cultura somera y ademanes
finústicos, que, tras una temporadilla de dos semanas en París, volvió
acá reventando de exquisitismo europeo. Su refinamiento no excluía el
gusto extravagante de algunos manjares españoles tan ordinarios como
sabrosos. En suma, que le gustaba con delirio el plato llamado
\emph{callos}. Entró a cenar con varios amigos en uno de los mejores
restaurantes de Madrid; mas no se atrevió a pedir el comistraje de su
gusto con el nombre español, que a su parecer era lo más contrario al
buen tono. Después que sus amigos pidieron lo que les vino en gana, él
dijo al mozo: «Para mí traiga usted\ldots{} A ver, a ver\ldots{} ¿Cómo
se llama eso?\ldots{} Ya, ya\ldots{} \emph{tripe à la mode de Caen}.»

\hypertarget{viii}{%
\chapter{VIII}\label{viii}}

Confundidos Casiana y yo entre el gentío fastuoso y el de medio pelo que
paseaba en la Castellana o el Retiro solíamos encontrarnos con
\emph{Leona la Brava}, acompañada de su amiga María Ruiz. Una tarde,
bajando de la Casa de Fieras al \emph{Parterre}, nos sorprendió la voz
de Leonarda, a quien vimos bebiendo un vaso de agua en la Fuente
Egipcia. No iba con María Ruiz sino con una doncella de servir llamada
Pilar, que a Casiana conocía por haber dado juntas no pocos pasos en las
correrías mundanas. Reunidas las tres mujeres y yo, seguimos
deambulando.

\emph{Leona}, que en otras ocasiones había mostrado simpatía por
Casiana, estuvo aquella tarde más expresiva, diciéndole entre otras
cosas amables: «Mujer, no te des tanto tono. ¿Por qué no has ido a mi
casa como me prometiste aquella noche que nos vimos a la salida de la
Zarzuela? Tendré mucho gusto en que comas conmigo. Después de comer
iremos al teatro, donde se nos agregará tu gallardo caballero, que no
vive separado de ti.»

Contestaba Casiana modosita y con infantil cortedad\ldots{} Balbuciente,
ya se excusaba con finura encogida, ya contemporizaba prometiendo
acceder a la invitación. La Pilar, aunque se hallaba en servidumbre,
miraba con cierta protección compasiva a la pobre Casiana,
considerándose como término medio entre el esplendor de su ama y la
obscuridad de la que en otros tiempos fue su igual en la vida galante.

Desmedido era el contraste entre la vestimenta magnífica y un poquito
estrepitosa de \emph{Leona} y los trapos caseros de mi humilde amiguita.
Esta me había dicho mil veces que no sentía envidia de la dama de Mula,
a pesar del rumbo que gastaba, y andando el tiempo me dio pruebas mil de
su encantadora modestia. Cuando salíamos del Paseo de las Estatuas a la
calle de Alfonso XII, me dijo \emph{La Brava} con su poquito de
misterio:

«Este año tardaré un poco en salir a mi veraneo, porque Alejandrito
tiene un asunto\ldots{} un negocio\ldots{} un proyecto de ferrocarril
que ha de ir por Miraflores a Segovia y La Granja\ldots{} ya te
contaré\ldots{} y hasta que no se lo despachen no saldremos\ldots{} No
sé si sabes que los moderadotes están que echan bombas: todo lo quieren
para sí, \emph{les belles places, les gros affaires, la lune et le
soleil}\ldots{} Y a propósito: Alejandrito les ha vuelto la espalda,
arrimándose a Romero Robledo y a López de Ayala, que le han prometido
echar los bofes para sacar adelante su asuntillo. Cuando esto sea,
\emph{nous partirons pour la France}. Pasaremos una temporadita en
Arcachón y luego nos vendremos a Biarritz.»

Terminó \emph{Leona} sus confidencias diciéndome que Carlota Pastrana se
iría pronto a San Juan de Luz, y que María Ruiz estaba \emph{aux abois},
porque \emph{el suyo}, que era empresario de casas de juego, dio el
trueno gordo y tuvo que salir escapando de Madrid para que no le
matasen.

En la Cibeles nos separamos. Cuando íbamos hacia nuestra casa, la
discreta Casiana consagró a la dama de Mula estos juicios sinceros:
«Leonarda es linda, simpática y cariñosa. Viste muy bien y tira el
dinero que es un gusto\ldots{} Pues con todo eso, yo no quiero parecerme
a ella. Según tú, \emph{La Brava} y yo nos asemejamos en que las dos
hemos querido instruirnos para pasar de burras a personas. Pero no es lo
mismo, Tito. La de Mula hipa por la grandeza, aprendió el habla fina,
luego el francés, y todo su aquel es tratarse con hombres ricos. Busca
el boato, la bambolla, y así como otras se pintan la cara para ser más
bonitas, \emph{Leona} se pinta el alma con la ilustración para que se
enamoren de ella los Duques, los Príncipes y hasta los mismos Reyes.

»Yo soy de otra manera; no pretendo más que saber leer y escribir, y
unas miajas de Aritmética para llevar las cuentas de mi casa. Muy corto
es mi genio, pero más cortos son mis deseos. Con un poquitín de lo que
Dios reparte a sus criaturas tengo asegurada la felicidad: un hombre
bueno que me quiera, una casa modesta y limpia, un pasar mediano y sin
ahogos, un vivir tranquilo, cuidar a mi hombre y tenerle todo a punto y
muy arregladito, y para colmo de contento mi plancha, mi aguja y mi
estropajo.»

Entre San Juan y San Pedro, entrada de verano, cambiamos Casiana y yo el
escenario en que exhibíamos nuestras bien aderezadas personas.
Abandonamos la Castellana y el Retiro, y vestidos cómodamente y sin
pretensiones nos íbamos por las tardes a la Fuente de la Teja o a la
Pradera del Corregidor. La libertad del vivir plebeyo al aire libre nos
encantaba, mayormente cuando llevábamos merienda o cena y nos la
comíamos tumbaditos sobre la hierba.

Era nuestra delicia la sociedad de los ventorrillos, donde escuchábamos
las conversaciones más graciosas; los musiquejos mendicantes nos
divertían, y el vocerío alegre regocijaba nuestros corazones. Por cierto
que una tarde encontramos a María Ruiz, una de las amigas de
\emph{Leona}, paseando del brazo de un gallardo sargento de Caballería.
Al poco rato bailaban una mazurca, bien agarrados, al son de los
atronadores organillos. Otra tarde se nos apareció el masón llamado
burlescamente \emph{Epaminondas}, a quien conocí en la tertulia de
\emph{Candelarita Penélope}. Le convidamos a merendar en un ventorro;
aceptó, y apenas nos sentamos los tres, empezó a discursear de esta
manera:

«Ya tenemos a Periquito hecho fraile, ya tenemos a Sagasta metido en la
legalidad. ¿No leíste la semana pasada el artículo de \emph{La Iberia}?
Pues bien claro lo dice. Los elementos procedentes del amadeísmo y del
unionismo, juntamente con los restos del antiguo progresismo que no
están con Zorrilla, quieren ahora formar un partidito que a un tiempo se
llame liberal y borbónico. ¿Entiendes esto; lo entiende usted, señora?

---Sí que lo entiendo, querido \emph{Epaminondas}---respondí yo.---Ni el
\emph{elemento} liberal ni el \emph{elemento} borbónico quieren perecer.
Para vivir y pescar lo que se pueda, se alían, se juntan, y buscan
\emph{un dogma} que encuentran en seguida\ldots{} Aquí hay \emph{dogmas}
para todo, hasta para las combinaciones y mezcolanzas más
extravagantes\ldots{} Encontrada la fórmula, se aprestan todos a
\emph{comulgar} en la iglesia alfonsina que hoy abre de par en par sus
puertas al culto del Funcionarismo. No te asustes de nada,
\emph{Epaminondas}. Sagasta formará un partido liberal dinástico que
alterne con el de Cánovas en la gobernación de estos Reinos venturosos.

---A eso iba---prosiguió el masón, mostrando en su rostro el júbilo y la
vanagloria de contar un suceso que él solo sabía.---Óyeme. Puedo
asegurarte, como si lo hubiera visto, que ayer y hoy se han reunido
Sagasta y Cánovas en casa de este último, Fuencarral, 2. Encerrados
estuvieron más de dos horas cada día, tratando de\ldots{} La
conversación entre ambos prohombres no he de referírtela, porque no la
oí\ldots{} Pero te diré, si te interesa saberlo, la hora exacta con
minutos en que entró Sagasta y la hora en que salió. Lo sé por Ramón, el
ayuda de cámara de don Antonio, que es paisano y amigo mío, y todo me lo
cuenta\ldots{} Total, es claro como el agua que los empingorotados
corifeos conferenciaron acerca de la forma y modo de fundar el nuevo
partidito, \emph{bajo la base} del equilibrio de los \emph{elementos}
dinásticos, conforme al \emph{credo} borbónico.

---En mi sentir---respondí yo---todo lo que me has dicho es la pura
realidad. Por mi parte, debo declarar que no patrocino el nuevo partido
ni me opongo a su creación, y así lo hago por dos razones: la primera es
que sucederá lo que debe suceder, y la segunda, que todo ello me tiene
sin cuidado.»

Disertamos un poco más sobre el asunto, cada cual según su temperamento
y estilo, hasta que el amigo \emph{Epaminondas} se fue con unas mozas
barbianas que salieron del merendero próximo.

Transcurrieron días calurosos, tardes de holganza placentera en las
soledades campesinas, noches serenas que empezaban tibias y concluían
con dulce frescura matinal. Más de una vez, la aurora risueña nos
acompañó a Casiana y a mí al tornar a nuestra vivienda.

El primer suceso público que relatan mis crónicas en la declinación del
verano fue la recrudescencia de las sofoquinas que a don Antonio daban
los \emph{moderados}. Los antagonismos en el seno del Ministerio
parecían ya irreductibles. Se tiraban los trastos a la cabeza por si las
primeras elecciones de la Restauración habían de hacerse con el sufragio
universal o con el restringido. Cánovas del Castillo, que a sus grandes
talentos unía un arte sutil para deshacerse de los revoltosos y amansar
a los díscolos con el sencillo gesto de abandonar el Poder, dejando tras
sí como emblema de castigo el vacío de su persona, inventó un Ministerio
Jovellar que fue plasmado rápidamente en esta forma: Romero Robledo,
Ayala y Salaverría conservaron sus carteras de Gobernación, Ultramar y
Hacienda. En Guerra, con la Presidencia, quedó Jovellar. Y entraron: en
Estado, don Emilio Alcalá Galiano, Vizconde del Pontón; en Fomento, don
Cristóbal Martín Herrera; en Gracia y Justicia, don Fernando Calderón
Collantes, y en Marina, Durán y Lira.

Heroico remedio fue para la turbada política el mutismo de don Antonio,
mejor dicho, medio mutis como los que en las acotaciones de las comedias
se designan con la siguiente fórmula: \emph{hace que se va y se queda}.
Para estos pasos escénicos tenía el maestro Cánovas una singular
destreza, casi estoy por decir travesura, y de ello dio nuevos ejemplos
en posteriores épocas de su mando. El flamante Ministerio correspondió
dócilmente a los fines que motivaron su presencia en el retablo
político, y el 1.º de Octubre, tras una gestación que no debió ser muy
laboriosa, la señora \emph{Gaceta} dio a luz un decreto estableciendo
que el nuevo Parlamento se formaría con arreglo a la ley electoral de
1870. El sufragio universal había vuelto a levantar la cabeza, y los
\emph{moderados}, con excepción del inflexible don Claudio Moyano,
bajaron la cresta convencidos de que se quedarían fuera de la
circulación política si continuaban encerrados en las covachas del
tiempo viejo.

Desembarazado de los engorrosos obstáculos que le ocasionó la cuestión
electoral, Cánovas volvió a ser cabeza visible de la Situación en la
Presidencia del Consejo. A Jovellar dio el mando supremo de Cuba,
prebenda que fue muy del agrado del General. En Guerra entró Ceballos;
en Fomento el Conde de Toreno. Martín Herrera pasó a Gracia y Justicia,
y don Fernando Calderón Collantes a Estado. Los demás Ministros, excepto
Alcalá Galiano, siguieron en sus puestos.

Ante un público de amigos inquietos y ambiciosos, congregado en el Circo
del Príncipe Alfonso el 7 de Noviembre, celebró Sagasta con endechas
tribunicias el advenimiento del partido liberal monárquico y la
felicidad que había de resultar del turno pacífico, del equilibrio, del
balanceo metódico entre los dos \emph{elementos} que diferenciaban e
integraban la política general, sirviendo a la Nación y al Rey cada cual
con su \emph{credo}, cada cual con su \emph{dogma}, sin perjuicio de
\emph{comulgar} ambos en el ideal común, en el ideal dinástico,
etc\ldots{} No expresó don Práxedes su pensamiento con los vocablos y
frasecillas que aquí empleo. Yo no asistí a la reunión; pero creo
interpretar fielmente la substancia del discurso utilizando las notas
tomadas al oído que me trajo el diligente informador \emph{Epaminondas}.

Que Sagasta puso en las nubes la Constitución del 69 y pisoteó la del
45, no hay para qué decirlo. Hizo un discreto elogio de los derechos
individuales y de la libertad de conciencia, armonizando estas
conquistas con el estricto mantenimiento del orden, y concertó las notas
chillonas del \emph{Himno de Riego} con la grave salmodia de la
\emph{Marcha Real}. El Partido Constitucional combatiría con el mismo
ardimiento los excesos de la demagogia y las atrocidades de la
reacción\ldots{} Todo iba bien, muy bien. Los liberales dinásticos,
provistos ya de las necesarias recetas para entrar con salud en la
política activa, andaban por Madrid a fines del 75 como chiquillos con
zapatos nuevos. Faltaba que el Gobierno convocase al pueblo a los
comicios, que se efectuaran las elecciones, y que se supiera quiénes
salían triunfantes del seno hermético de las urnas.

Perdonadme, lectores de mi alma, que pase como gato fugitivo por este
período de una normalidad desaborida y tediosa, días de sensatez
flatulenta, de palabras anodinas y retumbantes con que se disimulaba el
largo bostezar de la Historia. Todo este fárrago de convencionalismos
resobados pasó de las manos caducas del año 75 a las tiernas manecitas
del 76. Funcionó el artefacto electoral, y para haceros comprender su
eficacia me bastará decir que Romero Robledo estrenó entonces su
extraordinaria maestría en la fabricación de Parlamentos. Con tiempo y
saliva designó y encasilló a los padres de la Patria, formando a su
gusto el montón grande de la mayoría conservadora y el montón chico de
la minoría liberal dinástica, sin olvidar unas cuantas figuras sueltas,
sacadas de las urnas o de los cubiletes con un fin ornamental y
pintoresco. Fue al Congreso Emilio Castelar por el cariño que Cánovas le
tenía, y para que no estuviera solo pusieron a su lado al señor Anglada.
Una vez más, y aquella vez más que otras, lució sobre Madrid y España la
espléndida mentira de la Soberanía Nacional.

Ya sé, ya sé que mis lectores me agradecen mucho que no les cuente la
teatral apertura de las Cortes el 15 de Febrero de 1876, con la fastuosa
mascarada palatina, ni el discurso del Rey, ni los subsiguientes
trámites rutinarios de elección de Mesa, examen de actas y constitución
definitiva en las dos Cámaras. Todo esto, visto a cierta distancia, es
aburridísimo, letal, y el que lo contase de buena fe o lo leyere con
paciencia moriría de un ataque agudo de fastidio. Las Cortes alfonsinas
habían de empezar sus tareas pergeñando una nueva Constitución, pues la
del 12, la del 37, la del 45, la del 54 y la del 69, todas incumplidas,
o \emph{barrenadas} como suele decirse, estaban ya inservibles.

Aunque el pío lector no me lo agradezca, doy de lado la discusión del
Mensaje, juego de pirotecnia verbosa en el cual cada orador respiraba
por sus heridas, conforme a la postura política en que le habían dejado
los sucesos de los últimos años. Pidal se revolvía contra don Antonio
por no haber traído este a la Restauración las furias ultramontanas;
Moyano execraba la Revolución de Septiembre, pintándola como un criminal
esparcimiento demagógico; Sagasta, cantando por todo lo alto, izaba el
gallardete de la Soberanía Nacional; Castelar y Pavía disertaron
extensamente sobre el pro y el contra del 3 de Enero del 74; Cánovas,
con derroches de lógica elocuente, contestaba a unos y otros
requiriéndoles a la paz y concordia en los altares de la legalidad
alfonsina; todos, en fin, se encastillaban en las ficciones o decorosas
pamplinas que les servían de plataforma en aquella encrucijada de los
destinos de España.

Sospecho que estas páginas tendrán más amenidad hablando en ellas de mí
mismo, de la honda depresión de mi ánimo en aquellos días de amodorrante
sensatez. Sin que pudiera decir que estaba enfermo, yo me sentía
desganado y triste; apenas salía de mi casa; ni una sola vez traspasé la
puerta del Congreso; huía de la rarificada atmósfera de los que llaman
\emph{Círculos}, y para colmo de mi desdicha, en los meses transcurridos
del año 76 no me visitó la vaga \emph{Efémera}, ni tuve más relaciones
con mi adorada Madre que la cobranza de mi asignación en la portería de
la Academia de la Historia, sin que a la entrega de fondos acompañara
carta ni referencia directa de la divina \emph{Clío}. Llegué a creer que
mi Madre yacía en grave postración espiritual o que se hallaba en estado
de catalepsia, única enfermedad que acomete a los Dioses cuando no
tienen nada que hacer, o se creen dispensados de intervenir en las
acciones humanas.

También la vida de este pobre Tito había llegado a ser vida de durmiente
o cataléptico. Sus horas se deslizaban una tras otra lentas, pardas y
sin ruido. El ayer, el hoy y el mañana eran un solo día: esfumábanse los
recuerdos, extinguíase la esperanza\ldots{} De improviso, una noche me
sacudió y me puso en pie restituyéndome bruscamente a mi ser normal un
suceso inopinado, un relámpago de vida, la visita de un amigo
queridísimo a quien yo no había visto en algunos años. Este amigo era
Segismundo García Fajardo, el rebelde más tenaz y el revolucionario más
gracioso que ha existido bajo el limpio cielo de los Madriles.

En los días trágicos de la muerte de Prim y en todo el año 70, fecundo
en emociones y disturbios, derrochó Segismundo su agudeza satírica y los
donaires de su feliz ingenio en soliviantar las masas populares de
Lavapiés y las Peñuelas. Grande amigo de Romualdo Cantera, recibió de
este albergue y sustento en los azares de la vida más desordenada y
tormentosa que cabe imaginar. Aquel trueno de la política, bala perdida
en la sociedad, era como sabéis sobrino carnal del Marqués de Beramendi,
caballero talentudo y de alta posición, que se cansó de proteger al mozo
cuando las extravagancias de este llegaron a ser escandalosas.
Abandonado del tío y de sus padres, Segismundo se dejó arrastrar por la
desesperación revolucionaria, y aunque no tuvo arte ni parte en el
conato de regicidio contra don Amadeo, fue perseguido con tanta saña que
salió por pies y no paró hasta París. En aquella capital permaneció
largo tiempo entre los innúmeros españoles que conspiraban para cambiar
radicalmente las cosas de España.

Cansado, al fin, de soportar humillaciones, hambre y desnudeces, se
valió de sutiles arbitrios para repatriarse. Atravesó toda Francia
empleando los más inverosímiles medios de locomoción gratuita, y
protegido por un fogonero vino de Irún a Madrid\ldots{} Cuando ante mí
se presentó, su rostro estaba tan desfigurado por la miseria y su
vestimenta era tan haraposa que hubo de decirme su nombre más de una vez
para que yo pudiera reconocerle\ldots{} Le abracé conmovido, hícele
sentar a mi lado, y él, con voz doliente y asmática, eco de un cuerpo
vacío, me dijo: «No vengo a pedirte albergue, querido Proteo, que ese,
aunque no mejor que la guarida de una bestia, ya lo tengo. Vengo a
pedirte un pedazo de pan\ldots»

\hypertarget{ix}{%
\chapter{IX}\label{ix}}

Mi respuesta fue dar voces llamando a Ido para que nos sirviera al
instante la cena. «Cenarás conmigo---dije a Segismundo,---y con esta
señorita, Casiana \emph{Coelho}, que si no es ya una profesora de
instrucción primaria, lo será muy pronto. Ya sabes que diariamente,
desde esta noche, habrá siempre en mi mesa humilde un plato para ti.»
Por causa de la turbación de su ánimo, o quizás por la vacuidad de su
estómago, el pobre Segismundo no pudo expresar su gratitud más que con
truncadas frases expresivas.

Apenas tragó García Fajardo las primeras cucharadas de sopa y media copa
de vino, pudo advertirse que recobraba su perdido vigor. Ya era otro
hombre, y a medida que avanzaba en la ingestión de alimento, su gesto
hacíase menos desmayado y su voz más segura y vibrante. «Gracias a mi
antigua camarera y aposentadora, la benéfica \emph{Señángela}---nos
dijo,---no duermo a la intemperie. Aquella fiera, tan deslenguada como
caritativa, me ha dado cobijo en un cuchitril inmundo de la calle de
Cabestreros. Allí tengo unos palmos de terreno donde estirarme, sobre un
montón de trastos y rollos de esteras. El amigo Balbona ya no está en la
taberna de la calle de Toledo, y Romualdo Cantera se ha ido a vivir
lejos de Madrid\ldots{} Todo mi guardarropa se reduce hoy a estos
venerables guiñapos que ves colgados sobre mi cuerpo.

---No te apures, noble hijo de España---le contesté yo.---Nosotros te
proveeremos de ropa con algunas prendas mías y otras del amigo Ido, que
próximamente mide tu estatura. Todo es cuestión de tijera y aguja. Aquí
tenemos a Casianita, que es una gran sastra y arregladora de vestimentas
para todos los gustos. Te adecentaremos\ldots{} no te rías\ldots{} y
podrás salir a la calle con elegancia de figurín barato. Ya sabes que la
elegancia es el signo de los tiempos. Bien apañadito, como un estirado
señorete que viene de París, podrás presentarte a tu ilustre tío el
Marqués de Beramendi, y a tu amigo Vicente Halconero.»

Poniendo breves pausas en el buen comer, mi huésped replicó así: «En el
fondo y aun en la superficie de su espíritu, mi tío Beramendi es un
rebelde a macha martillo; pero su mujer, sus hijos y la sociedad en que
vive no le permiten sustraerse a esta atmósfera de artificios
convencionales y de mentiras aparatosas. Los hombres de ideas más
avanzadas se vuelven suspicaces y medrosicos, y se acomodan a vegetar
dentro de esta cárcel fastidiosa de la sensatez monárquica, mayormente
si poseen buenas rentas para tratarse a cuerpo de rey mientras dure su
cautiverio. En cuanto los jesuitas establezcan aquí esos Colegios
elegantes de que ya se habla, los primeros niños que entren en ellos
serán los de mi tío Pepe. Así lo quiere María Ignacia y así será.

»Lo mismo te digo de Vicentito Halconero. Es un chico excelente, talento
claro de los que miran al porvenir y a la regeneración de este pobre
pueblo. Pues hostigado por su madre, Lucila, y por sus suegros los
Calpenas, solicitó el acta de La Guardia; le encasilló Romero Robledo, y
ahí le tienes, entre los borregos de Cánovas\ldots{} no, me
equivoco\ldots{} entre los de Sagasta, que viene a ser lo mismo. Te diré
ahora que la hermosa Lucila, al cabo de los años, se siente un poco
ultramontana y papista. No hace mucho tiempo hizo un viaje a Roma con su
esposo don Ángel Cordero, el sutil economista, sin otro objeto que besar
la sandalia de Pío IX, y recibir la bendición pontificia\ldots{} Con que
ya sabes, a esta sociedad que me execra y me maldice, no puedo yo
acercarme sin recibir desaires y sofiones.»

Avanzada ya la cena, añadió Segismundo a las manifestaciones anteriores
confidencias de un orden más delicado. Poniendo en su acento el respeto
que a su madre debía, díjome que esta, Segismunda Rodríguez, esposa del
primogénito de los García Fajardo, se había dedicado en los últimos años
al negocio de préstamos usurarios, y laboraba sigilosamente tras la
pantalla de testaferros sin conciencia. Amasado un grueso capital
desplumando lindamente al prójimo, la buena señora hipaba por la
grandeza y era rabiosa alfonsina. Se desvivía por pescar un título
nobiliario, y no siéndole fácil conseguirlo de los de Castilla
resignábase a tenerlo pontificio, que como es sabido resultan muy
económicos.

De sobremesa volvimos a tratar la cuestión de indumentaria. Casiana,
movida de repentina inspiración, sacó de su cesta de costura la
cinta-metro que usan los sastres y modistas, y puesto en pie Segismundo,
le tomó las medidas a lo ancho y a lo largo. La señorita de
\emph{Coelho} cantaba los números y yo los iba apuntando en un papelejo.
Hecho esto, y cuando Segis se despidió con demostraciones de gratitud,
bien provisto de tabaco, le aseguré que a la tarde siguiente encontraría
en mi casa el remedio de su indecorosa desnudez.

Coincidiendo en una resolución práctica, habíamos pensado Casiana y yo
que la más expedita obra de misericordia era vestir al desnudo con un
traje de \emph{El Águila}. En efecto, a la mañana siguiente adquirimos,
por las medidas que llevábamos, un terno modestito y de buen ver. Luego,
en la calle de Toledo, compramos tres camisas y otras prendas
interiores, a las cuales agregamos un sombrerete blando adquirido en
\emph{Las Tres B B B} de la Plaza Mayor\ldots{} Con toda esta carga nos
volvimos a casa satisfechos y gozosos, pues nada era tan grato para mí,
y lo mismo para Casianilla, como aplicar nuestros limitados recursos a
una obra esencialmente cristiana y altruista.

Por la tarde, cuando se nos presentó el infeliz repatriado y le
mostramos las para él lujosas prendas de vestir, advertimos que se
humedecían sus ojos y que su boca tembliqueante no acertaba a formular
las oportunas frases de reconocimiento. Con un tonillo evangélico, que
maquinalmente me salía del pensamiento a los labios, le hablé de este
modo: «Amigo, mejor será decir hermano mío, coge estas ropas y tenlas
por tuyas sin reparar en la mano que te las entrega; corre a tu morada,
y una vez que purifiques tus carnes con santas abluciones, vístelas con
la decencia que Dios te ha deparado.»

El hombre infeliz, recogiendo parte de su equipo para hacer con él un
lío, me contestó en el tono más sencillo y familiar: «Benditos sean los
que practican el amor al prójimo con verdad y donosura. Muchos se
precian de socorrer a los desvalidos; pocos hay que posean el arte de la
caridad. Yo acepto estos dones y admiro la gracia con que se me
ofrecen\ldots{} Permitidme, mis queridos amigos, que no traslade a mi
casa toda la ropa interior; me llevo sólo una muda; lo demás aquí queda,
pues mi desmantelado cubil se me antoja que no es, no ya el
\emph{Puerto}, sino el Golfo \emph{de Arrebatacapas}.»

Con toda la presteza que su contento le infundía, el desgraciado y ya
favorecido Segismundo partió, llevándose su ropa envuelta en un pañuelo.
Casiana y yo nos quedamos discurriendo nuevas manifestaciones del arte
de la caridad. Al otro día sorprendíamos al menesteroso caballero con
una pañosa nuevecita y unas botas de becerro mate adquiridas en un bazar
de calzado. Todo resultó a las mil maravillas: cuando resurgió a media
mañana el amigo, bien lavoteado y vestido de limpio, parecía otro.
Obsequiole Casiana con unas corbatitas de colorines en las que había
trabajado la noche anterior. El espléndido regalo final de la capa y
botas puso al buen Segismundo en un estado de beatitud seráfica. Yo
reventaba de gozo, Casianilla no cesaba de reír, y los dos creíamos
hallarnos en presencia de un muerto a quien acabábamos de resucitar.

Tras un largo rato de ocioso charloteo, en que intervino Ido con su
cándido filosofismo, nos sentamos a la mesa. El muerto resucitado, dueño
ya de los varios registros de su inteligencia, nos contó interesantes
casos y episodios del vivir azaroso de los emigrados españoles en París.
Habíalos allí de todas castas y procedencias: republicanos federales del
73, zorrillistas de la última extracción con afiliados civiles y
militares, carlistas de todas las épocas, especialmente de la última,
pues la causa de la legitimidad iba de capa caída y muchos partidarios
del Pretendiente pasaban la frontera ansiosos de buscarse la vida en un
país pacífico y libre. El \emph{Pasaje Jouffroy} y el \emph{Café de
Madrid} hervían de españoles aburridos y famélicos. Algunos, embozados
en sus capitas, acechaban el paso de un amigo que les diera un Napoleón
o les convidase a un almuerzo de \emph{dos francos cincuenta}; otros se
instalaban en las mesas del café, y allí pasaban largas horas en tristes
añoranzas, o planeando medios de trabajo para poder matar el gusanillo.
Los más prácticos apencaban con los rudos oficios y se metían en una
cerrajería, en una tahona o en talleres de encuadernación.

«Me han contado---dije yo---que republicanos y carlistas fraternizan
allí, unidos por la común desgracia, y se buscan la vida dando lecciones
de español.

---Así es---prosiguió Segis.---Yo me asocié con un ex-capitán carlista,
natural de Azpeitia, excelente chico, que no hablaba bien más que el
vascuence. Pereciendo de hambre, anunciamos una \emph{Gran Academia de
Lenguas} en la cual, el vascongado y yo, y un andaluz muy despierto que
se nos agregó, ofrecíamos dar lecciones de español, de latín y de
griego. El resultado fue desastroso\ldots{} Debo añadir que de la
emigración zorrillista poco podíamos esperar, porque los prosélitos de
don Manuel, mal que bien, tenían para vivir y se cuidaban poco de los
demás, como no fuera para darnos de vez en cuando un corto auxilio.

»De Ladevese recibí yo algún socorro que le agradeceré toda mi
vida\ldots{} La conspiración zorrillista labora en España tratando de
mover las fuerzas militares para producir los tan acreditados
pronunciamientos. En París se manifiestan con un \emph{ojalaterismo}
rosado y transparente que a muchos deslumbra, a mí no, pues de los
pronunciamientos no espero nada bueno para mi Patria\ldots{} Desesperado
de la inutilidad de mis esfuerzos para resolver el problema vital,
abandoné el \emph{Pasaje Jouffroy}, donde todo se volvía cháchara sin
substancia, y planté mis reales en el \emph{Café Cluny, Boulevard Saint
Michel, Barrio Latino}.

---Dime, Segis, ¿no has visto por allí a Estévanez?

---Sí; pocos días antes de mi salida, llegó de Portugal. Está muy
desalentado, y cree que todo intento revolucionario, ya sea zorrillista,
ya sea de otro orden, quedará hecho polvo bajo el peso de esta
oligarquía de tres cabezas: la femenina aristocrática, la militar
masculina y la papista epicena\ldots{} Como decía, me instalé muy a
gusto en el \emph{Barrio Latino}, que es para mí el París luminoso, la
urbe de la ciencia y el arte. Allí están todos los focos del saber y de
la enseñanza pública; allí están la Sorbona, el \emph{Collège de
France}, la Universidad; allí las Escuelas Superiores de Medicina, de
Farmacia, de Ingenieros, el Observatorio Astronómico, innumerables
Institutos, Laboratorios y Bibliotecas; allí todos los grandes editores
de París; allí, en fin, la inmensa cátedra de escolares, estudiosos los
unos, otros afiliados a la graciosa hermandad que llaman \emph{bohemia}.
Sobre este inquieto y juvenil personal flota la nube de poetas más o
menos \emph{parnasianos}, y de pintores más o menos
\emph{impresionistas}.

---¡Hermosa y florida República---exclamé yo,---esperanza de un gran
pueblo!

---En el \emph{Café Cluny} y en otro que está junto al \emph{Odeón},
tenía yo mis Círculos predilectos. Hice amistad con unos chicos
mejicanos y chilenos, pensionados para estudiar Medicina. Sociedad más a
mi gusto jamás la conocí. Los americanillos eran estudiosos, y de la
piel del diablo. Ellos, y un pintor español que hacía paisajes
melancólicos, me arrastraron a la \emph{bohemia}, para lo cual es
condición precisa tener los bolsillos vacíos. Gocé y me divertí cuanto
pude, y mis calaveradas extravagantes dejaron memoria en aquel rincón
del París ático y bullicioso. Para que nada me faltase, tuve mi
\emph{griseta}, que me adoró durante dos días y medio.

»También aquel barrio era campo de acción de muchos expatriados
españoles, que se administraban por un presupuesto absolutamente
negativo. Con algunos de estos me lié yo en sociedad comanditaria al
objeto de \emph{arbitrar recursos} honradamente. Un tal Boneta,
cantonal, me propuso un negocio que consideraba de resultados
infalibles. ¡A trabajar se ha dicho! Alquilamos una tienda en la
\emph{rue Grenelle}, y nos instalamos en ella sin muebles ni cosa
alguna. Pero en la fachada pusimos este anuncio sugestivo,
\emph{Misterios de la vida parisién}, y en la puerta un rotulillo que
decía en letras bien claras, \emph{Entrada, un franco}. A mi cargo
corría la cobranza, mientras Boneta se paseaba en el salón vacío. El
primer día cayeron algunos incautos, que al ver aquellas paredes
desnudas preguntaban: «¿Pero qué es lo que se enseña aquí?» Boneta
contestaba con voz estruendosa: \emph{¡Rien!} Intervino la Policía
obligándonos a cerrar \emph{el establecimiento}. Con los francos
recaudados tuvimos para cenar algunas noches.

---Esa broma o ese timo, querido Segis---repuse yo,---no habríais podido
darlo en Madrid.

---Claro es---siguió diciendo el pícaro.---Pero tú no sabes que París es
el pueblo más novelero del mundo. Verás ahora otro caso de la
maravillosa inventiva de un emigrado español muerto de hambre. Un tal
Catuelles, carlista, anunció en la prensa que estaba dispuesto a
reconocer todos los hijos ilegítimos no reconocidos por sus padres. En
el anuncio, redactado con frases muy patéticas, declaraba que lo hacía
por lástima de las pobres criaturas, y deseoso de que estas pudieran
entrar decorosamente en la vida social. Lo demás ya se supone:
\emph{precios convencionales}. Pues este hombre que en España habría
pasado por loco, en París y en poco más de seis meses, reconoció ciento
dieciocho hijos y ganó doce mil duros.

---¡Ay qué gracioso, qué hombre más listo!---exclamó Casiana riendo a
carcajadas.---Pero usted, don Segis, ¿qué intentaba para ganar dinero y
salir de su miseria?

---¡Ah, hija mía! Yo no tenía la travesura de Boneta ni el genio de
Catuelles. Cuando llegué a los extremos de la necesidad me dejé llevar
por dos amigos, uno cantonal y otro carcunda, a las conferencias
religiosas que en cierta calle próxima a \emph{San Sulpicio} daba una
Sociedad Catequista. Aunque mis dos compañeros eran librepensadores,
casi ateos, y yo no tengo creencias religiosas, apencábamos con aquella
farsa porque los catequizadores recompensaban nuestro falso catolicismo
con un modesto socorro. Por las noches nos hacían oír unas pláticas
estúpidas y soporíferas. Pero ¡ay!, esto no bastaba: querían los señores
dar público espectáculo de nuestra piedad y mansedumbre, como éxito
notorio de la labor catequizante y triunfo de Nuestra Santa Madre
Iglesia. Eramos como unos doscientos, entre hombres menesterosos y
beatas vejanconas. Todas las mañanas nos llevaban a confesar y comulgar
en \emph{San Sulpicio}, y hasta que ingeríamos el pan espiritual no nos
daban el franco, óbolo remunerador de nuestras edificantes devociones.

---¡Pero tú comulgabas, Segis, tú\ldots!---exclamé yo, vacilando entre
la incredulidad y la risa.---¿Es posible?

---¡Ya lo creo! Como que si no comulgaba no comía\ldots{} ¡Ay, amigos
del alma! Si ahora que estoy decentito me decido a presentarme a mi
madre, ya sé lo primero que me dirá. Me parece que la estoy oyendo:
«Hijo mío, ¿vienes dispuesto a sentar la cabeza y a enmendarte de tus
errores? Si así es, tu madre te bendice, y lo primero que te recomienda
es que entres resueltamente en la grey cristiana y cumplas con la
Iglesia.» Yo le responderé: «¡Ah, madre querida!; bien \emph{cumplido} y
purificado vengo de París. Traigo \emph{cumplimiento} para lo que me
resta de vida.»

\hypertarget{x}{%
\chapter{X}\label{x}}

Desde aquel día, el náufrago salvado de las olas del infortunio quedó
unido a mí por vínculos fraternales. Casiana y yo partíamos el pan y la
sal con Segismundo, y él nos mostraba un cariño respetuoso que más
parecía veneración. Juntos salíamos los tres de paseo, tranquilos,
alegres, \emph{ni envidiados ni envidiosos}, y por las noches no
perdonábamos nuestra partidita de café en los de Zaragoza, Venecia o San
Sebastián donde poníamos el paño al púlpito despotricando, ora en tonos
enérgicos, ora en sarcástico estilo, contra la oligarquía dominante.
Aunque perorábamos para una posteridad remota, los parroquianos que nos
oían con la boca abierta celebraban nuestras locas arengas, cual si en
ellas viesen una palpitante actualidad.

En nuestra casa teníamos luego una segunda \emph{soirée} más interesante
y divertida, porque en ella gozábamos la inefable libertad del disparate
sin acortar el vuelo de nuestros arrebatados pensamientos. Reforzada
nuestra trinca con la conspicua personalidad de Ido del Sagrario y la de
un estudiantillo muy despierto llamado Gayoso, recorríamos hasta lo
infinito los espacios quiméricos.

Allí se oyeron afirmaciones aplastantes y atrevidísimas hipótesis. Por
ejemplo, oid a Segismundo: «Si en España viniera un cataclismo,
\emph{pongo por caso}, como dice Orovio en sus discursos\ldots{} un
cataclismo, \emph{es un suponer}, que decía el General Infante, y
fuéramos llamados Tito y yo a ejercer la dictadura, ¿qué haríamos?» El
estudiante Gayoso saltó en seguida sosteniendo que no dominaríamos la
situación si no consagrábamos los tres primeros días de mando a cortar
cabezas, la mar de cabezas\ldots{}

De esto protestaba Sagrario, movido de un alto espíritu de humanidad, y
decía con enfático acento: «No se cuiden los señores dictadores de
cortar cabezas, sino de cortar abusos, y esto se hará fácilmente
blandiendo en una mano el cetro de la Ley y en la otra la antorcha de la
Verdad. Sí; con ley, verdad, justicia y honradez ciudadana todo irá como
una seda. Matar no, no. Me opongo a la horca y a la guillotina. Todo lo
más que admito es el cartel que diga \emph{pena de muerte al ladrón},
sólo como amenaza contra los timadores y descuideros.»

A esto repliqué yo adoptando un término medio entre los feroces
procedimientos de Gayoso y la indulgencia de don José. Este me
interrumpió con atinadas razones: «Yo lo fío todo \emph{al progreso}, y
harto saben los \emph{preopinantes} que \emph{el progreso} es benigno,
suave, mirando siempre a la Voluntad Nacional\ldots{} Ya que los señores
se dignan escucharme, les diré que no veo más dictadura que la del
denodado señor Duque de la Victoria.»

Tomó entonces la palabra Segismundo para expresar estas ideas, propias
de su elevado cacumen: «Yo, conforme con el sesudo Sagrario, enarbolo
los pendones de la ley, la verdad y la justicia; pero ¿cómo hemos de
salvar el espacio mediante entre los furores del cataclismo y la
normalidad fundada en esos ideales? Al constituirnos necesitamos
Ejército. ¿Cómo pasamos del pretorianismo indisciplinado a la posesión
de una fuerza regular que apoye la acción gubernativa? Será
indispensable conciliar los intereses de los ricos con el bienestar
relativo de los menesterosos. Hemos de crear un presupuesto novísimo,
descargando las cifras asignadas al Clero y Milicia para reforzar las
dotaciones de Enseñanza y Obras Públicas. Y yo pregunto a \emph{los
preopinantes}: ¿Cómo nos defenderemos de las fieras que, azuzadas por
esta radical alteración del presupuesto, caerán sobre nosotros ansiosas
de devorarnos? Por todo lo dicho y por algo más que se me queda en el
magín, yo renuncio a la dictadura que galantemente me ha ofrecido el
amigo Proteo, y la transfiero, como propone el señor Ido, al Príncipe de
Vergara, Duque de la Victoria y Conde de Morella.»

Casianilla, que había permanecido muda y atenta ante el varonil senado,
se arrancó al fin con este juicio tan tímido como discreto: «Déjenme
pedir a los señores \emph{opinantes} que no se devanen los sesos por la
incumbencia \emph{del dictado}, que entiendo es el encaminar a la Nación
para que del tumulto pase a la paz\ldots{} Porque yo digo, del mucho
orden sale siempre el desorden, \emph{es a saber}, los motines y la
rabia del pueblo, y de esto sale siempre la tranquilidad o \emph{verbo y
gracia} quedarse todo como una balsa de aceite. Dios Nuestro Señor ha
dispuesto que tras de la calma tengamos las tempestades y tras de las
tempestades la calma y el cielo sereno. ¿Qué viene cataclismo? Pues que
venga. El cataclismo se encargará de volver las cosas a la norma\ldots{}
o como se diga. ¿Me explico?»

Los cuatro le aseguramos que la entendíamos muy bien, y ella, cobrando
ánimos, concluyó de este modo: «No quiero que Tito ni Segismundo se
metan \emph{a dictar} estas cosas. Si España se alborota, ya sabrá ella
desalborotarse, y por lo que voy viendo, buen desalborotador será ese
Duque mentado por don José y que, según yo calculo, no es otro que el
señor de Espartero.»

Aplaudimos todos, y disolví la reunión. El primer suceso memorable del
día siguiente fue que Segismundo, al venir a casa, se encontró a
\emph{Sebo}, el cual ya tenía conocimiento de que en su repatriación
García Fajardo había mudado de piel como las culebras. Díjole Telesforo
del Portillo que el señor Marqués de Beramendi deseaba ver a su sobrino,
y que él tenía orden terminante de llevarle a su presencia de grado o
por fuerza. Yo aconsejé a Segis que se dejara querer, pues algo bueno
resultaría de su entrevista con el bondadoso prócer oligarca.

El segundo suceso histórico de aquel día fue la terminación de la guerra
civil. Desde fines del año anterior andaban muy atropellados los
carlistas. No tenían dinero, no tenían generales de empuje. El atontado
Carlos VII puso al frente de sus tropas a don Alfonso de Borbón y de
Habsburgo, Conde de Caserta, hermano del ex-Rey de Nápoles Francisco II,
e hijo en segundas nupcias de Fernando, el llamado \emph{Rey Bomba}. El
pobre Conde de Caserta, con toda la hinchazón de su regia prosapia,
carecía de dotes para regir una poderosa hueste en quien iba faltando la
interior satisfacción. En tanto, el Gobierno de Cánovas, viendo ya
maduro el fruto de la paz, organizó dos grandes Ejércitos con nutrido
contingente de todas armas, mandado el uno por Martínez Campos y el otro
por Quesada. El primero llevaba consigo a los Generales Blanco y Primo
de Rivera; Quesada iba en la compañía de hombres tan expertos y
conocedores del territorio como Moriones, Loma, Villegas y otros.

Ambos Ejércitos adquirieron fáciles ventajas, así en el suelo navarro
como en el país vascongado y límites de Santander. Martínez Campos
emprendió su famosa marcha hacia el Baztán, iniciando el movimiento
envolvente a lo largo de la frontera que pronto dio sus frutos. Primo de
Rivera, después de sacudir duras palizas a las partidas facciosas, no ya
Cuerpos de Ejército, en Santa Bárbara de Oteiza, La Solana y línea del
río Egea, entró en Estella el 19 de Febrero del 76. Tan importante
suceso y la victoria alcanzada por el General Blanco en Peña Plata
determinaron la desbandada de las tropas carlistas. Estas gritaban
\emph{¡traición, traición!} y en grupos salían por pies hacia el
Pirineo.

Segismundo García Fajardo, después de hablar con su tío el Marqués de
Beramendi, me refirió las opiniones de este sagaz hombre de mundo que
sabía poner la realidad por encima de los engañosos convencionalismos.
Según el Marqués, las ventajas obtenidas se debían en primer término a
la eficacia de las armas liberales, después al influjo de la plata
repartida entre los pobres carlistas, descalzos, hambrientos, aburridos
ya de un heroísmo inútil. Viendo ya seguro el fin de la guerra, Cánovas
dispuso que don Alfonso fuese al Norte a recoger abundante cosecha de
laureles. Entró el Rey en Tolosa el 21 de Febrero, aclamado por
alfonsinos y carlistas. Un batallón guipuzcoano se sublevó en Leiza a
los gritos de \emph{¡Mueran los traidores! ¡Nos han vendido!}, teniendo
que retirarse Carasa con su Estado Mayor y escolta, no sin que le
insultaran. El batallón de Guernica se insurreccionó contra sus jefes, y
en todas partes se repetía: \emph{Esto se ha concluido}.

Completo esta página histórica con otra que me dictó Segis. Dando a tal
página toda la importancia que merece, la copio al pie de la letra: «Mi
tío Pepe me recibió con benévola conmiseración. Oyó el relato que tuve
que hacerle de mis andanzas y miserias, y al reprenderme por mi vida
borrascosa, atenuaba su severidad con inflexiones regocijadas. Harto
conocía yo la rebeldía interna, así en lo político como en lo social, de
mi señor tío; pero yo era pobre y él rico, yo no tenía casa ni hogar y
él vivía en la dorada farsa de un mundo artificioso. Por esta
fundamental diferencia, la rebeldía y el dogmatismo revolucionario de
Beramendi eran no más que un adorno mental, florecillas del espíritu que
el buen prócer sacaba a relucir tan sólo en la intimidad de sus amigos.

»También María Ignacia, que al oír mi voz entró en el despacho, mostrose
conmigo indulgente y compasiva. Tratando ante mí de aliviar mi
desdichada suerte en la forma más práctica, Beramendi me notificó que
estaba dispuesto a pagarme pupilaje decoroso y buena comida en cierta
casa de huéspedes regida por una señora llamada doña Leche. Añadió que
hoy mismo daría a Telesforo del Portillo las órdenes oportunas para que
fuera yo recibido sin dilación en mi nueva morada, Relatores, 4. Acto
seguido, María Ignacia puso en mi mano dos dobloncitos de a cuatro, para
mis gastos menudos de tabaco y café, advirtiéndome con sequedad
melindrosa que si yo no era económico y sensato no repetiría la dádiva.»

Cuando esto decía el buen Segis, sacó las moneditas de oro con el aleve
intento de pasarlas de su bolsillo al mío. Como yo me resistiera
enérgicamente, intentó ponerlas en la mano de Casianilla; pero esta
rechazó la oferta con más jovialidad que indignación, diciendo: «Eso es
para usted, don Segis; Tito y yo somos ricos por nuestra casa, ya usted
lo sabe, y del amigo queremos la amistad y el cariño, no el vil metal,
como dice don José cuando se le habla de oro.»

Pasados unos días, el 20 de Marzo de 1876, propuse a Segismundo que
fuésemos los tres a presenciar la entrada de Alfonso XII en Madrid al
frente de las tropas victoriosas en el Norte, pues según anunciaba la
Prensa tendríamos un acontecimiento grandioso, vibrante, solemne, un
himno a la paz cantado al unísono por el pueblo y las altas clases
sociales. Esta indicación mía dio motivo a un sustancioso juicio
histórico del rebelde, que merece el honor de la letra de molde. Ahí va:

«Detesto la guerra civil dinástica, y es tan vivo mi odio a ese medio
siglo de lucha fratricida sin gloria y sin fruto, que nada encuentro en
él que pueda contentarme. Tanto me amarga esa guerra que me incomodan
hasta las victorias, me carga el heroísmo y me revientan los laureles.
Para mí, la contienda de familia debió quedar acabada y finiquita el
mismo 34, a los pocos meses de entrar en España por Elizondo el inmenso
mentecato don Carlos María Isidro, cuando Martínez de la Rosa lanzó la
frase de \emph{un faccioso más}. En este desdichado país no había
entonces sentido político ni militar sentido, ni el vigoroso estímulo de
la conservación nacional. Por la flaqueza de estos sentimientos, los
españoles no supieron extirpar el mal aplicando con dureza implacable el
procedimiento quirúrgico. La querella dinástica se hizo crónica, y la
repugnante dolencia creció invadiendo el cuerpo social en el curso del
siglo. Todavía ¡pobre España!, todavía tienes sarna que rascar para
largo tiempo.

»En vez de resolver a rajatabla el problema \emph{Vendeano}, diose
tiempo a los carlistas para que se tomaran la beligerancia, para
reclutar hombres y allegar dinero formando ejércitos casi regulares,
para proveerse de una pequeña Corte y erigir un Estado minúsculo, dotado
con todos los engorros burocráticos y administrativos. Los liberales, a
su vez, se preparaban apercibiendo los resortes complejos del viejo
mecanismo histórico. En seguida empezaron los encuentros, las
batallitas, el correr y perseguirse por los ásperos montes y los verdes
oteros, que fueron y son campos del fanatismo. Para mayor desdicha de la
Patria, ambos Ejércitos eran valientes, incansables. Los triunfos y los
descalabros se compartían por igual. El heroísmo flameaba en uno y en
otro bando; victorias hubo aquí, victorias allá, mas ninguna bandera
logró desgarrar definitivamente la bandera contraria.

»En el rápido crecimiento de la grey militar, muchos veían ventajas
positivas. Si acertaban estos ilusos España era un país felicísimo y
envidiable, pues en los fatídicos tiempos de la guerra civil, las
frecuentes concesiones de grados por méritos efectivos multiplicaron
profusamente la cifra de Oficiales y Jefes. Muchos, hermanando el valor
con la fortuna, pasaron muy pronto de Tenientes a Generales. De esta
categoría teníamos caudillos bastantes para mandar los Ejércitos de
Napoleón. Naturalmente, bromas tan sangrientas en el campo de la
Historia no podían ser de larga duración. A los siete años de un
batallar tenacísimo, los dos Ejércitos, fatigados y anhelantes de la
paz, cayeron en la cuenta de que lo más conveniente y positivo para
entrambos era pactar franca reconciliación, abrazarse y lanzar el
\emph{Todos somos unos}. Tal como lo pensaron lo hicieron, conviniendo
en mantener y dar valor efectivo a los grados, empleos y condecoraciones
ganados por una y otra hueste en siete años de rabiosa porfía. ¿Por qué,
Señor, a santo de qué? Por si debía reinar varón o hembra.

»El huevo de Vergara fue ciertamente un huevo de paz. Pero de él, al
calor de nuestras incurables tonterías políticas, ha salido una gusanera
que es incubación de todo aquello que creíamos muerto y sepultado. Te
dije antes que en las guerras intestinas me cargan los heroísmos, los
laureles marchitos apenas ganados, y ahora te digo que me carga también
la paz, porque aquí la paz es el huevo de que sale otra generación con
la misma estúpida manía del pleito familiar dinástico, de la demencia
bélica, de la multiplicación de Generales\ldots{} Ya ves lo que ha
pasado en los últimos años. Otra vez parece que tenemos paces. Pero no
te fíes\ldots{}

---En este momento entra don Alfonso en Madrid---dijo Casiana.---¿No
oyen ustedes los tambores y cornetas que suenan lejos, lejos?

---Oímos, sí---prosiguió Segis.---Además de oír, desde aquí veo yo el
contento del Rey y el júbilo del pueblo inocente y confiado que le
aclama. ¡Pobrecitos! Llaman paz a una tregua cuya duración no podemos
apreciar todavía.

---Tienes razón---afirmé yo,---y es posible que los carlistas no vuelvan
a tomar las armas, porque verdaderamente no lo necesitan. Los vencedores
se han traído acá las ideas de los vencidos, creyendo que en ellas
consolidarán el trono flamante.

---Todo queda lo mismo---continuó García Fajardo, con gran seguridad en
su juicio.---El Borbonismo no tiene dos fases, como creen los
historiadores superficiales, sino una sola. Aquí y allá, en la guerra y
en la paz es siempre el mismo, un poder arbitrario que acopla el Trono y
el Altar para oprimir a este pueblo infeliz y mantenerlo en la pobreza y
en la ignorancia. Lo único positivo en ese cortejo brillante que ahora
atraviesa las calles de Madrid es un sinfín de Generales, Jefes y
Oficiales nuevos, agregados a los que ya teníamos, una caterva de
funcionarios viejos o novísimos que fundarán sobre el doble catafalco,
Altar y Trono, una política de inercia, de ficciones y de fórmulas
mentirosas extraídas de la cantera de la tradición. Todo esto va
decorado con el profuso reparto de honores, distinciones y títulos
nobiliarios. Pronto veréis, amigos míos, el Anuario de la Grandeza
empedrado de Condes y Marqueses. En lo de acuñar nobles al por mayor y
en la prodigalidad de los \emph{Excelentísimos, Ilustrísimos} y
\emph{Reverendísimos}, no hay país en el mundo que nos iguale. ¡Oh
desmedrada España! Cada día pesas menos, y si abultas más atribúyelo a
tu vana hinchazón.»

\hypertarget{xi}{%
\chapter{XI}\label{xi}}

Ya supondrán los píos lectores que habiendo paz en España ardió Madrid
en fiestas, conforme al ceremonial de alegría pública que amenizaba
nuestra Historia desde que volvió del destierro Fernando \emph{el
Deseado} en 1814. Vestían los balcones abigarradas percalinas, las más
de ellas de respetable ancianidad, pues ya figuraron en el regocijo de
1860, cuando entraron las tropas vencedoras en África, y en el regocijo
del 68, entrada de Serrano vencedor en Alcolea. De noche fulguraban las
hileras de gas en los edificios públicos, y en el caserío lucían de
trecho en trecho los farolitos de aceite con parpadeo mustio y
lacrimoso. La iluminación pública era la misma que esmaltó las noches en
diferentes ocasiones de júbilo, como el nacimiento del Príncipe y las
Infantas, o la traída de aguas del Lozoya.

Salimos una noche a ver los festejos los tres inseparables; mas no
tuvimos paciencia ni valor para correr el largo trayecto desde la
Cibeles a Palacio, entre un gentío espeso, silencioso y embobado, que a
mi parecer personificaba de un modo gráfico el aburrimiento nacional.
Nos dijeron que en algún sitio de la carrera se alzaba un armatoste de
pintados lienzos. Era sin duda lo que llaman un arco de triunfo, quizá
un templete del \emph{género clásico fastidioso} como el que pusieron en
el popular regocijo de 1830, cuando María Cristina vino a casarse con
Fernando VII. Toda esta balumba de tonterías no nos interesaba y la
dimos por vista, acogiéndonos a la sociedad amable, risueña y chispeante
del café de Las Columnas.

Y ahora, lector mío, a mi modo \emph{continuaré la Historia de España},
como decía Cánovas. En cuanto terminaron los desaboridos festejos, las
Cortes enredáronse en el arduo trajín de fabricar la nueva Constitución,
la cual si no me sale mal la cuenta, era la sexta que los españoles del
siglo XIX habíamos estatuido para pasar el rato. Naturalmente, se nombró
una Comisión cuyos individuos trabajaban como fieras para pergeñar el
documento, y a este propósito os diré que la última nota del regocijo
público, en los jolgorios de la paz, la dio don Antonio Cánovas con una
frase graciosísima que vais a conocer. Hallábase una tarde en el banco
azul el Presidente del Consejo, fatigado de un largo y enojoso debate,
cuando se le acercaron \emph{dos señores de la Comisión} para
preguntarle cómo redactarían el artículo del Código fundamental que
dice: \emph{son españoles los tales y tales}\ldots{} Don Antonio,
quitándose y poniéndose los lentes, con aquel guiño característico que
expresaba su mal humor ante toda impertinencia, contestó ceceoso:
«Pongan ustedes que son españoles\ldots{} los que no pueden ser otra
cosa.»

Cuando ya conocimos la letra y el espíritu de la Constitución,
Segismundo recitaba algunos fragmentos dándoles un sentido contrario al
que textualmente tenían. El tercer párrafo del famoso artículo 11, que
trata de la cuestión religiosa, lo volvía del revés en esta forma: «Todo
ciudadano será molestado continuamente en el territorio español por sus
opiniones religiosas y por el ejercicio de su respectivo culto, conforme
al menosprecio debido a la moral universal.» Otras cláusulas del mismo
Código ponía mi amigo en solfa, asegurándonos que a tales burlas le
incitaba una vena profética posesionada de su espíritu. Sin atormentar
su fantasía contemplaba en los días futuros la sistemática violación de
aquella Ley, como violadas y escarnecidas fueron las cinco
Constituciones precedentes. En el propio estado de pérfida legalidad
seguiría viviendo nuestra Nación año tras año, hasta que otros hombres y
otras ideas nos trajeran la política de la verdad y la justicia,
gobernando, no para una clase escogida de caballeros y señoras, sino
para la familia total que goza y trabaja, triunfa y padece, ríe y llora
en este pedazo de tierra feraz y desolado, caliente y frío, alegre y
tristísimo que llamamos España.

Del pesimismo profético de Segis participaba yo, haciéndolo aún más
lúgubre por la negra melancolía que empezó a invadir mi alma poco
después de las fiestas de la paz. Rápidamente creció aquel malestar
insufrible, no sé si cerebral o nervioso, que en años anteriores me
llevó a los mayores delirios. Durante algunos días conseguí sobreponerme
a los fenómenos más enojosos de la dolencia, como la percepción de voces
susurrantes que atormentaban mis oídos. Los seres invisibles hurtábanme
el sosiego, y en giros vertiginosos se revolvían en torno mío,
diciéndome palabras dulces, palabras tétricas o burlonas.

Cuando me encontraba junto a Casiana y Segis, apetecía la soledad, y si
estaba solo deseaba cualquier compañía, aunque fuera la de la
insignificante Nicanora. Enfadábanme la casa, y al buscar alivio en el
aire libre y en el bullicio de la muchedumbre, la calle se me hacía
también insoportable. En mi turbación hondísima, discurría yo que una de
las causas de aquel desvarío borrascoso era el abandono en que me tenía
mi divina Madre, pues aunque puntualmente me entregaba la portera de la
Academia mi estipendio, ya no venía este acompañado de cartita o
mensaje, y para mayor soledad no volvió a llegarse a mí la espiritual
mandadera de \emph{Clío}, la voladora \emph{Efémera}.

Los cuidados y mimos de Casiana y las gracias de Segis me aliviaron un
tanto a la entrada de verano. Llevábanme a dar largos paseos por las
afueras, y alejándome del caserío de la Villa y Corte notaba yo en mis
nervios efecto sedante. Un día nos íbamos por el Abroñigal, otros por
Bellas Vistas, Amaniel y Arroyo de San Bernardino, o bien Manzanares
arriba hasta cerca de El Pardo, o Manzanares abajo más allá del Canal.
Aunque prohibí a Segismundo que me hablase de política, este no podía
contenerse, y en forma jovial y guasona me daba cuenta de sucesos en los
cuales yo no vi ningún interés. Con prodigiosa memoria repetía trozos
del Breve que largó el Papa condenando el artículo 11 de la
Constitución. Sus chanzas no me divertían; mandábale yo callar
diciéndole que, pues éramos más súbditos de Pío IX que de Alfonso XII,
debíamos concretarnos a gemir bajo la sandalia que nos aplastaba.

Ni la cólera pontificia, ni la promulgación del sexto Código
fundamental, producto de los ocios políticos, ni el presupuesto
alfonsino, ni la cuestión foral, atraían mi dislocado
pensamiento\ldots{} Pasaron tardos y tediosos los meses caniculares con
suave mejoría de mi dolencia, y a la entrada de otoño creí notar que lo
que ganaba en salud física lo perdía en facultades mentales, pues
sentíame tonto, muy lento en el discurrir y en formar juicio de las
cosas. En la soledad de mi casa, suspendidas ya las caminatas
campestres, el buen Segis trataba de sacudir mi pereza mental
refiriéndome pormenores de la maquinación sediciosa. En París habían
llegado a un acuerdo Salmerón y Ruiz Zorrilla, concertando un pacto del
cual esperaban grandes frutos los amigos de don Manuel. Contra este
convenio tronó Emilio Castelar en carta dirigida a Morayta desde
Garrucha. En tanto, los zorrillistas seguían conspirando de lo lindo en
Francia y en Madrid. Segis me aseguró que en una vivienda obscura de la
calle de la Aduana tenían Ladevese y Santamaría la oficina
revolucionaria, en que tramaban un alzamiento combinado de paisanaje y
tropa. Llegaron al Gobierno soplos de esta conjura, y una mañana fueron
presas más de doscientas personas entre civiles y militares.

Escuchaba yo esto como quien oye llover, y no presté mayor atención a
las parrafadas de Segis comentando el \emph{bill de indemnidad} (dicho a
la inglesa para entenderlo mejor) que Cánovas pidió a las Cortes en
Noviembre. Sagasta y el Duque de la Torre, capitaneando con bravura el
Partido Constitucional recién empollado, pedían ya el Poder, que era
como pedir la luna. Al discutirse la reforma de las leyes municipal y
provincial del año 70, don Antonio se batió con ellos, con Castelar y
con los \emph{moderados}, en memorables sesiones de indudable interés
teatral.

Leíame Casiana los discursos del malagueño; decía Segis a este propósito
cuantos disparates se le ocurrían, y yo, recobrando por un momento la
lucidez de mi espíritu, pude aventurar esta gallarda opinión, que mis
interlocutores oyeron estupefactos: «Conozco el pensamiento de Cánovas;
penetro en su cerebro por privilegio que me ha dado mi excelsa Madre. El
hombre de la Restauración sacude a un lado y otro los latigazos de su
potente oratoria porque ve en peligro su obra, la ensambladura del Altar
y el Trono; sospecha que los enemigos del régimen se preparan a
reconquistar por la fuerza el Poder que por la fuerza se les arrebató en
Sagunto.

»Advierto que me miráis con incredulidad un poquito burlona. ¿No sabéis
que puede existir y en mil casos existe el contacto espiritual entre
dos, tres o más cerebros situados a larga distancia? Pues si esto
ignoráis, yo lo sé y os lo digo para que lo creáis como artículo de fe,
y no se os ocurra tomar estas cosas a broma. La vibración pensante se
comunica de aquel cerebro al mío por arte magnético desconocido de los
tontos, y aquí tenéis al pobre Tito fiel transmisor de las ideas del
Jefe del Gobierno.»

Pausa expectante y fúnebre. Casianilla y Segis se miraron perplejos, y
luego volvieron sus ojos hacia mí con expresión de lástima cariñosa.
Creían sin duda que yo no estaba en mis cabales, o que mi dolencia
nerviosa derivaba marcadamente hacia la locura. Los dos llevaron la
conversación a un tema jovial, como para desviar mi mente de las
obsesiones monomaníacas\ldots{} Debo añadir que empezaba yo a tomar
entre ojos al buen Segismundo, por su insistencia en contrariarme y por
su afán de traerme noticias que, a mi parecer, eran más que Historia
chismografía. También Casiana me causaba cierto enojo y fastidio por la
prolijidad de sus cuidados, que los enfermos solemos ser ingratos con
las personas que nos asisten.

Una tarde, a la hora del crepúsculo, salimos de paseo los tres. Casiana
y Segis iban delante, yo detrás, por la calle de las Huertas abajo.
Fuera porque ellos se adelantasen o porque yo me retrasara, lo cierto es
que les perdí de vista. Avancé hacia el Prado revolviendo mis ojos de
una parte a otra, y al llegar cerca de la fuente de las Cuatro
Estaciones vi un grupo de niñas grandullonas que, cantando y cogiditas
de la mano, jugaban al corro. El ruedo era muy extenso: formábanlo unas
veinte o veinticinco rapazuelas, vestidas con luengos ropajes flotantes
de distintos colores. Acerqueme, y creyendo reconocer a una de aquellas
ninfas juguetonas, la saqué violentamente del corro y le dije:

---Ven aquí; tú eres \emph{Efémera}.

---Sí, sí---me contestó.---Todas las del corro somos \emph{Efémeras}.

---¡Ah! Sí, sois muchas. Ya lo sabía yo. ¿Tú me has visitado algunas
veces?

---No puedo asegurártelo. Mensajeras veloces, tenemos alas eternas, pero
nuestra memoria no dura más que un día\ldots{} Y cuando no nos mandan a
recorrer las esferas jugamos, ya lo ves.

---Hijas del aire, ¡sed compasivas conmigo! Cogedme entre todas, que
bien podéis hacerlo, y llevadme adonde está mi divina Madre.»

Prorrumpió en alegres risas la sílfide picaresca, y desprendiéndose de
mi mano volvió al corro con sus gráciles hermanas. Corrí yo hacia ellas;
pero a mis primeros pasos me cegó una ráfaga de luz vivísima, sulfúrea,
violácea, y tuve que detenerme. No vi más a las \emph{Efémeras}; oía su
canto, un murmullo ciclónico que se desarrollaba en espirales cada vez
más lejanas. Mi oído pudo percibir estas cláusulas: \emph{En el Salón
del Prado---no se puede jugar---porque hay muchos mocosos---que vienen a
estorbar.---Con un cigarro puro---vienen a presumir:---más vale que les
dieran---un huevo y a dormir\ldots{}}

Andando a tropezones, medio ciego y en un estado de turbación indecible,
traté de orientarme para volver a mi vivienda, sin pretender encontrar a
Segis y Casiana. Mis ojos, encandilados por aquel resplandor
intensísimo, no me guiaban bien en mi camino. Era la hora en que los
faroleros corrían encendiendo los mecheros de gas. Por la Plaza de las
Cortes, calle de San Agustín y otras que seguí con andadura maquinal,
llegué a mi casa, donde me encontré solo. ¡Solo, Dios mío! No puedo
expresar la tristeza que invadió mi alma al hallarme sin Casianilla.
Cuando advertí que transcurría el tiempo sin verla entrar, mi tristeza
se trocó en ira. Tumbado en el sofá esperé, esperé. Al cabo de media
hora larga que me pareció un siglo, llegó mi compañera, inquieta y
turbada. Antes que pudiese darme explicaciones de su desaparición en la
calle, la increpé con voces ásperas y descompuestas. Mis gritos
atronaron la casa. La pobre mujercita, que jamás me vio en estado tan
contrario a mi natural mansedumbre, rompió a llorar amargamente,
balbuciendo entre gemidos estas atropelladas razones:

«¡Ay, Tito mío; yo no tengo la culpa!\ldots{} No me riñas así\ldots{}
Cuando te echamos de menos volvimos atrás. No te encontramos. Adelante
otra vez\ldots{} Como a ti te gusta ir hacia el Botánico, allá nos
fuimos\ldots{} ¡Ay Dios mío!\ldots{} Tampoco estabas allí\ldots{}
Segismundo dijo que habrías ido hacia el Museo\ldots{} ¡Ah!, en el Museo
tampoco te hallamos\ldots{} Por mi salud, yo estaba loca, no sabía lo
que me pasaba\ldots{} Buscándote por un lado y otro del Prado seguimos
hasta la Cibeles\ldots{} Aturdidos, y sin saber ya qué hacer, subimos
por la calle de Alcalá, entramos por la del Turco. Me dio una
corazonada. Yo dije: \emph{Al ver que nos perdíamos se habrá ido a la
plazuela de las Cortes, y allí estará sentadito en un banco, al pie de
la estatua de}\ldots{} No sé, no sé cómo se llama aquel hombre\ldots{}
No encontrándote, me dio otra corazonada, puedes creérmelo como Dios es
mi padre, y dije: \emph{Apuesto a que se ha metido en casa. Voy
corriendo, voy volando}. Y volando vine acá\ldots{} ¡Tito, por la Virgen
Santísima, no me digas esas cosas!\ldots{} ¡Ay, yo me muero si tú no me
quieres!

---¿Y Segismundo?---pregunté con acento agresivo, de suprema
desconfianza.

---Pues cuando llegábamos a la plazuela de las Cortes se nos presentó de
repente aquel señor \emph{Sebo}, ya sabes, y le dijo a Segis que tenía
que hablarle\ldots{} que si el señor Marqués o la \emph{señá}
Marquesa\ldots{} En fin, Tito, que yo eché a correr dejándolos con la
palabra en la boca.»

Pasado un rato se calmaron mis irritados nervios. La fiel Casiana, con
sinceras razones y blandas caricias, me devolvió la perdida
tranquilidad. Hicimos las paces. Volví a mi quietud enfermiza, no sin
que me atormentaran horas de insomnio, dudas, tristezas y alucinaciones
horribles.

No aquella noche, ni la siguiente, sino tres o cinco noches después (que
la cronología por entonces era problema insoluble para mí), hallándonos
Casiana y yo de sobremesa pensando mucho y hablando poco, se llegó a
nosotros Ido del Sagrario con paso grave y actitud sacerdotal.
Imponiéndonos silencio con marcada rigidez de su dedo índice, para que
oyéramos las campanadas del reloj de San Juan de Dios, alargó la nuez y
en tono sibilítico nos dijo: «Excelentísimo Señor, señorita \emph{de
Coelho}, en este momento ha fenecido el año de 1876 y ha entrado a
presidir nuestra existencia el 1877. \emph{Laus Deo}.»

\hypertarget{xii}{%
\chapter{XII}\label{xii}}

¡1877! La cifra pasó fugaz por mi mente. Menos que los años me
interesaban los meses y los días, pues el Tiempo había llegado a ser
para mí un concepto caótico\ldots{} Volvió Segismundo a mi compañía y
tertulia con la cordialidad de amigo verdadero y de hombre agradecido.
Una mañana (averigüe la fecha quien tenga empeño en conocerla) se
presentó ante nosotros con un chaleco rameado y un pantalón de género
inglés. Antes que me lo dijese comprendí que aquellas prendas eran el
desecho del rico guardarropa de Beramendi.

«Hemos de mostrar prácticamente---me dijo el rebelde con sorna
sutil---que nos asimilamos la característica elegancia de la sociedad
alfonsina. Otra característica de los tiempos es que estos se retrotraen
y vuelven las cosas al estado que tenían años ha. Sabrás, querido Tito,
que el hombre del día es Montpensier. Por las calles le he visto con su
tradicional paraguas y su aire de Príncipe acomodaticio y contento de la
vida. Sus querellas con la Reina doña Isabel, a quien quiso destronar;
el duelo trágico con el Infante don Enrique y los trabajos de zapa para
cargarse la corona democrática que las Constituyentes otorgan a don
Amadeo, han pasado al cesto en que arroja la Historia los papeles
inútiles. Busca y obtiene la reconciliación con los Borbones reinantes,
moviéndole a ello las gracias de su linda hija Mercedes. Te diré, si lo
ignoras, que el simpático Alfonso se ha enamorado perdidamente de su
primita.»

Otro día (indagad la fecha por el curso de los astros o el vuelo de las
aves), se nos apareció el pícaro Segis con un precioso alfiler de
corbata en que lucían dos perlitas y un rubí, y me dijo, poniendo en sus
palabras tanta seriedad como gracejo: «Vivimos en la época del fausto
insolente y de los grandes negocios. No se habla de otra cosa que de
capitales extranjeros que afluyen aquí buscando empleo y beneficios
pingües, de grandiosas empresas industriales, de ferrocarriles más
largos que la cuaresma, y de otros cortos y ceñidos al interés
particular. La alta banca se mueve; el dinero se desentumece, y corre a
donde lo llaman el crédito y el trabajo.

»España renace; pero los provechos de este resurgir de la vida económica
no alcanzan todavía más que a las clases opulentas. Y yo pregunto:
\emph{¿Por qué lo que llamamos capas inferiores de la sociedad no ha de
agregarse también a esta corriente financiera?} Si bien se mira, la
multitud es rica por solo el hecho de ser tal multitud. Los \emph{muchos
pocos}, alineados en cifra, representan ¡oh Tito!, suma considerable. Ha
llegado, pues, el momento de crear los \emph{Bancos Populares}, que
recojan los ahorros del pobre y se los devuelvan multiplicados. De tal
modo, entiendo yo que laborando de consuno las capas de abajo y las
capas de arriba se abrigarán recíprocamente. ¿No crees tú lo mismo?»

Le contesté que sí, sin añadir observación alguna. Había yo notado que
Segismundo, habitualmente muy diestro en el uso de la ironía, la
sutilizaba entonces hasta hacer de ella un arte maravilloso\ldots{}
Pasadas dos semanas, se nos presentó Fajardo mejor apañado de indumento:
traía botas de charol y un gabancete, no nuevo pero en buen uso, prenda
de fijo adquirida en un establecimiento de compraventa mercantil. A mis
felicitaciones por su buen porte, y a las preguntas que le hice, me
contestó que había mejorado de posición gracias a la buena amistad del
insigne \emph{Sebo}, quien le había conseguido empleo modesto y decoroso
en un \emph{Banco Popular}\ldots{} Relacioné al instante las referencias
de Fajardo con una entidad de crédito establecida no hacía mucho en la
Plaza de la Cebada, y cuyas operaciones daban que hablar a la gente.

«Sí, querido Proteo---me dijo Segis;---trabajo en las oficinas de ese
\emph{Banco}, fundación admirable que no viene a vaciar un lleno sino a
llenar un vacío en la sociedad española, porque ha de traer la sangre
plebeya a vigorizar el cuerpo financiero de la Nación\ldots{} Sangre
nueva, sangre fresca: el ahorro menudo, el globulillo rojo circulando
por las venas de este país anémico\ldots{} Por último sabrás, si ya no
lo sabes, que la creadora de esta institución benéfica y patriótica es
una dama ilustre en quien yo veo el símbolo de la raza hispana, mujer de
un vigor mental extraordinario cual nunca se vio en hembras de nuestra
tierra, portento de sagacidad, clarividencia y maestría en el arte o
ciencia de las finanzas, bonita y graciosa de añadidura; es, en fin,
doña Baldomera Larra, hija del gran \emph{Fígaro}.»

En conversaciones posteriores, me contó mi amigo que la gente de la
Plaza de la Cebada, y todos los lugareños que se albergaban en los
paradores de la calle de Toledo y adyacentes, hacían cola a la puerta
del \emph{Banco Popular} para imponer \emph{sus monises} en las cajas de
doña Baldomera. Aquello era un jubileo, era un escándalo, y la policía
tenía que intervenir para poner orden. Se contaba que en los pueblos
vendían las fincas con objeto de hacer imposiciones en el flamante
\emph{Banco}. La genial hacendista, persona muy sugestiva y de
fenomenales dotes oratorias, echaba discursos a la entusiasta y
codiciosa plebe, y al darles el primer plazo de los cuantiosos
intereses, les ofrecía ganancias pingües, colosales. La garantía de tan
inaudito negocio ¿cuál era? Pues unas minas de plata, de oro o de
piedras preciosas radicantes en el suelo virgen de América, minas de
incalculable riqueza cuya explotación multiplicaría los parneses
depositados en las arcas Baldomeriles.

En las visitas que casi diariamente me hacía el buen Segis, contábame el
asunto en cierto modo fundamental y étnico del \emph{Banco Popular}.
Sostuve yo que la credulidad candorosa del pueblo español y las artes
hipnóticas de la hija de Larra eran, como signo indudable del estado
mental de la raza, más dignos del fuero de \emph{Clío} que las ficciones
vanas en que se agitaban nuestros políticos; en suma, que la Historia
debía consagrar más páginas al zurriburri de las finanzas plebeyas que
al barullo retórico de las Cortes, y al trajín de quitar y poner
Constituciones que no habían de ser respetadas.

Acorde con cuanto yo dije, Segis me manifestó que estaba contento en su
destinillo. La dama banquera le consideraba, mostrándole un afecto casi
maternal, al que correspondía el funcionario con su puntual asistencia y
el esmero y pulcritud de su trabajo de contabilidad. Iba, pues, muy a
gusto en el machito, y como los Marqueses de Beramendi le aseguraban su
hospedaje y manutención, el duro diario que en el \emph{Banco} percibía
destinábalo a mejorar su vestimenta. Cada vez que se nos presentaba con
algo nuevo en su atavío, ya fuese prenda de ropa, ya un relojito barato,
nos decía:

«Ved aquí el positivo producto de las minas de América, de esos ricos
yacimientos de metales preciosos ¡ay!, que han venido a ser la felicidad
del pueblo madrileño. Adelante con la ilusión, vida y encanto de las
naciones pobres. Tú, buen Proteo, que a ratos escribes o garabateas en
las tabletas de la divina \emph{Clío, continúa la Historia de España},
como dice Cánovas, transmitiendo a la posteridad estos actos de fe
candorosa y de sutil taumaturgia; añade a ello la fiebre taurina, la
ciencia recóndita de esos que llaman \emph{los apóstoles}, y que andan
por los barrios bajos curando todas las enfermedades con agua más o
menos limpia, y habrás hecho el retrato fiel de la España de la
Restauración.»

No tenía yo ánimos en aquellos días para \emph{continuar la Historia de
España}, ni conforme al canon político, ni acogiéndome al rico tema de
la ilusión plebeya que me recomendaba Segis, deseoso de arrastrarme al
concepto irónico de la psicología nacional. Declaro que el acto del Rey
poniendo la primera piedra de la Cárcel Modelo en las proximidades de la
Moncloa, las sesiones de las Cámaras, el cambio de Ministro de Hacienda,
así como el viaje que emprendió don Alfonso para visitar las provincias
de Levante y Mediodía, no me interesaban poco ni mucho. Cuando mis
amigos me contaban estas menudencias históricas sonábame todo a hueco.
La tristeza invadió nuevamente mi alma, complicándose con un malestar
físico que me llenó de inquietud, avanzados ya los días tibios de la
primavera.

Después de Semana Santa empecé a notar que mi vista se nublaba; sentía
como arenillas en los ojos, sin que de ello me aliviasen los cuidados de
Casiana, que dos o tres veces al día bañaba con agua de rosas mis
pupilas enfermas. Los patrones me recomendaron ejercicio y distracción.
Conforme con este tratamiento elemental, mi compañera sacábame de paseo
todas las tardes; pero mi vista mermaba tan rápidamente, que a los pocos
días de estas divagaciones por el Botánico y Ronda de Atocha, tuve que
agarrarme al brazo de mi leal Casianilla para no tropezar con los
transeúntes. Al propio tiempo crecía la fotofobia, y ni aun amparando
mis ojos con gafas negras érame posible resistir la viveza de la luz en
plena calle. Fue menester reducir los paseos a la hora crepuscular,
motivo mayor de tristeza y abatimiento. Siguieron a esto dolores en las
sienes, vascularización en la córnea, que perdía su brillo, tomando
según me dijeron un aspecto mate, sanguíneo.

Tanto Segis como los demás amigos que me acompañaban en mis largas horas
tediosas, convinieron en familiar consulta que era forzoso acudir a la
Ciencia. Agravado el mal en breve tiempo, hasta el punto de que ya no
distinguía más que los objetos próximos y de mucho bulto, se trató en mi
casa de elegir el médico que había de curarme, y Pablo Nougués, doliente
también de la vista, llevó a mi casa una tarde para que me examinase al
doctor Albitos. Era este un oculista joven, inteligentísimo en su
profesión, de trato muy ameno y agradable, discípulo del famoso Delgado
Jugo. Examinó el doctor mis dolidos ojos con escrupulosa atención y
cariño; enterose de cuanto en mi naturaleza y en mis costumbres pudiera
ser considerado como antecedente de la enfermedad. Sus palabras dulces
me consolaron; mi sufrimiento sería tal vez un poco largo; pero si no me
faltaba la virtud puramente medicatriz de la paciencia, él respondía de
mi curación. Terminó el diagnóstico con el nombre científico y un tanto
enrevesado de lo que yo padecía. No se me olvida aquel nombre, que fue
como un rótulo, clavado por el médico en mi frente: \emph{Queratitis
Parenquimatosa}.»

Desde aquella tarde quedamos unidos con vínculo estrecho mi
\emph{Queratitis} y yo, cual un matrimonio doloroso que había de durar
hasta que la ciencia del oculista nos divorciara. Fortalecido por mi
paciencia, de la que hice acopio exuberante, cargaba mi cruz y con ella
recorría el agrio camino de la vida hora tras hora, semana tras semana.
Recluso en mi habitación, sumido en intensa obscuridad, yo no distinguía
los días de las noches, ni un día de otro, ni apreciaba el principio y
fin de cada semana. Era para mí el tiempo un concepto indiviso, una
extensión sin grados ni dobleces. Las únicas interrupciones de la
continuidad eran los momentos en que me hacían la cura de los ojos el
doctor o su ayudante.

En aquel lúgubre rodar de mi existencia notaba yo menos constancia en
las visitas de los amigos. Hasta el propio Segis se me antojó poco
asiduo: casi siempre tenía perentorias ocupaciones que le obligaban a
retirarse pronto. Sólo la fiel Casiana permanecía junto a mí superándome
en paciencia, y llevando a los límites de lo sublime la humanidad, el
amor y la misericordia.

Compadecedme ahora más que nunca, piadosos lectores, pues encontrábame
ya en el período más doloroso y tétrico de mi largo padecer. Mi ceguera
llegó a ser absoluta, mis ojos inflamados dábanme la sensación de dos
ascuas mal contenidas dentro de las órbitas. Los fomentos calientes y
las duchas de vapor, que me administraba el ayudante del oculista,
aliviábanme a ratos. Casianilla me servía con puntual solicitud la
medicación interna, mercuriales, antisépticos\ldots{} Cuando a mis oídos
llegaba el tintín de la cucharilla revolviendo las dosis terapéuticas en
el vaso de agua, sentía yo cierto regocijo. Aquel rumor cristalino era
mi único reloj, y por él tenía yo un vago conocimiento de las
horas\ldots{} En cierto modo imitaba el ritmo de la \emph{Queratitis},
arrullándome en sus duros brazos\ldots{}

Mi existencia no era más que una sombra encerrada en ancha caverna, que
ya me parecía roja, ya de un tinte violáceo surcado de ráfagas verdes.
En tal estado llegué a perder, según después he podido apreciar, la
conciencia de la realidad. Una tarde o una noche, no sé precisarlo,
sintiendo junto a mí rumorcillo de faldas, alargué la mano y dije:
«Casiana, ven, siéntate a mi lado.» Y una voz tenue, con leve inflexión
burlona, me contestó: «Tonto, no soy Casiana. Soy \emph{Efémera}.»

No me dio tiempo a expresar mi alborozo porque, apenas oí la voz
primera, otras voces sonaron en alegre y voluble cháchara, y al par de
esta, rumor de pisaditas como de seres alados que juegan y revolotean
rozando apenas el suelo con blandos pies. «Ya os siento, ya os escucho,
mensajeras de mi Madre---exclamé.---¿Venís a consolarme?\ldots{} ¿Me
traéis nuevas de la que es vuestra Señora y Señora mía?»

Las ninfas juguetonas siguieron revoloteando a mi alrededor, y el aire
que movían sus flotantes túnicas me daba en el rostro. Del murmullo
picaresco destacose una voz que claramente me dijo: «Somos las
\emph{Efémeras} ociosas que hoy están libres, dueñas de los aires y del
tiempo\ldots{} La Madre, que se halla lejos, lejos, y también ociosa,
nos ha mandado que juguemos y nos divirtamos sin más ley que nuestro
albedrío. Venimos de embromar a Cánovas, y ahora la emprendemos con el
buen Tito. \emph{(Risillas mal sofocadas.)} Nos ha dicho Cánovas que
quiere consultar contigo el problema matrimonial de don
Alfonsito\ldots{} Ja, ja, ja\ldots{} Ji, ji, ji\ldots»

El giro vertiginoso de las sílfides me mareaba, me volvía loco\ldots{}
Algunas, al pasar junto a mí, dábanme papirotazos en la cabeza con sus
manos livianas y frías\ldots{} Arreció el murmullo reidor, chancero.
Levanteme frenético, empecé a dar voces, traté de coger a una de las
ninfas, creí agarrar su ropa, tiré fuertemente y la traje hacia mí
diciendo: «Ven, \emph{Efémera}, quédate aquí.» Pero ella se escapó
susurrando: «Volveré, Tito. Soy tu amiga.» En esto oí la voz de mi
compañera que a mi lado dormitaba y que a mis gritos habíase
despabilado. Abrazándome tiernamente me dijo: «¿Qué te pasa, muñeco mío?
¿Sueñas, deliras? ¿Por qué llamas \emph{Efémera} a tu Casianilla?»

\hypertarget{xiii}{%
\chapter{XIII}\label{xiii}}

Contra lo que sin duda creerán mis compasivos lectores, aquel delirio me
sentó muy bien. Acostome Casiana y me dormí con sueño tranquilo y
reparador. Al despertarme, no sé a qué hora, sentí notorio alivio en mi
estado general\ldots{} La oleada de ambiente quimérico me refrescaba el
alma y producía en mis pobres vísceras acción más eficaz que los
antisépticos y calomelanos\ldots{} Cuando el bendito don José vino a
preguntarme cómo me encontraba, le dije: «Muy bien, amigo Sagrario.
Fíjese ahora en lo que voy a encargarle. Si vienen a visitarme las
señoritas \emph{Efémeras}, o una \emph{Efémera} sola, no haga la
tontería de cerrarles la puerta; páseme aviso inmediatamente, que estoy
dispuesto a recibirlas. Mucho cuidado, don José, mucho cuidado.»

Casiana y el patrón callaron. Yo, sin ver gota, comprendí que se miraban
alarmados y compasivos, como diciendo: \emph{Nuestro pobre Tito, a
fuerza de sufrir ha perdido la chaveta}\ldots{} Omito los pormenores del
proceso patológico, hora tras hora y día tras día, en aquella existencia
de clínica, monótona y triste\ldots{} Debo añadir que la imaginación
endulzaba mis males, ora tiñendo de color rosa las paredes de mi
caverna, ora dejándome ver con los ojos cerrados objetos y figuras
enteramente arbitrarios y convencionales. De esta labor anárquica de mi
fantasía resultó que, hallándome despierto en mi sillón de paciente
resignado, paseábame por las calles viendo todas las cosas como las
viera en mis tiempos de perfecta salud, hablaba con los amigos, hacía
visitas, y a mi casa tornaba tranquilamente con un paquetito de dulces
para Casiana.

Si este regalo de vida ilusoria dábame la imaginación hallándome
despierto, ¿qué no me daría en las horas del descanso nocturno, bien
arrebujado entre las sábanas?\ldots{} Una noche de furiosa tormenta con
desaforados truenos y copiosa lluvia, que azotaba las paredes y sacudía
los cristales de mi ventana, entraron en mi habitación tres
\emph{Efémeras}. Saltonas, risueñas y parlanchinas, tomaron asiento en
los bordes de mi cama. Asustado me incorporé y les dije: «¿Por dónde
entrasteis, picaronas?» Y una de ellas, acercándose tanto a mí que su
aliento frío me dio en la cara, contestó: «Entramos por un cristal roto
de la claraboya de la escalera, y aquí nos tienes.» Suscitose entonces
un vivo diálogo que transmito a la posteridad en la forma más concisa:

«\emph{Yo}.---¿Sois espíritus traviesos, maleantes, desligados del
gobierno y autoridad de la Madre?

\emph{Efémera 1.ª}---Somos ninfas libres y desocupadas, dueñas del
espacio.

\emph{Efémera 2.ª}---Llevamos de un confín a otro las razones de la
sinrazón.

\emph{Efémera 3.ª}---Nos divertimos despertando a los dormidos, y
adormeciendo a los que se tienen por muy despabilados.

\emph{Yo}.---\emph{(Defendiéndome de los pellizcos y estrujones con que
me atormentaban las seis manos de aquellas malditas hembras.)} ¿Qué
queréis de mí, espíritus desmandados, aviesos? Idos de mi casa, dejadme
en paz.»

Furioso me arrojé del lecho gritando: «¡Casiana, Casiana, despierta,
levántate, que hay duendes en la casa!» Y las raudas féminas, que ya me
parecían harpías, brincaban por la habitación y chillaban
desaforadamente. En su algarabía de aves parleras destacose este
concepto: «No busques a tu Casiana. Tu dulcísima compañera se divierte
ahora con otro muñeco\ldots» Como loco me abalancé hacia el lecho de
Casianilla, colocado en otro testero de la estancia, y palpando en las
ropas revueltas advertí que estaba vacío\ldots{} Desaparecieron las
diablesas con revoloteo susurrante, y yo, medio desnudo, caí fatigado en
el sillón de la paciencia, sin cesar en mis alaridos angustiosos:
«¡Casiana, don José, Nicanora!\ldots»

La primera que vino en mi auxilio fue Casiana, haciéndose de nuevas y
asegurando que se levantaba en aquel instante. «Tú no dormías en esa
cama---le dije, rechazando sus caricias.---Tú, ausente de mí, te
divertías con otro muñeco\ldots» Disputamos un rato. Yo callé, al fin,
guardando mis recelos, con la idea de observar en noches y días
sucesivos\ldots{} Desde aquel inaudito suceso, real o imaginario, el
monstruo de los celos empezó a morderme el corazón\ldots{}

Al siguiente día, el doctor Albitos, después de un largo cuchicheo que
tuvieron con él apartados de mí don José y mi costilla, me recetó
bromuro en frecuentes dosis, y cuando me lavaba los ojos con la ducha de
vapor y me ponía colirio de atropina para impedir que se soldasen los
bordes del iris, díjome cariñosamente: «No sólo hay que proveerse de
paciencia, querido, sino también de serenidad y de sentido común para no
dejarse arrebatar por ideas insanas, que insubordinan el sistema
nervioso y dan al traste con la acción medicatriz. Ánimo, amigo.
Resígnese a no ver nada por ahora, que mejor está ciego que el que ve
visiones.»

Me convenció Albitos por el momento; mas no tardé en volver a mi
horrible pesimismo. Creí notar en Casiana cierta displicencia o
cansancio, que atenuaba su celo de enfermera\ldots{} Aplicando después
toda mi observación a Segismundo, traté de escrutar por sus palabras y
actitudes el estado de su conciencia. Advertí en él menos acritud en la
ironía, y un delicado estudio para medir los conceptos y darles
estructura familiar y una intención candorosa. Oyéndole, yo decía para
mí: «Tú conciencia se ha impurificado. Ya no eres el mismo. Quieres
engañarme y no lo conseguirás.»

Con ánimo de sondearle le dije: «Segis, alguna noche de estas has estado
tú en casa sin entrar a verme, y has permanecido en una habitación
interior hasta la madrugada o hasta el día siguiente.» La contestación
fue un reír descompuesto de Segismundo, y el sostener que yo desatinaba.
Pero bien conocí que su risa era fingida, como de histrión que no domina
su papel, y del mismo modo aprecié las burlas que, por lo que dije,
hicieron de mí Casiana y Nicanora, allí presentes. Ocurrió entonces un
hecho que hubo de aumentar mi escama. García Fajardo varió sutilmente de
conversación, largándome estas parrafadas que me dejaron atónito:

«Se me olvidaba decirte, querido Tito, que un periódico de gran tirada
viene publicando hace días unos artículos, muy bien escritos, que llaman
grandemente la atención. No se habla de otra cosa en Madrid.

---¿Y a mí qué me importa que hablen o no hablen de artículos de
periódico que yo no he leído ni podré leer en mucho tiempo? ¿Para qué me
cuentas esas cosas, tontaina?

---Te las cuento porque todo el mundo dice que esos artículos son tuyos,
y verdaderamente, su estilo y gracia delatan el ingenio de Proteo
Liviano.

---¡Qué desatino!\ldots{} ¿Y de qué tratan los articulejos, que por lo
visto son anónimos?

---El asunto, interesantísimo, está tratado de una manera magistral. La
tesis es que el Gobierno español no procede con altas miras patrocinando
el casamiento del Rey Alfonso con su prima Mercedes. Si Cánovas, como
dice la voz pública, sabe ver el porvenir y presiente la España futura
redimida de tanta barbarie, debe entablar negociaciones para enlazar a
don Alfonso XII con la princesa Beatriz de Inglaterra, hija menor de la
Reina Victoria. En las estipulaciones matrimoniales se reconocería a
Beatriz el derecho de mantener viva su fe protestante al venir a ocupar
el trono de España. De este modo se planteaba sobre sólida base el
problema de la libertad confesional, y pronto entraríamos en una vida de
tolerancia, de cultura, dejando de ser rebaño predilecto del Romano
Pontífice.

---Yo no escribí eso, yo no sé nada de eso---exclamé, en tono
descompuesto y airado.---Tales enredos son invención tuya para
mortificarme.

---No, no. Todos creen que tú eres el autor de los artículos. Por cierto
que en uno de ellos dices que ya hubo conatos de negociaciones en la
primavera del año pasado, cuando estuvo en Madrid el Príncipe de Gales.

---Quizás cuando vimos aquí a ese Príncipe dije yo algo de eso. Pero no
fue más que una idea, un decir, nada\ldots{} Ahora estoy pensando que
toda esa monserga la has escrito tú, Segismundo, y que me la atribuyes a
mí para aumentar mis cavilaciones, mis sobresaltos, y hacerme más viva y
patente la sensación de mi inutilidad.»

Comprendiendo Segis que yo me excitaba demasiado guardó silencio,
dejando el asunto para mejor coyuntura. Con ligeros descansos, mis
inquietudes tomaron cuerpo en los días subsiguientes. Mi caverna se
teñía de un azul intenso algunas veces, otras de un rojo de
sangre\ldots{} Despierto creía notar que eran demasiado largas las
ausencias de Casiana. A lo mejor venía con la historia de que su tía
Simona estaba enferma del hígado. ¡Así reventara!\ldots{} Dormido, o a
medio dormir, adquiría la certidumbre de que estaba vacío el lecho de la
que fue mi dulce compañera\ldots{} Mi corazón era ya una piltrafa,
destrozado por la mordedura de los celos\ldots{}

Una tarde siniestra de soledad y sufrimientos, mi exaltación fue tan
grande que salí por los pasillos dando gritos y tropezando en las
paredes. Ido vino a mi encuentro para contenerme y llevarme de nuevo a
mi cuarto, y las expresiones melifluas de su filosofismo angelical
fueron el fulminante que hizo estallar mi cólera: «Déjeme usted\ldots{}
No me toque\ldots{} Usted me ha vendido, usted es un traidor\ldots{}
Quítese de mi presencia. En su casa se ha labrado mi deshonra\ldots{} Le
tenía a usted por un santo, y resulta usted un alcahuete\ldots{} Atrás,
villano\ldots{} Déjeme en paz.» Me arrojé en la cama, ocultando mi
rostro entre las almohadas, y oí los gemidos del pobre Sagrario que
lloraba como una Magdalena.

Pasado un mes, pienso que no entero, de sufrimientos horribles más en lo
moral que en lo físico, sobrevino el extraño incidente que a
continuación se narra. Antes debo indicar que a ratos iniciábase ligero
alivio en mi dolencia de los ojos. La percepción luminosa cada vez era
mayor, y refugiándome en una casi obscuridad podía distinguir vagamente
los objetos de más bulto. El amable y gracioso Albitos me vaticinó que
antes de tres o cuatro semanas mi retina cumpliría como buena ejerciendo
las funciones que le asignó la Naturaleza. Pero no contaba el buen
doctor con las aventuras de mi dislocada imaginación, lanzándose sin
freno ni paracaídas a los espacios novelescos. Una tarde o noche, no lo
sé, hallándome solo en mi caverna teñida de color violeta con franjas de
oro, vi que a mí se llegaba una mujer. ¡Ay!, era \emph{Efémera}, la
buena, la estatuaria, la que en Tafalla y Madrid me trajo dulces
mensajes de mi adorada Madre. La reconocí al sentir en mi hombro su mano
marmórea. Alargué la mía para coger su túnica, y advertí que sobre esta
llevaba un delantal casero.

«Aunque te has puesto el delantal de Casiana---dije yo,---bien te
reconozco, \emph{Efémera}.» Tras breve pausa, la fantasma pronunció
estas apagadas voces: «No soy \emph{Efémera}. Tampoco soy Casiana,
aunque lleve su delantal para ser tu servidora y enfermera.» Yo callé,
atontado y confuso, y mi perplejidad subió de punto cuando escuché este
otro concepto: «¿No me conoces por el acento, pobre Tito? ¿Tendré que
decirte mi nombre? Soy \emph{Leona la Brava}.

La gentil aparición se sentó junto a mí y, echándome su brazo por encima
de los hombros, me habló de esta manera: «Vengo a tu lado para cuidarte
y servirte en sustitución de la mujer desleal que te abandona seducida
por el ingrato Segismundo\ldots» Algo debí yo de responderle, quizás
expresando consternación o vergüenza por la desdicha que me anunciaba.
Insistió ella en su afirmación, prosiguiendo así: «A tu lado me tendrás,
si quieres, hasta que recobres la vista y la salud. Si una compañera de
amor y de caridad has perdido, en mí tienes otra más solícita y fiel que
esa desventurada recogida por ti del arroyo.» Tuve un momento de
horrorosa duda; pero no tardé en recobrar toda la fuerza de mi
arrebatada inventiva genial. Como yo me asombrase de que \emph{Leona}
descendiera de su posición rumbosa, para unir su existencia a la de un
hombre enfermo y casi pobre, la dama de Mula me dio esta explicación de
su actitud humilde:

«Debí empezar por decirte, Titín salado, que hace algún tiempo me
despeñé de aquella cumbre de bienestar y lujo jactancioso en que me
viste antes de caer enfermo. Reñí con Alejandrito, ¿no lo sabías? Te
contaré el caso con descarnada sinceridad. \emph{J'adore la vérité. Je
haïs le mensonge}. Al dichoso Alejandrito le daré yo lo suyo, que no es
poco: hombre más impertinente y más chinche no ha nacido de madre; y a
mí me daré lo mío, que es más grave\ldots{} Pues, hijo, apestada de
\emph{mon bourgeois} tuve una tentación\ldots{} cosas del temperamento,
de la ociosidad\ldots{} En fin, chico, que me colé demasiado, y cuando
\emph{mon vieux} se enteró de que yo la había puesto en la cabeza unas
cositas puntiagudas\ldots{} que no traen gran malicia cuando los hombres
no son casados\ldots{} figúrate la trapatiesta que se armó. Total, que
caí de mi escabel dorado. Como yo me había hecho al lujo y a la
\emph{bonne chère}, me vi en el caso de vender algunos muebles y empeñar
alhajas para seguir viviendo a mi modo. Aunque aún no me tienes
enteramente tronada, camino de eso voy. A pesar de mis tropiezos, soy
siempre una mujer buena, y vengo a tu lado para renovar nuestro cariño y
practicar las obras de misericordia.»

Apenas empecé yo a comentar tan vulgar historia, \emph{Leona} me pidió
que siguiese escuchando, pues aún faltaba la segunda parte. «Es el juego
de la vida humana---dijo,---el eterno balancín, el vaivén de las
prosperidades y las miserias. Cuando yo me precipitaba en la desgracia,
tu Casianilla subía de golpe a grandezas que nunca pudo soñar. Has de
saber que tu dulcísima cuanto traidora compañera, inducida por esa
lagarta de Simona, ha cobrado a toca teja todos los atrasos de su sueldo
como Inspectora de Escuelas. Para ello ha tenido que mover ciertas
influencias altas y bajas el pillastre de Segismundo, \emph{le demon
ironique}.

»¡Menuda suerte la de esos bribones! Mientras la señorita Conejo
embolsaba buenos duros por un empleo que nunca desempeñó, Segis pescó un
magnífico destino en el Ministerio de Ultramar. ¿No lo sabías? Pues el
Marqués de Beramendi le pidió a Cánovas esa bicoca, y don Antonio al
instante\ldots{} pum, pum\ldots{} Como comprenderás, ahora están en
grande. La Conejo lleva brillantes en las orejas y García Fajardo fuma
puros de a peseta. Han tomado un piso en la Costanilla de los Ángeles.
¿Ves qué vueltas da el mundo, Tito?

---Sí, sí, qué de vueltas tan horribles\ldots---exclamé yo.---Vueltas
damos todos\ldots{} todos\ldots» Me sentí anonadado, me faltaba la
respiración\ldots{} Púseme en pie, giré sobre mí mismo y caí en redondo
al suelo\ldots{}

\hypertarget{xiv}{%
\chapter{XIV}\label{xiv}}

Después de aquel que yo no sabía si llamar suceso, fenómeno, pesadilla o
caso real, caí en un estado parecido a la idiotez. Hablaba muy poco, no
sólo por desgana de conversación sino porque sentía dificultad para
articular las palabras. Advertí que Albitos mostrábase intranquilo
respecto al curso de mi dolencia cerebral: la de la vista iba
indudablemente mejor. Ya no tenía yo dolor en las sienes ni escozor en
los ojos, ya veía un poco más. Pero hacíaseme imposible distinguir las
facciones de la mujer que me servía. ¿Era Casiana, era \emph{Leona}?
¿Era una sola que cambiaba de rostro a cada momento? Tocábale yo las
orejas para ver si tenía brillantes. Mi olfato buscaba en sus vestidos
el perfume que solía usar Leonarda. En las visitas de los amigos que
iban a mi casa tampoco pude discernir si entre ellos hallábase
Segismundo, pues las voces de todos me parecían la misma.

Una noche de largo insomnio me levanté a palpar el lecho de mi
enfermera. No estaba vacío.

Pregunté: «¿Eres \emph{Leona}?»

Y la respuesta fue: «Sí, soy \emph{Leona}. Déjame dormir.»

Las pérdidas de sueño durante la noche cobrábamelas por el día durmiendo
a pierna suelta. No sé cuándo me sacó de mi hondo letargo una mano que
tocaba mi frente, mano fría y marmórea.

«¿Eres Casiana?---pregunté a la persona que me despertó.

---No.~Casiana se fue de paseo con su marido.

---¿Eres Leonarda?

---No.~Leonarda ha salido a comprarte las medicinas que hoy recetó
Albitos.»

La mano de mármol cogió la mía, y tirándome del brazo me incorporó en la
cama. Al propio tiempo, una voz de dulcísimo timbre me dijo: «¿No me has
conocido? Soy \emph{Efémera}, la fiel y amable, la de Tafalla, la
mensajera de \emph{Clío}. Levántate y obedéceme.

---¿Qué tengo que hacer?

---Vestirte para una visita y venir conmigo adonde yo te lleve.

---¿Pero cómo he de salir yo, ciego, enfermo?

---Te digo que me obedezcas, que me sigas y calles.

---Mi ropa ¿dónde está?

---Aquí la tienes---dijo poniendo sobre la cama todas las piezas, sin
que faltase una.

Mientras me vestía vi muy clara la figura estatuaria, con su helénico
rostro y el sutil ropaje negro. Era mi \emph{Efémera}, la ninfa
predilecta, la que me llevaría quizás a los brazos de mi excelsa Madre.
Con arte mágico me vestí, sin que me faltara ninguna prenda ni se me
olvidase el menor detalle. Por el mismo arte maravilloso y taumatúrgico
me condujo \emph{Efémera} de la mano, sacándome no sé si escaleras
abajo, o escaleras arriba por la claraboya de cristales. Lo cierto fue
que me encontré en la calle bueno y sano, como en mis mejores tiempos,
viendo claramente todas las cosas, alegre y muy orgulloso de llevar en
mi compañía una estatua griega. Todo cuanto hallé a mi paso era de una
perfecta naturalidad. Tan sólo me parecía ilógico y absurdo que los
transeúntes no se fijaran en que iba yo acompañado de una señora de
mármol, sin más ropa que el vaporoso túnico negro.

\emph{Pian pianino}, cambiando frases cortas y vulgares, llegamos a la
calle de Alcalá y de rondón nos introdujimos en la Presidencia del
Consejo, sin que los guardias civiles que custodiaban la puerta pararan
mientes en el ser fantástico que iba conmigo. En la escalera obscura y
angulosa me encontré solo, y solito llegué a la puerta de la
Subsecretaría, a punto que por ella salía Fernández Bremón con un fajo
de papeles. «Qué caro te vendes, Tito---me dijo el sagaz
periodista.---Puedes pasar. Aunque Saturnino no está solo, él te dirá si
el Presidente te recibe al momento, o si tienes que esperar un rato.»

En el despacho de Esteban Collantes tertuliaban unos cuantos señores de
esos que van a las oficinas a matar el tiempo rumiando la comidilla de
la actualidad política. No más de un cuarto de hora permanecí en aquella
sociedad \emph{charlamentaria}, deslizando algunas palabritas en la
ociosa conversación. Cuando me llegó la vez, Esteban Collantes me
condujo al salón presidencial, al tiempo que se retiraban el Marqués de
Orovio y el Conde de Toreno. Y heme aquí, lectores cachazudos, crédulos
y traga bolas, en el despacho \emph{del monstruo}, hablando mano a mano
con él en el diván frontero a la mesa escritorio.

Empezaré por decir que olfateaba yo el ambiente, creyendo rastrear la
persona invisible de la Madre \emph{Clío}. Dábame en la nariz el
delicioso y peculiar olor suyo. No sé si os he dicho que mi Madre
gastaba en sus ropas un solo perfume, el aroma exquisito de los tomillos
del monte Hymeto.

Entrando en materia sin preámbulos, como buen tasador del tiempo, don
Antonio me dijo: He leído los artículos de usted. Yo leo todo escrito
que tiene entre sus líneas una intención recta y sana, aunque el autor,
dejándose arrastrar de las seducciones de la forma, no penetre en las
entrañas de la realidad, que no está nunca en la superficie. Ha tratado
usted con sumo arte y donosura el asunto del casamiento del Rey. Escribe
usted muy bien, y la gallardía con que eleva sus miras hacia la Historia
me encanta. Pero ha de permitirme que a sus opiniones oponga las
realidades indestructibles que para tan complejos problemas nos ofrece
la constitución interna de nuestro país.»

Bien claro vi que se trataba de los trabajos periodísticos cuya
paternidad me había colgado Segis. Con gran agilidad de espíritu me
declaré modestamente autor de los articulejos, sin que pudiera
percatarme de la ocasión y lugar en que hube de escribir semejantes
cosas. Para no hacer el ridículo dejé correr el engaño y seguí prestando
atención al gran don Antonio, que continuó de esta manera:

«Si en algunas afirmaciones se ha equivocado usted, en otras ha tenido
un feliz acierto. Hubo en efecto negociaciones para traer al solio de
España a la Princesa Beatriz de Inglaterra. Cuando tuvimos aquí al
Príncipe de Gales planteé yo el asunto. Pero debo decirle que lo inicié
tímidamente, movido de un ideal histórico que siempre me sedujo, aunque
nunca dejé de prever las dificultades de tan audaz empresa. No pasaron
aquellas tentativas de una exploración que pronto quedó terminada, pues
apenas llegamos a tratar del cambio de religión para que la Princesa de
Inglaterra pudiera ser Reina de España, se vio la imposibilidad de
llegar a un acuerdo. Nos hallábamos ante un nudo imposible de desatar,
porque el puritanismo protestante es tan fanático como nuestro
catolicismo. En cuanto la Reina Victoria se enteró de que su hija tenía
que hacerse papista para ser nuestra Soberana, cerró la puerta a toda
inteligencia. Esto no se hizo público; por el contrario, se guardó un
secreto escrupuloso para evitar el estallido de un turbón ultramontano
que sabe Dios a qué extremos de violencia habría llegado.

---¿Y cree usted, señor don Antonio---me atreví yo a decirle con el
mayor respeto,---que si la Reina Victoria hubiera mirado con buenos ojos
el cambio de religión de Beatriz habríase producido aquí alguna tormenta
clerical?

---Seguramente, sí. Pero ésa la hubiese sofocado yo. Respondo de ello.

---También he dicho en mis artículos---manifesté codeándome con el
ilustre estadista---que el matrimonio anglo-español ofrecía dificultades
con abjuración o sin ella; pero luego sostuve que el problema
confesional, el gran problema hispánico, no podía ser abordado y
resuelto aquí más que por un hombre que ha venido a ser dueño de todas
las voluntades: este hombre es don Antonio Cánovas del Castillo.

---Ay, amigo---dijo el jefe de la Situación, afirmando los lentes en el
caballete de su nariz;---no me suba usted un punto más de la altura en
que me han puesto las circunstancias, ni me atribuya un poder omnímodo
sobre la opinión, que no podrá nunca lograr quien no posea dotes
sobrenaturales. Abata usted un poco su fantasía, y véngase conmigo a
examinar de cerca el ser interno de nuestra patria. Esta vieja nación,
con sus glorias y sus tristezas, sus fuerzas y sus recuerdos, sus
instituciones aristocráticas y populares, y su extraordinario poder
sentimental, constituye un cuerpo político de tan dura consistencia que
los hombres de Estado, cualesquiera que sean sus dotes de voluntad y
entendimiento, no lo pueden alterar. El alma de ese cuerpo es igualmente
maciza, petrificada en la tradición y desprovista de toda flexibilidad.
El único gobernante capaz de llevar a esa alma y a ese cuerpo a un nuevo
estado de civilización es el Tiempo, y yo seré todo lo que usted quiera,
amigo Proteo, pero el Tiempo no soy.

---Me conformo con esa opinión fatalista por ser de usted. Pero es
triste cosa en verdad que España tenga que subsistir largo tiempo bajo
un poder extraterritorial que entorpece y ahoga todos sus alientos, y
ata sus manos y sus pies con el cordón dogmático, inutilizándola para
emprender nuevas direcciones de vida. Esto dije en mi último artículo, y
esto repito ante usted, suplicándole que sea benévolo con mis audacias.

---Admito las audacias como labor sintética y teorizante, como un
bosquejo artístico de la Historia del porvenir. Mas yo no teorizo, yo
gobierno, señor Liviano, y como gobernante estoy amarrado por los ciento
y tantos cordones de la realidad. De mi gestión depende que ese ser
interno que he descrito a usted no se convierta en elemento trágico. Mi
deber es sofocar la tragedia nacional, conteniendo las energías étnicas
dentro de la forma lírica, para que la pobre España viva mansamente
hasta que lleguen días más propicios. No podemos marchar a saltos, ni
con trompicones revolucionarios. Las algaradas y las violencias nos
llevarían hacia atrás en vez de abrirnos paso franco hacia un adelante
remoto.

---También escribí que aplicando con firmeza las Regalías de la Corona o
del Estado, un Gobierno fuerte y hábil podría contener al Papa dentro de
su esfera espiritual, y atajar sus intromisiones vejatorias en el
régimen interior de los pueblos.»

Cuando esto decía yo sentí más intenso el olor de la Madre, la fragancia
de los tomillos del monte Hymeto. Después de vacilar un instante, don
Antonio habló así: «Mucho tiento será menester hoy para desenvainar en
nuestra edad la espada que esgrimieron Carlos V, Felipe II y Carlos III
contra diferentes Papas, desde Clemente VII hasta Clemente XIV. Aquellos
Monarcas eran de más fuste que los que ahora tenemos, y el Papa de hoy,
desposeído del poder temporal, aprieta furiosamente las clavijas del
mecanismo dogmático con que gobierna las conciencias católicas. Yo
procuro por todos los medios fortalecer el poder real, debilitado por
las agitaciones revolucionarias y por las propagandas de los ambiciosos
de bajo vuelo. Y si en este reinado y en los siguientes mantiene su
fortaleza el poder real, será obra fácil reducir y someter al poder
eclesiástico.

»Por lo demás, hemos resuelto del modo más feliz el asunto interesante
del casamiento del Rey. ¿Qué nos importan las majaderías del inquieto
Montpensier, ni la palinodia que ha tenido que cantar para poner a su
hija en el trono de España? Hemos doblado esa hoja triste de las
querellas dinásticas. Los resquemores de doña Isabel han ido a parar a
la cesta de los papeles rotos. Esa buena señora no tiene derecho a
trazar una página rencorosa en los anales contemporáneos. Ningún efecto
nos han hecho las ridículas bravatas de mis buenos amigos los
\emph{moderados} de la vieja cepa, ni el discurso del pobre Moyano
sacando a relucir un texto arcaico y manido de Donoso Cortés. Lo
importante, lo definitivo es que la Infanta Mercedes, futura Reina de
España, atesora las cualidades más bellas: linda, modesta, dócil,
amable, inteligente, apenas lanzado su nombre en el remolino de la
opinión, se ha hecho popular. ¿Qué más podemos apetecer? Reina bonita,
discreta, popular\ldots{} \emph{Por lo demás}\ldots»

Dejé de percibir la voz de don Antonio. Después vi su figura en pie,
desvanecida, alejándose de mí. El grande hombre se hallaba en un salón
lujoso, rodeado de damas elegantes, Marquesas y Duquesas que le
agasajaban solicitando su conversación ingeniosa, amenísima, a veces
cáustica. Entre aquellas señoras creí ver a la dama de Mula, y
seguramente vi a \emph{Mariclío}, fastuosa, calzada con el alto coturno.
Pasó a mi lado inundándome con su fragancia helénica.

Lo más extraño fue que detrás de la Madre vino hacia mí Casiana. Al
verla empecé a dar voces, y entonces sentí que me sacudían los brazos
diciéndome: «Despierta, hijo, que ya has dormido más de la cuenta.» Mis
primeras palabras al abrir los ojos fueron: «¡Ah, qué delicioso olor a
tomillos!» Casiana me acercó al rostro un ramo de estas aromáticas
hierbas. «¡Déjame gozar de aroma tan delicioso!---exclamé yo.---¡Ay,
pero esas plantas no son del monte Hymeto!

---Son de la Casa de Campo.

---¿Vienes tú de allí, chiquilla?

---No, hijo, no. Esto me lo trajo Nicanora que fue allá con varias
amigas a visitar a un guarda, pariente suyo.

---¡Oh, la Casa de Campo! Allí estarían paseando la Infanta Mercedes y
el Rey Alfonso, que son novios y se van a casar pronto, ya lo sabes. La
futura Reina es simpática, humilde, linda, y apenas se habló de su boda
se hizo popular.

---Todos hablan bien de ella menos Segismundo, que está con la tecla de
que por ser hija de Montpensier debían haberla puesto a cien leguas del
trono de España. El demonio de Segis y otros tan locos como él, ya lo
oíste noches pasadas, querían que nos trajeran aquí una
\emph{protestanta} para casarla con don Alfonso.

---Cánovas me ha dicho que la idea es hermosa. Pero que se opone a
realizarla \emph{el ser interno}\ldots{} ¿lo entiendes?\ldots{} el
cuerpo y alma de esta Nación, que es Católica hasta los tuétanos. Don
Antonio teme que \emph{el ser interno} se le vuelva trágico, y trata de
irlo conllevando por lo lírico hasta que, fortalecido el poder real,
\emph{etcétera}\ldots{} En suma, Casianilla de mis pecados, que ha de
llover mucho hasta que los Gobiernos de esta tierra puedan decirle al
amigo Pío, o a sus sucesores: \emph{Tente allá, Papa, que los españoles
ya sabemos salvarnos cada cual a su modo}.»

\hypertarget{xv}{%
\chapter{XV}\label{xv}}

Desde aquel día, que en mi mente quedó marcado con el recuerdo de los
tomillos del monte Hymeto, avancé rápidamente en la curación de mi
vista. La horrenda \emph{Queratitis}, que había sido mi suplicio en gran
parte del año 77, se apartaba de mí, se retiraba, se iba. Tan acertado
estuvo Albitos en devolverme la luz de los ojos como en el régimen y
medicinas aplicadas para librar a mi cerebro del desorden anárquico.
Gracias a esto no tardaron en deshacerse por sí mismas las fábulas que
mi intelecto, lanzado a un delirio de Carnestolendas, forjó para
embromar a la razón.

La quimera que más tardó en disiparse fue la de \emph{Leona la Brava}.
Mas tuve la suerte de que esta viniera un día a visitarme, no habiéndolo
hecho antes por haber estado ausente de Madrid durante algunos meses.
Viéndola en su propio ser, sin ninguna mudanza en su estado de
prosperidad y rumbo, comprendí que era pura novela mía picaresca lo de
los cuernecitos que le puso a don Alejandro, novelón sentimental el
venir a ser mi enfermera, y terrorífico folletín por entregas el
truculento caso de la fuga de Casiana con Segismundo. Este buen amigo me
desengañó también con su asidua presencia, con la lealtad y gracejo de
su conversación amenísima. En cuanto a la entrevista con Cánovas, y a la
intervención de las \emph{Efémeras} buenas y malas, diré que esto lo
trasladaba yo a la esfera de mis relaciones ideológicas con
\emph{Mariclío}, estableciendo una especie de equilibrio entre lo cierto
y lo dudoso, y saboreando los puros goces que encontré siempre en la
verdad de la mentira.

Antes que se me olvide, debo anotar en los anales de mi Madre el
estrepitoso fin del drama económico de doña Baldomera, según me lo contó
testigo de tanta autoridad como Segismundo. Llegado el momento en que la
sutil arbitrista vio agotada la simplicidad de los imponentes, determinó
levantar el vuelo hacia una región lejana de la esfera terráquea. Los
mismos que en el fervor del entusiasmo la llamaron \emph{nuestra madre},
al ver en la casa señales de tronicio, no se contentaban con menos que
con arrastrar a su protectora por la Plaza de la Cebada y calle de
Toledo, hasta la Fuentecilla. Agregó Segis a sus noticias este
comentario fieramente sarcástico:

«Ved aquí, amigos míos, la mejor muestra de la injusticia del pueblo,
que si entregó sus ahorros a la genial banquera hízolo por ambición
canallesca y por su idea estúpida de la multiplicación del vil metal. Yo
sostengo que mi \emph{jefa} y \emph{principala} no engañó más que a los
que ya venían engañados y ciegos desde su nacimiento. Procedió como
hábil financiera que ve la parte suya en un negocio, sin cuidarse de la
parte de los que operan con ella. Según mi cálculo, la buena señora no
se ha llevado más que unos siete millones de reales, cantidad mezquina
si se compara con los millones desfalcados por agiotistas de más alta
categoría social.

---Ya lo creo---afirmé yo.---Ejemplos mil tenemos aquí del
\emph{Baldomerismo} en grande escala, de Sociedades de Seguros
inseguros, en las cuales, unos cuantos caballeros de muchas campanillas
han arramblado con los ahorros de una o dos generaciones, quedándose
luego tan frescos. A esos elegantes \emph{Baldomeros} les han dado
títulos de Condes y Marqueses, y andan por ahí con el rango y
tratamiento de \emph{Excelentísimos señores}.

---A la hija de Larra---prosiguió Segis con profunda convicción---le
daré yo el superlativo de \emph{archi-excelentísima}, pues era muy buena
para sus empleados, afable con los imponentes, a quienes llamaba sus
hijos, y observante del axioma de que \emph{la caridad bien entendida
empieza por uno mismo}. Si le dieron siete millones, qué había de hacer
la pobrecita más que cogerlos y decir: \emph{gracias, caballeros; me voy
a tomar aires}.

»Ahora os contaré la fuga de la banquera, que fue en la madrugada del 4
de Diciembre, día de Santa Bárbara, festividad muy del caso para esta
clase de catástrofes. La señora estuvo con unas amigas en el teatro de
la Zarzuela viendo la función, y concluida esta se fue a su casa, calle
del Sordo. Allí se preparó para el viaje, y antes de amanecer salió en
un coche de colleras camino de Pozuelo, donde tomó el tren mixto del
Norte y\ldots{} ¡Adiós, Madrid, que te quedas sin gente!

»El secretario de la dama, don Saturnino Iglesias, evaporose también. Se
ha dicho que un señor Pallares, que fue Jefe de Policía en tiempo de la
República, ha favorecido el mutis de la gran histrionisa de los números.
Por mi parte, no he tenido que \emph{desaparecerme}, ni temo que me
empapelen como funcionario modestísimo de aquella mágica oficina, porque
en el último día de Noviembre olí la quema, pedí mi cuenta y presenté la
dimisión, pretextando tener que ausentarme para un asunto de familia.»

El mutis de doña Baldomera en el escenario social tuvo, como supondréis,
sus naturales derivaciones. De ello se hablará cuando la sagaz
hacendista reaparezca en el campo de la actualidad. Por el momento, en
las agonías del 77 y primeros vagidos del 78, lo más importante para mí
era el acentuado restablecimiento de mis ojos, y la reconquista de la
facultad visual perdida en largos y dolorosos meses. Los que no han
vivido en tinieblas por más o menos tiempo no conocen el purísimo,
inefable gozo de ver y contemplar hombres y cosas, lo feo y lo bonito,
la Naturaleza toda en la plenitud de sus maravillosos aspectos. Es como
vivir de nuevo. Yo resucité, yo renací, y difícilmente puedo expresar mi
alegría.

Coincidió mi resurgimiento a la vida con los desposorios de Alfonso y
Mercedes, obligado motivo de festejos oficiales, palatinos, y en aquel
caso señaladamente populares. Yo no me acerqué a la basílica de Atocha,
teatro del espléndido ceremonial, ni vi el desfile de la procesión
epitalámica desde el templo a Palacio. Aunque frecuentaba ya la calle y
los paseos, no quise meterme en el remolino de las muchedumbres
regocijadas, ávidas de contemplar tan lucido espectáculo. Pero, sin
verlo, la frescura de mi imaginación permitíame apreciar el soberbio
cuadro, por el recuerdo de otras cabalgatas del propio estilo en
diferentes ocasiones de la Historia.

Desde el Retiro, donde me paseaba con Casianilla, veía yo en mi mente
las carrozas de la Casa Real, los arreos del guadarnés, los soberbios
caballos que pausadamente tiraban de los coches, el mover rítmico de las
cabezas de los brutos adornadas de vistosos plumachos, las bordadas
libreas, las blancas pelucas, el sinfín de jinetes palatinos y
militares, los timbaleros y clarines, reyes de armas, monteros de
Espinosa, caballerizos, correos y carreristas, los mancebos, lacayos y
palafreneros, y por fin, los regios novios y el acompañamiento de
coronadas testas, de Príncipes, embajadores y magnates, que componían el
cortejo nupcial. Si doña Isabel II brillaba por su ausencia, por su
presencia majestuosa resplandecía doña María Cristina, de albo cabello y
dulce sonrisa que el paso de los años no había logrado destruir. Don
Francisco de Asís ocupaba el puesto que por regia clasificación le
correspondía, y el suyo los Duques de Montpensier y las Infantas
hermanas de Alfonso XII.

Si aparté mis ojos, recién abiertos a la luz, de estas magnificencias
callejeras, no pude resistir la tentación de presenciar las dos corridas
de toros con caballeros en la plaza, que fueron el número popular en el
programa de los reales festejos. Obra fue del Municipio esta solemnidad
taurina. Por cierto que los ediles discutieron calurosamente si debía
celebrarse en la Plaza Mayor, teatro antaño de los regios torneos
taurómacos así como de los autos de fe, o utilizar para el caso la nueva
Plaza de Toros, inaugurada en 1874. Prevaleció por fin este criterio, y
yo, ávido de gozar el lindo espectáculo, tomé cuatro delanteras de
grada, pues además de Casiana convidé a Segis y a Ido del Sagrario.

Llegado el día feliz entré en la Plaza con mi pareja y mis dos amigos,
arrebatado de un gozo infantil que embellecía y agrandaba todas las
cosas. El nuevo Circo, que yo veía entonces por primera vez, se me
representaba superior en grandeza y hermosura a la idea que tenemos del
Coliseo de Roma, y el ornamento de banderolas, escudos, gallardetes,
guirnaldas, guardamalletas, lanzas de torneo y demás requilorios, se me
antojó lo más bello y gracioso que pudiera imaginarse. El alborozo de mi
espíritu convertía las flores de trapo en naturales y olorosas, los
tapices de percalina en ricos reposteros de seda y oro.

Si de tal modo transfiguraba mi fantasía las cosas materiales, imaginad
mi desenfreno optimista al contemplar el mujerío que en gradas y palcos
dábame la impresión de una corte celestial de belleza y amor. Desde
nuestros asientos veíamos perfectamente el palco regio; cuando en él
aparecieron Mercedes y Alfonso, rodeados de Majestades históricas aunque
cesantes y venidas muy a menos, y de las Princesas y Príncipes de Borbón
y Orleáns, estalló un ciclón de aplausos y aclamaciones que bramaba y
crujía como un cataclismo atmosférico.

Después de colocarse en el ruedo, debajo del palco de los Reyes, una
Compañía de Alabarderos en triple fila y en actitud de firmes, Mercedes
dio la señal para el comienzo del desfile. Tras de cinco alguacilillos
aparecieron por la puerta de caballos los timbaleros y clarines de la
Real Casa con uniforme de gala; seguía una carroza conduciendo a dos
caballeros en plaza, tirada por cuatro soberbios alazanes empenachados;
a los estribos marchaban a pie, como padrinos de campo, \emph{Frascuelo}
y otros dos lidiadores, que eran \emph{Regatero} y Hermosilla, según
alguien me dijo; venían luego dos pajes con rejoncillos, y cuatro más
conduciendo del diestro otros tantos caballos, enjaezados con montura de
raso y pasamanería de oro y plata.

Vi después lo que enumero con la prolijidad que me permite el continuo
pasar de figuras tan pintorescas: otro coche de gala con ocho corceles
empenachados, y lacayos ostentando las libreas de los grandes de España
que apadrinaban a los caballeros en plaza; gran carroza sobresaliente
con adornos y arabescos de plata en su caja, propiedad, según oí, del
Duque de Santoña; tiraban de aquel armatoste dos troncos de poderosos
potros morcillos, y en él iban dos caballeros, vestidos de azul y rojo y
de morado y blanco; marchaban al vidrio los espadas Cayetano Sanz,
Gonzalo Mora, Ángel Pastor y Francisco Sánchez; detrás, pajes con
caballos y rejoncillos, coche de respeto, carruajes de los padrinos
Condes de Bazalote y Superunda, escoltados por lacayos, mancebos y
palafreneros.

Concluían la relumbrante procesión las cuadrillas de lidiadores,
formadas por diecisiete espadas, cuarenta y ocho banderilleros, cuatro
puntilleros, tres chulos y veintisiete picadores, y a la cola iban mozos
de caballos, tiros de mulas de arrastre con preciosos arreos y
mantillas, ramaleros y mayorales luciendo ropa de terciopelo y fajas de
seda. Pensaba yo que humanos ojos no habían visto nunca mascarada tan
espléndida y suntuosa, desfile mareante por lo abigarrado de los
colorines, el esplendor del oro y la plata, el movible oscilar de
plumachos y el continuo pasar de figuras y figurillas, rígidas unas,
flexibles otras, y todas recargadas de tintas chillonas. Casianilla
estaba embobada; Ido del Sagrario abría un palmo de boca; Segis, siempre
descontento y mordaz, burlábase de aquel lujo estúpido y un tanto
chabacano; y yo, que al principio admiraba todo como un chiquillo, acabé
por atontarme ante las vueltas, revueltas y movibles luces de aquel
rutilante caleidoscopio.

La cabalgata dio la vuelta al redondel, y al llegar debajo del palco
real, apeáronse caballeros y padrinos, saludando todos a las Majestades
y Altezas. Los alabarderos abrieron filas, y por la puerta de Madrid
salió la brillante procesión, no quedando en el ruedo más que los
lidiadores y tres alguaciles a caballo.

Comenzada la lidia, los caballeros en plaza rejonearon sus toros. Era la
primera vez que yo veía tal juego, y fuera de la gallardía de los
jinetes y de la soberbia estampa de los bridones, no encontré en ello
gran emoción. El tercer toro rejoneado embistió a uno de los
alguacilillos, que fue a caer con caballo y todo entre los alabarderos,
produciendo algún estropicio. El mismo torito alcanzó a un caballero en
plaza cuando iba a clavar su rejoncillo, le volteó, matándole la
cabalgadura, y el airoso campeón, vestido a la chamberga, hubo de ser
retirado a la enfermería. La lidia ordinaria me interesó un poco al
principio; pero como no entiendo de toros ni frecuento este espectáculo,
acabé por sentir aburrimiento y ganas de que aquello terminara. Ido del
Sagrario, no más perito en tauromaquia, hacía de cuanto veíamos críticas
tan sesudas como la que podría yo hacer de la \emph{Ilíada} de Homero.

En los ratos de hastío convertía mis ojos del ruedo a los palcos y
gradas, para pasar revista al pintoresco público. La hilera de palcos
ofrecía un aspecto deslumbrador. Allí estaban la Navalcarazo, la Belvís
de la Jara, Luisa Campoalange, la Perijaa, y las más admiradas
hermosuras de la Grandeza, luciendo albas mantillas y adorno de camelias
y gardenias en la cabellera y en el seno. No lejos del montón
aristocrático vi a \emph{Leona la Brava} con Carolina Pastrana y otras
amigas del género \emph{demi-mundano}. Ocasión es de decir que, en
aquella época de sus progresos en el arte social, daba la dama de Mula
la mejor prueba de su talento vistiéndose con modestia, procurando
obscurecerse y pasar inadvertida.

En un palco fronterizo entre sombra y sol vi una tanda de mujeres,
ataviadas estrepitosamente con pañolones de Manila, mantillas de
madroños, altas peinetas y gran carga de flores en el pelo. Eran las que
el año 72 hicieron en la Castellana, a las órdenes de Ducazcal, la
famosa manifestación contra la dinastía de Saboya: la \emph{Moño
Triste}, la \emph{Condesa del Real Cuño}, la \emph{Sílfide, Pepa la
Sastra}, la \emph{Cacharrito}, Rosa Huertas, la \emph{Napoleona, Paca la
Alicantina}, la \emph{Eloísa}, la \emph{Clotildona}, etcétera.

Retrocediendo con mi atenta observación hacia la grey aristocrática, vi
en dos palcos a Vicente Halconero y al Marqués de Beramendi con sus
familias. En las gradas, no lejos de nosotros, había tres muchachas
picoteras, inquietas y reidoras, que a ratos miraban hacia mí,
saludándome con lindas garatusas formuladas con los morros y con los
abanicos. «¿Ves aquellas tres chicas que vuelven hacia acá sus rostros
picarescos como haciéndonos burla?---dije a Casiana.---Pues son tres
\emph{Efémeras} que han venido disfrazadas de personas, dejando en
alguna percha de los espacios sus túnicas flotantes. Pertenecen al grupo
de las malas, traviesas y enredadoras. No mires hacia ellas; no les
hagamos caso.» Casiana, sin comprender bien lo que yo decía, se dio por
enterada.

Observamos luego que, en los tendidos, hombres y mujeres comían a
mandíbula batiente y empinaban botellas o zaques, sin desatender los
incidentes de la corrida. La razón de estas merendonas era que,
empezando las corridas a las doce y terminando a las cuatro por causa de
la cortedad de los días, trastornábanse las ordinarias horas de almorzar
y comer.

Entre los accidentes restantes de la lidia ordinaria, el que más
presente ha quedado en mi memoria es el achuchón que dio un toro a los
alabarderos, apostados al pie del palco Real. Rechazaron estos con sus
hierros la embestida del morucho, que volvió a la carga con más coraje,
abriendo brecha. La res sufrió terribles lanzazos, rompiéronse bastantes
alabardas, dobláronse otras, volaron los tricornios por el aire, y
muchos Guardias sufrieron el destrozo de sus uniformes. Pero ni los
alabarderos abandonaron su puesto de honor y de peligro, ni el cornúpeto
se mostraba propicio a terminar la desigual pelea. Fue preciso que el
espada Felipe García colease al codicioso bruto para hacerle abandonar
el campo.

Llegó el momento final, que yo vi con gusto porque ya me cansaba fiesta
tan prolija y fatigosa por el vértigo de sus complicadas emociones. La
inmensidad de la concurrencia dificultaba la salida; largo rato
empleamos en pasar de la Plaza a la calle, y en las apreturas de aquel
atranco, Segis comentaba con negro humorismo el festejo, en su doble
aspecto popular y aristocrático.

«¡Cuánto nos hemos divertido!---exclamó.---¿Verdad, Casiana, que tenemos
retortijones de tripas para todo el año? Me alegro de haber venido para
no verme obligado a leer en la prensa taurina la descripción de esta
chocarrería sublime\ldots{} Si me dieran el dinero que gastó el de
Santoña en esa carroza de cuento de hadas, lo emplearía en comprarle una
chichonera de oro, recamada de esmeraldas y brillantes, al Alcalde que
inventó esta mojiganga de \emph{Las mil y una noches}\ldots{}
aburridas\ldots{} Me ha entusiasmado Manzanedo, me han hecho tilín los
padrinos de la Grandeza, y entre las brutalidades de los lidiadores y
las \emph{finustiquerías} de los caballeros en plaza, me quedo con las
primeras.

»Los alabarderos han estado monísimos; merecen la Gran Cruz de San
Fernando por el \emph{canguelo} que pasaron. Y si hubiera que dar un
premio a las figuras culminantes del \emph{jembrerío} de los palcos, yo
agraciaría con la \emph{Jarretiera} inglesa a la \emph{Moño Triste},
obligándola a enseñar la pierna para que el público viese imponer entre
aplausos la insignia de tan ilustre Orden. Yo hubiera organizado este
espectáculo en la Plaza Mayor, abriéndolo con un torneo y cerrándolo con
un auto de fe, para que la fiesta fuese más nacional y castiza. El
último y más lucido número habría sido quemar en elegantes hogueras al
Duque de Sexto, a Manzanedo, a los Grandes y pequeños de España, a
Cánovas, Ducazcal, Romero Robledo, Varagua, Saltillo, y el \emph{Marqués
del Bacalao}\ldots{} en efigie, por supuesto.»

Cuando ya pasábamos de las apreturas a sitio de algún desahogo, nos
encontramos con Celestina Tirado, buscando a Fructuoso y
\emph{Graziella} que se le perdieron en el tumulto de la salida. Tiempo
hacía que no nos veíamos: noté a la mujer \emph{dantesca} más vieja,
huesuda y barbuda que en los días de mi última visita al laboratorio de
la italiana. Interrogada por Casianita sobre la corrida regia, la
zurcidora de voluntades nos dijo:

«A ratos me ha parecido comitiva de boda, a ratos acompañamiento de
entierro, porque\ldots{} créanlo, yo me fijo en todo\ldots{} algunas de
las carrozas eran coches de la funeraria, pintados de colorines para dar
el pego a los bobalicones\ldots{} La Corte muy brillante; la Reina
Mercedes linda y triste\ldots{} Motivos tiene para ello\ldots{}
\emph{Graziella} y yo examinamos detenidamente el pañuelo que agitaba
para cambiar los tercios de la lidia\ldots{} ¡ay qué pena!\ldots{} Por
el movimiento que hacían en el aire las puntas del pañuelo, y por los
giros y pliegues de la tela junto a la carita de Su Majestad, vinimos a
conocer como este es día que la pobre Mercedes vivirá muy poco.

---¡Quita allá, bruja indecente!---exclamé yo indignado.---No nos vengas
con vaticinios ni sandeces.

---Por la luz del santo día, Tito; créanlo, que estos signos no fallan:
la hija de Montpensier no llegará a San Juan.»

\hypertarget{xvi}{%
\chapter{XVI}\label{xvi}}

Al abrirse las Cortes el 15 de Febrero ya pude yo decir que había
recobrado completamente la salud. Pero como me enojaba el barullo del
Congreso no asistí jamás a las sesiones. Las únicas noticias
parlamentarias que puedo daros son que, por renuncia de Posada Herrera,
fue elegido don Adelardo López de Ayala Presidente de la Cámara popular,
y que desde los primeros días arreciaron su oposición los sagastinos.
Todo ello es, históricamente considerado, flojo, anodino y sin
substancia.

Más interés tuvo la conspiración zorrillista, que desde París enviaba
sordos mugidos, llenando de zozobra los corazones monárquicos. Hablábase
mucho de los Generales Villacampa y Lagunero, y los más timoratos les
veían aparecer aquí y acullá como fantasmas sediciosos, capitaneando
soldados o paisanaje. Renegaba yo de la vana y artificiosa política de
aquellos tiempos, y cuidábame tan sólo de darme buena vida y de pasar el
tiempo plácidamente en teatros y honestas diversiones. El 30 de Marzo
fui con Casiana al estreno de la comedia de Ayala, \emph{Consuelo}, en
el Español, y ocupamos dos modestas delanteritas en el anfiteatro
principal. La sala rebosaba de selecto público, descollando en sus
palcos los Reyes, los Duques de Montpensier y un lucido acompañamiento
de magnates y fantasmones.

Casianilla y yo no apartábamos los ojos de la simpática Merceditas, que
en el teatro como en la Plaza de Toros, en los paseos y en todas partes,
se llevaba tras sí los corazones. La obra del gran Ayala gustó mucho,
sin llegar al éxito clamoroso y entusiasta de \emph{El tanto por
ciento}. Pasaje culminante de la representación fue el monólogo del
actor segundo, que dijo Vico de un modo magistral. Aclamado el insigne
poeta con aplauso ardoroso se presentó en el palco escénico, no
ciertamente cogido de la mano de los actores como es costumbre en estas
solemnidades, sino solo, enteramente solo, pues su categoría de
Presidente de las Cortes le obligaba, según se dijo, a recibir los
homenajes teatrales en un decoroso aislamiento. La eminente actriz Elisa
Mendoza Tenorio subió a las más altas cumbres del arte en la creación
del carácter de la protagonista.

Como antes indiqué, yo no perdía ripio para gozar de todo espectáculo
artístico de noble cultura. En años anteriores fui parroquiano ferviente
de la \emph{Sociedad de Conciertos}, que celebraba sus fiestas los
domingos de Cuaresma en el Teatro-Circo del Príncipe Alfonso. La
incomparable orquesta que primero dirigió Barbieri, luego Monasterio,
Mariano Vázquez y otros maestros, ha sido y es la gran educadora del
pueblo de Madrid en el clásico y supremo arte musical. Por ella han
venido a ser el más puro recreo de nuestras almas las monumentales, las
soberanas sinfonías de Beethoven y lo mejor del repertorio de Haydn,
Mozart, Mendelssohn, Weber, Handel, Schubert, y demás genios de la
gloriosa pléyade germánica. Después de educarme yo quise iniciar a
Casiana en los misterios de la santa religión de Euterpe. Durante las
primeras audiciones, la pobrecilla no lograba tomar gusto al intrincado
lenguaje de aquella teología del sonido. Pero poco a poco iba entrando,
y acabó por deleitarse con el andante de la \emph{Sinfonía Pastoral} y
el \emph{allegretto scherzando} de la \emph{Octava}.

Cuidábame yo mucho de dar al espíritu de Casianilla un matiz de cultura,
sacándola de la rusticidad y ordinariez en que se había criado. Sus
nobles sentimientos, y los estímulos de su alma querenciosa de un vago
ideal, me ayudaron en mi tarea. Firme en mi propósito, llevábala con
frecuencia al Museo del Prado, y a los tres o cuatro días de andar por
aquellas salas mi compañera se asimiló el valor estético de la pintura,
supo apreciar a los maestros, y distinguía perfectamente a Velázquez del
Tiziano y a Murillo de Rubens, dando a cada uno lo suyo.

Una mañana, cuando nos hallábamos en la Rotonda recreándonos en la
variada colección de obras capitales, que no tiene igual en el mundo,
sorprendiome la presencia de Vicentito Halconero, que con su mujer y su
suegra se deleitaba como nosotros en aquel Olimpo pictórico. En cuanto
me vio el simpático amigo vino a saludarme muy cariñoso, y me presentó a
su familia; yo, naturalmente, no les presenté a Casiana, y esta se
mantuvo cohibida y avergonzadita, fijos los ojos en el suelo, cual si
quisiera recatarse con el invisible manto de su modestia.

Insinuante y efusivo, Halconero me dijo así: «¡Caramba, Tito, cuánto me
alegro de verle! Hasta hace muy poco no supe que ha estado usted enfermo
de los ojos\ldots{} Ya me extrañaba a mí no encontrarle por ninguna
parte\ldots{} Pero lo que es ahora, ya no se me escapa usted, querido.
Tenemos que hablar. Usted es un hombre que vale mucho, y no debe estar
obscurecido, huyendo de la gente y malogrando en la inacción sus
extraordinarias dotes de talento y cultura. Eso no puede ser, no puede
ser. Es preciso que hablemos, amigo mío.»

Contestele yo, con mi habitual llaneza, que me encontraba muy bien en la
obscuridad y que me infundía temor la idea de salir de ella. Disertamos
un rato, y al llegar el momento de la despedida me dijo Vicente: «Mala
cosa es la obscuridad, y ello tiene usted ejemplo en la dolencia que
acaba de padecer. Los hombres que valen deben vivir en plena luz. De eso
hemos de tratar detenidamente. ¿Quiere usted que vaya yo a su casa, o
vendrá usted a la mía?» Le contesté que tendría mucho gusto en
visitarle, y con esto nos despedimos. Casiana y yo continuamos admirando
a Van Dick, Correggio, Velázquez, Rafael y el delicioso y minúsculo
cuadro del Mantegna \emph{Las exequias de la Virgen}.

Ocurrió esto a fines de Abril o principios de Mayo, no me acuerdo bien.
En lo restante de Mayo llevé a Casiana a la Armería Real, donde le fui
mostrando uno por uno los soberbios arneses, y dándole a conocer los
altos héroes que habíanlos llevado sobre su cuerpo en famosas batallas.
Visitamos también el Museo Naval, y allí vio Casianilla despojos
gloriosos de Trafalgar y los modelos de las antiguas y modernas naves de
guerra. En el Museo de Artillería contemplamos recuerdos agradables o
lastimeros de la vida de la Patria, y en el de Historia Natural, mi
compañera se deleitó contemplando los fósiles gigantescos y el rico
muestrario de la fauna felina, de la ornitológica y de los organismos
inferiores.

\emph{Continuando la Historia de España} os diré que la mozuela que yo
recogí del arroyo adelantaba con seguro paso en sus conocimientos.
Dominada prodigiosamente la lectura y escritura, don José y yo le
dábamos lecciones de Aritmética, de Geografía y de Historia compendiada.
Había leído ya el \emph{Quijote}, el \emph{Gil Blas} y algunos libros
modernos de poesía o amena literatura. Su instrucción era gradual, lenta
y práctica; expresaba su gozo por cada conocimiento recién adquirido
huyendo de las demostraciones pedantescas, todo ello sin olvidar los
trajines caseros que constituían su mayor deleite. Modista de sí propia,
vestía con suma sencillez, evitando las formas llamativas y de
relumbrón. Como yo, se encontraba muy bien en la obscuridad y le
infundía temor la idea de salir de ella.

A principios de Junio circularon por Madrid rumores de que la Reina
Mercedes no gozaba de buena salud. En nuestras divagaciones por la
Castellana y el Retiro, Casiana y yo la veíamos pasar en coche con su
esposo, y en efecto, notamos en su linda carita palidez, tristeza, un
indeciso mohín que a mí me pareció algo como despego de la vida. Nos
interesábamos por la joven Soberana como si fuera de nuestra familia, y
el propio sentimiento creo yo que alentaba en todo el pueblo de Madrid.
Vino Mercedes al trono de España como símbolo de paz, sin odios por su
parte, sin ningún recelo por parte de la Nación. Merecía reinar, merecía
vivir\ldots{}

Después de San Antonio, festividad del padre de la Reina, fue más denso
el rumor de la enfermedad de esta, y ya no se ocultaba lo grave del
caso. Quién decía que era una afección al pecho, quién que una fiebre
maligna; muchos recordaban que otros hijos de Montpensier habían muerto
en plena juventud, de calenturas infecciosas, contra las cuales nada
pudo la ciencia; algunos, desviando los hechos del terreno lógico al de
las conjeturas supersticiosas, afirmaban que sobre don Antonio de
Orleáns pesaba una maldición: no podía ser feliz en su vida doméstica el
que había sido en la pública desleal, ingrato y locamente ambicioso. Era
el Duque una capacidad administrativa, hombre ordenadísimo, económico,
buen esposo, buen padre, y a despecho de estas apreciables dotes nadie
le quería. En la mente popular se claveteaba con remaches duros la idea
fatalista de que los hijos inocentes han de expirar las culpas de los
padres pecadores.

El 22 de Junio aumentó tanto la gravedad de la Reina infeliz, que se
desconfiaba de salvarla. En la Mayordomía de Palacio agolpábase el
gentío aristocrático y oficial, cubriendo de firmas tal número de
pliegos que pronto se formaron montes de papel en las anchas mesas. El
pueblo soberano, que no firmaba porque no sabía o no le dejaban, hizo
pública demostración de su afecto a la Reina ocupando silencioso y
triste la Plaza de Oriente y sus avenidas. Casiana, Segis y yo
recorríamos los grupos de aquella plebe consternada y ansiosa que,
clavando sus ojos en los balcones de Palacio, firmaba según su peculiar
modo de escritura. Las impresiones que recogimos aquí y allá pueden ser
sintetizadas en esta forma: Merceditas era la cándida paloma que trajo a
España el ramo de oliva. Mientras ella calentó el nido huyeron
espantadas las víboras de la trágica escandalera dinástica en el siglo
XIX.

El día 23 llegaron de París los Duques de Montpensier, llamados por un
angustioso telegrama del Rey Alfonso. Ante la hija herida de muerte
disimularon su consternación, y a espaldas de Mercedes pidieron que
fuese llamado a consulta el célebre médico republicano Federico
Rubio\ldots{}

El 24 arreció la gravedad de la enferma con síntomas y caracteres que
inducían a la desesperación; se creyó que la Reina terminaría su vida en
el aniversario de su natalicio: el día de San Juan Bautista cumplía
Mercedes de Orleáns diez y ocho años. Contra este horrible sarcasmo del
Destino protestaron la familia de la moribunda, el mundo palatino, las
clases altas y bajas de Madrid y el pueblo entero de España, elevando al
cielo todas las formas de plegaria, desde las más solemnes a las más
humildes. Hiciéronse rogativas en innúmeros templos, catedrales,
parroquias, conventos, santuarios y ermitas; enronquecieron frailes,
monjas, capellanes y canónigos de tanto pedir a Dios la vida de la joven
Reina; y hasta las pobrecitas presas de la Cárcel de Mujeres reunieron,
cuarto a cuarto, suma bastante para mandar decir una misa rezada con el
mismo piadoso objeto.

En la noche del 24 al 25 se inició ligera remisión en la enfermedad. Las
salas próximas a la regia alcoba parecían un campamento; aquí y allá,
recostados en los lujosos divanes, daban descanso a sus fatigados huesos
Montpensier, la Princesa de Asturias, los Cardenales Moreno y Benavides,
y los palatinos de servicio. Las personas que no se movían a ninguna
hora de junto al lecho de Mercedes eran don Alfonso, la Marquesa de
Santa Cruz y la Infanta Luisa Fernanda.

El 25 renació la confianza. Federico Rubio dijo que no se debía tener
por imposible la salvación de la Reina. A propósito del doctor Rubio
referiré las voces que aquel día corrieron por Madrid. Según el rumor
público, el famoso médico se presentó en Palacio vestido de americana y
se le dijo que no podía penetrar en la Cámara Real sin ponerse levita, a
lo que don Federico respondió que él no entraba en aquella casa por su
voluntad, que le habían llamado para ver un enfermo, y que iba con el
traje que usar solía en el ejercicio de su profesión\ldots{} Después
supe por el propio Federico Rubio que todo aquello era una fábula, que
fue a Palacio como le exigían su dignidad, su educación y el respeto a
los compañeros.

Llegada la noche del 25 al 26 disipáronse las esperanzas rápidamente. No
había salvación para la Reina. Extendida la triste noticia por todo
Madrid, el público abandonó los teatros, los cafés y los círculos de
recreo. Grandes muchedumbres acudieron a Palacio, invadiendo el patio y
galerías bajas. La guardia exterior tuvo que desalojar el edificio; pero
el gentío siguió estacionado en la Plaza de la Armería y en la de
Oriente\ldots{}

Desde las primeras horas de la mañana del 26, entrañaba la situación de
Mercedes una definitiva, inevitable desesperación. Todas las personas
que rodeaban el lecho mortuorio, hijas de Reyes las más, magnates o
Príncipes de la Iglesia las otras, presenciaron enmudecidas por la
congoja el lento descender de la Reina a la región de la eterna
sombra\ldots{} Mercedes expiró a las doce y cuarto.

En pleno día, el vecindario de Madrid llenaba las calles; se oían más
las pisadas que las voces\ldots{} A punto de las tres de la tarde, el
insigne Ayala, desde su sitial de la presidencia del Congreso,
pronunciaba una corta oración fúnebre, de la cual entresaco lo que a mi
parecer expresa con más delicadeza y ternura el duelo de España en aquel
luctuoso día:

«Ya lo oís, señores Diputados: nuestra bondadosa Reina, nuestra cándida
y malograda Reina Mercedes, ya no existe. Ayer celebramos sus bodas; hoy
lloramos su muerte. Tan general es el dolor como inesperado ha sido el
infortunio; a todos alcanza; todos lo manifiestan; parece que cada uno
se encuentra desposeído de algo que ya le era propio, de algo que ya
amaba, de algo que ya aumentaba el dulce tesoro de los afectos íntimos;
y al verlo arrebatado por tan súbita muerte, todos nos sentimos como
maltratados por lo violento del despojo, por lo brusco del engaño.

»Joven, honesta, candorosa, coronada de virtudes antes que de la Real
diadema, estímulo de halagüeñas esperanzas, dulce y consoladora
aparición\ldots{} ¡quién no siente lo poco que ha durado!\ldots{} No sé,
señores Diputados, si la profunda emoción que embarga mi espíritu en
este momento me consentirá decir las pocas palabras con que pienso, con
que debo cumplir la obligación que este puesto me impone. No es porque
yo crea sentir más vivamente el funesto suceso que ninguno de los que me
escuchan; porque son tan variadas, tan acerbas las circunstancias que
contribuyen a hacer por todo extremo lamentable la desgracia presente,
que no hay alma tan empedernida que le cierre sus puertas. Pero concurre
una tristísima circunstancia, que nunca olvidaré, a que yo la sienta con
más intensidad en este momento.

»Testigo presencial de los últimos instantes de nuestra Reina sin
ventura, aún tengo delante de mis ojos el lúgubre cuadro de su agonía;
aún está fresca en mi mente la imagen de la pena, de la horrible y
silenciosa pena que, con varios semblantes y diversas formas, rodeaba el
lecho mortuorio: he visto el dolor en todas sus esferas. Allí, nuestro
amado Rey, hoy más digno de ser amado que nunca, apelaba a sus deberes,
a sus obligaciones de Príncipe, a todo el valor de su magnánimo pecho,
para permanecer al lado de la que fue la elegida de su corazón, y para
reprimir, aunque a duras penas, el alma conturbada y viuda que pugnaba
por salir a sus ojos. Allí, los aterrados padres de la ilustre
moribunda, viva estatua del dolor, inclinaban su frente ante el Eterno,
que a tan dura prueba les sometía, y con cristiana resignación le
ofrecían en holocausto la más honda amargura que puede experimentarse en
la vida. Incansables en su amor, la Princesa de Asturias y sus tiernas
hermanas seguían con atónita mirada todos los movimientos de la doliente
Reina, como ansiosas de acompañarla en la última partida. Allí, la
presencia del Gobierno de Su Majestad representaba el duelo del Estado;
los Presidentes de los Cuerpos Colegisladores el luto del país\ldots»

A estas expresiones elevadas, patéticas, que revelaban al orador
elocuente y al poeta eximio, añadió Ayala otras que podríamos llamar de
literatura oficial, proponiendo que enmudeciera la tribuna parlamentaria
hasta que el cuerpo de la infortunada Reina recibiese cristiana
sepultura.

El suceso del día siguiente fue la exposición pública del cadáver de
Mercedes en el Salón de Columnas. No exagero al decir que medio Madrid
desfiló por la capilla ardiente. Las apreturas fueron horribles; se
entraba por la Plaza de la Armería y se salía por la puerta del
Príncipe. El sentimiento, derivando a la curiosidad, convertíase en
fuerza irresistible que todo lo arrollaba: hubo desmayos, síntomas de
asfixia, magulladuras y estrujones tan violentos que muchas personas
hubieron de ser auxiliadas en la Casa de Socorro o en las farmacias
próximas.

Casiana y yo llegamos a la Plaza de Oriente, y viendo el tumulto no nos
atrevimos a meternos en tan terribles angosturas. Minutos después nos
encontramos a Celestina Tirado que salía de Palacio, desgreñada,
sudorosa, jadeante. Antes que yo le hablara, llegose a nosotros con esta
retahíla:

«La he visto, la he visto. ¡Qué dolor de niña! Está ya medio
descompuesta, vestidita con el hábito de la Merced, en una caja de tisú
de oro. Por cierto, Tito salado, que cuando en la Plaza de Toros solté
la profecía, sacada de los signos y \emph{céteras} que nunca fallan, me
equivoqué en el santo, nada más que en el santo\ldots{} Quise decir San
Pedro y dije San Juan\ldots{} Desde que ando en este oficio se me
trabucan los santirulicos.»

\hypertarget{xvii}{%
\chapter{XVII}\label{xvii}}

Una tarde de Julio, paseando por el Prado, oímos estas coplas, cantadas
por las tiernas niñas que jugaban al corro: \emph{¿Dónde vas, Alfonso
XII? ---¿Dónde vas, triste de ti?---Voy en busca de Mercedes,---que ayer
tarde no la vi.---Si Mercedes ya se ha muerto;---muerta está, que yo la
vi:---cuatro Duques la llevaban---por las calles de Madrid}. La
simplicidad candorosa de estos versos, en boca de inocentes criaturas,
se me metía en el corazón avivando la doliente memoria de la Reina sin
ventura, muerta en la flor de la edad.

Otro día, en Recoletos, oí las mismas coplas, continuadas de este modo:
\emph{Su carita era de Virgen,---sus manitas de marfil,---y el velo que
la cubría---era un rico carmesí.---Los zapatos que llevaba---eran de
rico charol,---regalados por Alfonso---el día que se casó}. Recreándonos
con tan ingenua cantata dimos la vuelta al corro, y pudimos enriquecer
el poema infantil con esta otra cuarteta: \emph{El manto que la
cubría---era rico terciopelo,---y en letras de oro decía:---Ha muerto
cara de cielo}.

«Fíjate---dije a Casiana,---y convendrás conmigo en que esos lindos
cantares contienen más inspiración y mayor encanto que las odas
hinchadas y las elegías lacrimosas con que los poetas de oficio
lamentaron el prematuro fin de Merceditas, apedreándonos con ripios
duros y aburriéndonos con el desfile monótono de imágenes sobadas y
terminachos rimbombantes.»

Opinó como yo Casianilla y me dejó estupefacto al preguntarme: «Dime,
Tito: ¿tú conoces a los poetas que hacen esos cantares? ¿Quiénes son,
dónde están?

---No lo sé, hija mía---contesté.---Sólo te digo que el pueblo hace las
guerras y la paz, la política y la Historia, y también hace la poesía.»

Si no referí antes mi primera visita a Vicentito Halconero, fue porque
en ella nada hubo digno de mención. Redújose a cortesías de ritual y a
remembranzas de sucesos que se desvanecieron en el tiempo. Las
posteriores entrevistas tuvieron más interés. Vivía mi amigo en la calle
de San Quintín, Plaza de Oriente, y cuando le visitaba por la tarde,
como a esas horas salía yo siempre con Casiana, quedábase mi compañera
sentadita en un banco de los jardinillos entrando yo solo en la casa.

Requería Vicente mi persona un día y otro para convencerme de la
necesidad de que yo me lanzase de lleno a la política activa,
afiliándome con él al partido de Sagasta. Apuró Halconero sus razones
sin persuadirme, y entre otras cosas me dijo que el propio don Práxedes
le manifestó deseos de tenerme a su lado, porque ansiaba fortalecer el
Partido Constitucional con gente moza, atraer a todos los jóvenes de
mentalidad a la moderna, aunque hubiesen sido revolucionarios y
alborotadores en días no lejanos. El relleno de sus adeptos, consistente
en progresistas acartonados, necesitaba renovación.

Después de hablar por boca de Sagasta, habló Vicente por la suya
diciéndome que si me determinaba podía contar desde luego con un
distrito seguro para salir diputado, bien cediéndome el suyo, La
Guardia, bien Villarcayo, el de su suegro, pues este ansiaba retirarse
de la vida pública. «Como ve usted---añadió,---tengo dos distritos.
Escoja el que quiera.»

Contestele que yo agradecía mucho su generoso interés, pero que me
repugnaba el \emph{cunerismo} y nunca pasó por mi mente pertenecer a
esos rebaños parlamentarios que forma el Ministro de la Gobernación como
Dios hizo el mundo, de la nada. Sostuve que en España no existe la
representación nacional, y que los diputados no expresan más opinión que
la de unos cuantos señores; que en las Cortes no reside ninguna parte de
la soberanía, y que la ley fundamental del Estado no es más que una
edición bonita y esmerada de las coplas de Calaínos. Todos los poderes
residen en el Rey y en las camarillas, a las que están subordinados los
jefes de las ganaderías políticas.

De estas afirmaciones surgió una discusión entre cómica y seria, y
Halconero acabó por arrancarme la promesa de que iría yo con él a ver a
Sagasta. Al salir de casa de mi amigo y entrar en los jardinillos para
reunirme con Casiana, vi un ruedo infantil que cantaba con dulces
vocecitas las coplas que en otra página he transcrito, y estas que ahora
copio: \emph{Los faroles de Palacio---ya no quieren alumbrar,---porque
Mercedes se ha muerto---y luto quieren guardar.---Junto a las gradas del
trono---una sombra negra vi,---cuanto más me retiraba---más se
aproximaba a mí.---No te retires, Alfonso;---no te retires de mí, ---que
soy tu esposa querida---y no me aparto de ti}.

Cumplí a Vicente Halconero mi promesa de visitar a Sagasta, y una mañana
fui con él a casa del jefe de los Constitucionales, Alcalá, 52. Había yo
tratado superficialmente a don Práxedes en años anteriores. Antes que
Vicentito me presentase, Sagasta me reconoció, saludándome como si
nuestro trato hubiera sido frecuente y nunca interrumpido. Ya sabéis que
la característica de aquel hombre realmente extraordinario era el don de
simpatía, el don de gentes, la flexibilidad del ingenio y de la palabra,
sin que por ello dejase traslucir su pensamiento en la conversación.
Entendía yo que en su afable sonrisa no debíamos ver un accidente, sino
un estado constitutivo de la personalidad, y además la máscara
impenetrable de su genial astucia.

Don Práxedes rompió la conversación sacando a relucir diabluras y
extravagancias de mi temprana juventud, y no fue poco mi asombro al ver
que tales simplezas conocía y recordaba. Pronto comprendí que trataba de
ganar mi voluntad y atraerme a su esfera por la afinidad de los
caracteres y la semejanza de nuestros respectivos modos de expresión. De
frase en frase nos metimos en la política, y Sagasta hizo el panegírico
de la Monarquía constitucional, prometiendo a España días muy felices.
La buena crianza obligome a una delicada conformidad con las opiniones
del riojano, y al observar yo que recogía la sonrisa en su larga boca
para departir con grave estilo, pensaba que seguía riéndose por dentro.

Una observación del amigo Halconero llevó a don Práxedes a tocar el tema
de mi incorporación a su partido. Yo me excusé declarándome inepto para
la vida pública, tal como aquí se practicaba entonces; y él, entre
severo y festivo, me habló de este modo:

«Ya sé, ya sé que a usted las cavilaciones le han hecho algo metafísico
y que los desengaños han matado su optimismo. Déjese de tonterías,
amigo, que por ese camino no se va a ninguna parte. Usted sostiene que
vivimos en un mundo de ficciones; que la representación nacional, base
del régimen, será una farsa mientras hagamos los diputados por un
sistema de moldes y cubiletes. Algo hay de verdad en todo lo que usted
dice, lo reconozco; pero también afirmo que semejantes males sólo puede
remediarlos el Partido Constitucional, maridaje perfecto entre el poder
real y la soberanía del pueblo\ldots{} No lo dude usted, amigo Liviano,
pues mi partido, en la oposición, está haciendo ya una gran obra
política. El porvenir es nuestro. Si usted no lo reconoce todavía, lo
reconocerá bien pronto. Yo he de intentar la regenaración de este país.
¿Fracasaré? Allá veremos. Lo que aseguro es que si mis esfuerzos
resultan fallidos y sucumbo en la demanda, caeré siempre del lado de la
libertad.»

Con esto y poco más, terminó mi primera visita a don Práxedes. El rápido
avance del verano interrumpió mis relaciones con Halconero porque este
se fue a La Guardia, Vitoria y San Sebastián\ldots{} Casiana y yo, no
queriendo infringir la moda de la emigración estival, partimos para
nuestras posesiones de La Sagra, radicantes en el término de un
desconocido pueblo llamado Borox. Reducíase el patrimonio mísero de los
Conejos a unas tierrucas de pan llevar y a una casucha propiedad de la
tía Simona. Encantadas entraron Simonica y Casiana en su pueblo natal;
pero a mí me pareció muy desagradable. En Borox no se conocía el árbol;
había una sola fuente, y el agua de esta no servía para cocer los
garbanzos: utilizábase en tales usos la que brotaba de un manantial
distante cinco kilómetros del pueblo, y era transportada por
arrieros-aguadores que surtían a todo Borox y sus aledaños.

Aunque la pobreza y sequedad de aquel suelo eran lo más apropiado a
nuestra ingénita cursilería, yo no me conformé con tan ruin
\emph{villeggiatura}, y nos fuimos a Esquivias, lugar próximo donde
Simona tenía parentela. Por mediación de esta alquilamos una hermosa
casa, con huerta, rodeada de viñedos y frutales. Ya sabéis que Esquivias
es la patria de doña Catalina de Salazar, esposa de Cervantes, y que
allí vivió algún tiempo el Príncipe de nuestros ingenios. Gozábamos el
alto honor de veranear en una villa famosa en los anales de las Letras
patrias. El pueblo era cómodo y alegre, y en su vecindario encontramos
muchas personas de buena crianza, y algún señorío. Había no pocos
veraneantes de Madrid, gente de medio pelo, pero campechana y cortés.

Tan bien nos fue en Esquivias que nos quedamos hasta la vendimia, muy
entretenidos y gozosos. En aquella temporada placentera no teníamos más
relación con el resto del mundo que las cartas que de vez en cuando nos
escribía nuestro amigo Segis, desde San Sebastián primero, después desde
Zaragoza y Barcelona. Al llegar a Madrid me enteré de acaecimientos que
surgían y pasaban sin dejar tras sí más que el comentario fugaz de las
lenguas ociosas: que Martos, después de entenderse con Ruiz Zorrilla,
logró catequizar al Duque de la Torre y llevarlo a las trincheras
revolucionarias; que los tres celebraron una conferencia en Biarritz, de
la cual, según los \emph{ojalateros} de Madrid, resultaría muy pronto el
triunfo de la República. Estas ilusiones y otras de rosados matices se
desvanecieron en la normalidad perezosa de la vida política en aquellos
tiempos de glacial positivismo.

La intentona revolucionaria de Navalmoral de la Mata fue otro caso de la
vacuidad histórica que caracterizó aquellas décadas. El 25 de Octubre
regresó el Rey Alfonso de un viaje que hizo a las provincias del Centro,
y al pasar en coche por la calle Mayor, cerca ya de los Consejos, un
jovenzuelo disparó contra él dos pistoletazos, sin causarle daño alguno.
El agresor, detenido al instante, se llamaba Juan Oliva Moncasi, era
natural de Cabra (Zaragoza), y según dijo, estaba afiliado a la
Internacional. La emoción de este suceso no duró mucho. El tal Oliva era
indudablemente un fanático; pero con menos visos de locura que de
tontería. Según mi leal entender, en aquella época de una insipidez mal
azucarada, hasta el regicidio era tonto, desaborido y sin picante. Del
desdichado Oliva se habló un poco en aquellos días, y otro poco cuando
le dieron garrote en Enero del año próximo.

El mundo marchaba, dejando atrás a personalidades ilustres que habían
cumplido ya su misión en la vida. En Agosto del 78 falleció la que fue
Reina Gobernadora, doña María Cristina; en Diciembre perdió la
democracia al famoso tribuno don Nicolás María Rivero; y a principios
del año siguiente, 1879, acabó sus días Espartero, Duque de la Victoria
y Príncipe de Vergara, que durante un cuarto de siglo llenó con su
nombre la Historia de España.

Mientras llega ocasión de traer a estas páginas las cosas de Cuba, os
diré que la llamada paz del Zanjón (más bien tregua o convenio, al
estilo del de Vergara) pactada entre Martínez Campos y los jefes de la
insurrección, no era del gusto del Partido Peninsular Español de la Gran
Antilla. Sonaron con mayor estridencia que antes las declamaciones
patrióticas; Martínez Campos, viendo que el Gobierno de Madrid se
mostraba esquivo para realizar lo pactado con los insurrectos, se
\emph{atufó}, dio de lado al Capitán General Jovellar y a los
\emph{españoles incondicionales}, y se vino a Madrid decidido a plantear
la grave cuestión ante el Rey, el Gobierno y las Cortes.

Cánovas del Castillo, estimando con razón o sin ella que el horno
político de España no estaba para bollos autonómicos ni otras zarandajas
ofrecidas a los cubanos, mostró su repugnancia a convertir en leyes las
estipulaciones del convenio del Zanjón, y para salir de aquel convenio
puso en práctica la consabida artimaña del medio mutis, que había
empleado con éxito en los comienzos de la Restauración.

El 27 de Febrero planteó don Antonio la crisis total, aconsejando al Rey
que encargase de formar Gobierno a Martínez Campos. ¡Lástima grande que
un hombre como Cánovas desestimara el alto ideal que Martínez Campos
defendía; error funesto que don Antonio, por falta de valor para
imponerse a los patrioteros, entregase el Poder a un hombre que si en lo
militar era eminente, en lo político carecía de trastienda y travesura
para luchar con las pasiones humanas! ¡Fatalidad inexorable! Cánovas, no
atreviéndose a resolver el gran problema antillano, cedía los trastos de
gobernar a quien, sobrado de valor para todo, no podía consumar la magna
empresa por falta de aptitudes políticas. De este modo, entre un sabio
que no quiere y un valiente que no puede, decretaron para un tiempo no
lejano la pérdida de las Antillas.

Llevó Martínez Campos al Ministerio de la Gobernación a \emph{Paco}
Silvela, el más joven de los tres hermanos de este ilustre apellido,
todos muy notables en la jurisprudencia, la literatura y la política.
Fuera de disolver las Cortes y convocar otras nuevas, el Gabinete
Campos-Silvela poco o nada hizo, a no ser que se tenga en cuenta su obra
negativa. Las reformas políticas de Cuba, que se había comprometido a
realizar don Arsenio, pasaron suavemente al panteón del olvido, y ni aun
se trató de sacar adelante el proyecto de ley de abolición de la
esclavitud que parecía lo de más urgencia.

En cambio, los Ministros pusieron toda su atención en el proyecto que
daba por quebrada a la Compañía constructora de las líneas férreas del
Noroeste, facultando al Gobierno para otorgar por concurso lo que
restaba por construir. De ello resultó que adjudicaron el bonito negocio
a un afortunado francés llamado \emph{Monsieur} Donon, a quien, según se
dijo, protegían altísimas personalidades.

Pasando de lo colectivo a lo personal, os contaré que Halconerito
insistió en sus deseos de sacarme diputado, aprovechando aquellas
elecciones. Yo me negué en absoluto, y nunca me pesó este apego a la
dorada obscuridad: así lo digo, porque en mi salvaje independencia llevo
dentro una luz espiritual que me hace amable y placentera la vida.

A los que se hayan sorprendido de no ver en mi compañía hace algún
tiempo la figura de García Fajardo, les diré que poco después de irme yo
al veraneo de Esquivias mi grande amigo se reconcilió con su madre,
Segismunda Rodríguez, señora de circunstancias, dotada de no comunes
talentos para traer dineritos de los bolsillos ajenos al suyo propio, y
para decorar su vanidad con fáciles blasones. De esta dama os hablé hace
algún tiempo, y aquellas referencias las completo ahora diciendo que
doña Segismunda había realizado su dorado sueño de poseer un título
nobiliario, aunque fuera pontificio: desde el verano anterior titulábase
Condesa de Casa Pampliega.

Satisfecho este anhelo, y viéndose ya en la madurez de la vida, sin más
afecto que el de su hijo, requirió la compañía de Segis con el ansia de
completar su corrección teniéndole siempre consigo. Sacó al rebelde del
poder de doña Leche, y firme en la idea de apartarle de las malas
compañías de Madrid, emprendió con él largos viajes que fueron a un
tiempo de recreo y de vanidad. Pasaron sus temporaditas en los
balnearios y playas del Norte, visitaron después Barcelona, Zaragoza y
otras capitales, y llegado el invierno se fueron a Andalucía, terminando
su agradable excursión con la temporada de Semana Santa y Ferias en
Sevilla.

En cuanto supe el regreso de Segismundo a Madrid me fui a verle a su
casa, y lo encontré más reformado de indumento que de lenguaje. La madre
de García Fajardo, en el descenso de la vida, conservaba la siniestra
hermosura de su rostro ceñudo y desapacible. En otro tiempo compararon
su cabeza con la de Medusa, y aún podía sostenerse la comparación; sólo
que su cabellera de serpientes había blanqueado. Al visitar por primera
vez a mi amigo hablamos de sus recientes viajes, y la señora Condesa de
Casa Pampliega se despachó a su gusto, contando con prolijidad enfadosa
las preciosidades que había visto en el Norte y Sur de las Españas.

A la tarde siguiente volví a casa de Segismundo, y puedo aseguraros que
esta segunda visita fue memorable, digna ciertamente de ser marcada con
piedra blanca en mis historias. Al entrar yo se despedía una dama
elegantísima, guapetona, de grandes ojos negros fulgurantes, carnosa,
espléndida en hechuras, bien plantada\ldots{} Quedé absorto ante tan
seductora belleza, y dije para mí: «Sin saber quién es esta mujer, sé
que la he visto en alguna parte. ¿Dónde, Señor, dónde?\ldots{} No me
acuerdo.»

Cuando Segis volvió de despedir a la linda señora, notando mi asombro y
perplejidad, me dijo: «¿Pero no la conoces? Parece que estás tonto. Es
Elena Sanz.

\hypertarget{xviii}{%
\chapter{XVIII}\label{xviii}}

---¿Elena Sanz?\ldots{} ¡Ah!\ldots{} sí\ldots{} sí---exclamé yo
golpeándome la frente,---la hermosa cantante española\ldots{} Nunca la
vi fuera de la escena; por eso la desconocía.

---En el teatro, querido Tito---me dijo Segis,---su belleza entra en el
orden de lo monumental, y al pasar del escenario a la vida es un
conjunto de gracias y seducciones que quitan el sentido. Recordarás que
la aplaudimos en el Real por primera vez, interpretando el carácter de
Leonor de Guzmán, favorita del Rey don Alfonso XI y madre del bastardo
Trastamara y de sus hermanos, que tanta guerra dieron en estos Reinos.

---Ya, ya me acuerdo---contesté.---Luego la vimos en la \emph{Azucena}
de \emph{El Trovador}, tipo musical a que da extraordinario relieve su
potente voz de contralto.»

Queriendo mostrar sus conocimientos en el arte del \emph{bel canto}
aplicado a la ópera, doña Segismunda intervino en la conversación con
estas sensatas razones: «Entiendo yo que eso de \emph{contralto} es lo
mismo que \emph{barítona}, o como quien dice, el barítono de las
mujeres. Recuerden lo bien que estaba Elenita, vestida de muchacho, en
esa ópera tan preciosa\ldots{} no me acuerdo\ldots{} ¿Cómo se llama?

---\emph{Lucrecia Borgia}---contestó Segis.---El papel de \emph{Maffeo
Orsini} le va que ni pintado. ¡Qué elegante mozo, qué frescura, qué
gracia!\ldots{} Como dice \emph{Asmodeo} en sus formas críticas, Elena
Sanz \emph{rayó a gran altura} en el \emph{racconto} del primer acto y
en el brindis del tercero.

---Pero donde está incomparable, ideal, es en \emph{Aida}---afirmé
yo.---¡Qué \emph{Amneris}! Diríase que es la auténtica hija del Rey de
Egipto\ldots{} Cuando entra en escena parece que viene de dar un paseíto
por el Nilo y de echar un vistazo a las Pirámides.

---Todas esas óperas y otras le hemos oído en Sevilla---me dijo
Segismundo.---Cada vez está mejor.

---Además---añadió la Condesa de Casa Pampliega,---como vivíamos en el
Hotel de París, donde ella moraba, nos hicimos muy amigas. Elenita es
una mujer simpatiquísima, buena como el pan, toda pasión, generosidad,
ternura.»

Hijo y madre siguieron bosquejando con cariñosa benevolencia el retrato
de la \emph{diva} guapetona y adorable, y yo me retiré porque tenía que
hacer en mi casa. Al bajar la escalera pareciome sentir leves pasos al
compás de los míos; volví el rostro y nada vi. Cuando llegué a la calle,
además de los pasos oí una voz tenue que deslizó en mi oreja estas
dulces palabras: «Soy la \emph{Efémera} a quien nuestra Madre augusta
confía las comunicaciones de índole más delicada. ¿No me ves?

---Vagamente, como un espectro engendrado por la luz solar, veo tu
perfil de mármol y tu ropaje azul.

---No es azul; es verde con grecas de plata, fíjate bien. Y la región
espiritual que cruzamos con fugaz vuelo mis hermanas y yo es aquella
inmensa esfera encendida por el fuego de amor, que crea o destruye las
familias humanas\ldots{} Cuando hablabas con tu amigo y su madre estaba
yo presente, pero no pudiste verme. Cuando salías te seguí para
comunicarte el pensamiento de la divina \emph{Clío}: ella movió la
voluntad de tus amigos a fin de que te dieran a conocer a la gentil
artista que, con su gallarda persona y sus acendrados sentimientos ha de
ocupar grande espacio de la Historia\ldots{} pero entiéndase bien, en
los anales \emph{del ser interno} de la Nación. Demasiado sabes tú que
la vida externa y superficial no merece ser perpetuada en letras de
molde. Lo que aquí llaman política es corteza deleznable que se llevan
los aires. Desea \emph{Mariclío} que te apliques a la Historia interna,
arte y ciencia de la vida, norma y dechado de las pasiones humanas.
Estas son la matriz de que se derivan las menudas acciones de eso que
llaman \emph{cosa pública}, y que debería llamarse \emph{superficie de
las cosas}.»

Aplicando toda mi atención a las palabras de aquella fémina incorpórea,
pude hacerme cargo de las excelsas órdenes que me transmitía. «Bueno
---le dije.---Ya sé que la hermosa \emph{diva} de los ojos de fuego
trae, además de sus papeles de teatro, otro muy importante en la
Historia. Dispuesto estoy a escribir lo que, tocante a esa señora, sea
digno de pasar a la posteridad; pero ¿de dónde voy a sacar los
pormenores y noticias de una vida que desconozco? ¿Ha de relatarme ella
misma su propia biografía? Los amigos suyos que también lo sean míos,
¿podrán contarme el pasado de esa mujer seductora, algo de su presente,
y revelarme los pensamientos y propósitos con que intenta elaborar su
porvenir?»

Ibamos por la Plaza de Santa Ana, y al atravesar el jardincillo donde
años después se colocó la estatua de Calderón, la infantil y grácil
\emph{Efémera} brincaba, separándose por momentos de mí para pisotear el
césped y volver luego a mi lado con paso de cabritilla juguetona. De
pronto me cogió de la mano, y como yo le manifestase de nuevo mi
perplejidad ante la falta de datos para escribir la \emph{Vida y Hechos}
de la bella cantatriz, obligome a sentarme en un banco y me dijo: «No te
apures, Titín, que aquí tengo yo, y voy a dártelo, el remedio de tu
ignorancia.»

Acto seguido sacó del seno un cartuchito de papel, y de este una pluma
que me entregó, acompañando la acción con las siguientes diabólicas
palabras: «Tu Madre te envía la péñola que ella usó algunas veces para
apuntar los nombres de los Reyes enamorados que por sus liviandades
perdieron el trono, y los de otros que por las mismas o parecidas
flaquezas lo ganaron. Todo lo que con ella se escribe es verdad, aunque
otra cosa quiera el que la coge en su mano para llenar de letras un
blanco papel. ¿Te vas enterando? Si te propusieras escribir con esta
pluma una mentira, ella no te obedecería y pondría la verdad.»

Pronunciando la última palabra, introdujo la pluma en el bolsillo
interior de mi levita y desapareció de mi vista\ldots{} Apenas percibí
un rumor, un viso verde rasgando el aire.

Sin detenerme a reconocer la dirección que por el alto espacio seguía la
mensajerita de mi Madre, emprendí presuroso el camino de mi casa,
espoleado por la inquietud y confusión que la presencia de la linda
\emph{Efémera} me causara, y con la esperanza de que cesarían mis dudas
en cuanto pudiese probar la maravillosa virtud de la pluma que a
despecho del escritor escribía siempre la verdad. Pocos minutos me
bastaron para llegar a mi vivienda, y segundos tan sólo tardé en
sentarme junto a mi mesa, requiriendo con ágil mano tintero y papel.

Púseme inmediatamente al trabajo, entregándome al arbitrio de la mágica
péñola, la cual empezó a traducir mi pensamiento, o más bien a sugerirme
el suyo en esta forma: «Elena Sanz nació en Castellón de la Plana por
los años de 1852 ó 53, y no doy más referencias de su progenie, ni
puntualizo la fecha de su nacimiento, porque ello ni quita ni pone un
ardite en el valor documental de esta verídica historia. Os diré tan
sólo que a mediados del 63 ingresó con su hermanita Dolores en el
Colegio de las Niñas de Leganés, sito, como saben hasta los más
indoctos, en la calle de la Reina, a mano derecha bajando de la calle
del Clavel a la de San Jorge.

»Acreditados autores dan a entender que la gentil Elenita y su hermana
entraron a recibir educación en aquel benéfico instituto por los
auspicios o voluntad expresa del representante del Patronato señor
Marqués de Leganés, más conocido por los ilustres títulos de Duque de
Sexto y Marqués de Alcañices. Cuestión es esa que dejo al libre criterio
de los lectores, limitándome a consignar que la nueva colegiala se
distinguió por su belleza, por su aplicación al estudio, y singularmente
por su magnífica voz y extraordinarias aptitudes para la música y el
canto. El maestro don Baltasar Sardoni, profesor del Colegio en las
clases de solfa, vaticinó a Elenita un porvenir brillante y provechoso
si consagraba su florida juventud y su admirable órgano vocal a la ópera
italiana.

»Todo Madrid sabe que en algunas tardes y noches de Semana Santa, acude
gran gentío al Colegio de Niñas de Leganés para oír cantar a las
educandas motetes, misereres, y otras piezas religiosas propias de tales
solemnidades. A fuer de historiador de indiscutible veracidad, aseguro
que la voz angélica de Elena Sanz, sobreponiéndose a la de sus
compañeras, subyugó al público, y que este llevó de la iglesia a la
calle y de la calle a diferentes Círculos y salones el nombre de la
precoz \emph{niña de Leganés}, que anunciaba la extraordinaria mujer de
teatro en un porvenir próximo. También sostengo, sin temor de ser
desmentido, que el año 66, cuando salió Elena del Colegio, era una moza
espléndida, admirablemente dotada por la Naturaleza en todo lo que atañe
al recreo de los ojos, completando así lo que Dios le había dado para
goce y encanto de los oídos. Muchas familias aristocráticas se la
disputaban para gozar de su canto en reuniones y tertulias. Por fin, en
alas de su incipiente nombradía, fue llamada a Palacio por la Reina
Isabel, que la oyó, la celebró, ofreciéndole su protección
gallardamente, como siempre lo hizo, para que pudiera llegar pronto a
las cumbres más excelsas del arte.

»Por conveniencia o por capricho, averígüelo Vargas, el historiador os
anuncia que para seguir su relato dará un formidable salto en el tiempo,
omitiendo no pocos episodios de la vida de Elena Sanz, que si para ella
entrañan indudable importancia, no han de traer ningún hilo nuevo al
sutil tejido de la historia presente. No tengo por qué decir, ni ello
hace al caso, cómo fue Elena Sanz a Italia para perfeccionarse en el
arte del canto; cómo se dio a conocer en los teatros de aquellos Reinos,
obteniendo ruidosos éxitos por su belleza y su arte; cómo recorrió
triunfalmente varias capitales de Europa y América; y cómo, en fin,
volvió a París el año 73, en la plenitud de su hermosura y de su talento
musical. En uso del sagrado derecho de preterición me callo lo que
importa poco a mis fines, y me apresuro a consignar que uno de los
primeros cuidados de Elenita en la capital de Francia fue visitar a su
protectora y amiga la Reina Isabel en el Palacio Basileusky\ldots»

Cuando a este punto llegaba, acercóseme Casianilla muy quedito, y
mirando por encima de mis hombros lo que yo escribía, me dijo: «Pero
¿qué haces, Titín? No has levantado mano del papel desde que entraste en
casa. Eso que escribes, ¿es Historia o qué demonios es?

---Novela, chiquilla, novela---repliqué un tanto confuso.---Ahora me da
por ahí. Pero esta invención supera en verdad a la misma Historia.

---¡Bonita cosa será!---exclamó Casiana pasando sus ojos por las
cuartillas.---Ya veo que sacas una heroína y que esta se llama Elena.

---Nombre supuesto, convencional. Mi heroína es \emph{Doña Leonor de
Guzmán}, señora muy bella y frescachona, que cantaba como los ángeles y
que tuvo amores con el Rey don Alfonso.

---¿Con este Rey de ahora, con el viudo de Mercedes?

---No, mujer, no digas desatinos. Fue con otro Rey, a quien llamaban
Alfonso onceno allá en los tiempos de Maricastaña, siglo XIV.

---¿Y esa \emph{Doña Leonor} era cantante?\ldots{} ¿De malagueñas, de
jotas, o de\ldots?

---De ópera, hija mía. Uno de sus mayores triunfos era \emph{La
Favorita}. ¡Qué arias se cantaba ella sola, qué dúos con el Rey!

---Explícame, explícame eso. ¿Dices que el Alfonso cantaba también?

---No, Casiana, no es eso. Déjame ahora. Temo que se me vaya el santo al
cielo si me entretengo en hablar contigo\ldots{} Vete a tus
quehaceres\ldots{} Esta noche te contaré todo el argumento.»

Seguí mi trabajo con febril actividad, y la mágica pluma, que ya iba
concordando sus verdades con la inspiración mía, trazó estas
interesantes cláusulas: «Que doña Isabel II recibió a su amiga Elenita
con la efusión más cariñosa, no hay para qué decirlo. La convidó a
comer; llevola en su coche a los paseos por el \emph{Bois}; y para que
la oyeran cantar invitó en repetidas \emph{soirées} a sus amigas, entre
las cuales estaba la famosa soprano Ana de Lagrange, tan querida del
público de Madrid. Aplaudida y celebrada pomposamente fue la Sanz en
aquella linajuda sociedad. Todo esto es corriente y vulgarísimo. Lo que
sigue, no sólo es interesante, sino que pertenece al orden de las cosas
de indudable trascendencia en la vida de los pueblos\ldots{} No reírse,
caballeros\ldots{}

»Ello fue que al ir Elenita a despedirse de Su Majestad, pues tenía que
partir para Viena, donde se había contratado por no sé qué número de
funciones, Isabel II, con aquella bondad efusiva y un tanto candorosa
que fue siempre faceta principal de su carácter, le dijo: «¡Ay, hija,
qué gusto me das! ¿Conque vas a Viena? Cuánto me alegro. Pues mira, has
de hacer una visita a mi hijo Alfonso, que está, como sabes, en el
Colegio Teresiano. ¿Lo harás, hija mía?» La contestación de la gentil
artista fácilmente se comprende: con mil amores visitaría a Su Alteza;
no, no, a Su Majestad, que desde la abdicación de doña Isabel se
tributaban al joven Alfonso honores de Rey.

»Como testigo de la pintoresca escena, aseguró que la presencia de Elena
Sanz en el Colegio Teresiano fue para ella un éxito infinitamente
superior a cuantos había logrado en el teatro. Salió la \emph{diva} de
la sala de visitas para retirarse en el momento en que los escolares se
solazaban en el patio, por ser la hora de recreo. Vestida con suprema
elegancia, la belleza meridional de la insigne española produjo en la
turbamulta de muchachos una impresión de estupor: quedáronse algunos
admirándola en actitud de éxtasis; otros prorrumpieron en exclamaciones
de asombro, de entusiasmo. La etiqueta no podía contenerles. ¿Qué mujer
era aquella? ¿De dónde había salido tal divinidad? ¡Qué ojos de fuego,
qué boca rebosante de gracias, qué tez, qué cuerpo, qué lozanas curvas,
qué ademán señoril, qué voz melodiosa!\ldots{}

»En tanto, el joven Alfonso, pálido y confuso, no podía ocultar la
profunda emoción que sentía frente a su hechicera compatriota\ldots{}
Partió \emph{la diva}\ldots{} Las bromas picantes y las felicitaciones
ardorosas de \emph{los Teresianos} a su regio compañero quedaron en la
mente del hijo de Isabel II como sensación dulcísima que jamás había de
borrarse\ldots{} Una de las primeras óperas que Elenita cantó en Viena
fue \emph{La Favorita}.»

Escrito lo que antecede, suelto la mágica pluma y me permito obsequiar a
los conspicuos lectores con este monólogo de mi propia cosecha:

«¡Bien haya, oh tierna Isabel, Majestad bondadosa y desdichada, aquel
filósofo-político que añadió a tu nombre el lastimero mote de \emph{La
de los tristes destinos}!\ldots{} Digo esto porque en tu larga vida de
Soberana pusiste siempre tu corazón blando sobre tu inteligencia, y
abusaste irreflexivamente del poder afectivo y lo extendiste fuera de tu
órbita personal, llevándolo a trastornar y corromper la vida del
Régimen\ldots{} ¿Quién te inspiró la diabólica idea de enviar a la linda
histrionisa al Colegio Teresiano, donde tu hijo educábase para Rey
constitucional, grave, reflexivo, guardador de las leyes, primer
ciudadano de un país ávido de acomodar su vida a la virtud y a las
buenas costumbres? ¿No pensaste que Alfonso se hallaba en la edad
crítica de la formación del carácter, expuesto a llevar a la existencia
del hombre los arrebatos de la edad juvenil? Sin darte cuenta de ello,
¡oh Reina!, movida de tu ardorosa ternura, cumpliste tu sino, en el cual
hemos de ver siempre una modalidad incendiaria. Con la tea del buen
querer pegaste fuego al templo del Estado.»

Esto pensé, y por lo que valiera aquí lo digo, entre dos parrafadas de
la divina péñola forjada por los geniecillos que a su servicio tiene la
Verdad.

\hypertarget{xix}{%
\chapter{XIX}\label{xix}}

Puestos los puntos de la pluma sobre el papel, rápidamente fue tomando
estado caligráfico la vida de Elena Sanz. De las notas que aparté,
creyéndolas de escaso valor para mi objeto, se me antoja sacar alguna en
estas páginas para que los lectores se hagan cargo de la grandeza de
alma de mi heroína. «Hallándose de paso en París durante la tremenda
explosión revolucionaria de la \emph{Commune}, apareció en los sitios de
mayor peligro recogiendo y curando a los heridos, y cuando las tropas de
Thiers acometieron y destrozaron a los valientes comunistas, la
intrepidez de \emph{la diva} tocó los linderos de lo sublime. Más tarde
le concedió la villa de París distinciones y diplomas por su ejemplar
conducta, y de permitirlo la ley se la hubiera condecorado con la Legión
de honor. Añadiré a esto que en todo tiempo distinguió a Elena Sanz una
generosidad inaudita; no se presentaba a sus ojos ningún infortunio que
no fuese al momento espléndidamente socorrido; el pueblo la titulaba,
con sobrada razón, la madre de los pobres.

»De un brinco me planto en el año 79 para deciros\ldots» Al llegar este
punto advertí que no necesitaba de la milagrosa pluma para continuar
historiando, pues los hechos que ahora relataré fueron apreciados
fácilmente por mi propio conocimiento, o por fidedignas referencias de
los amigos. Guardé en lugar seguro el cálamo de la verdad, y con el mío,
vulgarísimo y comprado en la tienda, seguí pergeñando los anales de la
vida hispana, sin distinguir lo interno de lo externo.

Según los verídicos informes de Segis y de su madre, en Sevilla dejaron
de ser platónicas las relaciones de Alfonso XI con \emph{Doña Leonor de
Guzmán}. Durante algún tiempo permaneció esquiva la hechicera cantatriz,
encendiendo más con sus desdenes la exaltada pasión del Monarca. Pero al
fin, de tal modo extremó Alfonso sus delicadas artes de seducción, artes
realmente soberanas, que la pobre Elenita, quebrantada en su tesón de
mujer y de artista, \emph{cayó del lado de la libertad}.

Declaro que al saber esto tuve lástima de la hermosa y popular artista.
A mi modo de ver, fue gran necedad preferir el título de favorita del
Rey al de favorita del público. Pronto habría de serle imprescindible el
abandono de su brillante carrera teatral. Ved aquí el triste balance:
pérdida de doscientos o trescientos mil francos anuales con que le
pagaba el público sus gorgoritos; ganancia de una obvención de amor
relativamente miserable. A este desnivel lastimoso habría de añadir la
obscuridad, la social anulación a que fatalmente la condenaba el
implacable principio de la Razón de Estado.

¡Oh, la Razón de Estado! Esta pícara norma del vivir de los Reyes, no
siempre compatible con los sentimientos humanos, vino a truncar la dicha
de la \emph{bella del Rey}. Cánovas, y todos los hombres importantes que
con él dirigían la política de la Restauración, creyeron indispensable
para la felicidad de España que Alfonso XII contrajera segundas nupcias,
y que estas fuesen con Princesa católica de la más alta estirpe
reinante. Busca buscando, encontraron en la familia de Hapsburgo una
joven Archiduquesa de la empingorotada parentela del Emperador de
Austria Francisco José. Nuestros palaciegos se hacían lenguas de la
distinción, talento y virtudes de la que habían elegido para compartir
con Alfonso el solio de España.

Entabladas las negociaciones, pronto se llegó a un felicísimo acuerdo.
Decidiose celebrar las acostumbradas \emph{vistas} que preceden a los
desposorios regios, y este trámite tuvo efecto en Arcachón, a donde
acudió la novia con su madre la Archiduquesa Isabel, y don Alfonso con
el séquito correspondiente a su alta jerarquía. Resultaron las
\emph{vistas} conforme a lo previsto en el Protocolo, es decir, que
fuéronse gratos el uno al otro. ¡Ya teníamos Reina!

Un detalle que no debe preterirse es que el Rey fue a la entrevista de
Arcachón con el brazo derecho en cabestrillo. En la temporada estival de
La Granja sufrió Alfonso aquel año un accidente de caza, que le estropeó
la mano, imposibilitándole para escribir durante muchos días. Por cierto
que Su Majestad, hombre poco sufrido y algo voluntarioso, no quería
someterse al sistema de quietud y recogimiento que le impusieran los
médicos para curarle. Ninguna de las personas que le rodeaban conseguía
que el Rey refrenase su impaciencia por lanzarse a la vida ordinaria.

Sólo el criado de confianza de Alfonso, llamado Prudencio Menéndez,
discreto mediador en las relaciones del Monarca con \emph{Doña Leonor de
Guzmán}, logró someter a su Señor a las prescripciones facultativas,
gracias a este arbitrio de mágico efecto. Escribió a \emph{La Favorita}
una sentida carta. Entre otras cosas, le decía: «Cumpliendo mi primer
deber os comunico, doña Elena, la verdad sobre la importancia que tiene
el accidente sufrido por el Señor, para vuestra tranquilidad y para que
no creáis tantas mentiras como os contarán. Le ruego, señora mía, que
cuando le escriba le encargue por Dios no haga ningún esfuerzo hasta que
la cura esté \emph{echa}, pues de hacer ensayos podría quedar mal,
\emph{digáselo} usted por Dios, que a usted le hará caso.» Para mayor
exactitud no he querido alterar la ortografía arbitraria del documento.

Pertenece esta incidencia al ser interno de España. Ved de qué manera
tan chusca el cabestrillo de Alfonso entrelaza la protocolaria etiqueta
del ser externo, en \emph{las vistas} de Arcachón, con el influjo
decisivo de Elena Sanz. Después de lo que relatado queda, el Duque de
Bailén partió para Viena al frente de una lucida Embajada, con objeto de
pedir al emperador Francisco José la mano de la Archiduquesa María
Cristina. Mientras tanto, se preparaban en Madrid los imprescindibles y
tan acreditados festejos reales, con iluminaciones, fuegos de artificio,
corridas de toros con caballeros en plaza y demás requilorios que los
esponsales de la Majestad requieren.

Enorme angustia produjo a toda España la inundación de Murcia, en la
noche del 14 al 15 de Octubre de 1879. Desde que reventó el pantano de
Lorca en el siglo XVIII, no se había visto en aquella comarca catástrofe
tan terrible. Innumerables familias perecieron arrastradas por las
aguas. Fue una especie de parodia del Diluvio Universal, sin arca de
Noé, pero con aluvión de suscripciones, rifas, espectáculos, y sinfín de
arbitrios que se idearon en toda Europa y en América, para socorrer a
los infelices huertanos supervivientes de aquel espantoso cataclismo.
Aún duraban las tómbolas y las cuestaciones cuando la Razón de Estado, y
su inseparable compañera la Iglesia, unieron con lazos indisolubles al
Rey don Alfonso de Borbón y a la archiduquesa doña María Cristina de
Hapsburgo-Lorena.

Suprimo la cansada letanía de los festejos: el coruscante cortejo
nupcial, las áureas carrozas, los pintorescos palafrenes, el derroche de
percalinas, arcos de embadurnadas lonas, farolillos pitañosos y demás
garambainas para recreo de transeúntes aburridos. Apenas efectuadas las
nupcias mayestáticas, Martínez Campos y Silvela, que no habían hecho
cosa de fundamento en la esfera gubernamental, se retiraron por el foro,
volviendo Cánovas a ocupar el Poder con su inseparable acólito Romero
Robledo. Reanudadas las tareas parlamentarias, empeñáronse vivas
discusiones políticas por si fuiste o no fuiste, y por si hicimos o
dejamos de hacer. En una de aquellas sesiones ocurrió el famoso
incidente llamado \emph{el sombrerazo}. Hallábase no sé qué diputado
contendiendo con don Antonio Cánovas, cuando este, dejándose arrebatar
de su altanería, agarro el sombrero, y con mirada despectiva y ademán
impropio de aquel lugar que algunos llamaban augusto, salió del Salón
seguido de los demás Ministros, dejando al orador con la palabra en la
boca. Gran escándalo, desenfreno de vocablos no muy parlamentarios, y
retirada de todas las minorías.

Quedaron los ánimos un tanto agriados\ldots{} La muerte no quiso que
terminara el año sin arrebatarnos algunas personalidades ilustres. El 29
de Diciembre murió el General Zabala, una de las glorias más puras de
nuestro Ejército, y el 30, Adelardo López de Ayala, Presidente del
Congreso y figura culminante en el Parnaso español. Más le lloró la
Patria como poeta que como político. El mismo día 30 quiso hacer de las
suyas el fanatismo sectario: al entrar en coche por la Puerta del
Príncipe del Palacio Real Alfonso XII con su esposa María Cristina, les
disparó dos tiros un vesánico, Francisco Otero González, natural de
Santiago de Nantín, aldea de la provincia de Lugo. Las alevosas balas no
tocaron a los Reyes. El criminal fue detenido en el acto. Revelose como
un inconsciente, incurso cual su precursor Oliva en el pecado de
estupidez. Repito que los regicidas de aquellos tiempos, en que hasta la
exaltación política era rutinaria y pedestre, más bien parecían
engendros del Limbo que del Infierno.

En los comienzos del año 1880, hízose más patente la invasión del
positivismo en las almas de los afortunados políticos que entonces
estaban en candelero. El sabio consejo de un estadista francés que dijo
a sus contemporáneos \emph{enriqueceos, que ningún hombre público
agobiado por la pobreza puede hacer la felicidad de su Patria}, fue
tomado al pie de la letra por los que aquí pastoreaban el rebaño
nacional. El bendito \emph{Monsieur} Donon, a quien se adjudicó en
concurso la terminación de las líneas férreas del Noroeste, dio pruebas
de ser hombre sagaz, y al propio tiempo muy agradecido. Al constituir su
Consejo de Administración repartió las plazas de Consejeros, dotadas
espléndidamente, entre lo más granado de la Situación conservadora,
dando también su poquito de turrón a los liberales, y mucho más a la
gente palatina.

Recuerdo ya las caras risueñas y complacidas que tenían en aquel tiempo
todos los agraciados con los premios gordos de la lotería
\emph{Dononiana}. Recuerdo también que un conspicuo gacetillero hizo un
chiste que ha quedado de repertorio. Disputaban varios amigos en el
Salón de Conferencias del Congreso para determinar cuáles eran los
segundos apellidos de las dos ramas borbónicas. Alguien dijo que todos
llamábanse Borbón y Este, y nuestro gacetillero contestó en el acto que
el Rey de España se llamaba don Alfonso de Borbón y del Noroeste.

Platicando yo un día de tales cosas con mi amigo Segis, recordamos el
caso de doña Baldomera. La sagaz arbitrista, cuya fuga relaté a su
tiempo, había vivido tranquila en Ginebra, comiendo el fruto de sus
ardides financieros. \emph{Libre, feliz e independiente} permaneció en
Suiza amparada por las leyes de aquel país, donde no había extradición.
Alguien le hizo creer que en España ya no se acordaban de ella, y que
podía recorrer a su antojo toda Europa si así le venía en gana.
Alucinada por esta idea marchó a París. En mal hora lo hizo. Cuentan que
por denuncia de su hermana Adela, \emph{la dama de las patillas}, fue
doña Baldomera Larra detenida y puesta a buen recaudo. Tramitada la
extradición, trajeron a la pobre señora a Madrid entre gendarmes y
guardias civiles.

Díjome Segismundo que solía visitar a la cautiva en la Cárcel de
Mujeres, por agradecimiento a las bondades que tuvo con él en los días
felices del \emph{Banco Popular}. Últimamente habíala encontrado
sosegada, risueña, expresándose con el donaire y afabilidad que usar
solía tiempos atrás en su conversación. Creyó entender Segismundo por el
tono y actitud de la sutil financiera, que esta, repartiendo con arte y
discreción los dineritos que aún poseía, esperaba ser absuelta
libremente. «Pues nada más justo---dije yo.---¿Qué razón hay para
condenar a esa señora? La cárcel debe ser para todos o para ninguno. Sí;
que la absuelvan, y en cuanto esté libre que restablezca su
\emph{Banco}, y otra vez se le llenará la casa de dinero.»

Los progresos del positivismo en nuestra sociedad conocíanse, no sólo en
las caras sonrosaditas y alegres de los que se procuraban enormes
sueldos para dulcificar la vida, sino en las incorporaciones de diversos
grupos al Partido Constitucional, de que resultó el inmenso conglomerado
llamado \emph{Fusionismo}. Antes de esto, Martínez Campos, procediendo
con gallardo desinterés y harto de las arrogancias de don Antonio
Cánovas del Castillo, se agregó a la hueste sagastina.

Tales movimientos del ánimo pertenecen al ser interno de la Nación,
preferente objeto de mis investigaciones en la tarea histórica.
Cultivando gozoso el huerto de la vida intrínseca seguiré el cuento de
Elenita, que en este año de 1880 me ofrece particularidades de incitante
interés. Ya sabía yo que la simpática y bondadosísima doña Isabel II no
veía con malos ojos los deslices de su hijo Alfonso con Elena Sanz, y
que no había retirado a esta el cariño que le profesaba desde que fue
lucida colegiala en las Niñas de Leganés. Nacido el primer hijo de aquel
idilio morganático, doña Isabel hizo manifestaciones muy sinceras y
expresivas, aunque reservadas, en favor de Elena Sanz. A este primer
vástago le pusieron el nombre de Alfonso.

Robustecí mi conocimiento de tales cosas requiriendo la maravillosa
péñola, que un día me escribió este trozo de palpitante verdad: «La
Reina Madre Isabel II comisionó a un venerable sacerdote que había sido
su confesor, don Bonifacio Marín, para que visitase a don Alfonso XII,
interesándole por la que ella llamaba \emph{su nuera ante Dios}. El
dichoso cura expresó a Elena Sanz sus impresiones de la visita en una
carta fechada en 4 de Abril, de la que transcribo este sustancioso
parrafito: He sido recibido y oído con gratitud y amabilidad
inexplicables, cuyo júbilo particular le comunico \emph{por orden
expresa}, a la par que con toda mi espontaneidad.»

La pluma me ha suministrado referencias de otra carta del criado y
confidente del Rey, Prudencio Menéndez, en la que este, después de
notificar a Elenita que \emph{el Señor} se proponía escribirle con
extensión, terminaba así, haciendo referencia al bastardo Alfonso:
«Celebro mucho que esté tan bueno el \emph{Señorito}, y que la distraiga
a usted, que bien lo necesita\ldots» La péñola me dio asimismo noticia
de otra epístola del Marqués de Alta Villa, fechada en el \emph{Palais
de Castille} de París, en la que se lee un membrete que dice:
\emph{Grand Maître de la Real Casa de doña Isabel II}. En esta epístola,
el \emph{Grand Maître} pide a Elena Sanz que recomiende con eficacia al
Rey una porción de cosas de mucho interés para él, para el señor
Marqués, naturalmente. Luego hace referencia a una cestilla de dulces
que Elenita le envió para doña Isabel, y concluye con estas cariñosas
admoniciones: «Adiós, Elena. Tengan ustedes juicio. Acuérdese usted y
tenga \emph{él} presente que puede usted perder su voz y su
carrera\ldots{} y esto tiene consecuencias bien desagradables.»

La Razón de Estado, sorda y ciega ante los casos idílicos tocantes al
augusto fuero de la pasión humana, continuaba elaborando tranquilamente
la vida externa de España, ora con hechos de carácter político, ora con
otros de un orden familiar. Entre estos debo señalar el parte que
publicó en la \emph{Gaceta} la Facultad de Medicina de la Real Cámara,
notificando al país con tonos jubilosos que Su Majestad la Reina doña
María Cristina se hallaba en estado interesante.

\hypertarget{xx}{%
\chapter{XX}\label{xx}}

En los mismos días en que la pregonera del vivir oficial comunicaba al
pueblo español albricias y congratulaciones, por la probable felicidad
de que nuestros Reyes tuvieran pronta y quizá masculina sucesión, empezó
a correr por Madrid rumor muy denso de los amores de Alfonso con
\emph{doña Leonor de Guzmán}, y hasta llegó a decirse que había nacido
el primer bastardo, el primer \emph{Trastamara}. ¡Bonito porvenir te
esperaba, oh Nación española!

Revolviendo en mi mente tan inauditos casos, y pensando en las
complejidades que podían ocasionar en tiempos próximos o lejanos,
despertose en mí cierta conmiseración simpática por la Reina doña María
Cristina. ¿Tendría conocimiento la augusta señora de los hechos que
delataba el obstinado mosconeo popular? Sospechaba yo que sí. La
sospecha se trocó en certidumbre un día que me encontré con mi antiguo
amigo Quintín González, esposo de la sensible planchadora Nieves, con la
que yo tuve algo que ver en los tiempos para mí venturosos de don Amadeo
I. Quintín ya no era portero de Palacio, sino ujier de antecámara, cargo
cuyas funciones le aproximaban a las reales personas. Díjome \emph{que
la Señora lo sabía}. Pero que se encastillaba dentro de su dignidad como
Reina de cuerpo entero, no dejando traslucir agravios de cierta índole,
que rebajan más al que los manifiesta que a quien los infiere.

Deseaba yo ver de cerca a la Reina María Cristina. Una tarde, mi buena
suerte me deparó la ocasión de satisfacer esta curiosidad en el Real
Sitio de Aranjuez. Fuimos Casiana y yo a pasar el día en aquellos amenos
lugares, y un amigo residente en el pueblo nos proporcionó papeletas,
con las cuales podíamos ver los jardines y la casita de abajo, no el
Palacio, por estar allí los Reyes. Paseamos tranquilamente por la Isla,
y el señor que nos acompañaba nos dijo que no veríamos a Sus Majestades,
pues desde por la mañana hallábanse en \emph{La Flamenca}, con los
Duques de Fernán Núñez y unos Príncipes austriacos.

Admirábamos Casianilla y yo los gigantescos álamos que parecían tocar
las nubes, las copiosas y murmurantes aguas que por una y otra parte
embelesaban la vista, cuando divisamos a los Reyes con lucido
acompañamiento, que en dirección contraria a la nuestra venían. Al
llegar las regias personas cerca de nosotros, nos detuvimos para
dejarles paso y saludar con todas las ceremonias que nuestra buena
educación, a falta de monarquismo, nos exigía.

La Reina pasó muy cerca de mí, y en su elegante persona se saciaron mis
ojos. Agradome en extremo su porte señoril y su aire de dignidad y
nobleza. A nuestro saludo contestó la Soberana con una reverencia
graciosa y afable. Casianilla, con la boca abierta y los ojos
espantados, veía alejarse a María Cristina, admirando tanto su persona
como su ropaje. Luego me dijo: «Bien se le conoce el nacimiento, la
estirpe que es, como tú dices, la más encumbrada del mundo.»

De regreso del paseo di a mi compañera una compendiosa lección histórica
de la Casa de Austria. Rebañando en mis vagos recuerdos hablé del Rey de
Romanos, del entronque de la Casa de Borgoña con la de Castilla, de doña
Juana la Loca, del Emperador Carlos V, de su hermano don Fernando,
heredero de la Corona imperial, y luego de toda la serie de Hapsburgos y
Hapsburgos-Lorenas hasta la familia reinante a la sazón en Austria.

Aquel verano nos arrastró a San Sebastián y a sus baños de ola la
Condesa de Casa Pampliega. No me pesó ir con Segis y su madre, porque
así nos dimos el pisto de veranear en el sitio de moda y de refrescar
nuestra sangre con las aguas cantábricas. Fueron muy de mi gusto la
frescura del ambiente, la belleza del país, la cultura de la ciudad, la
buena educación de sus habitantes. En cambio, no me hizo maldita gracia
la sociedad que allí se congregaba, que era la misma gente frívola de
Madrid, con sus cargantes etiqueteos, sus rutinas y su cursilería.

Al volver a la Villa y Corte me encontré sorprendido por el fausto
suceso del alumbramiento de la Reina María Cristina, en 11 de
Septiembre. El parto fue muy feliz, según los luminosos dictámenes de la
Facultad de Medicina de la Real Cámara y los concienzudos informes de la
Prensa. Mas como vino al mundo una niña, quedaron chasqueados y
cariacontecidos los que esperaban anhelantes sucesión masculina para la
Corona de España. Apenas nacida la tierna criatura, descendiente de
tantos Reyes y Emperadores, su dorada cuna se meció en un campo de
Agramante, por el recio altercado que sostuvieron políticos y palatinos
sobre si correspondía o no a la nueva Infanta el título de Princesa de
Asturias. Contra el sentir general, Cánovas sostuvo la negativa,
robusteciéndola con los grandes elementos de su vasta erudición. El
heráldico litigio encendió los ánimos de toda la gente ociosa y
formulista, y nunca hubiera terminado a no cortar la cuestión Alfonso
XII con fallo inapelable.

Mayores disturbios y disputas más agrias produjeron las ridículas
cuestiones de etiqueta suscitadas en las solemnidades de la presentación
y bautizo de la Infanta, a quien dieron el nombre de María de las
Mercedes. Los Cardenales Moreno, Primado de las Españas, y Benavides,
Patriarca de las Indias, se tiraron las mitras a la cabeza---valga la
figura---por si correspondía al uno o al otro el honor de administrar el
Sacramento. Ambos Prelados y sus parciales se lanzaron a enfadosas
polémicas en lo restante del año 80, sosteniendo cada cual sus
pretendidos derechos.

Contienda tan ridícula no había yo visto en mi vida. Me divirtió de lo
lindo. Pero aún me regocijó más el enojo de los Capitanes Generales
porque, habiendo tomado asiento en no sé qué banco preferente de la Real
Capilla, un palatino obligoles a cambiar de sitio diciendo que aquel era
el puesto de los mitrados. ¡Jesús, la que se armó! Los Príncipes de la
Milicia, así como los de la Iglesia, que en este pobre Estado español no
tenían nada que hacer, pues sus funciones eran puramente decorativas y
pintureras, mantuviéronse alborotados y de puntas hasta el año
siguiente, sin que les aplacaran las gracias y mercedes que el Gobierno
derramó sobre ellos a manos llenas.

¡Delicioso país este rincón occidental de Europa! Da grima leer la
Prensa en aquellos meses. Todos los periódicos llenaron columnas y
columnas con los piques de este General y de aquel Obispo, con las
conferencias y cabildeos entre los agraviados y el Jefe Superior de
Palacio o el Presidente del Consejo de Ministros, para domesticar a las
fieras de la vanidad. Por si fuera poco esto, los Consejeros de Estado
elevaron una imponente protesta a Su Majestad el Rey por habérseles dado
un puesto poco decoroso en le Real Capilla, y, si no estoy equivocado,
también los claros varones de la Sociedad Económica de Amigos del País
solicitaron mayores preeminencias en los actos de fanfarronería oficial.
Yo dije a Casiana: «Un país sin ideales, que no siente el estímulo de
las grandes cuestiones tocantes al bienestar y a la gloria de la Nación,
es un país muerto. La Prensa, consagrada a glosar y a comentar los
incidentes de estas chabacanas querellas, exhala de sus columnas un olor
cadavérico. Prensa, Gobierno, Partidos, altos y bajos Poderes, todo ello
anuncia su irremediable descomposición.»

Para mayor ignominia, las mercedes concedidas por el Rey en celebración
del natalicio de la Infantita, ofrecen nuevo ejemplo de la degradante
frivolidad a que habían llegado las clases superiores del Estado. El
reparto de dos Toisones, de no sé cuántos collares de Carlos III, de
grandes cruces, encomiendas, bandas de María Luisa, Grandezas de España
y títulos de Castilla, dio margen a una rebatiña vergonzosa. Tal
espectáculo era el signo más característico de unos tiempos en que las
\emph{turbas} que se llamaban directoras no tenían otros móviles que el
egoísmo, la farsa y el delirio de las distinciones farandulescas.

Con la feria de fatuidades coincidió aquel año la era de las expansiones
gastronómicas. Todos los españoles grandes o mediocres que tenían algo
que manifestar a sus amigos o al pueblo, derramaban su elocuencia sobre
los blancos manteles, ante unos comistrajes indigestos y mal servidos.
Balaguer en Valencia, Barcelona y Lérida, Vega de Armijo en Córdoba,
Romero Robledo en Sevilla, Castelar en Alcira, y Carvajal en Málaga,
lanzaron sus trenos patéticos o jocosos tras el solemne momento de
\emph{descorchar el champagne}. Luego \emph{gemían las prensas}
reproduciendo en largas columnas toda esta caudalosa palabrería que, con
excepción del verbo soberano de Castelar, era como remolinos de
hojarasca que se lleva el viento.

Mis relaciones con Segis y con su madre se estrecharon más en aquel
Otoño. La Condesa de Casa Pampliega, a pesar de su finchación
nobiliaria, no repudiaba el trato con mi pobre Casianilla. Cierto que no
la presentó en sus salones heteróclitos, a donde concurrían familias de
nobles tronados y de tenderos enriquecidos. Pero cuando yo iba con mi
compañera por las tardes a la mansión condal, recibía su visita la
señora con mucho agrado, gustosa de la llaneza, buen apaño y suave
condición de la señorita de Conejo. Indudablemente, doña Segismunda,
mujer desprovista de toda cultura, simpatizaba con Casiana al verla tan
instruidita y al oírla expresarse con un claro sentido, que para ella
era el colmo de la sapiencia. Excuso decir que la improvisada Condesa se
había hecho conservadora furibunda, y que sentía por don Antonio Cánovas
un entusiasmo delirante.

«¡Qué hombre, qué talento, qué elocuencia!---solía exclamar.---¿Y dicen
que es bizco? No, señor. ¡Qué bizco ni qué niño muerto! Es un caballero
que ve largo y mira muy por derecho.»

Desde que volvió de San Sebastián, la Condesa de Casa Pampliega
frecuentaba el santuario y colegio de las Hermanas del Corazón de Jesús,
en le calle del Caballero de Gracia. A esto la movía, más que su propio
misticismo, el afán de codearse con damas de la más alcurniada sociedad
de Madrid. Por hacer el papelón apencaba con los enfadosos ejercicios
espirituales, y asiduamente se dejaba ver en las diarias solemnidades de
Novenas, Triduos, Cuarenta Horas, \emph{etcétera}. En este trajín hizo
amistades con varias señoras beatas y con algunos de los jesuitas
predicadores, que constantemente estaban metidos en aquella santa casa.
Por cierto que, según oí, un Padre de los más sagaces puso los puntos a
doña Segismunda para sacarle dinero; pero a tanto no llegaba la piedad
\emph{fashionable} de la flamante Condesa. La discreta y astuta dama
paró el golpe\ldots{} Mas ya se lo dirían \emph{de misas} cuando se
hallase \emph{in articulo mortis}\ldots{} Entonces sí que no se
escapaba\ldots{} ¡Pobre Segis! Como se descuidara le dejarían en cueros
vivos.

A propósito de Segis diré que su indómita rebeldía se iba modificando
por las flexibilidades de aquella época positivista. Evolucionó con
suavidad hacia el arte o ciencia del buen vivir, y acabó por entregarse
a un filosofismo atrozmente cínico. Dejábase llevar por la Condesa a las
beaterías del Caballero de Gracia, y de otras iglesias de moda,
afectando cierta contrición y propósito de enmienda que a muchos
engañaba, y a mí, que tan bien le conocía, causábame el efecto más
cómico que puede imaginarse. El principal objeto de esta farsa era
vigilar constantemente a su madre, para estar al quite de los ataques
con que los sagaces caballeros de la faja negra amenazaban al saneado
caudal de Casa Pampliega.

En las francas expansiones que conmigo tenía Segismundo, se quitaba la
máscara hipócrita para revelarme con esta leal llaneza los móviles de su
conducta: «Ni tú ni yo, querido Tito, podemos esperar nada del estado
social y político que nos ha traído la dichosa Restauración. Los dos
partidos, que se han concordado para turnar pacíficamente en el Poder,
son dos manadas de hombres que no aspiran más que a pastar en el
Presupuesto. Carecen de ideales, ningún fin elevado les mueve, no
mejoraran en lo más mínimo las condiciones de vida de esta infeliz raza,
pobrísima y analfabeta. Pasarán unos tras otros dejando todo como hoy se
halla, y llevarán a España a un estado de consunción que de fijo ha de
acabar en muerte. No acometerán ni el problema religioso, ni el
económico, ni el educativo; no harán más que burocracia pura,
caciquismo, estéril trabajo de recomendaciones, favores a los amigotes,
legislar sin ninguna eficacia práctica, y adelante con los
farolitos\ldots{} Si nada se puede esperar de las turbas monárquicas,
tampoco debemos tener fe en la grey revolucionaria. ¿Crees tú, Titillo,
en la revolución?

---Yo no---contesté resueltamente.---No creo ni en los revolucionarios
de nuevo cuño ni en los antediluvianos, esos que ya chiflaban en los
años anteriores al 68. La España que aspira a un cambio radical y
violento de la política se está quedando, a mi entender, tan anémica
como la otra. Han de pasar años, lustros tal vez, quizá medio siglo
largo, antes que este Régimen, atacado de tuberculosis étnica, sea
sustituido por otro que traiga nueva sangre y nuevos focos de lumbre
mental.

---De acuerdo, querido---dijo Segis.---Por eso yo he cambiado mi
rebeldía por un epicureísmo que me asegure el regalo y el reposo del
presente y el porvenir. Quiero vivir bien y sin fatigas; quiero asegurar
la posesión venidera del caudal que afanó mi madre\ldots{} como Dios le
dio a entender; quiero construirme, en fin, un bello refugio contra la
miseria. ¿Qué me importa doblegar la frente ante un curángano vestido de
ropones negros o colorados, ni prestarme a prácticas de puro formulismo
y exterioridad, si esto que yo llamo etiqueta litúrgica, no exenta de
belleza en algunos casos, jamás penetra en mi libre espíritu? Al
principio me violenté no poco para lograr acomodarme a las beaterías de
mi señora madre. Pero luego fui entrando por grados,
insensiblemente\ldots{} Todo se reduce a una farándula más entre las
múltiples que regulan la conducta social del hombre civilizado, como por
ejemplo, la buena educación, el respeto a las personas que ostentan
alguna dignidad aunque sean unos gaznápiros, el someterse a las modas
del comer, del beber, del vestir y del calzar, y otras tonterías que
hacemos de continuo, sin parar mientes en nuestra imbecilidad.»

No iba descaminado el amigo García Fajardo en su apreciación de las
cosas de España; pero las ideas que expresó para justificar su proceder,
me parecieron más ingeniosas que razonables. Pocos días después de lo
que acabo de contaros, supe que la infatuada Condesa de Casa Pampliega
había concebido el plan de casar a su hijo con una señorita honesta y de
buen ver, hija única de opulento matrimonio, muy notado por su
catolicismo a macha martillo y por sus conexiones con toda la gente de
la Iglesia. Nació este proyecto de las amistades que doña Segismunda
contrajo en el Sagrado Corazón con damas ilustres y con algunos
Reverendos de la Compañía.

La candidata a la mano de Segis llamábase Ritita, y en sus padres se
habían reunido los linajes de Erro, Sureda, Socobio y Landázuri, todos
ellos, como sabéis, rabiosamente absolutistas. Parentesco tenía también
Rita con los Emparanes, Trapinedos y Pipaones, y llamábase sobrina de
los Marqueses de Beramendi y de la Marquesa de Villares de Tajo. Andando
días me aseguraron que la boda de Segis era un hecho. Directamente acudí
a mi amigo para que me sacase de dudas diciéndome la verdad, y con gran
estupor mío habló de esta manera:

«No es todavía un hecho, querido Tito; pero podrá serlo pronto, muy
pronto. He consagrado largas cavilaciones a madurar el asunto, y al fin,
tanto se ha obstinado mi madre y tales razones me han expuesto mi tío
Beramendi y mi tía María Ignacia, que he acordado rendirme a discreción.
La muchacha es buena, muy rezadora y amiga de comerse los santos. En su
vida leyó más libro que \emph{El Año Cristiano}. Pero a mí ¿qué me
importa? Parece que le he caído en gracia, y que me quiere un poquitín.»

Contagiado del fantástico catolicismo de Segis, me persigné, diciéndome
con picante ironía: «¡Alabado sea Dios! Ya veo bien clara la \emph{lenta
pero continua} evolución de nuestra bendita sociedad hacia las ollas del
ultramontanismo.»

\hypertarget{xxi}{%
\chapter{XXI}\label{xxi}}

Tratábamos una mañana Segis y yo de esta interesante y hasta cierto
punto divertida mudanza, cuando se llegó a nosotros la Condesa de Casa
Pampliega cargada con un rimero de polvorientos librotes, que puso sobre
un velador, diciendo: «Mi marido, que en gloria esté, heredó de su
hermano Ramón la mar de libros viejos que yo he conservado largo tiempo
en la bohardilla, entre los montones de trastos inservibles. Ayer mandé
a Micaela que los bajase para dárselos al trapero con unos miriñaques
míos, y los bragueros y otras prendas de mi difunto. Pero cuando la
chica y yo quitábamos la mugre a los librachos, pensé que estos
mamotretos son muy del gusto de don Antonio Cánovas, el cual tiene en su
casa gran acopio de ellos y los cuida como a las niñas de sus ojos. Se
me ha ocurrido que debo, no vendérselos, sino regalárselos, pues
seguramente estimará mucho el obsequio. Si te parece bien, Segismundo,
llévaselos tú mismo y ofréceselos en mi nombre, poniendo en cada uno
tarjetas de las nuevas que ayer me trajiste con mi nombre, título y
corona condal.»

A esto dijo García Fajardo con agria displicencia, que aunque él se
dejaba llevar del curso evolutivo de las aguas sociales, no tenía
maldita gana de presentarse a don Antonio, ni a ningún otro fantasmón de
la ganadería conservadora. En tanto, yo levantaba las tapas de pergamino
para ver los títulos de aquellos vetustos infolios, y leí los rótulos
que siguen: \emph{Diversas fazañas y Tractado de los rieptos y
desafíos}, por Mosén Diego de Varela, cronista de la Reina
Católica.---\emph{Memorial en detestación de los grandes abusos en los
trajes y adornos nuevamente introducidos en España}, por Alfonso Carraza
(Madrid 1640). ---\emph{Clavellinas de recreación}, por Ambrosio de
Salazar (Ruan 1614). ---\emph{Geometría y trazas pertenecientes al
oficio de sastre}, por Martín de Andújar (Madrid 1640).---\emph{Diálogo
de la verdadera honra militar}, por don Hierónimo de Urrea (Venecia
1566), y otros rarísimos títulos, entre los cuales distinguí el de la
obra del Reverendo Padre Hernando de Talavera, primer Arzobispo de
Granada, \emph{Tractados de la mesa, del vestir e calçar e de la
mormuración}.

Examinados los libros, dije a doña Segismunda que no tenía yo
inconveniente en ofrecer a don Antonio las obras con que la señora
Condesa le obsequiaba. Dos veces había visitado yo a Cánovas y sin duda
me acogería con agrado, pues, a pesar de su fama de mal genio, era
hombre cortés y de cortesana educación. Conformes hijo y madre en darme
credenciales de embajador de los Casa Pampliega cerca del Presidente del
Consejo, me personé en el número 2 de la calle de Fuencarral el segundo
domingo de Adviento, 5 de Diciembre, porque me constaba que las mañanas
de los días festivos pasábalas el gran don Antonio en el recreo de su
magnífica biblioteca. Recibiome con gran displicencia el famoso criado
Ramón, dándome a entender que era notoria osadía intentar acercarse al
Presidente sin traer etiqueta o marchamo de personaje muy calificado de
la Situación. Con risita guasona levanté el papel que era envoltura de
los librotes, para que Ramón viese el título con que yo pretendía ser
llevado a la presencia del grande hombre. En cuanto el fámulo vio los
arrugados pergaminos, desarrugó el entrecejo y me dijo:

«¿Viene usted a vender al señor sus libros?

---No, no. Vengo a regalárselos de parte de la Excelentísima señora
Condesa de Casa Pampliega. Son obras muy raras, y pienso que algunos de
estos incunables no figuran en la biblioteca del Presidente.»

Suplicándome que esperase un momento se internó Ramón en la casa, para
anunciar a su amo la visita de un bibliófilo. Instantes después me
encontraba en la presencia del insigne político y erudito historiógrafo.
Había yo entrado con cierto temor en la morada del estadista, pensando
que mis anteriores visitas \emph{al monstruo} fueron fantásticas, obra
de mi desbordada imaginación o artífice dispuesto por las
\emph{Efémeras} obedientes a misteriosos dictados de mi divina Madre.
Contra lo que yo esperaba, don Antonio me reconoció al instante, y con
llaneza y afecto me dijo:

«Hola, señor Liviano\ldots{} Mucho gusto en verle\ldots{} ¡Ah!,
¿libritos viejos? ¿También padece usted mi chifladura? Veamos, veamos
qué es eso.»

Con ágil mano alzó Cánovas las tapas de los volúmenes para examinarlos,
y al llegar al de Fray Hernando de Talavera, exclamó lleno de júbilo:
«¡Ay\ldots{} esto no lo tengo, no lo tengo! Conocía la obra por citas
que de ella hacen otros autores\ldots{} \emph{Tractados de la mesa, del
vestir e calçar e de la mormuración}. Es un libro interesantísimo.
¡Cuánto se lo agradezco!\ldots{} Los demás que me trae usted creo que
los tengo todos, menos este: \emph{Carro de las dona}, por Fray
Francisco Ximénez, Obispo (Valladolid 1542)\ldots{} ¡Ah! Tampoco poseía
este otro: \emph{De las cosas que traen de las Indias que sirven al uso
de la Medicina}, por Monardes (Sevilla 1569)\ldots{} En cambio poseo una
edición lindísima del \emph{Libro del arte de las comadres}, por Damián
Carbón, y dos ejemplares, uno de Venecia y otro de Amberes, del
\emph{Diálogo de la verdadera honra militar}, de Hierónimo de
Urrea\ldots{} Difícilmente podrá usted traerme una obra de arte militar
que yo no tenga\ldots{} Deme usted ahora las señas de la señora Condesa
de Casa Pampliega, que quiero ofrecerle personalmente mis respetos y
darle las gracias por su valioso regalo.»

Pensaba yo en el loco entusiasmo de la vanidosa doña Segismunda al saber
que sería visitada por el Presidente del Consejo, cuando este,
reteniéndome con bizarra cortesía, se dignó mostrarme los primores de su
rica biblioteca. Vi preciosos incunables, manuscritos de inmenso valor,
y los cuadernos de las Cortes de Castilla, Aragón, Valencia y Navarra,
con las pragmáticas y cédulas reales emanadas de sus acuerdos.
Convencido regalista, Cánovas puso ante mis ojos un verdadero tesoro
diplomático y bibliográfico de las cuestiones habidas entre España y
Roma desde los Reyes Católicos, Carlos V y Felipe II, hasta Felipe V y
Carlos III.

A propósito de esto, entablamos una conversación, iniciada por él
gallardamente. Sentados junto a la gran mesa central del salón de la
biblioteca, don Antonio me honró más de lo que yo merecía, oyendo mis
opiniones sobre la independencia del poder civil. Orgulloso de la
gentileza con que me hablaba, considerándome equivocadamente como
historiador de la actualidad palpitante, me atreví a expresar esta idea:

«¿Y qué me dice usted, señor don Antonio, de la irrupción de los frailes
expulsados de Francia por las leyes y edictos del pasado Noviembre?

---Reconozco la gravedad del problema que se nos presenta---me contestó
Cánovas, mordiéndose el bigote y afirmándose los lentes sobre el
caballete de su nariz.---Pero ha de reconocer usted, como historiador
imparcial, atento a la circunstancialidad de las cosas públicas y a la
estructura interior de cada partido, que yo no soy el llamado a cerrar
el paso a la caterva de regulares despedidos de Francia. Por ahí se dice
que los constitucionales, llamados ahora fusionistas, verán calmada muy
pronto su impaciencia por gobernar a la Nación. Créame usted: no
encontrarán en mí esos señores la menor resistencia para sustituirme en
el puesto que ocupo. Dos cosas deseo: el descanso mío, y ver el estreno
del nuevo partido en las funciones del Gobierno. Si Sagasta no reniega
de su historia, su primer cuidado al llegar al poder será poner diques a
la inundación frailesca, ateniéndose estrictamente a la letra del
Concordato. Cada cual debe permanecer en su terreno propio, gobernando
conforme a sus ideales y a sus compromisos. La realidad histórica, el
carácter y sentido de las fracciones políticas que me han dado su apoyo
para consolidar la Restauración, me impiden realizar con acento vigoroso
la política regalista. Sagasta es el llamado\ldots{} ¿no lo cree usted
así?»

Con expresivas cabezadas asentí a las observaciones del Presidente, el
cual siguió mostrándome curiosos ejemplares de su soberbia librería.
Cual padre amoroso encariñado con sus tiernas criaturas, me presentó el
precioso incunable \emph{Coronación de D. Íñigo López de Mendoza y
coplas de Juan de Mena}, editado en 1489. Después admiré el
\emph{Doctrinal de Caballeros}, del Obispo de Burgos don Alonso de
Cártagena, impreso en 1487, fijándome en las anotaciones que el propio
don Antonio puso en las guardas de tan interesante y arcaico libro. Vi
también la \emph{Invención liberal y arte del juego de axedrez}, por Ruy
López de Segovia, clérigo, \emph{vecino de la Villa de Çafra}, dado a la
imprenta en Alcalá de Henares el año 1561, y otras joyas preciadísimas
del arte de imprimir en los siglos XV y XVI.

En este punto hirió mi olfato un fuerte aroma de tomillos. ¿Eran los
tomillos del monte Hymeto?\ldots{} Creí entrar en la esfera de las
alucinaciones: al olfato se agregaron los ojos haciéndome ver una figura
de mujer, arrogante, de luengos paños negros vestida, que de las
estanterías sacaba los libros para ponerlos en las manos del poseedor de
tanta riqueza tipográfica. Entregado de lleno al trastorno de mis
sentidos o a la percepción del vidente que explora el mundo
ultraterreno, reconocí a mi excelsa Madre que hacía el servicio de
auxiliar de bibliotecaria. \emph{Mariclío} clavó en mí una mirada de
fuego, transmitiéndome los pensamientos que literalmente traslado:

«Toda esta ciencia arcaica y este fárrago que tuvieron su porqué y sazón
en siglos remotos, ¿le sirven al buen don Antonio para consumar y
sutilizar sus artes de estadista y gobernador de los Reinos hispanos, o
sería el mismo sujeto, que descuella hoy al frente de los negocios
públicos, si estuviera privado del continuo trato con los treinta mil
volúmenes que adornan las paredes de esta noble vivienda? Las venerables
antiguallas de arte de guerra, y de las armas e ingenios militares de
tiempos remotos, ¿ayudan al conocimiento y régimen de los Ejércitos de
nuestros días? Voy creyendo que esto no es más que un bello delirio de
coleccionista, ávido de gozar tesoros raros no poseídos por otro alguno,
monomanía que satisface los amores de la erudición platónica, con poca o
ninguna eficacia en el arte de aplicar las sabidurías trasnochadas al
vivir contemporáneo.»

Llegó el momento de despedirme del patriarca de la Restauración, el cual
me reiteró su afecto, invitándome a repetir mis visitas en su casa o en
la Presidencia, donde esperaba recibir poco tiempo más.

Al salir yo de la biblioteca repitiéronse los fenómenos
peri-espirituales, pues si no me engañaron mis ojos, la divina
\emph{Clío}, gallarda y bien oliente, despidiendo de su ropaje el aroma
de las hierbas del monte Hymeto, me condujo de la mano hasta el
vestíbulo, entregándome al celoso guardián de su Excelencia, conocido en
el mundo político por su nombre de pila.

Ramón, más complaciente a mi salida que a mi entrada, me abrió la
puerta, y tranquilamente descendí la escalera, satisfecho de haber
aumentado el tesoro bibliográfico de don Antonio Cánovas del Castillo.

\hypertarget{xxii}{%
\chapter{XXII}\label{xxii}}

En la calle me esperaba Casiana, algo inquieta por mi tardanza.

«Ya sabes---me dijo---que doña Segismunda está en ascuas por saber cómo
ha recibido este buen señor los librotes del tiempo de
\emph{Maricastaña}. ¿Nos volvemos allá?

---No---repliqué.---Vámonos calle arriba para que se me despeje la
cabeza. Estoy un poco mareado de ver infolios y legajos, que a mi
parecer no sirven más que para llenar de telarañas el
entendimiento\ldots{} Nos llegamos hasta la \emph{Era del Mico} o el
\emph{Campo del Tío Mereje}, y confortaremos nuestros cuerpos ateridos
con la benéfica luz del sol. No nos faltará espacio para pasear a gusto
y charlar sabrosamente cuanto nos dé la gana.

---Por esos lugares no me lleves, Tito---indicó mi Casiana un tanto
medrosa.---Allí se reúnen las brujas, según me has dicho, y yo no quiero
trato con esa caterva.

---No temas nada, chiquilla---le repondí riendo.---Una mujer ilustrada
como tú no debe asustarse ante las viejas carroñas que, ya cabalgando en
sus escobas, ya montadas una sobre otra, acuden a la cita del Gran
Cabrón. Fíjate además en que los aquelarres son funciones esencialmente
nocturnas, y a estas horas, en pleno mediodía, no hay que temer las
visitas de las almas del otro mundo ni de las vejanconas puercas que
hociquean con el diablo.

---Pues vamos allá, que aunque no tengo la debida ilustración, donde tú
estés yo no me asusto de nada.

---Muy bien. Pero no me niegues la verdad de tu cultura, Casiana mía,
que anoche bien te luciste en la tertulia íntima de la señora Condesa,
cuando contendías discretamente con aquellas dos damas de las
aristocracia \emph{que acaba de salir ahora}, una de las cuales soltó el
disparate de que los Reyes Católicos eran los padres de Felipe II y de
Fernando VII.

---Fue la que llaman Marquesa de San Epifanio la que echó de su bonita
boca ese garrafal desatino. Yo no me atreví a corregirla más que con una
frase por tabla, y tú remataste la suerte. La otra, señora muy entonada,
que se enriqueció con el comercio de petróleo, lleva el apellido de
Cucúrbitas, es muy redicha y punto fuerte en las modas del vestir, y no
se le escapa ninguno de los requilorios y perendengues que \emph{ahora
se llevan}. Sus lindas niñas se educan en el Sagrado Corazón.

---Donde aprenden Catecismo a todo pasto, nociones incompletas de
Aritmética y Geografía, mascullar el francés, un machaqueo de piano para
romper los oídos de toda la familia, y etiquetas y saluditos a estilo de
\emph{París de Francia}\ldots{} Al cuidado de los buenos Padres, estos
aguardan a que las educandas sean señoras para meter las narices en sus
hogares, adueñándose del marido y de los hijos, y por fin, esperan
cachazudos y tenaces a que se hagan viejas idiotas para quitarles todo
lo que tienen.

---Así es y así será. Y ahora te digo que la de San Epifanio anda muy a
la cola en ortografía. Ayer vi casualmente una tarjeta que escribió a
doña Segismunda, en la cual noté que pone hombre sin hache y ayer con
hache y elle. La de Cucúrbitas dice \emph{ivierno, ferroscarriles} y
\emph{Espirituisanto}.

---Ya lo ves, Casianilla: con lo poquito que tú sabes eres muy superior
a esas señoronas hartas de dinero, que nos miran a nosotros por encima
del hombro. Compárate, y verás bien claro tu superioridad. Vuelve la
vista al pasado, y te harás cargo del inmenso adelanto que has
conseguido desde que te saqué de la abyección y la miseria para elevarte
hasta donde ahora te encuentras. Ido te enseñó a leer y escribir, y
entre ese buen hombre y yo te dimos las nociones elementales con que
apareces superior a todo este señorío hecho de pronto que sólo brilla
por el oro ganado sabe Dios cómo.»

Andando, andando, y cuando íbamos frente al Hospicio, pasó junto a
nosotros rapidísima una figura de mujer, que me tocó en el codo y siguió
su camino con la velocidad del viento. De lejos me miró sonriente: era
una \emph{Efémera}. No bien rebasamos el terreno antaño llamado los
\emph{Pozos de Nieve}, donde a la sazón se construían hermosas casas,
pasaron con loca presteza y travesura, no una, sino dos o tres
\emph{Efémeras}, rozándome con dedos ligerísimos como para hacerme
cosquillas. Desparecieron delante de nosotros, perdiéndose entre los
grupos de transeúntes, y dejando tras sí ecos de risas livianas y de
interjecciones burlescas.

En estos prodigios del orden quimérico no se fijó Casiana, y sí lo hizo
con atención discreta en que era la hora de comer y debíamos volvernos a
casa. Aferrado a una idea tenazmente alojada en mi cerebro, propuse
hacer rabona en nuestra hospedería, y retrocediendo algunos pasos nos
metimos en el bodegón llamado \emph{La Criolla}. Pedimos para
sustentamos dos raciones de \emph{batallón}, un besugo, vino y café.

O yo me había vuelto tarumba, o en una mesa no distante de la nuestra
estaban dos \emph{Efémeras} vestidas de negra túnica, manducando
tortilla con jamón, a la que siguieron sendas raciones de pepitoria. En
lo restante del local almorzaban tranquilamente hombres y mujeres, sin
reparar en las fantásticas hembras que eran tal vez proyección de mis
alborotados pensamientos.

Mientras comíamos con buen apetito, di a Casiana una lección de
Historia, enlazando, como es uso y costumbre de todo buen narrador de
las cosas públicas, lo presente palpitante con lo pretérito fosilizado
ya en las capas geológicas del Tiempo.

He aquí fielmente copiados mis pinitos históricos: «Nuestra respetable
amiga doña Segismunda, la Marquesa de San Epifanio, la de Cucúrbitas y
otras tales, están locas de contento con la venida de los frailes que,
lanzados de las Galias a puntapiés, pasan la frontera esperando
encontrar aquí comederos bien provistos por la piedad española. Esas y
otras damas de la misma flaca mentalidad, se aprestan a rascarse el
bolsillo para favorecer a los inmigrantes consagrados al servicio de
Dios Nuestro Señor. Doña Segismunda entiendo que no se correrá mucho,
porque es larga en el prometer y muy encogida en el dar. Otras señoras,
las antes citadas así como las Emparanes, Zuredas y Landazuris, serán
algo más pródigas en el socorro de la frailería galicana. Pero todas
ellas juntas no llegarán a la inaudita magnanimidad de la eximia Duquesa
de Pastrana, que ha legado íntegramente los cuantiosos bienes raíces,
urbanos y suntuarios de su ilustre Casa, opulenta rama del árbol del
Infantazgo, a los caballeros de Loyola. Esta sacra y militar Orden ha
venido a ser casi tan poderosa como el Estado mismo.

»Constituyen el cuantioso donativo el soberbio palacio donde moró
Napoleón I cuando vino a poner sitio a Madrid en Diciembre de 1808,
inmensos terrenos de labor y de monte en el término de Chamartín de la
Rosa, donde ya se trata de formar una población suburbana, otro palacio
en la Plaza de Leganitos esquina a la calle de los Reyes, las casas de
la calle de Isabel la Católica y de la Flor Baja, fincas rústicas en la
provincia de Guadalajara, una millonaria riqueza mobiliaria y muchos
cuadros de mérito, entre los cuales había uno de Rubens, muy famoso, que
los felices herederos vendieron a Rostchild en tres millones de reales.

---¿Pero esa señora---dijo Casianilla espantada---no tenía parientes a
quien legar su riqueza?

---Sí que los tenía. A unos sobrinos, no sé si en segundo o tercer
grado, les favoreció la Duquesa con piadosas mandas para que no les
faltase un cocido. No hizo más la señora por la prisa que tenía en subir
al cielo para recoger el galardón de su extremada santidad. Los
ignacianos, caballeros y caritativos en este caso, determinaron educar
gratuitamente a los hijos de la olvidada parentela, y a una sobrina de
la santa testadora quieren casarla con un caballero chileno muy rico,
para que todos queden contentos.»

Despachado el \emph{batallón}, y antes de emprenderla con el besugo,
proseguí mi leccioncita con el siguiente paralelo histórico, que a mi
parecer no carece de enjundia: «Recordarás, Casianilla de mis
entretelas, que cuando comencé tu educación hice que te fijaras en las
correrías de diferentes pueblos por el territorio de esta península.
Bien enterada quedaste de la entrada de los fenicios, de los romanos, de
los cartagineses, de los visigodos, y por fin, de los árabes. Luchó la
primitiva raza española con tales pueblos, sin lograr impedir que
ocuparan y explotaran una parte o el todo de nuestro suelo durante años,
lustros o siglos. Determinan dichas ocupaciones las diferentes etapas o
períodos históricos de España. Pues bien, el regalo que ha hecho la
Duquesa de Pastrana a los caballeros de San Ignacio, marca el dominio de
estos en el solar hesperio por un lapso de tiempo que nadie puede
precisar. En la santísima dama linajuda y generosa tienes otro
Midácrito, otro Asdrúbal, otro Sertorio, otro Ataúlfo, otro Tárik, y
ella nos trae una nueva intrusión de gente, a la cual habrá que vencer y
despedir como fueron vencidos y mandados a paseo los anteriores
bárbaros.

»Presumo yo que los guerreros de la faja negra, traídos ahora por una
dama, cuando se aseguren en el territorio recientemente adquirido,
extenderán su dominio a todas las esferas y serán nuestros amos.
Fortalecerán su poder educando a las generaciones nuevas, interviniendo
la vida doméstica, y organizando sus ejércitos de damas necias y
santurronas, paulatinamente dotadas con el armamento piadoso que les
llevará a una fácil conquista. Preparémonos, ¡oh Casiana de mis
pecados!, y pues sufrimos esclavitud, seamos cautos y comedidos con
nuestros dominadores, hasta que llegue, si es que llega en vida nuestra,
el momento de darles la zancadilla. Cuando salgamos de paseo y nos
encontremos con un ignaciano, yo me quitaré el sombrero y tú darás una
discreta cabezada en señal de aparente sumisión, rezongando para nuestro
sayo: \emph{Adiós, Reverendo, vive y triunfa, que ya te llegará tu
hora}.»

\hypertarget{xxiii}{%
\chapter{XXIII}\label{xxiii}}

Mientras tomábamos café salieron presurosas las dos \emph{Efémeras}, y
una de ellas, en quien creí reconocer a la que me dio la pluma milagrosa
en la plazuela de Santa Ana, dijo, tocándome en el codo: «Aprisita, que
es tarde\ldots» Al pasar las dos rapazuelas del bodegón a la calle,
advertí que sus flotantes túnicas se trocaron de negras en verdes.

Reparadas las fuerzas con el sabroso condumio, Casiana y yo seguimos
paseando. Nuestra lenta y maquinal andadura nos llevó por los
\emph{Pozos de Nieve} y la antigua Ronda de Santa Bárbara hasta
encontrarnos, sin saber cómo ni por qué, en el \emph{Campo del Tío
Mereje}, lugar asoleado y polvoriento que en verano suele ser invadido
por los jayanes que apalean alfombras, y en todo tiempo es academia
donde maestros de tambor enseñan a los quintos el paso redoblado, el
paso lento, y demás fililíes del sonoro parche guerrero.

Al llegar nosotros al ejido, que antaño debió de ser Eras de Madrid,
vimos tan sólo unos hombres que machacaban cañas para tejer cañizos de
cielo raso. Nos entreteníamos en contemplar aquella ruda faena cuando
Casianilla, mirando al cielo, exclamó asustada: «¡Cristo bendito! ¿No
ves el sin fin de aves que giran en el aire trazando círculos con aleteo
y greguería infernal? Parece que bajan hacia nosotros. ¿Serán estas las
brujas, que de día vienen a reconocer el lugar donde han de reunirse por
la noche en juntas y concilios demoníacos?»

Alcé yo mis ojos al cielo y dije a mi amiga: «No son brujas, Casiana.
Son las \emph{Efémeras}, espíritus mensajeros de lo que en el mundo
llamamos la Actualidad. Traen y llevan el suceso del día. Aquí se
congregan sin duda para distribuirse el trabajo y ver a dónde transmiten
sus raudas informaciones. No tengas miedo, que aunque algunas veces son
portadoras de mentirijillas o falsedades inocentes, no hacen daño a los
mortales, sino antes bien los entretienen y halagan. ¿Ves cómo abaten el
vuelo, acercándose cada vez más a nosotros? Parece que quieren
conversación. Has de saber, hija mía, que son muy traviesas y
habladoras.»

Gradualmente descendían las sílfides en su giro vertiginoso, y nos
aturdían con aquel rumor, que no sé si era cháchara o graznido, bullanga
de risas o estridentes exclamaciones de alegría burlesca. Con rápida
inspiración pedí a los tejedores de cañizo que nos prestasen dos cañas,
y pertrechados Casiana y yo con estas inocentes armas acometimos a
cañazo limpio a las \emph{Efémeras}, cuando ya pasaban rozando nuestras
cabezas.

Por fin logré atrapar a una, cogiéndola por la túnica, y la traje al
suelo. Era lindísima, sus mejillas coloradas echaban fuego, sus ojos
luz, sus cabellos negros y rizados delataban las manos del viento
juguetón.

«¿De dónde vienes tú?---le dije.---¿Has visto entrar en España muchos
frailes?

---Sí, señor don Tito---respondió ella con amable donosura.---Yo
pertenezco al grupo \emph{Céfiro}, y trabajo en la parte de los aires
que ustedes llaman Noroeste. En Coruña vi entrar una partida de
hombrachos vestidos de estameña y con unas correas llenas de nudos. Eran
franciscanos. Llegaron en un vapor. Salieron a recibirles muchos señores
beatos, y las damas pías les enviaron a su alojamiento jamones y tortas
de dulce. Al día siguiente desembarcó otra caterva de frailes, con
diferentes vestiduras, y marcharon a Santiago llamados por el Arzobispo,
que les tenía dispuesto un hermosísimo convento. Mi hermana, que estaba
en Vigo viéndoles venir, presenció el desembarco de \emph{un porción} de
gandules que dijeron ser de los de Santo Domingo. Al instante partieron
para Pontevedra, donde ya les tenían apercibida casa cómoda y mesa bien
provista de cuanto Dios crio.»

Casiana logró atrapar otra ninfa, rubia como las espigas, de ojos
azules, la cual, antes que la interrogaran, se arrancó con esta graciosa
respuesta: «Yo soy del grupo \emph{Boreas}, que vosotros decís Norte, y
en la frontera de Irún he visto entrar una patulea sin fin de frailucos.
Unos traían baberos blancos, melenitas que les tapaban las orejas y
sombreros tricornios que parecían cosa de máscara. Dijeron que venían a
España para poner escuelas y enseñar a los niños. ¡Bonitas cosas les
enseñarán!\ldots{} Luego entraron otros, vestidos de blanco y canelo,
lucios y fornidos como mozos de cuerda. Parece que estos son carmelitas.
Salieron a recibirles la mar de señoras aristocráticas y ricachonas, que
les besaban los rosarios, popándoles y haciéndoles fiesta como si les
hubieran conocido toda la vida. A ellos se les saltaban las lágrimas de
contento, y miraban a todos lados en busca de alguna mesa donde pudieran
matar el hambre atrasada que de Francia traían\ldots{} ¡Pobre España:
buena nube de langosta te ha caído!»

Sin necesidad de esgrimir nuestras cañas, otras \emph{Efémeras} fueron
bajando, alegres y decidoras. Una de ellas, de cabello castaño y ojos
verdes, ondulante y saltarina, vestida de túnica roja, nos dijo: «Mi
puesto de vigilancia está entre las regiones de \emph{Coecias} y
\emph{Apellotes}, que es como decir Nordeste y Este. Vi entrar por el
golfo de Rosas una barcada de dominicos, y otra de trinitarios, que
fueron bien acogidos en la playa y marcharon a ponerse bajo la custodia
de los obispos de Gerona y de Vich. Mis hermanas y yo presenciamos en
Barcelona la llegada de una banda de capuchinos procerosos, bien cebados
y con unas barbas hasta la cintura. Al pasar por la Rambla les arrearon
una silba espantosa. Los frailes barbudos, azuzados por mujeres y
chiquillos, tuvieron que buscar refugio en le iglesia del Pino, a donde
acudió el Gobernador con policía para sacarlos de aquel trance y
llevarles con mucho mimo al Palacio episcopal. El señor Prelado, después
de tenerlos varios días en su casa a mesa y mantel, les alojó solícito
en varios conventos de Cataluña.»

Otra de las mensajeritas aéreas nos contó que en Tortosa dieron fondo
unos benedictinos jacarandosos que, según se dijo, venían a montar en
Tarragona fábricas de licores tan ricos y celebrados como los que en
Francia elaboraban\ldots{} Compadeció seguidamente una nueva
\emph{Efémera} de túnico negro recamado de oro, quien, después de
declarar que venía de la región del \emph{Eurus} (Sudoeste), nos informó
de que en Cartagena habían penetrado mesnadas de agustinos-recoletos,
los cuales tomaron al punto el caminito de Orihuela, donde el Obispo les
tenía prevenido un holgado monasterio. Allí se instalaron todos los que
en él cabían. Los demás recibieron albergue en el Seminario, hasta que
se les habilitara definitiva vivienda en un convento de Alicante. Añadió
la informadora que, tras de los agustinos-recoletos, llegó un nutrido
cargamento de los frailecitos de babero y tricornio. Parte de estos
quedaron en Cartagena, bajo la tutela y amparo de una junta de damas
sumamente pías y rezadoras, y los otros tomaron el tren para irse a
Murcia, pues allí les esperaban con los brazos abiertos individuos del
Comité conservador y el Prelado de la diócesis.

Recorriendo el cuadrante hacia la región \emph{Notus}, entiéndase Sur,
otra ninfa de los aires, no menos graciosa que sus hermanas y muy
bachillera, nos contó que por Almería había penetrado un buen golpe de
monjas, llamadas descalzas aunque todas llevaban medias y zapatos.
Venían afligidas del mareo y de la inanición. Pero al punto se las
socorrió con cuanto pudieran necesitar. Con ellas desembarcaron unos
frailucos mal trajeados, desnudos de pie y pierna, \emph{si que también}
muertos de hambre. Las esposas del Señor encontraron su nido y agasajo
en la propia ciudad de Almería, y los frailachos se metieron tierra
adentro a la querencia del Obispo de Guadix.

Con todo lo referido no es completa la información \emph{efemerídea}. Yo
la resumo y sintetizo, agregando otras noticias y datos que nos dieron
las vagarosas hijas del viento. Por Sevilla hubo también inundación de
religiosas clarisas; a Valencia llegaron trapenses y paúles; la frontera
de Francia, por Navarra y la Seo de Urgel, dio paso a espesas caravanas
de salesianos, premonstratenses, terciarios, redentoristas, adoratrices,
trinitarias, capuchinas, ursulinas y otras muchas castas y familias del
inmenso mundo monástico.

Cuando ya las aladas mensajeras comenzaban a remontarse de nuevo en los
aires, apareció la \emph{Efémera} mía, la de Tafalla, que en aquella
ocasión me pareció capitana de todas ellas, la que al pisar el suelo
tomaba apariencias marmóreas y formas del más puro helenismo.

«¿A dónde vais ahora?---le pregunté tembloroso.

Ella me contestó con suprema tranquilidad: «Vamos a llevar por todo el
mundo las nuevas de esta plaga de insectos voraces que devastará tu
tierra.»

Y quitándole a Casianilla la caña que esta conservaba en sus manos, la
figura estatuaria azuzó a las \emph{Efémeras} rezagadas. Todas
remontaron el vuelo en alegre remolino bullicioso.

\hypertarget{xxiv}{%
\chapter{XXIV}\label{xxiv}}

Las vimos subir rápidamente hasta una región muy alta del espacio, donde
se fraccionó la bandada en grupos que partieron hacia distintos puntos
del horizonte.

Emprendimos Casiana y yo nuestro regreso al centro de Madrid, buscando
la vuelta de Recoletos por la Ronda de este nombre y las inmediaciones
de lo que fue huerta de las Salesas. Por aquella parte, la Villa trataba
de embellecerse, y abría en los solares polvorosos la cimentación para
nuevas y elegantes casas de vecindad. Charlando de las peregrinas cosas
que habíamos visto y oído, caminábamos a la ventura, guiados, más que
por la intención, por el instintivo movimiento de nuestros propios
pasos.

Sin darnos cuenta de ello, costeamos la maciza fundación de doña Bárbara
de Braganza, y por calles a medio construir llegamos a internarnos en el
Parque de Buenavista. Hicimos alto para descansar en un banco de las
rampas que dan a la calle de Alcalá, frente al palacio de Alcañices.
Aunque el sol picaba templando el ambiente invernal, yo sentía un frío
que no pude mitigar embozándome en mi capa hasta las narices, porque
aquella tiritona era síntoma febril de mi estado anímico al considerar
la invasión monástica, principio de un período histórico desastroso para
nuestra pobre España.

A mis quejas lastimosas contestó Casianilla: «Como nosotros no podemos
impedir que España se convierta en un gran monasterio, nuestro papel es
ver y esperar. Si llega el caso de que no haya más remedio que ser yo
monja y tú fraile, no te apures, Tito, que ya encontraremos conventos
donde convivan ambos sexos.

---Así tendrá que ser, nenita---dije yo, y como estaba helado propuse
que siguiéramos andando hasta la calle de Sevilla, y que allí tomásemos
la dirección de nuestra casa, con escala en algún café para matar las
horas de la tarde.

Por ambas aceras de la calle de Alcalá bajaba un tropel de paseantes que
iban a tomar el sol en el Prado y el Retiro. Eran a mi parecer
funcionarios que abandonaban la ociosa oficina para espaciarse con la
señora y los niños, pensionistas de poco pelo, tenderos desocupados,
rentistas de mediano pasar, provincianos con dinerito fresco, que
practicaban la deambulación como un obligado empleo de la actividad en
los días serenos.

Por el centro de la calle rodaban los mismos carruajes que habíamos
visto el día anterior y todos los días, conduciendo a las damas de
siempre, bien emperifolladas, y a los señores del margen que acompañaban
a sus esposas en el asiento zaguero de las carretelas. Acrecían el
tumulto los gallardos jinetes y los caballos que guiaban faetones o
tílburis con la pericia de consumados aurigas. En las caras de toda esta
gente, así la de a pie como la de coche, así la de alto como la de
rastrero pelaje, observé una tranquilidad paradisíaca. Sus cabezas no
alojaban otra idea que la del momento presente, el goce del paseo al
sol, la vanidad de exhibirse con galas y arreos de distinción
fantasiosa.

¡Pobres majaderos! Desconocían en absoluto la gravísima situación de
nuestro país, el momento histórico, semejante a la entrada de los
cartagineses ávidos de riqueza, de los bárbaros visigodos o de los
insaciables y feroces agarenos. Nada sabían, nada sospechaban: se
enterarían de la nueva esclavitud cuando esta ya no tuviese remedio. Me
costó trabajo contener este grito de alarma: «¡Bobalicones, despertad de
vuestra modorra estúpida! ¡No tenéis gobernantes que sepan contener, ya
que no extirpar, la horrible plaga que se os viene encima!»

Al pasar por la calle de Sevilla entramos en la tienda de mi amigo
Matías Luengo, sobrino del famoso comerciante, parlanchín y entrometido
don Plácido Estupiñá, de quien tanto hablé en diferentes ocasiones.
Traficaba Matías en objetos de escritorio. Comprámosle un paquete de
sobres, charlamos, le pregunté si estaba contento de su negocio, y me
contestó que de sus ventas no sacaba más que lo preciso para mal vivir.
El Cielo le había dado cuatro hijos, y su mujer, que era una coneja, le
traería el quinto retoño para Febrero próximo. En vista de este
crecimiento del familiaje, pensaba añadir a su tráfico el de
devocionarios, florilegios, novenas, cilicios, recordatorios de
difuntos, estampitas de todos los santos del cielo, escapularios y demás
chirimbolos pertinentes a la santa Religión.

Yo le felicité, palmoteándole en los hombros, y le dije: «Eres un genio,
Matías. Has previsto el fetichismo farandulero a que nos llevará la
maldita Restauración. Ahora empieza, fíjate bien, ahora empieza el
reinado de la Muerte y de las santurronerías bobaliconas. Tú serás rico.
Haz todos los hijos que puedas, que el negocio místico te dará pan para
ellos, y para tus nietos y biznietos, hasta la cuarta generación. Adiós,
chico. El Espíritu Santo ha entrado en tu casa. Adiós.»

A lo largo de la calle íbamos tropezando con cómicos y toreros, y en
ellos vi caras satisfechas aunque perecían de hambre por la falta de
contratas. A mi paso por diferentes tiendas vi también sastres, joyeros
y perfumistas, que parecían muy contentos viviendo al día con menguadas
transacciones. Junto a nosotros pasaron dos curas, ante los cuales me
quité el sombrero haciendo acto de sumisión y reverencia. Era muy cuerdo
y saludable vivir en santa paz con nuestros opresores.

En la esquina del callejón de Gitanos encontramos a Delfina Gil. Después
de saludarme con rígida frialdad, me dijo que iba a poner una nueva
Funeraria de gran lujo en la propia Carrera de San Jerónimo, y que
introduciría en España las últimas novedades en féretros de cinc
sobredorados y en carrozas-estufas a la \emph{gran Daumont}. Pensaba
adornar su escaparate con espléndido surtido de coronas fúnebres de hilo
de cristal, elegantísimas, y con unos angelitos, arrodillados, que daban
el opio. La colmé de parabienes, vaticinándole un éxito formidable.
Merecía enterrar la vida española con todo el boato y \emph{chic} de las
artes mortuorias.

Seguimos, y al embocar la Carrera de San Jerónimo, tropecé de manos a
boca con Vicente Halconero, que salía del Casino. Cortés y afable como
siempre estrechó mis manos, no escatimando un gentil saludo ceremonioso
a mi compañera humilde.

«Ya sabrá usted---me dijo---que está próximo el advenimiento de los
Constitucionales al Poder. El turno se impone, y la tocata liberal ha de
sustituir a la tocata conservadora. Espero yo que entre ambas músicas
haya bastante diferencia, así en lo fundamental como en lo
externo\ldots{} Entiendo que tendremos elecciones generales en Febrero o
Marzo, y usted no me negará entonces lo que tantas veces le pedí.
Aceptará usted un acta de diputado, y en los escaños de la mayoría
lucharemos juntos por el progreso, con su poquito de morrión y sus
toques democráticos, todo ello dentro del orden más perfecto.

---Sí, sí, Vicentito---le contesté, con la socarronería que en aquella
hora dominaba en mi ánimo.---Puede usted hacer de mí lo que quiera. Y si
tocan a repartir algunos destinillos denme a mí el de Inspector de
Monjas, quiero decir, de los monasterios que han de ser creados para
reunir los dos sexos en la vida contemplativa.

---¿Pero qué dice el amigo Tito? ¿Se ha vuelto loco?\ldots{} ¡Ah! Es que
a usted le solivianta lo que se cuenta por ahí de si vienen o no vienen
los religiosos regulares expulsados de Francia. No haga usted caso.
Ataremos corto a los que vengan no más que a darse buena vida, y
recibiremos con estimación a los que traigan la idea de establecer en
España buenos Colegios, donde podamos dar decorosa educación a nuestros
hijos.»

No quise hablar más y me despedí de Halconero con breves razones
amistosas, lamentando que un caballerete tan espiritual no apreciara el
feo cariz del nublado cartaginés y agareno que entenebrecía el cielo
español, ni viera claramente que se iniciaba un período de larga y
pavorosa esclavitud. ¡Pobre Vicentito, tan joven, tan simpático, y ya
contagiado del negro y pestilente virus!

\hypertarget{xxv}{%
\chapter{XXV}\label{xxv}}

Casiana y yo nos colamos en el café de \emph{La Iberia}, dirigiéndonos a
las mesas donde habitualmente concurrían mis amigos. En efecto, allí
estaban Campo y Navas, Llano y Persi, Casalduero, y Carratalá. En una
piña inmediata vi a Díaz Quintero, republicano, que alternaba con
Fernández Bremón y Mariano Zacarías Cazurro, conservadores, y con Pablo
Cruz, León y Llerena, Zoilo Pérez y Cándido Martínez, sagastinos.

Apenas cambié con ellos los primeros saludos, algunas palabras
referentes a sucesos de actualidad, comprendí que ninguno de aquellos
esclarecidos ciudadanos paraba mientes en el capital suceso histórico
que a mí me volvía tarumba. O lo ignoraban, o las menudencias y
chismorreos políticos les impedían fijarse en los hechos que, afectando
intensamente al porvenir de la Patria, se nos presentan revestidos de
una insignificancia traicionera. Los afectos a la Situación imperante
aseguraban que había Gobierno de Cánovas para rato. Al proclamarlo así,
reforzaban su opinión con apuestas humorísticas de cinco duros contra
dos reales. Los otros, entonando con diferentes inflexiones el
\emph{esto se va}, vaticinaron rotundamente que antes de dos meses
cogería Sagasta las riendas y la tralla del Poder.

De pronto llegaron a nuestras mesas otros dos individuos, cuyos nombres
no son del caso. Con frase tajante y enfática \emph{sostuvieron la
tesis} de que don Antonio se había hecho imposible por su soberbia, y
porque no supo desprenderse a tiempo de los pulpos del
\emph{moderantismo}. Un tercer sujeto, que presuroso vino de las mesas
interiores, nos dijo en tonillo parlamentario: «¡Ah, señores! Mi
\emph{teoría} es que política nueva pide hombres nuevos. Las cosas caen
del lado a que se inclinan. O la regia prerrogativa no sabe lo que se
pesca, o ha de poner en seguida en manos de don Práxedes el timón de la
nave del Estado.»

Reunidos todos, enzarzaron sus ágiles lenguas en el discreto político
sin tocar ningún punto de interés público, picoteando tan sólo en las
cuestiones de orden burocrático, que eran para los Fusionistas o
Constitucionales el único imán de sus pueriles ambiciones. Diferentes
nombres sonaron de mesa en mesa para distribuir entre ellos los cargos
políticos de la nueva Situación, Direcciones generales y Gobiernos de
provincia. Entre aquellos ociosos charlatanes no faltaron algunos vivos
que graciosamente se adjudicaron las mejores prebendas. A la entrada de
los agarenos, o si se quiere cartagineses, no consagró ninguno de los
allí reunidos, hombres de diferente cartel político, una sola palabra.

Asqueado de la frivolidad de tales majaderos, que con raras excepciones
sólo apreciaban la vida pública por los apremios de su vanidad o de su
flaco peculio, pretexté para retirarme un repentino dolor de estómago
con ganas de vomitar, y cogiendo del brazo a Casianilla nos plantamos en
la calle. ¿A dónde iríamos? A casita, a mi caverna solitaria, o a darle
albricias a nuestra coruscante amiga la Excelentísima señora Condesa de
Casa Pampliega.

Ibamos por la calle del Lobo, y en los extremos de ella vimos lujosa
berlina parada junto a una puerta humilde. De esta salió una dama en
quien al punto reconocí a la Marquesa de Villares de Tajo, mujer
talentuda y de historia, vistosa todavía y de buen talle aunque había
rebasado con creces las fronteras del medio siglo. En su coche partió
hacia la Carrera de San Jerónimo. ¡Pobrecilla! Venía de parlotear con
los \emph{Caballeros de la Tenaza}, albergados a espaldas de la iglesia
de San Ignacio. Pensé que ya le estaban ajustando las cuentas para
mandarla al otro mundo bien limpia de pecados, y aliviada del peso de
sus cuantiosos intereses.

Permanecíamos Casiana y yo junto a la puerta mísera, contemplando la
lobreguez del hondo zaguán, cuando vimos que de aquellas tinieblas
salían un cura joven, gallardo, desenvuelto, y una señora hermosísima.
¡Oh asombro de los asombros! La señora era Lucila Ansúrez, más conocida
en estas historias por el lindo mote de \emph{La Celtíbera}.

\hypertarget{xxvi}{%
\chapter{XXVI}\label{xxvi}}

La nieve que blanqueaba el cabello de la viuda de Halconero no era
estorbo de su belleza, que se defendía bravamente contra la edad,
frisante ya en los cincuenta años si no fallan mis cómputos
cronológicos. Apenas me vio en la calle, honrome Lucila con expresivo
saludo, presentándome incontinenti al clérigo, mocetón elegante, limpio,
y cumplido galán por su melosa cortesía.

«El Padre Garrido---dijo \emph{La Celtíbera} en la ceremonia de la
presentación.---Don Proteo Liviano\ldots»

Al pronunciar Lucila mi nombre se arrancó el jesuita con estas
hiperbólicas alabanzas: «¡Ah, el señor Liviano! Mucho gusto en verle. Ya
le conocía y le admiraba como historiógrafo eminente. Yo también soy
aficionado a la Historia, y en el nuevo Colegio de Chamartín tendré a mi
cargo esa importante asignatura. Mi ciencia es corta; pero supliré la
escasez de conocimientos con mi firmeza de voluntad, imitando en lo
posible al maestro que me escucha\ldots»

Intervino Lucila con esta donosa corrección: «No se achique, Padre
Garrido\ldots{} Y usted, amigo Tito, no le haga caso, que la más alta
virtud de este santo varón es la modestia, una modestia verdaderamente
angelical.»

Al protestar el clérigo de los elogios de \emph{La Celtíbera}, llegó
hasta ruborizarse, y yo, penetrando en la médula de aquel carácter más
fino que el coral y con más conchas que un galápago, le devolví sus
lisonjas con este golpe de incensario:

«Bien sé con quién hablo, reverendo Padre. He leído en el \emph{Iris de
Paz} la respuesta que da usted a las diatribas con que \emph{La Ciudad
de Dios}, el periódico de los agustinos, trata de mermar las glorias de
La Compañía. Es usted escritor de primer orden y dialéctico formidable.
Así como suena\ldots{} En esfera humilde, hago yo lo que puedo por la
ilustración del pueblo español, tan católico como desgraciado\ldots{}
Esta señora que a mi lado está es mi esposa, doña Casiana \emph{Coelho},
insigne pedagoga, maestra en todas las artes y ciencias, de quien tomo
ejemplo, apropiándome su saber al mismo tiempo que imito sus
virtudes\ldots{} virtudes excelsas, noble señora y caballero tonsurado,
pues en mi dulce cónyuge se confunden y amalgaman la prudencia, la
castidad, la paciencia, la caridad, las artes caseras, el filosofismo
más espiritual y el don de escudriñar las obscuridades del
porvenir\ldots»

Colorada y balbuciente, Casianilla quiso desmentir los embustes que en
honor suyo desembuché, y en el rostro del clérigo advertí un ligero
mohín de desconfianza: sin duda interpretaba en sentido burlesco mi
lenguaje hiperbólico. Lucila, también un poquito recelosa, inició la
marcha hacia la calle del Prado. Detrás fuimos los tres, y yo,
arrimándome al Padre Garrido, de quien no quería separarme sin soltarle
alguna barbaridad, acaricié su tímpano con esta blanda ironía:

«Dios me ha deparado el placer de ofrecer a usted hoy mis respetos,
Padre Garrido\ldots{} Ya sé, ya sé que ayer llegó usted de un corto
viaje a París, a donde fue con el mandato de organizar la nueva traída
de jesuitas para el Colegio de Chamartín de la Rosa, institución
educatriz que será el coronamiento de la sublime longanimidad de la
señora Duquesa de Pastrana.

---El objeto de mi viaje a Francia no está bien que yo lo diga
---replicó el clérigo un tanto amoscado.---Sólo indicaré a usted que
hace tres días estaba ya de regreso en la Villa y Corte, donde seguiré
hasta que lo disponga quien puede hacerlo, consagrado al servicio del
Señor y a la salvación de las almas españolas.

---A lo mismo nos dedicamos nosotros---dije, poniéndome la mano, no
precisamente en el corazón, pero muy cerca de él.---Mi esposa y yo
también servimos a Dios y salvamos almas cuando se tercia\ldots{} En la
persona de usted, Padre Garrido, reverenciamos a la milicia cristiana, a
quien el Altísimo otorga el mandato de gobernar a los pueblos y
conducirlos a la eterna gloria. Ya nuestra España es de ustedes. Aquí no
reina Alfonso XII sino el bendito San Ignacio, que a mi parecer está en
el cielo, sentadito a la izquierda de Dios Padre\ldots{} Los españoles
somos católicos borregos, y sólo aspiramos a ser conducidos por el
cayado jesuítico hacia los feraces campos de la ignorancia, de la santa
ignorancia, que ha venido a ser virtud en quien se cifra la paz y la
felicidad de las naciones\ldots{} Nos prosternamos, pues, ante el negro
cíngulo, y rendimos acatamiento al dulcísimo yugo con que se nos oprime
\emph{ad majorem Dei gloriam}.»

No se le escapó al ladino y sutil clérigo el saborete irónico que ponía
yo en mis palabras. Con forzada sonrisa y frunciendo el ceño, doble y
equívoca expresión facial de su índole solapada, el joven Padre me
alargó la mano buscando la fórmula de despedida. También Lucila mostraba
deseo de cortar nuestra conversación, poniendo tierra entre los dos
grupos, y así me dijo:

«Sigan ustedes paseando, Tito; el Padre y yo tenemos que ir a la
Nunciatura para un asunto\ldots{}

---La Virgen les acompañe, reverendo caballero y señora ilustre---dije
yo destapando mi cabeza.---Y si se acuerdan de estos pobres pecadores,
tengan la bondad de implorar para nosotros la bendición apostólica, por
mediación del santísimo Nuncio\ldots{} Adiós, adiós.»

\hypertarget{xxvii}{%
\chapter{XXVII}\label{xxvii}}

Viéndoles partir hacia la Plaza del Ángel, Casianilla, súbitamente
alterada y colérica, me dijo: «Si estuviéramos en descampado les
apedrearíamos. ¿No te parece?

---No, hija mía, no---repliqué yo, cogiéndole el brazo con que imitaba
el manejo de la honda.---Modera tu arrebato bélico, que los tiempos son
más de paciencia solapada que de fiereza impulsiva. Si apedreáramos,
podría suceder que nuestros tiros no dieran en la cabeza del Reverendo,
que bajo la capa de su finura exquisita esconde las intenciones de un
grandísimo bellaco, y fuesen a descalabrar a la hermosa
\emph{Celtíbera}, persona ciertamente estimable y digna de
respeto\ldots{} Esta buena señora fue en sus días juveniles la corza
ligera y elegante que a todos cautivaba; ahora es la oveja tarda y
simplísima que no puede con el peso de sus lanas\ldots{} No hemos de ver
en las beaterías de Lucila un movimiento espontáneo de su ánimo, el
cual, digan lo que quieran, aún conserva la independencia celtíbera. Sus
concomitancias con lo que podríamos llamar \emph{el elemento} jesuítico,
son puro artilugio para ponerse a tono con la caterva elegante y
santurrona que hoy rige los destinos de España. A tal comedia la mueve
el amor de su hijo Vicente, y el anhelo de empujar al chico en su
carrera política. Ya verás, ya verás cómo, auxiliada por \emph{los
padres, las madres y las tías}, consigue hacer Ministro a Vicentito, con
Sagasta o con el demonio coronado\ldots{} \emph{¿Entiendes, Fabia, lo
que voy diciendo?}\ldots{} No debemos acometer a nuestros enemigos con
palo ni piedra. Esperemos a que tomen posiciones y nos manifiesten el
poder de sus armas, y la eficacia de sus ingenios de guerra.

---Está muy bien, Tito mío---dijo Casiana agarrándose de mi brazo.---Y
ahora decidamos si nos metemos en casa o nos vamos a visitar a la señora
Condesa. Quiero ver la cara que pone doña Segismunda cuando se le diga
que el grande hombre del siglo, don Antonio Cánovas, irá pronto a
ofrecerle sus respetos y a darle las gracias por los librachos del
tiempo de la Nanita.

---Yo también deseo contemplar el cariz de nuestra \emph{Medusa} y su
cabellera de serpientes---contesté.---Pero antes, si te parece, debemos
personarnos en la Academia de la Historia, que está muy cerca como
sabes. ¿Te olvidas de que hace unos días tengo allí mi asignación, y aún
no he ido a cobrarla? Lo primero es lo primero, Casianilla. Vamos allá,
vamos.»

Minutos después estábamos en el ancho zaguán de la Academia. Mas no
hallándose presente la señora portera, que según nos dijeron había
subido al segundo piso llamada por el Bibliotecario para que le prestase
servicios de cocina y despensa, aguardamos sentaditos en la modesta
estancia conserjeril, donde pasamos el rato en vagos comentarios sobre
nuestra situación económica, que no era en aquellos días muy despejada.

Llegó en esto el anciano portero, a quien yo con caprichosa travesura
imaginativa daba el nombre de \emph{Tucídides}, por su puesto en aquella
Casa y por el trazo helénico de su rostro visto de perfil. Lamentose el
buen hombre de la ausencia de su esposa, secuestrada por las
impertinencias del señor Bibliotecario, hombre excelente, pero un tanto
enfadoso. Diciéndolo, puso en mis manos el pliego de mi Madre\ldots{}
¡Ay! Fue cual onda luminosa que súbitamente disipó las tinieblas de mi
espíritu.

Retirose \emph{Tucídides}, que tenía precisión de arreglar la Sala de
Juntas para la \emph{tenida} de aquella noche, y nos dejó en la portería
indicándonos que estábamos en nuestra casa y podríamos permanecer allí
todo el tiempo que quisiéramos. Solitos Casiana y yo, abrimos el pliego
y\ldots{} ¡Oh inefable sorpresa y alegría! La Musa excelsa me mandaba
doble suma de la presupuesta para cada mensualidad.

\hypertarget{xxviii}{%
\chapter{XXVIII}\label{xxviii}}

Después de justificar este doble socorro, enumerándome las privaciones y
agobios que había yo de sufrir si me conservaba incorruptible y puro en
medio del general positivismo, la Madre exponía su pensamiento acerca
del porvenir de España en la forma elocuente y profética que traslado a
mis buenos lectores:

«Hijo mío: cuando a fines del 74 te anuncié en una breve carta el suceso
de Sagunto, anticipé la idea de que la Restauración inauguraba \emph{los
tiempos bobos}, los tiempos de mi ociosidad y de vuestra laxitud
enfermiza. La sentencia de mi buen amigo Montesquieu, \emph{dichoso el
pueblo cuya Historia es fastidiosa}, resulta profunda sabiduría o
necedad de marca mayor, según el pueblo y ocasión a que se aplique.
Reconozco que en los países definivamente constituidos, la presencia mía
es casi un estorbo, y yo me entrego muy tranquila al descanso que me
imponen mis fatigas seculares. Pero en esta tierra tuya, donde hasta el
respirar es todavía un escabroso problema, en este solar desgraciado en
que aún no habéis podido llevar a las Leyes ni siquiera la libertad del
pensar y del creer, no me resigno al tristísimo papel de una sombra
vana, sin otra realidad que la de estar pintada en los techos del Ateneo
y de las Academias.

»La paz, hijo mío, es don del cielo, como han dicho muy bien poetas y
oradores, cuando significa el reposo de un pueblo que supo robustecer y
afianzar su existencia fisiológica y moral, completándola con todos los
vínculos y relaciones del vivir colectivo. Pero la paz es un mal si
representa la pereza de una raza, y su incapacidad para dar práctica
solución a los fundamentales empeños del comer y del pensar. Los
\emph{tiempos bobos} que te anuncié has de verlos desarrollarse en años
y lustros de atonía, de lenta parálisis, que os llevará a la consunción
y a la muerte.

»Los políticos se constituirán en casta, dividiéndose hipócritas en dos
bandos igualmente dinásticos e igualmente estériles, sin otro móvil que
tejer y destejer la jerga de sus provechos particulares en el telar
burocrático. No harán nada fecundo; no crearán una Nación; no remediarán
la esterilidad de las estepas castellanas y extremeñas; no suavizarán el
malestar de las clases proletarias. Fomentarán la artillería antes que
las escuelas, las pompas regias antes que las vías comerciales y los
menesteres de la grande y pequeña industria. Y por último, hijo mío,
verás si vives que acabarán por poner la enseñanza, la riqueza, el poder
civil, y hasta la independencia nacional, en manos de lo que llamáis
vuestra Santa Madre Iglesia.

»Alarmante es la palabra Revolución. Pero si no inventáis otra menos
aterradora, no tendréis más remedio que usarla los que no queráis morir
de la honda caquexia que invade el cansado cuerpo de tu Nación.
Declaraos revolucionarios, díscolos si os parece mejor esta palabra,
contumaces en la rebeldía. En la situación a que llegaréis andando los
años, el ideal revolucionario, la actitud indómita si queréis,
constituirán el único síntoma de vida. Siga el lenguaje de los bobos
llamando paz a lo que en realidad es consunción y acabamiento\ldots{}
Sed constantes en la protesta, sed viriles, románticos, y mientras no
venzáis a la muerte, no os ocupéis de \emph{Mariclío}\ldots{} Yo, que ya
me siento demasiado clásica, me aburro\ldots{} me duermo\ldots»

\flushright{Madrid-Santander.—Marzo-Agosto de 1912.}

~

\bigskip
\bigskip
\begin{center}
\textsc{fin de cánovas}
\end{center}

\end{document}
